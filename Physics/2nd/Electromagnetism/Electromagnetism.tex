\documentclass[class=article,10pt,crop=false]{standalone}
\usepackage{standalone}
\usepackage{preamble}

\begin{document}
\begin{multicols}{2}[\section{Electromagnetism}]
\subsection{Electrostatics}
\begin{concept}[Columb's law]
Let $q_1,q_2$ be two point charges at positions $\textbf{r}_1,\textbf{r}_2$, respectively. Then the force $\textbf{F}_2$ experienced by $q_2$ in the vicinity of $q_1$ is given by $$\textbf{F}_{12}=kq_1q_2\frac{\textbf{r}_1-\textbf{r}_2}{\|\textbf{r}_1-\textbf{r}_2\|^3},$$ where $k=\frac{1}{4\pi\varepsilon_0}$ and $\varepsilon_0=8,854\;F/m$ is the vacuum permittivity.
\end{concept}
\begin{concept}[Electric field]
We define the \textit{electric field} $\textbf{E}$ as the force per unit of charge. For a point charge, we have that the electric field created by $q_1$ i the position of $r_2$ is $$\textbf{F}_2=q_2\textbf{E}_1(\textbf{r}_2),\qquad\textbf{E}_1(\textbf{r}_2)=kq_1\frac{\textbf{r}_1-\textbf{r}_2}{\|\textbf{r}_1-\textbf{r}_2\|^3}.$$
\end{concept}
\begin{concept}[Superposition principle]
Let $\rho(\textbf{r})=\frac{dq}{d\mathcal{V}}$ be the volume charge density of an object. Then we have that $$\textbf{F}=\int_\mathcal{V}\rho(\textbf{r})\textbf{E}(\textbf{r})d^3r,\qquad\textbf{E}(\textbf{r})=k\int_\mathcal{V}\rho(\textbf{r}')\frac{\textbf{r}-\textbf{r}'}{\|\textbf{r}-\textbf{r}'\|^3}d^3r'.\footnote{Analogously we can define $\sigma(\textbf{r})=\frac{dq}{d\mathcal{S}}$ to be the surface charge density and $\lambda(\textbf{r})=\frac{dq}{d\ell}$ to be the linear charge density, and the integrals become as expected.}$$
\end{concept}
\begin{concept}[Gau\ss's\space theorem]
\begin{equation}
    \nabla\cdot\textbf{E}=\frac{\rho}{\varepsilon_0}\iff\oint_\mathcal{S}\textbf{E}\cdot\textbf{n}d\mathcal{S}=\frac{1}{\varepsilon_0}\int_\mathcal{V}\rho(r)d^3r=\frac{Q_T}{\varepsilon_0},
    \label{gauss}
\end{equation} where $Q_T$ is the total charge enclosed within $\mathcal{V}$.
\end{concept}
\begin{concept}[Electric potential]
The electric potential $\phi$ in a point $\textbf{r}$ is defined as: \begin{equation}
    \textbf{E}=-\nabla\phi,\qquad\phi(\textbf{r})=k\int\frac{\rho(\textbf{r}')}{\|r-r'\|}d^3\textbf{r}'.
    \label{potential}
\end{equation} And then, $$\phi_a-\phi_b=-\int_b^a\textbf{E}\cdot d\ell=\int_a^b\textbf{E}\cdot d\ell.$$ Alternatively we can define the potential from the electric energy. The work required to move a charge $q$ from $b$ to $a$ is $$W_{b\to a}=-q\int_b^a\textbf{E}\cdot d\ell.$$ And also if we consider the point $P$ as a reference point we can define the \textit{electric energy} $U_a$ as follows $$U_a-U_b=-q\int_P^a\textbf{E}\cdot d\ell+q\int_P^b\textbf{E}\cdot d\ell=-q\int_b^a\textbf{E}\cdot d\ell=W_{b\to a}.$$ An finally we get $$\Delta U(\textbf{r})=q\Delta\phi(\textbf{r}).$$
\end{concept}
\begin{concept}[Poisson and Laplace equations]
Having in account formulas \ref{gauss}, \ref{potential}, we get Poisson's equation $$\nabla^2\phi=-\frac{\rho}{\varepsilon_0}.$$ If $\rho=0$, we obtain Laplace's equation: $$\nabla^2\phi=0.$$
\end{concept}
\end{multicols}
\end{document}