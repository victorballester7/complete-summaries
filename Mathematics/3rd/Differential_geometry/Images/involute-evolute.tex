\documentclass{standalone}
\usepackage{../../../../preamble_formulas}

\def\n{1}
\pgfplotsset{
  layers/standard/.define layer set={
      background,
      pre main,
      main,
      foreground
    }{},
  set layers=standard, % set my custom layer set
}
\begin{document}
\begin{tikzpicture}
  \begin{axis}[
      xmin = -3, xmax = 3, %
      ymin = -0.5, ymax = 5, %
      axis equal image,
      clip mode=individual,
      legend columns = 2,
      legend entries={Catenary,Tractrix},
      legend to name=named,
      legend cell align = {left}, %
      legend image code/.code={%
          \draw[dashed] (0cm,-0.1cm) -- (0.575cm,-0.1cm);
          \draw[solid] (0cm, 0.1cm) -- (0.575cm, 0.1cm);
        },
      %legend style={at={(0,-0.5)},anchor=center}, % position of the legend box and anchor is the point on the box to be fitted exactly at the point of cs:<>,<>. Options are anchor=center,south west,south east,north west,north east,north,south,west...
      title={Tractrix as involute of the catenary},
      name=left plot,
    ]
    \addplot[
      domain=-4:4, %Domain of the fucntion
      samples=200, %This parameter determines the number of point to be plotted for the function, while bigger the number better looks the function.
      smooth, %f we use this option, the compiler makes an interpolation between the point plotted to get a soft appearance for the function.
      thick, %Stroke of the function. Options: ultra thin, very thin, thin, semithick, thick, very thick, ultra thick.
      red, %Color of the function.
      dashed,
      name path=tractrix,
    ]({x-tanh(x)},{1/cosh(x)});
    \addplot[
      domain=-2.5:2.5, %Domain of the fucntion
      samples=200, %This parameter determines the number of point to be plotted for the function, while bigger the number better looks the function.
      smooth, %f we use this option, the compiler makes an interpolation between the point plotted to get a soft appearance for the function.
      thick, %Stroke of the function. Options: ultra thin, very thin, thin, semithick, thick, very thick, ultra thick.
      blue %Color of the function.
    ]{cosh(x)};
    \begin{pgfonlayer}{background}
      \foreach \i in {0,...,\n}{
          \addplot[
            domain=-3:3, %Domain of the fucntion
            samples=10, %This parameter determines the number of point to be plotted for the function, while bigger the number better looks the function.
            very thin, %Stroke of the function. Options: ultra thin, very thin, thin, semithick, thick, very thick, ultra thick.
            blue!20, %Color of the function.
            name path=linetan,
            on layer=background,
          ]{cosh(4/\n*\i-2)+sinh(4/\n*\i-2)*(x-(4/\n*\i-2))};
          %\draw[name intersections={of=tractrix and linetan,by=P}] [draw=blue!30, fill=blue!30] (P) circle (1.25pt);;
        }
    \end{pgfonlayer}
    \legend{Catenary,Tractrix}
    %\node[anchor=center] at (axis cs:0,-1.25){Tractrix as involute of the catenary};
  \end{axis}
  \hspace{1cm}
  \begin{axis}[
      xmin = -3, xmax = 3, %
      ymin = -0.5, ymax = 5, %
      at={(left plot.south east)},
      %anchor=south west,
      %xshift=0.2cm,
      axis equal image,
      clip mode=individual,
      title={Catenary as evolute of the tractrix},
    ]
    \addplot[
      domain=-4:4, %Domain of the fucntion
      samples=200, %This parameter determines the number of point to be plotted for the function, while bigger the number better looks the function.
      smooth, %f we use this option, the compiler makes an interpolation between the point plotted to get a soft appearance for the function.
      thick, %Stroke of the function. Options: ultra thin, very thin, thin, semithick, thick, very thick, ultra thick.
      red, %Color of the function.
      name path=tractrix,
    ]({x-tanh(x)},{1/cosh(x)});
    \addplot[
      domain=-2.5:2.5, %Domain of the fucntion
      samples=200, %This parameter determines the number of point to be plotted for the function, while bigger the number better looks the function.
      smooth, %f we use this option, the compiler makes an interpolation between the point plotted to get a soft appearance for the function.
      thick, %Stroke of the function. Options: ultra thin, very thin, thin, semithick, thick, very thick, ultra thick.
      dashed,
      blue, %Color of the function.
      name path=cate,
    ]{cosh(x)};
    \begin{pgfonlayer}{background}
      \foreach \i in {0,...,\n}{
          \addplot[
            domain=-3:3, %Domain of the fucntion
            samples=10, %This parameter determines the number of point to be plotted for the function, while bigger the number better looks the function.
            very thin, %Stroke of the function. Options: ultra thin, very thin, thin, semithick, thick, very thick, ultra thick.
            red!20, %Color of the function.
            name path=linetan,
            on layer=background,
          ]{cosh(4/\n*\i-2)+sinh(4/\n*\i-2)*(x-(4/\n*\i-2))};
          %\draw[name intersections={of=cate and linetan,by=P}] [draw=red!30, fill=red!30] (P) circle (1.25pt);;
        }
    \end{pgfonlayer}
    %\legend{Catenary,Tractrix}
    %\node[anchor=center] at (axis cs:0,-1.25){Catenary as evolute of the tractrix};
  \end{axis}
  %\centering
  %\pgfplotslegendfromname{named}

\end{tikzpicture}
\end{document}