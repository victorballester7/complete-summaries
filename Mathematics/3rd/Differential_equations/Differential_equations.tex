\documentclass[../../../main.tex]{subfiles}

\begin{document}
\begin{multicols}{2}[\section{Differential equations}]
  \subsection{Ordinary differential equations}
  \subsubsection*{Introduction}
  \begin{definition}
    An \textit{ordinary differential equation (ODE) of $m$ unknowns and of order $n$} in \textit{implicit form} is an expression of the form: $$f\left(x,\vectorfunction{y}(x),\vectorfunction{y}'(x),\vectorfunction{y}''(x),\ldots,\vectorfunction{y}^{(n)}(x)\right)=0$$
    where $\vectorfunction{y}:U\subseteq\RR\rightarrow\RR^m$ is a vector-valued function of one variable $x\in\RR$ (which is called \textit{independent variable}) and $f:\Omega\subseteq\RR\times\RR^{m(n+1)}\rightarrow\RR$. The same ordinary differential equation in \textit{explicit form} is an expression of the form: $$\vectorfunction{y}^{(n)}(x)=\vectorfunction{g}\left(x,\vectorfunction{y}(x),\vectorfunction{y}'(x),\vectorfunction{y}''(x),\ldots,\vectorfunction{y}^{(n-1)}(x)\right)$$
    where $\vectorfunction{g}:\Omega\subseteq\RR\times\RR^{mn}\rightarrow\RR^m$.
  \end{definition}
  \begin{definition}
    Consider the following ODE of $m$ unknowns and of order $n$:
    \begin{equation}\label{DE_ode1}
      \vectorfunction{y}^{(n)}(x)=\vectorfunction{f}\left(x,\vectorfunction{y}(x),\vectorfunction{y}'(x),\ldots,\vectorfunction{y}^{(n-1)}(x)\right)
    \end{equation}
    We say that $\vectorfunction{\varphi}:I\subseteq\RR\rightarrow\RR^m$ is a \textit{solution of the ODE} if:
    \begin{itemize}
      \item $\vectorfunction{\varphi}$ is $n$-times differentiable on $I$.
      \item $\displaystyle\left\{\left(x,\vectorfunction{\varphi}(x),\vectorfunction{\varphi}'(x),\ldots,\vectorfunction{\varphi}^{(n-1)}(x)\right):x\in I\right\}\subseteq\domain \vectorfunction{f}$
      \item For all $x\in I$ we have:
            $$\vectorfunction{\varphi}^{(n)}(x)=\vectorfunction{f}\left(x,\vectorfunction{\varphi}(x),\vectorfunction{\varphi}'(x),\ldots,\vectorfunction{\varphi}^{(n-1)}(x)\right)$$
    \end{itemize}
    The set of all solutions of the ODE is called \textit{general solution of the ODE}.
  \end{definition}
  \begin{prop}
    Consider the following ODE of $m$ unknowns and of order $n$ of the form of \eqref{DE_ode1}. Then, we can transform this ODE to an ODE of $m\cdot n$ unknowns and order 1 in the following way. Define $\vectorfunction{z}_i=\vectorfunction{y}^{(i-1)}$ for $i=1,\ldots,n$\footnote{Therefore, we will mainly study the ODEs of order 1.}. Therefore the functions $\vectorfunction{z}_i$ must satisfy:
    \begin{equation*}
      \left\{
      \begin{aligned}
        {\vectorfunction{z}_1}'     & =\vectorfunction{z}_2                                                                                            \\
        {\vectorfunction{z}_2}'     & =\vectorfunction{z}_3                                                                                            \\
                                    & \;\;\vdots                                                                                                       \\
        {\vectorfunction{z}_{n-1}}' & =\vectorfunction{z}_{n-2}                                                                                        \\
        {\vectorfunction{z}_n}'     & =\vectorfunction{f}\left(x,\vectorfunction{z}_1(x),\vectorfunction{z}_2(x),\ldots,\vectorfunction{z}_2(x)\right) \\
      \end{aligned}
      \right.
    \end{equation*}
  \end{prop}
  \begin{definition}
    We say that an ODE is \textit{autonomous} if it doesn't depend on the independent variable, that is, if it is of the form: $$\vectorfunction{y}'=\vectorfunction{f}(\vectorfunction{y})$$ Analogously, we say that an ODE is \textit{non-autonomous} if it does depend on the independent variable, that is, if it is of the form: $$\vectorfunction{y}'=\vectorfunction{f}(x,\vectorfunction{y})$$
  \end{definition}
  \subsubsection*{Initial value problem}
  \begin{definition}
    Let $U\subset\RR\times\RR^n$ be an open set and $\vectorfunction{f}:U\rightarrow\RR^n$ be a function. Given $(x_0,y_0)\in U$, the \textit{initial value problem (ivp)} (or \textit{Cauchy problem}) consists in finding a solution of the ODE $$\vectorfunction{y}'=\vectorfunction{f}(x,\vectorfunction{y})$$ with initial conditions $\vectorfunction{y}(x_0)=y_0$.
  \end{definition}
  \subsection{Existence and uniqueness theorems}
  \begin{prop}
    Let $f:(a,b)\rightarrow\RR$ be a continuous function such that $f(x)\ne 0$ $\forall x\in(a,b)$. Then, the ivp
    $$
      \left\{
      \begin{aligned}
         & x'      =f(x) \\
         & x(t_0)  =x_0
      \end{aligned}
      \right.
    $$
    has a unique solution $\forall t_0\in\RR$ and $\forall x_0\in(a,b)$.
  \end{prop}
  \begin{prop}
    Let $f:(a,b)\rightarrow\RR$, $g:(c,d)\rightarrow\RR$ be continuous functions such that $f(x)\ne 0$ $\forall x\in(a,b)$. Then, the ivp
    $$\left\{
      \begin{aligned}
         & x'      =f(x)g(t) \\
         & x(t_0)  =x_0
      \end{aligned}
      \right.$$
    has a unique solution $\forall t_0\in(c,d)$ and $\forall x_0\in(a,b)$.
  \end{prop}
  \begin{prop}
    Let $I\subseteq\RR$ be an interval and $a:I\rightarrow\RR$ and $b:I\rightarrow\RR$ be continuous functions. Then, the ivp
    $$\left\{
      \begin{aligned}
         & x'      =a(t)x+b(t) \\
         & x(t_0)  =x_0
      \end{aligned}
      \right.$$
    has a unique solution $\varphi(t)$ $\forall t_0\in I$ and $\forall x_0\in\RR$. Furthermore, this solution is given by:
    $$\varphi(t)=\exp{\int_{t_0}^ta(s)\dd s}\left(x_0+\int_{t_0}^tb(u)\exp{-\int_{t_0}^ua(s)\dd s}\dd u\right)$$
  \end{prop}
  \subsubsection*{Picard theorem}
  \begin{definition}
    Let $\vectorfunction{f}:U\subseteq\RR\times\RR^n\rightarrow\RR^m$ be a function. We say that $\vectorfunction{f}$ is \textit{Lipschitz continuous with respect to the second variable} if $\exists L\in\RR_{>0}$ such that: $$\|\vectorfunction{f}(t,x)-\vectorfunction{f}(t,y)\|\leq L\|x-y\|\qquad\forall (t,x),(t,y)\in U$$
  \end{definition}
  \begin{definition}
    Let $\vectorfunction{f}:U\subseteq\RR\times\RR^n\rightarrow\RR^m$ be a function. We say that $\vectorfunction{f}$ is \textit{locally Lipschitz continuous with respect to the second variable} if $\forall (t_0,x_0)\in U$ there exists a neighbourhood $V$ of $(t_0,x_0)$ such that $f_{|V}$ is Lipschitz continuous with respect to the second variable.
  \end{definition}
  \begin{prop}
    Let $U\subseteq\RR\times\RR^n$ be an open set and $\vectorfunction{f}:U\rightarrow\RR^n$ be a continuous function. Let $I\subseteq\RR$ be an open interval, $t_0\in I$ and $x_0\in\RR^n$ be such that $(t_0,x_0)\in U$. Then, a continuous function $\vectorfunction{\varphi}:I\rightarrow\RR^n$ is a solution of the ivp
    \begin{equation}
      \left\{
      \begin{aligned}
         & \vectorfunction{x}'=\vectorfunction{f}(t,\vectorfunction{x}) \\
         & \vectorfunction{x}(t_0)=x_0
      \end{aligned}
      \right.
      \label{DE_ivp}
    \end{equation}
    if and only if $$\vectorfunction{\varphi}(t)=x_0+\int_{t_0}^tf(s,\vectorfunction{\varphi}(s))\dd s\quad\forall s\in I$$
  \end{prop}
  \begin{definition}
    An \textit{operator} is a function whose domain is a set of functions or other structured objects.
  \end{definition}
  \begin{definition}
    Let $U\subseteq\RR\times\RR^n$ be an open set, $(t_0,x_0)\in U$, $\vectorfunction{f}:U\rightarrow\RR^n$ be a continuous function and $I$ be a closed interval. We define the operator
    \begin{align*}
      T:\mathcal{C}(I,\RR^n)   & \longrightarrow\mathcal{C}(I,\RR^n)                                                                           \\
      \vectorfunction{\varphi} & \longmapsto T(\vectorfunction{\varphi})(t)=x_0+\int_{t_0}^tf(s,\vectorfunction{\varphi}(s))\dd s\footnotemark
    \end{align*}
  \end{definition}
  \begin{prop}\footnotetext{Note that the fixed points of this operator are precisely the solutions of the ivp \eqref{DE_ivp}.}
    Let $X=\mathcal{C}([a,b],\RR^n)$ and define a distance $d$ in $X$ in the following way. For all $\vectorfunction{\varphi},\vectorfunction{\psi}\in X$: $$\|\vectorfunction{\varphi}\|:=\sup\{\|\vectorfunction{\varphi}(t)\|:t\in[a,b]\}\qquad d(\vectorfunction{\varphi},\vectorfunction{\psi}):=\|\vectorfunction{\varphi}-\vectorfunction{\psi}\|$$ Then, $(X,d)$ is a complete metric space. Moreover, if $D\subset\RR^n$ is a closed set and $X=\mathcal{C}([a,b],D)$, then $(X,d)$ is also a complete metric space.
  \end{prop}
  \begin{theorem}[Banach fixed-point theorem]
    Let $(X,d)$ be a complete metric space and $f:X\rightarrow X$ be a contraction. Then, $f$ has a unique fixed point $p\in X$\footnote{Furthermore, $p$ can be found as follows: start with an arbitrary element$x_0\in X$ and define a sequence $(x_n)$ by $x_n=f(x_{n-1})$ for $n\geq 1$. Then, $\lim_{n\to\infty} x_n=p$.}.
  \end{theorem}
  \begin{corollary}
    Let $(X,d)$ be a complete metric space and $f:X\rightarrow X$ be a function. If there exists $m\in\NN$ such that $f^m$ is a contraction, then $f$ has a unique fixed point $p\in X$.
  \end{corollary}
  \begin{definition}
    Let $t_0\in\RR$, $b\in\RR^n$ and $a,b\in\RR_{>0}$. We define the following sets: $$I_a(t_0):=[t_0-a,t_0+a]\subset\RR\;\;\text{and}\;\;\overline{B}_{b}(x_0):=\overline{B}(x_0,b)\subset\RR^n$$
  \end{definition}
  \begin{theorem}[Picard theorem]\label{DE_picard}
    Let $t_0\in\RR$, $x_0\in\RR^n$, $a,b\in\RR_{>0}$, $\vectorfunction{f}:I_a(t_0)\times\overline{B}_{b}(x_0)\subset\RR\times\RR^n\rightarrow\RR^n$ be a continuous function and Lipschitz continuous with respect to the second variable, and define: $$M:=\max\{\|\vectorfunction{f}(t,x)\|:(t,x)\in I_a(t_0)\times\overline{B}_{b}(x_0)\}$$ Then, the ivp \eqref{DE_ivp} has a unique solution $\vectorfunction{\varphi}:I_\alpha(t_0)\rightarrow\RR^n$, where $\alpha:=\min\left\{a,\frac{b}{M}\right\}$.
  \end{theorem}
  \begin{prop}[Simplified Picard theorem]
    Let $I\subset \RR$ be a closed interval, $t_0\in I$, $x_0\in\RR^n$, $a,b\in\RR_{>0}$ and $\vectorfunction{f}:I\times\RR^n\rightarrow\RR^n$ be a continuous function and Lipschitz continuous with respect to the second variable. Then, the ivp \eqref{DE_ivp} has a unique solution $\vectorfunction{\varphi}:I\rightarrow\RR^n$.
  \end{prop}
  \begin{corollary}[Picard iteration process]
    Suppose we want to solve the ivp \eqref{DE_ivp}. That is, we look for a solution $\vectorfunction{\varphi}(t)$. Let $\vectorfunction{\varphi}_0$ be a fixed function and define
    $$\vectorfunction{\varphi}_{n+1}(t)=T(\vectorfunction{\varphi}_n)(t)=x_0+\int_{t_0}^tf(s,\vectorfunction{\varphi}_n(s))\dd s$$
    for all $n\geq 1$. Then, $\displaystyle\vectorfunction{\varphi}=\lim_{n\to\infty}\vectorfunction{\varphi}_n$.
  \end{corollary}
  \begin{corollary}
    Let $U\subseteq\RR\times\RR^n$ be an open set and $\vectorfunction{f}:U\rightarrow\RR^n$ be a continuous function and locally Lipschitz continuous with respect to the second variable. Then, $\forall(t_0,x_0)\in U$, there exists a neighbourhood $V_{(t_0,x_0)}=I_{a(t_0,x_0)}(t_0)\times\overline{B}_{b(t_0,x_0)}(x_0)$ (following the notation of theorem \ref{DE_picard}) of $(t_0,x_0)$ in $U$ such that the ivp \eqref{DE_ivp} has a unique solution $\vectorfunction{\varphi}_{(t_0,x_0)}$ defined on $I_{a(t_0,x_0)}$ with $\graph(\vectorfunction{\varphi}_{(t_0,x_0)})\subset V_{(t_0,x_0)}$.
  \end{corollary}
  \begin{prop}
    Let $I\subseteq\RR$ be an interval and $\vectorfunction{f}:I\times\RR^n\rightarrow\RR^n$ be a continuous function and Lipschitz continuous with respect to the second variable. Then, $\forall(t_0,x_0)\in I\times\RR^n$ there is a unique solution of the ivp \eqref{DE_ivp} defined on $I$.
  \end{prop}
  \begin{corollary}
    Let $I\subseteq\RR$ be an interval and $\vectorfunction{A}:I\rightarrow\mathcal{L}(\RR^n,\RR^n)$ and $\vectorfunction{b}:I\rightarrow\RR^n$ be continuous functions. Then, for all $(t_0,x_0)\in I\times\RR^n$ the ivp
    $$
      \left\{
      \begin{aligned}
         & \vectorfunction{x}'=\vectorfunction{A}(t)\vectorfunction{x}+\vectorfunction{b}(t) \\
         & \vectorfunction{x}(t_0)=x_0
      \end{aligned}
      \right.
    $$
    has a unique solution defined on $I$.
  \end{corollary}
  \subsubsection*{Peano theorem}
  \begin{definition}
    Let $(X,d)$ be a metric space and $F\subset\mathcal{C}(X,\RR^n)$ be a subset. We say that $F$ is \textit{pointwise bounded} if: $$\forall x\in X\;\exists M_x>0\text{ such that }\|\vectorfunction{f}(x)\|\leq M_x\quad\forall\vectorfunction{f}\in F$$
    We say that $F$ is \textit{uniformly bounded} if: $$\exists M>0\text{ such that }\|\vectorfunction{f}(x)\|\leq M \quad\forall\vectorfunction{f}\in F\text{ and }\forall x\in X$$
  \end{definition}
  \begin{definition}
    Let $(X,d)$ be a metric space and $F\subset\mathcal{C}(X,\RR^n)$ be a subset. We say that $F$ is \textit{equicontinuous at a point $x_0\in X$} if $\forall \varepsilon>0$ $\exists \delta>0$ such that $\forall x\in X$ with $d(x,x_0)<\delta$ we have $$\|\vectorfunction{f}(x)-\vectorfunction{f}(x_0)\|<\varepsilon\quad\forall\vectorfunction{f}\in F$$
    We say that $F$ is \textit{pointwise equicontinuous} if it is equicontinuous at each point of $X$. Finally, we say that $F$ is \textit{uniformly equicontinuous} if $\forall \varepsilon>0$ $\exists \delta>0$ such that $\forall x,y\in X$ with $d(x,y)<\delta$ we have $$\|\vectorfunction{f}(x)-\vectorfunction{f}(y)\|<\varepsilon\quad\forall\vectorfunction{f}\in F$$
  \end{definition}
  \begin{theorem}[Arzelà-Ascoli theorem]
    Let $(X,d)$ be a compact metric space and $(\vectorfunction{f}_m)$ be a sequence of functions such that $\vectorfunction{f}_m\in\mathcal{C}(X,\RR^n)$ $\forall m\geq 1$. If the sequence is pointwise bounded and equicontinuous, then there exists a subsequence $(\vectorfunction{f}_{m_k})$ that converges onn $\mathcal{C}(X,\RR^n)$.
  \end{theorem}
  \begin{corollary}
    Let $(X,d)$ be a compact metric space, $D\subset\RR^n$ be a closed set and $(\vectorfunction{f}_m)$ be a sequence of functions such that $\vectorfunction{f}_m\in\mathcal{C}(X,D)$ $\forall m\geq 1$. If the sequence is pointwise bounded and equicontinuous, then there exists a subsequence $(\vectorfunction{f}_{m_k})$ that converges onn $\mathcal{C}(X,D)$.
  \end{corollary}
  \begin{theorem}[Peano theorem]
    Let $t_0\in\RR$, $x_0\in\RR^n$, $a,b\in\RR_{>0}$, $\vectorfunction{f}:I_a(t_0)\times\overline{B}_{b}(x_0)\subset\RR\times\RR^n\rightarrow\RR^n$ be a continuous function, and define: $$M:=\max\{\|\vectorfunction{f}(t,x)\|:(t,x)\in I_a(t_0)\times\overline{B}_{b}(x_0)\}$$ Then, the ivp \eqref{DE_ivp} has at least one solution $\vectorfunction{\varphi}:I_\alpha(t_0)\rightarrow\RR^n$, where $\alpha:=\min\left\{a,\frac{b}{M}\right\}$.
  \end{theorem}
  \begin{corollary}
    Let $U\subseteq\RR\times\RR^n$ be an open set, $K\subset U$ be a compact set and $\vectorfunction{f}:U\rightarrow\RR^n$ be a continuous function. Then, $\exists\alpha\in\RR_{>0}$ such that $\forall (t_0,x_0)\in K$, the ivp \eqref{DE_ivp} has a solution defined in $I_\alpha(t_0)$.
  \end{corollary}
  \subsubsection*{Maximal solutions}
  \begin{definition}
    Let $U\subseteq\RR\times\RR^n$ be an open set, $(t_0,x_0)\in U$ and $\vectorfunction{f}:U\rightarrow\RR^n$ be a continuous function. We define the set $A(U,f,t_0,x_0)$ as:
    \begin{multline*}
      A(U,\vectorfunction{f},t_0,x_0):=\{(I,\varphi):I\subseteq\RR\text{ is an interval},t_0\in I\\\text{and }\varphi:I\rightarrow\RR^n\text{ is a solution of the ivp \eqref{DE_ivp}}\}
    \end{multline*}
  \end{definition}
  \begin{definition}
    Let $\leq$ be a relation\footnote{It can be seen that, in fact, $\leq$ is a partial (but not total) order relation.} defined on $A(U,\vectorfunction{f},t_0,x_0)$. For $(I,\vectorfunction{\varphi}),(J,\vectorfunction{\psi})\in A(U,\vectorfunction{f},t_0,x_0)$ we define: $$(J,\vectorfunction{\psi})\leq (I,\vectorfunction{\varphi})\iff J\subset I\text{ and }\vectorfunction{\varphi}|_J=\vectorfunction{\psi}$$ In this case, we say that $(I,\vectorfunction{\varphi})$ is an \textit{extension} of $(J,\vectorfunction{\psi})$.
  \end{definition}
  \begin{definition}
    Let $(A,\mathcal{R})$ be a poset. Then, $m\in A$ is a \textit{maximal element} if and only if $\forall a\in A$ with $m \mathcal{R} a$ we have $m=a$.
  \end{definition}
  \begin{definition}
    Consider the poset $(A(U,\vectorfunction{f},t_0,x_0),\leq)$. We say that a solution $(I,\vectorfunction{\varphi})$ is \textit{maximal} if for all extensions $(J,\vectorfunction{\psi})$ of $(I,\vectorfunction{\varphi})$ we have $I=J$ and $\vectorfunction{\varphi}=\vectorfunction{\psi}$.
  \end{definition}
  \begin{definition}
    Let $(A,\mathcal{R})$ be a poset and $C\subset A$ be a subset of $A$. We say that $C$ is a \textit{chain} if it is totally ordered in the inherited order, that is, if it is partially ordered and $\forall x,y\in C$ we have either $x\mathcal{R}y$ or $y\mathcal{R}x$.
  \end{definition}
  \begin{definition}
    Let $(A,\mathcal{R})$ be a poset and $x\in A$. $x$ is an \textit{upper bound of $A$} if and only if $a\mathcal{R}x$ $\forall a\in A$.
  \end{definition}
  \begin{definition}
    Let $(A,\mathcal{R})$ be a poset. Then, $m\in A$ is a \textit{greatest element} if and only if $\forall a\in A$ we have $a \mathcal{R} m$.
  \end{definition}
  \begin{lemma}[Zorn's lemma]
    Let $(A,\mathcal{R})$ be a poset. If every chain $C\subset X$ has an upper bound in $A$, then $A$ contains at least one maximal element.
  \end{lemma}
  \begin{theorem}
    Let $U\subseteq\RR\times\RR^n$ be an open set, $(t_0,x_0)\in U$ and $\vectorfunction{f}:U\rightarrow\RR^n$ be a continuous function. Consider the poset $(A(U,\vectorfunction{f},t_0,x_0),\leq)$. Then, $A(U,\vectorfunction{f},t_0,x_0)$ has maximal elements. Furthermore, if $(I,\vectorfunction{\varphi})$ is a maximal solution, then $I$ is open.
  \end{theorem}
  \begin{prop}
    Let $U\subseteq\RR\times\RR^n$ be an open set and $\vectorfunction{f}:U\rightarrow\RR^n$ be a continuous function such that $\forall(t_0,x_0)\in U$ the ivp \eqref{DE_ivp} has a unique solution defined in a neighbourhood of $t_0$. Then, $\forall(t_0,x_0)\in U$ the ivp \eqref{DE_ivp} has a unique maximal solution.
  \end{prop}
  \begin{lemma}[Wintner lemma]
    Let $U\subseteq\RR\times\RR^n$ be an open set, $\vectorfunction{f}:U\rightarrow\RR^n$ be a continuous function, $\vectorfunction{\varphi}:I\rightarrow\RR^n$ be a solution of $\vectorfunction{x}'=\vectorfunction{f}(t,\vectorfunction{x})$ and $(b,y)\in U$ be an accumulation point of $\vectorfunction{\varphi}$. Then, $\displaystyle\lim_{t\to b}\vectorfunction{\varphi}(t)=y$ an the solution can be extended up to $b$.
  \end{lemma}
  \begin{prop}
    Let $U\subseteq\RR\times\RR^n$ be an open set, $\vectorfunction{f}:U\rightarrow\RR^n$ be a continuous function and $\vectorfunction{\varphi}:(a,b)\rightarrow\RR^n$ be a maximal solution of $\vectorfunction{x}'=\vectorfunction{f}(t,\vectorfunction{x})$. If $b<\infty$, then for all compact set $\forall K\subset U$, $\exists t_k<\infty$ such that $\forall t\in[t_k,b]$, $(t,\vectorfunction{\varphi}(t))\notin K$.
  \end{prop}
\end{multicols}
\end{document}