\documentclass[../../../main.tex]{subfiles}

\begin{document}
\begin{multicols}{2}[\section{Differential equations}]
  \subsection{Ordinary differential equations}
  \subsubsection*{Introduction}
  \begin{definition}
    An \textit{ordinary differential equation (ODE) of $m$ unknowns and of order $n$} in \textit{implicit form} is an expression of the form: $$f\left(x,\vectorfunction{y}(x),\vectorfunction{y}'(x),\vectorfunction{y}''(x),\ldots,\vectorfunction{y}^{(n)}(x)\right)=0$$
    where $\vectorfunction{y}:U\subseteq\RR\rightarrow\RR^m$ is a vector-valued function of one variable $x\in\RR$ (which is called \textit{independent variable}) and $f:\Omega\subseteq\RR\times\RR^{m(n+1)}\rightarrow\RR$. The same ordinary differential equation in \textit{explicit form} is an expression of the form: $$\vectorfunction{y}^{(n)}(x)=\vectorfunction{g}\left(x,\vectorfunction{y}(x),\vectorfunction{y}'(x),\vectorfunction{y}''(x),\ldots,\vectorfunction{y}^{(n-1)}(x)\right)$$
    where $\vectorfunction{g}:\Omega\subseteq\RR\times\RR^{mn}\rightarrow\RR^m$.
  \end{definition}
  \begin{definition}
    Consider the following ODE of $m$ unknowns and of order $n$: $$\vectorfunction{y}^{(n)}(x)=\vectorfunction{f}\left(x,\vectorfunction{y}(x),\vectorfunction{y}'(x),\ldots,\vectorfunction{y}^{(n-1)}(x)\right)$$
    We say that $\vectorfunction{\varphi}:I\subseteq\RR\rightarrow\RR^m$ is a \textit{solution of the ODE} if:
    \begin{itemize}
      \item $\vectorfunction{\varphi}$ is $n$-times differentiable on $I$.
      \item $\displaystyle\left\{\left(x,\vectorfunction{\varphi}(x),\vectorfunction{\varphi}'(x),\ldots,\vectorfunction{\varphi}^{(n-1)}(x)\right):x\in I\right\}\subseteq\domain \vectorfunction{f}$
      \item For all $x\in I$ we have:
            $$\vectorfunction{\varphi}^{(n)}(x)=\vectorfunction{f}\left(x,\vectorfunction{\varphi}(x),\vectorfunction{\varphi}'(x),\ldots,\vectorfunction{\varphi}^{(n-1)}(x)\right)$$
    \end{itemize}
    The set of all solutions of the ODE is called \textit{general solution of the ODE}.
  \end{definition}
  \begin{prop}
    We can always transform a ODE of $m$ unknowns and order $n$ to an ODE of $m\cdot n$ unknowns and order 1\footnote{Therefore, we will mainly study the ODEs of order 1.}.
  \end{prop}
  \begin{definition}
    We say that an ODE is \textit{autonomous} if it doesn't depend on the independent variable, that is, if it is of the form: $$y'=f(y)$$ Analogously, we say that an ODE is \textit{non-autonomous} if it does depend on the independent variable, that is, if it is of the form: $$y'=f(x,y)$$
  \end{definition}
  \subsubsection*{Initial value problem}
  \begin{definition}
    Let $U\subset\RR\times\RR^n$ be an open set and $\vectorfunction{f}:U\rightarrow\RR^n$ be a vector field. Given $(x_0,y_0)\in U$, the \textit{initial value problem} (or \textit{Cauchy problem}) consists in finding a solution of the ODE $$\vectorfunction{y}'=\vectorfunction{f}(x,\vectorfunction{y})$$ with initial conditions $\vectorfunction{y}(x_0)=y_0$.
  \end{definition}
\end{multicols}
\end{document}