\documentclass[../../../main.tex]{subfiles}

\begin{document}
\begin{multicols}{2}[\section{Topology}]
  \subsection{Topological spaces}
  \subsubsection{Metric spaces}
  \begin{definition}
    Let $X$ be a set. A \textit{distance} (or \textit{metric}) \textit{in $X$} is a function $d:X\times X\rightarrow\RR $ such that $\forall x,y,z\in X$ the following properties are satisfied:
    \begin{enumerate}
      \item $d(x,y)=0\iff x=y$.
      \item $d(x,y)=d(y,x)$.
      \item $d(x,y)\leq d(x,z)+d(z,y)\quad$\textit{(triangular inequality)}.
    \end{enumerate}
    We define a \textit{metric space} as a pair $(X,d)$ that satisfy the previous properties.
  \end{definition}
  \begin{prop}
    Let $x,y\in\RR^n$ such that $x=(x_1,\ldots,x_n)$ and $y=(y_1,\ldots,y_n)$. The following functions are metrics in $\RR^n$.
    \begin{enumerate}
      \item \textit{Euclidean metric}: $$d(x,y)=\sqrt{\sum_{i=1}^n{(x_i-y_i)}^2}$$
      \item \textit{Taxicab metric}: $$d(x,y)=\sum_{i=1}^n|x_i-y_i|$$
      \item \textit{Maximum metric}: $$d(x,y)=\max\{|x_i-y_i|:i\in\{1,\ldots,n\}\}$$
    \end{enumerate}
  \end{prop}
  \begin{prop}
    Let $X$ be a set. Then, $(X,d)$ is a metric space, where: $$d(x,y)=
      \left\{\begin{array}{lcc}
        1 & \text{if} & x\ne y \\
        0 & \text{if} & x= y
      \end{array}\right.
    $$
    This metric $d$ is called \textit{discrete metric}.
  \end{prop}
  \begin{definition}
    Let $(X,d)$ be a metric space $a\in X$ and $r\in\RR$. We define the \textit{ball $B_d(a,r)$ with the metric $(X,d)$ of center $a$ and radius $r$} as: $$B_d(a,r)=\{x\in X:d(x,a)<r\}$$
  \end{definition}
  \begin{definition}
    Let $(X,d_X)$ and $(Y,d_Y)$ be two metric spaces and $f:(X,d_X)\rightarrow(Y,d_Y)$ be a function. We say that $f$ is continuous if $\forall a\in X$ and $\forall\varepsilon>0$, $\exists\delta>0$ such that $d_Y(f(x),f(a))<\varepsilon$ whenever $d_X(x,a)<\delta$ or, equivalently: $$f(B_{d_X}(a,\delta))\subset B_{d_Y}(f(a),\varepsilon)$$ which is equivalent to $B_{d_X}(a,\delta)\subset f^{-1}\left(B_{d_Y}(f(a),\varepsilon)\right)$.
  \end{definition}
  \begin{definition}
    Let $(X,d)$ be a metric space. We say that a subset $A\subseteq X$ is \textit{open} if $\forall a\in A$, $\exists\varepsilon>0$ such that $B_d(a,\varepsilon)\subset A$.
  \end{definition}
  \begin{prop}
    Let $(X,d)$ be a metric space. Then:
    \begin{itemize}
      \item $\varnothing$ and $X$ are open sets.
      \item If $I$ is an arbitrary index set and $\{U_i:U_i\subseteq X\;\forall i\in I\}$ is a collection of open sets, then $\bigcup_{i\in I}U_i$ is an open set.
      \item If $\{U_i:U_i\subseteq X\;\forall i\in \{1,\ldots,n\}\}$ is a collection of open sets, then $\bigcap_{i=1}^nU_i$ is an open set.
    \end{itemize}
  \end{prop}
  \begin{prop}
    Let $(X,d)$ be a metric space and $x\in X$. Then, the ball $B_d(x,r)$ is open $\forall r\in\RR$.
  \end{prop}
  \begin{prop}
    Let $(X,d)$ be a metric space and $A\subset X$ be a subset of $X$. Then, $A$ is open if and only if $A=\bigcup_{i\in I}B_d(a_i,\varepsilon_i)$, where $I$ is an index set, $a_i\in A$ and $\varepsilon_i>0$ for all $i\in I$.
  \end{prop}
  \begin{theorem}
    Let $(X,d_X)$ and $(Y,d_Y)$ be two metric spaces and $f:(X,d_X)\rightarrow(Y,d_Y)$ be a function. The following statements are equivalent:
    \begin{enumerate}
      \item $f$ is continuous.
      \item If $A\subset Y$ is open, then $f^{-1}(A)\subset X$ is also open.
    \end{enumerate}
  \end{theorem}
  \begin{prop}
    Let $(X,d)$ be a metric space with $|X|\geq 2$ and $x,y\in X$. Then, $\exists\delta>0$ such that $x\in B_d(x,\delta)$, $y\in B_d(y,\delta)$ and $B_d(x,\delta)\cap B_d(y,\delta)=\varnothing$.
  \end{prop}
  \subsubsection{Topological spaces}
  \begin{definition}
    Let $X$ be a set. A \textit{topology $\mathcal{T}$} on a set $X$ is a collection of subsets of $X$ (that is, $\mathcal{T}\subseteq\mathcal{P}(X)$) satisfying the following properties:
    \begin{enumerate}
      \item $\varnothing, X\in\mathcal{T}$.
      \item The intersection of any finite subcollection of $\mathcal{T}$ is in $\mathcal{T}$.
      \item The union of any subcollection of $\mathcal{T}$ is in $\mathcal{T}$.
    \end{enumerate}
    The ordered pair $(X,\mathcal{T})$ is called a \textit{topological space}. The elements of $X$ are called \textit{points} and the elements of $\mathcal{T}$, \textit{open sets}.
  \end{definition}
  \begin{definition}
    Let $(X,\mathcal{T})$ be a topological space such that $|X|\geq 2$. We say that $(X,\mathcal{T})$ is a \textit{Hausdorff space} if satisfies the \textit{Hausdorff property}: $\forall x,y\in X$ with $x\ne y$, there exist $U,V\in\mathcal{T}$ such that $x\in U$, $y\in V$ and $U\cap V=\varnothing$.
  \end{definition}
  \begin{definition}
    Let $(X,\mathcal{T})$ and $(X,\mathcal{T}')$ be topological spaces. We say that $\mathcal{T}$ is \textit{finer than} $\mathcal{T}'$ if $\mathcal{T}'\subseteq\mathcal{T}$.
  \end{definition}
  \begin{prop}
    Let $X$ be a set and $p\in X$ be a point of $X$ and $(X,d)$ be a metric space. Then, we can construct some topologies on $X$:
    \begin{itemize}
      \item \textit{Topology induced from the metric}: $$\mathcal{T}:=\{U\subseteq X:U\text{ is open with the metric }d\}$$
      \item \textit{Trivial topology}: $\mathcal{T}_\text{t}:=\{\varnothing,X\}$
      \item \textit{Discrete topology}: $\mathcal{T}_\text{d}:=\mathcal{P}(X)$
      \item \textit{Cofinite topology}: $$\mathcal{T}_\text{f}:=\{U\subseteq X:U=\varnothing\text{ or }X\setminus U\text{ is finite}\}$$
      \item \textit{Cocountable topology}:
            \begin{multline*}
              \mathcal{T}:=\{U\subseteq X:U=\varnothing\text{ or }X\setminus U\text{ is finite or}\\\text{ countable}\}
            \end{multline*}
      \item \textit{Particular point topology}: $$\mathcal{T}:=\{U\subseteq X:U=\varnothing\text{ or }p\in U\}$$
      \item \textit{Excluded point topology}: $$\mathcal{T}:=\{U\subseteq X:U=X\text{ or }p\notin U\}$$
      \item \textit{Sierpiński space}: If $X=\{0,1\}$, $$\mathcal{T}:=\{\varnothing,\{1\},\{0,1\}\}$$
    \end{itemize}
  \end{prop}
  \begin{definition}
    Let $(X,\mathcal{T})$ be a topological space and $C\subseteq X$. We say that $C$ is \textit{closed} if $X\setminus C\in\mathcal{T}$ is open.
  \end{definition}
  \begin{definition}
    Let $(X,\mathcal{T})$ be a topological space and $A\subseteq X$. We say that $A$ is \textit{clopen} if it is both open and closed.
  \end{definition}
  \begin{prop}
    Let $(X,\mathcal{T})$ be a topological space. Then:
    \begin{enumerate}
      \item $\varnothing$ and $X$ are closed.
      \item The union of any finite subcollection of closed sets in $(X,\mathcal{T})$ is a closed set in $(X,\mathcal{T})$.
      \item The intersection of any subcollection of closed sets in $(X,\mathcal{T})$ is closed in $(X,\mathcal{T})$.
    \end{enumerate}
  \end{prop}
  \subsubsection{Basis for a topology}
  \begin{definition}
    Let $(X,\mathcal{T})$ be a topological space and $\mathcal{B}\subseteq\mathcal{T}$ be a subset of open sets. We say that $\mathcal{B}$ is a \textit{basis of $\mathcal{T}$} if $\forall U\in\mathcal{T}$ and $\forall x\in U$, $\exists B\in\mathcal{B}$ such that $x\in B\subseteq U$.
  \end{definition}
  \begin{prop}
    Let $(X,\mathcal{T})$ be a topological space and $\mathcal{B}$ be a basis of $\mathcal{T}$. Then, for all $U\in\mathcal{T}$ we have: $$U=\bigcup_{x\in U}B_x$$ where $x\in B_x\subseteq U$ and $B_x\in\mathcal{B}$ $\forall x\in U$.
  \end{prop}
  \begin{lemma}
    Let $(X,\mathcal{T})$ be a topological space, $\mathcal{B}\subseteq\mathcal{T}$ be a basis of $\mathcal{T}$ and $\{B_i\in\mathcal{B}:i=1,\ldots,n\}$ be a collection of elements of $\mathcal{B}$. Then, $\forall x\in\bigcap_{i=1}^nB_i$, $\exists B'\in\mathcal{B}$ such that $x\in B'\subset\bigcap_{i=1}^nB_i$.
  \end{lemma}
  \begin{prop}
    Let $X$ be a set and $\mathcal{B}\subset\mathcal{P}(X)$ be a collection of subsets of $X$ such that:
    \begin{enumerate}
      \renewcommand{\labelenumi}{\alph{enumi})}
      \item $\displaystyle X=\bigcup_{B\in\mathcal{B}} B$
      \item $\forall U,V\in\mathcal{B}$ and  $\forall x\in U\cap V$, $\exists B\in\mathcal{B}$ such that $x\in B\subset U\cap V$.
    \end{enumerate}
    Then, there exists a unique topology $\mathcal{T}$ of $X$ such that:
    \begin{enumerate}
      \item $\mathcal{B}$ is a basis of the topology $\mathcal{T}$.
      \item $\mathcal{T}$ is the least finer topology that contains $\mathcal{B}$.
    \end{enumerate}
  \end{prop}
  \begin{definition}
    Let $X$ be a set and $\mathcal{B}\subset\mathcal{P}(X)$ be a collection of subsets of $X$. The \textit{topology $\mathcal{T}$ generated by $\mathcal{B}$} is: $$\mathcal{T}=\left\{U\subset X:U=\bigcup_{i\in I}B_i, B_i\in \mathcal{B}\right\}$$ Or equivalently: $$\mathcal{T}=\left\{U\subset X:\forall x\in U\;\exists B\in\mathcal{B}\text{ such that }x\in B\subseteq U\right\}$$
  \end{definition}
  \begin{definition}
    Let $$\mathcal{B}=\{[a,b)\subset\RR:a,b\in\RR\text{ and }a<b\}$$
    We define the \textit{lower limit topology} as the topology generated by $\mathcal{B}$.
  \end{definition}
  \begin{theorem}
    Let $(X,\mathcal{T}_X)$ and $(Y,\mathcal{T}_Y)$ be topological spaces, $f:(X,\mathcal{T}_X)\rightarrow (Y,\mathcal{T}_Y)$ be a function and $\mathcal{B}_Y$ be a basis of $\mathcal{T}_Y$. Then, $f$ is continuous if and only if $f^{-1}(B)$ is open $\forall B\in \mathcal{B}_Y$.
  \end{theorem}
  \begin{definition}
    Let $(X,\mathcal{T})$ be a topological space and $\mathcal{S}\subseteq\mathcal{T}$ be a subset. Then, $\mathcal{S}$ is a \textit{subbasis of $\mathcal{T}$} if $\forall U\in\mathcal{T}$, $U$ can be written as a union of finite intersections of elements of $\mathcal{S}$.
  \end{definition}
  \begin{prop}
    Let $X$ be a set and $\mathcal{S}\subset\mathcal{P}(X)$ such that $X=\bigcup_{S\in\mathcal{S}} S$. Then, there exists a unique topology $\mathcal{T}$ of $X$ such that:
    \begin{enumerate}
      \item $\mathcal{S}$ is a subbasis of the topology $\mathcal{T}$.
      \item $\mathcal{T}$ is the least finer topology that contains $\mathcal{S}$.
    \end{enumerate}
  \end{prop}
  \subsubsection{Interior, closure and boundary of a set}
  \begin{definition}[Interior]
    Let $(X,\mathcal{T})$ be a topological space and $A\subseteq X$ be a subset. The \textit{interior of $A$}, $\Int A$, is the largest open subset of $X$ contained in $A$.
  \end{definition}
  \begin{definition}[Closure]
    Let $(X,\mathcal{T})$ be a topological space and $A\subseteq X$ be a subset. The \textit{closure of $A$}, $\Cl A$, is the smallest closed subset of $X$ containing $A$.
  \end{definition}
  \begin{prop}
    Let $(X,\mathcal{T})$ be a topological space and $A\subseteq X$ be a subset. Then: $$\Int A=\bigcup_{\substack{U\subseteq A\\U\text{ is open}}}U\qquad\Cl A=\bigcap_{\substack{C\supseteq A\\C\text{ is closed}}}C$$
    Hence, we have the inclusions: $$\Int A\subseteq A\subseteq \Cl A$$
    And, furthermore:
    \begin{itemize}
      \item $\Int A=A$ if and only if $A$ is open.
      \item $\Cl A=A$ if and only if $A$ is closed.
    \end{itemize}
  \end{prop}
  \begin{definition}
    Let $(X,\mathcal{T})$ be a topological space and $A\subseteq X$ be a subset. $A$ is called \textit{dense in $X$} if $\forall U\in\mathcal{T}$ with $U\ne\varnothing$ we have $U\cap A\ne\varnothing$.
  \end{definition}
  \begin{prop}
    Let $(X,\mathcal{T})$ be a topological space and $A\subseteq X$ be a subset. Then, $A$ is dense in $X$ if and only if $\Cl A=X$.
  \end{prop}
  \begin{prop}
    Let $(X,\mathcal{T})$ be a topological space and $A\subset X$ be a subset. Then:
    \begin{itemize}
      \item If $U\subset A$ is open, then $U\subset\Int A$.
      \item If $A\subset C$ is closed, then $\Cl A\subset C$.
    \end{itemize}
  \end{prop}
  \begin{definition}
    Let $(X,\mathcal{T})$ be a topological space and $A\subset X$ be a subset. The \textit{boundary of $A$} is: $$\Fr A=\Cl(A)\cap\Cl(X\setminus A)$$
  \end{definition}
  \begin{definition}
    Let $(X,\mathcal{T})$ be a topological space and $x\in X$. We say that $N\subseteq X$ is a \textit{neighbourhood of $x$} if $\exists U\in\mathcal{T}$ such that $x\in U\subseteq N$.
  \end{definition}
  \begin{definition}
    Let $(X,\mathcal{T})$ be a topological space and $A\subseteq X$ be a subset. We say that $x\in X$ is an \textit{interior point of $A$} if $A$ is a neighbourhood of $x$.
  \end{definition}
  \begin{definition}
    Let $(X,\mathcal{T})$ be a topological space and $A\subseteq X$ be a subset. We say that $x\in X$ is an \textit{adherent point of $A$} if for all neighbourhood $N$ of $x$ we have that $N\cap A\ne\varnothing$.
  \end{definition}
  \begin{prop}
    Let $(X,\mathcal{T})$ be a topological space and $A\subseteq X$ be a subset. Then:
    \begin{enumerate}
      \item $\Int A$ is the set containing all the interior points of $A$.
      \item $\Cl A$ is the set containing all the adherent points of $A$.
    \end{enumerate}
  \end{prop}
  \begin{prop}
    Let $(X,\mathcal{T})$ be a topological space and $A,B\subseteq X$ be subsets.

    Properties regarding the interior:
    \begin{enumerate}[leftmargin=1.15cm]\renewcommand{\labelenumi}{1.\arabic{enumi}.}
      \item $\Int(\Int (A))=         \Int A$
      \item $A\subseteq B\implies    \Int A\subseteq\Int B$
      \item $\Int(X\setminus A)=     X\setminus \Cl A$
      \item $\Int(A\cap B)=          \Int A\cap\Int B$
      \item $\Int(A\cup B)\supseteq  \Int A\cup\Int B$
    \end{enumerate}
    Properties regarding the closure:
    \begin{enumerate}[leftmargin=1.15cm]\renewcommand{\labelenumi}{2.\arabic{enumi}.}
      \item $\Cl(\Cl (A))=          \Cl A$
      \item $A\subseteq B\implies   \Cl A\subseteq\Cl B$
      \item $\Cl(X\setminus A)=     X\setminus \Int A$
      \item $\Cl(A\cap B)\subseteq  \Cl A\cap\Cl B$
      \item $\Cl(A\cup B)=          \Cl A\cup\Cl B$
    \end{enumerate}
    Properties regarding the boundary:
    \begin{enumerate}[leftmargin=1.15cm]\renewcommand{\labelenumi}{3.\arabic{enumi}.}
      \item $\Fr A\cap\Int A=         \varnothing$
      \item $\Fr A=\Cl A\setminus\Int A$
      \item $\Fr A\cup\Int A=             \Cl A$
      \item $\Fr A\subseteq A\iff  A  \text{ is closed}$
      \item $\Fr A\cap A=\varnothing\iff  A\text{ is open}$
      \item $\Fr A=\varnothing\iff\text{$A$ is clopen}$
    \end{enumerate}
  \end{prop}
  \subsubsection{Functions between topological spaces}
  \begin{definition}[Continuous function]
    Let $(X,\mathcal{T}_X)$ and $(Y,\mathcal{T}_Y)$ be topological spaces and $f:(X,\mathcal{T}_X)\rightarrow(Y,\mathcal{T}_Y)$ be a function. We say that $f$ is continuous if for all $U\in\mathcal{T}_Y$, we have $f^{-1}(U)\in\mathcal{T}_X$.
  \end{definition}
  \begin{prop}
    Let $(X,\mathcal{T}_X)$ and $(Y,\mathcal{T}_Y)$ be topological spaces and $f:(X,\mathcal{T}_X)\rightarrow(Y,\mathcal{T}_Y)$ be a function. We say that $f$ is continuous if and only if for all closed sets $C\subseteq Y$, we have $f^{-1}(C)\subseteq X$ is closed.
  \end{prop}
  \begin{theorem}
    Let $(X,\mathcal{T}_X)$ and $(Y,\mathcal{T}_Y)$ be topological spaces and $f:(X,\mathcal{T}_X)\rightarrow (Y,\mathcal{T}_Y)$ be a function. Then, the following assumptions are equivalent:
    \begin{enumerate}
      \item $f$ is continuous.
      \item $f^{-1}(\Int(B))\subseteq \Int(f^{-1}(B))$ for all subsets $B\subseteq Y$.
      \item $f(\Cl(A))\subseteq \Cl(f(A))$ for all subsets $A\subseteq X$.
    \end{enumerate}
  \end{theorem}
  \begin{theorem}
    Let $(X,\mathcal{T}_X)$ and $(Y,\mathcal{T}_Y)$ be topological spaces and $f:(X,\mathcal{T}_X)\rightarrow (Y,\mathcal{T}_Y)$ be a function. Then, $f$ is continuous if and only if $\forall x\in X$ and $\forall U\in\mathcal{T}$ such that $f(x)\in U$, there exists a neighbourhood $N$ of $x$ with $f(N)\subset U$.
  \end{theorem}
  \begin{prop}
    Let $(X,\mathcal{T}_X)$, $(Y,\mathcal{T}_Y)$ and $(Z,\mathcal{T}_Z)$ be topological spaces and $f:(X,\mathcal{T}_X)\rightarrow (Y,\mathcal{T}_Y)$, $f:(Y,\mathcal{T}_Y)\rightarrow (Z,\mathcal{T}_Z)$ be continuous functions. Then, $g\circ f:(X,\mathcal{T}_X)\rightarrow (Z,\mathcal{T}_Z)$ is continuous.
  \end{prop}
  \begin{definition}
    Let $(X,\mathcal{T}_X)$ and $(Y,\mathcal{T}_Y)$ be topological spaces. A \textit{homeomorphism} between $(X,\mathcal{T}_X)$ and $(Y,\mathcal{T}_Y)$ is a bijective function that is continuous and whose inverse is also continuous. We say that $(X,\mathcal{T}_X)$ and $(Y,\mathcal{T}_Y)$ are \textit{homeomorphic}, denoted by $(X,\mathcal{T}_X)\cong(Y,\mathcal{T}_Y)$, if there exists a homeomorphism between them.
  \end{definition}
  \begin{definition}[Open function]
    Let $(X,\mathcal{T}_X)$ and $(Y,\mathcal{T}_Y)$ be topological spaces and $f:(X,\mathcal{T}_X)\rightarrow (Y,\mathcal{T}_Y)$ be a function. We say that $f$ is \textit{open} if $\forall U\in\mathcal{T}_X$, we have $f(A)\in \mathcal{T}_Y$.
  \end{definition}
  \begin{definition}[Closed function]
    Let $(X,\mathcal{T}_X)$ and $(Y,\mathcal{T}_Y)$ be topological spaces and $f:(X,\mathcal{T}_X)\rightarrow (Y,\mathcal{T}_Y)$ be a function. We say that $f$ is \textit{closed} if for all closed subsets $C\subseteq X$, we have $f(C)$ is closed.
  \end{definition}
  \begin{prop}
    Let $(X,\mathcal{T}_X)$ and $(Y,\mathcal{T}_Y)$ be topological spaces and $f:(X,\mathcal{T}_X)\rightarrow (Y,\mathcal{T}_Y)$ be a continuous bijective function. Then, the following statements are equivalent:
    \begin{enumerate}
      \item $f$ is a homeomorphism.
      \item $f$ is open.
      \item $f$ is closed.
    \end{enumerate}
  \end{prop}
  \begin{prop}
    Being homeomorphic as topological spaces is an equivalence relation.
  \end{prop}
  \subsection{Subspaces}
  \begin{definition}
    Let $(X,\mathcal{T})$ be a topological space and $A\subseteq X$ be a subset. We define the following set: $$\mathcal{T}_A=\{U\subseteq A:\exists V\in\mathcal{T}\text{ such that }V\cap A=U\}$$ Then, $(A,\mathcal{T}_A)$ is a topological space and $\mathcal{T}_A$ is called the \textit{subspace topology on $A$}. We will write $(A,\mathcal{T}_A)\subseteq (X,\mathcal{T})$ to denote that $(A,\mathcal{T}_A)$ is a topological subspace.
  \end{definition}
  \begin{prop}
    Let $(X,\mathcal{T})$ be a topological space and $(A,\mathcal{T}_A)\subseteq (X,\mathcal{T})$ be a topological subspace. Then, $C\subseteq A$ is closed if and only if $C=K\cap A$, where $K\subseteq X$ is a closed subset on $(X,\mathcal{T})$.
  \end{prop}
  \begin{prop}
    Let $(X,\mathcal{T})$ be a topological space and $(A,\mathcal{T}_A)\subseteq (X,\mathcal{T})$ be a topological subspace. Then:
    \begin{enumerate}
      \item If $A$ is open and $U\subseteq A$, then $U\in\mathcal{T}_A$ if and only if $U\in\mathcal{T}$.
      \item If $A$ is closed and $C\subseteq A$, then $C$ is closed on $(A,\mathcal{T}_A)$ if and only if $C$ is closed on $(X,\mathcal{T})$.
    \end{enumerate}
  \end{prop}
  \begin{prop}
    Let $(X,\mathcal{T})$ be a topological space and $(A,\mathcal{T}_A)\subseteq (X,\mathcal{T})$ be a topological subspace. Then, the inclusion $\iota:(A,\mathcal{T}_A)\hookrightarrow (X,\mathcal{T}_X)$ is continuous and $\mathcal{T}_A$ is the least finer topology where $\iota$ is continuous.
  \end{prop}
  \begin{corollary}
    Let $(X,\mathcal{T}_X)$ and $(Y,\mathcal{T}_Y)$ be topological spaces, $f:(X,\mathcal{T}_X)\rightarrow (Y,\mathcal{T}_Y)$ be a continuous function and $(A,\mathcal{T}_A)\subseteq (X,\mathcal{T}_X)$ be a topological subspace. Then, $f|_A$ is also continuous.
  \end{corollary}
  \begin{prop}
    Let $(X,\mathcal{T})$ be a topological space, $\mathcal{B}$ be a basis of $\mathcal{T}$ and $(A,\mathcal{T}_A)\subseteq (X,\mathcal{T})$ be a topological subspace. Then, $$\mathcal{B}_A=\{B\cap A:B\in\mathcal{B}\}$$ is basis of $\mathcal{T}_A$.
  \end{prop}
  \begin{prop}
    Let $(X,\mathcal{T}_X)$ and $(Y,\mathcal{T}_Y)$ be topological spaces and $(A,\mathcal{T}_A)\subseteq (Y,\mathcal{T}_Y)$ be a topological subspace. Let $f:(X,\mathcal{T}_X)\rightarrow (A,\mathcal{T}_A)$ be a function. Then, $f$ is continuous if and only if $\iota\circ f:(X,\mathcal{T}_X)\rightarrow (A,\mathcal{T}_A)\hookrightarrow (Y,\mathcal{T}_Y)$ is continuous.
  \end{prop}
  \begin{corollary}
    Let $(X,\mathcal{T}_X)$ and $(Y,\mathcal{T}_Y)$ be topological spaces, $f:(X,\mathcal{T}_X)\rightarrow (Y,\mathcal{T}_Y)$ be a continuous function. Then, $g:(X,\mathcal{T}_X)\rightarrow (f(X),\mathcal{T}_{f(X)})$ is also continuous.
  \end{corollary}
  \begin{prop}
    Let $(X,\mathcal{T})$ and $(Y,\mathcal{T}_Y)$ be topological spaces such that $X=A\cup B$, for some sets $A$, $B$. Consider the function $f:(X,\mathcal{T}_X)\rightarrow (Y,\mathcal{T}_Y)$ such that $f|_A$ and $f|_B$ are continuous. Then:
    \begin{enumerate}
      \item If $A$, $B$ are open, then $f$ is continuous.
      \item If $A$, $B$ are closed, then $f$ is continuous.
    \end{enumerate}
  \end{prop}
  \subsubsection{Cantor set}
  \begin{definition}
    Let $C_0=[0,1]$. Define $I_1:=\left(\frac{1}{3},\frac{2}{3}\right)$ and $C_1:=C_0\setminus I_1$. Then, define $I_2:=I_1\cup\left(\frac{1}{9},\frac{2}{9}\right)\cup\left(\frac{7}{9},\frac{8}{9}\right)$ and $C_2:=C_0\setminus I_2$. In general, define:
    \begin{gather*}
      I_{n+1}=I_n\cup\left[\bigcup_{k=0}^{3^n-1}\left(\frac{3k+1}{3^{n+1}},\frac{3k+2}{3^{n+1}}\right)\right]\\
      C_{n+1}=C_0\setminus I_{n+1}
    \end{gather*}
    Then, the \textit{Cantor set $\mathcal{C}$} is defined as: $$\mathcal{C}:=\bigcap_{n=0}^\infty C_n$$
  \end{definition}
  \begin{prop}
    The Cantor set $\mathcal{C}$ can be expressed as: $$\mathcal{C}=\{x\in[0,1]:x_3\footnote{Here, $x_3$ mean the expression of $x$ in base 3.}\text{ does not contain the digit 1}\}$$
  \end{prop}
  \begin{prop}
    The Cantor set $\mathcal{C}$ satisfies the following properties:
    \begin{enumerate}
      \item $\mathcal{C}\ne\varnothing$.
      \item $\mathcal{C}$ is closed in $\RR$.
      \item $\mathcal{C}$ does not contain any interval of $\RR$.
      \item $\Int\mathcal{C}=\varnothing$.
      \item $\mathcal{C}$ does not have the discrete topology.
      \item $\mathcal{C}$ is not countable.
    \end{enumerate}
  \end{prop}
  \subsection{Product topology}
  \subsubsection{Finite product}
  \begin{definition}
    Let $(X_i,\mathcal{T}_{X_i})$ be topological spaces for $i=1,\ldots,n$. We define the \textit{finite product topology on $X:=\prod_{i=1}^nX_i:=X_1\times \cdots\times X_n$}, denoted by $\mathcal{T}_X$, as the topology generated by $$\mathcal{B}=\{U_1\times\cdots\times U_n :U_i\in\mathcal{T}_{X_i},i=1,\ldots,n\}$$
  \end{definition}
  \begin{prop}
    Let $(X_i,\mathcal{T}_{X_i})$ be topological spaces for $i=1,\ldots,n$ and $X:=\prod_{i=1}^nX_i$. Then, $A\subseteq X$ is open on $(X,\mathcal{T}_X)$ if and only if $\forall a\in A$ $\exists U_i\in\mathcal{T}_{X_i}$ for $i=1,\ldots,n$ such that $a\in U_1\times\cdots\times U_n\subseteq A$.
  \end{prop}
  \begin{prop}
    Let $(X_i,\mathcal{T}_{X_i})$ be topological spaces for $i=1,\ldots,n$ and $X:=\prod_{i=1}^nX_i$. Then, the projection $$\pi_{X_i}:\left(X,\mathcal{T}_X\right)\longrightarrow (X_i,\mathcal{T}_{X_i})$$
    is continuous and open for $i=1,\ldots,n$.
  \end{prop}
  \begin{prop}
    Let $(X_i,\mathcal{T}_{X_i})$ be topological spaces for $i=1,\ldots,n$ and $X:=\prod_{i=1}^nX_i$. If $\mathcal{B}_{X_i}$ is a basis of $\mathcal{T}_{X_i}$ for $i=1,\ldots,n$, then: $$\mathcal{B}=\{U_1\times\cdots\times U_n :U_i\in\mathcal{B}_{X_i}\text{ for }i=1,\ldots,n\}$$
    is a basis of $\mathcal{T}_X$.
  \end{prop}
  \begin{prop}
    Let $(X_i,\mathcal{T}_{X_i})$ and $(Y,\mathcal{T}_Y)$ be topological spaces for $i=1,\ldots,n$, $X:=\prod_{i=1}^nX_i$ and $f:(Y,\mathcal{T}_Y)\rightarrow(X,\mathcal{T}_X)$ be a function. Then, $f$ is continuous if and only if $\pi_{X_i}\circ f$ is continuous for all $i=1,\ldots,n$.
  \end{prop}
  \begin{prop}
    Let $(X_i,\mathcal{T}_{X_i})$ and $(Y_i,\mathcal{T}_{Y_i})$ be topological spaces, $X:=\prod_{i=1}^nX_i$, $Y:=\prod_{i=1}^nY_i$ and $f_i:(X_i,\mathcal{T}_{X_i})\rightarrow(Y_i,\mathcal{T}_{Y_i})$ be a continuous function for $i=1,\ldots,n$. Then, $$f_1\times\cdots\times f_n:\left(X,\mathcal{T}_X\right)\longrightarrow\left(Y,\mathcal{T}_Y\right)$$ is also continuous.
  \end{prop}
  \begin{prop}
    Let $(X_i,\mathcal{T}_{X_i})$, $(A_i,\mathcal{T}_{A_i})\subseteq (X_i,\mathcal{T}_{X_i})$ be topological subspaces for $i=1,\ldots, n$. Consider the following topological spaces:
    \begin{enumerate}
      \item The topological space created from the product of subspaces $A_i$.
      \item The topological space created from the subspace $\prod_{i=1}^nA_i$ of the product $\prod_{i=1}^nX_i$.
    \end{enumerate}
    Then, these topological spaces are the same.
  \end{prop}
  \subsubsection{Arbitrary product}
  \begin{definition}
    Let $I$ be an index set, $\{(X_i,\mathcal{T}_{X_i}):i\in I\}$ be a collection of topological spaces and $X:=\prod_{i\in I}X_i$. We define the \textit{box topology on $X$} as the topology generated by $$\mathcal{B}=\left\{\prod_{i\in I}U_i:U_i\in\mathcal{T}_{X_i}\right\}$$
  \end{definition}
  \begin{definition}
    Let $I$ be an index set, $\{(X_i,\mathcal{T}_{X_i}):i\in I\}$ be a collection of topological spaces and $X:=\prod_{i\in I}X_i$. We define the \textit{infinite product topology on $X$}, denoted by $\mathcal{T}_X$, as the topology generated by
    \begin{multline*}
      \mathcal{B}=\Bigg\{\prod_{i\in I}U_i:U_i\in\mathcal{T}_{X_i}\text{ and $U_i=X_i$ except for}\\\text{a finite number of indexes}\Bigg\}
    \end{multline*}
  \end{definition}
  \begin{prop}
    Let $I$ be an index set, $\{(X_i,\mathcal{T}_{X_i}):i\in I\}$ be a collection of topological spaces and $X:=\prod_{i\in I}X_i$. Then, the projection $$\pi_{X_i}:\left(X,\mathcal{T}_X\right)\longrightarrow (X_i,\mathcal{T}_{X_i})$$
    is continuous and open for all $i\in I$.
  \end{prop}
  \begin{prop}
    Let $I$ be an index set, $\{(X_i,\mathcal{T}_{X_i}):i\in I\}$ be a collection of topological spaces and $X:=\prod_{i\in I}X_i$. If $\mathcal{B}_{X_i}$ is a basis of $\mathcal{T}_{X_i}$ $\forall i\in I$, then: $$\mathcal{B}=\left\{\prod_{i\in I}U_i:U_i\in\mathcal{B}_{X_i}\text{ for }i\in I\right\}$$
    is a basis of $\mathcal{T}_X$.
  \end{prop}
  \begin{prop}
    Let $I$ be an index set, $\{(X_i,\mathcal{T}_{X_i}):i\in I\}$ and $(Y,\mathcal{T}_Y)$ be a collection of topological spaces, $X:=\prod_{i\in I}X_i$ and $f:(Y,\mathcal{T}_Y)\rightarrow(X,\mathcal{T}_X)$ be a function. Then, $f$ is continuous if and only if $\pi_{X_i}\circ f$ is continuous for all $i\in I$.
  \end{prop}
  \begin{prop}
    Let $I$ be an index set, $\{(X_i,\mathcal{T}_{X_i}):i\in I\}$ and $\{(Y_i,\mathcal{T}_{Y_i}):i\in I\}$ be two collections of topological spaces, $X:=\prod_{i\in I}X_i$, $Y:=\prod_{i\in I}Y_i$ and $f_i:(X_i,\mathcal{T}_{X_i})\rightarrow(Y_i,\mathcal{T}_{Y_i})$ be a continuous function for all $i\in I$. Then, $$\prod_{i\in I}f_i:\left(X,\mathcal{T}_X\right)\longrightarrow\left(Y,\mathcal{T}_Y\right)$$ is also continuous.
  \end{prop}
  \begin{theorem}
    The function
    $$
      \begin{array}{r@{\hspace{0.5\tabcolsep}}c@{\hspace{0.5\tabcolsep}}c@{\hspace{0.5\tabcolsep}}c@{\hspace{0.5\tabcolsep}}c}
        \varphi: & \displaystyle\prod_{i=1}^\infty\{0,2\} & \longrightarrow & \mathcal{C}                                   \\
                 & (a_i)                                  & \longmapsto     & \displaystyle\sum_{i=1}^\infty\frac{a_i}{3^i}
      \end{array}
    $$
    is a homeomorphism.
  \end{theorem}
  % fins el Tema 10: acabat
\end{multicols}
\end{document}