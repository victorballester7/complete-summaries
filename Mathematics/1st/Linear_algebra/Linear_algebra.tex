\documentclass[../../../main.tex]{subfiles}

\begin{document}
\begin{multicols}{2}[\section{Linear algebra}]
\subsection{Matrices}
\subsubsection*{Linear systems}
\begin{definition}
    A \textit{linear equation} is an equation of the form $$a_1x_1+\cdots+a_nx_n=b$$ where $x_1,\ldots,x_n$ are the \textit{variables} or \textit{unknowns} and $a_i,b\in\RR$, $i=1,\ldots,n$, are the coefficients of the equation. The term $b$ is usually called \textit{constant term}.
\end{definition}
\begin{definition}
    A \textit{system of linear equations} is a collection of one or more linear equations involving the same set of variables.
\end{definition}
\begin{definition}
    Let
    \begin{equation*}
        \arraycolsep=1pt
        \left\{
        \begin{array}{ccccc}
            a_{11}x_1 & + \cdots + & a_{1n}x_n & = & b_1\\
            \vdots & \vdots & \vdots & & \vdots\\
            a_{m1}x_1 & + \cdots + & a_{mn}x_n & = & b_m
        \end{array}
        \right.
    \end{equation*}
    be a system of linear equations. A \textit{solution of a system of equations} is a set of numbers $c_1,\ldots,c_n$ such that $$a_{i1}c_1+\cdots+a_{in}c_n=b_i$$ for $i=1,\ldots,m$. A linear system may behave in three possible ways:
    \begin{enumerate}
        \item The system has a unique solution. 
        \item The system has infinitely many solutions.
        \item The system has no solution.
    \end{enumerate}
\end{definition}
\begin{definition}
    Two systems of equations are \textit{equivalent} if they have the same solutions.
\end{definition}
\subsubsection*{Matrices}
\begin{definition}[Matrix]
    A \textit{matrix $\mathblack{A}$ with coefficients in $\RR$} is a table of real numbers arranged in rows and columns. That is, $\mathblack{A}$ is of the form: 
    \begin{equation*}
        \mathblack{A}=(a_{ij})=
        \begin{pmatrix}
            a_{11} & \cdots & a_{1n} \\
            \vdots & \ddots & \vdots \\
            a_{m1} & \cdots & a_{mn} 
        \end{pmatrix}
    \end{equation*}
    for some values $a_{ij}\in\RR$, $i=1,\ldots,m$ and $j=1,\ldots,n$. The set of $m\times n$ matrices with real coefficients is denoted by $\mathcal{M}_{m\times n}(\RR)$\footnote{In the case when $m=n$ we will denote $\mathcal{M}_{n\times n}(\RR)$ by $\mathcal{M}_n(\RR)$.}.
\end{definition}
\begin{definition}
    Let $\mathblack{A},\mathblack{B}\in\mathcal{M}_{m\times n}(\RR)$ and $\alpha\in\RR$. If $\mathblack{A}=(a_{ij})$ and $\mathblack{B}=(b_{ij})$, we define the \textit{sum $\mathblack{A}+\mathblack{B}$} as: $$\mathblack{A}+\mathblack{B}=(a_{ij}+b_{ij})$$
    We define the \textit{product $\alpha\mathblack{A}$} as: $$\alpha\mathblack{A}=(\alpha a_{ij})$$
\end{definition}
\begin{prop}[Properties of addition and scalar multiplication of matrices]
    The following properties are satisfied:
    \begin{enumerate}
        \item Commutativity: $$\mathblack{A}+\mathblack{B}=\mathblack{B}+\mathblack{A}$$ for all $\mathblack{A},\mathblack{B}\in\mathcal{M}_{m\times n}(\RR)$.
        \item Associativity: $$(\mathblack{A}+\mathblack{B})+\mathblack{C}=\mathblack{A}+(\mathblack{B}+\mathblack{C})$$ for all $\mathblack{A},\mathblack{B},\mathblack{C}\in\mathcal{M}_{m\times n}(\RR)$.
        \item Additive identity element: $\exists\mathblack{0}\in\mathcal{M}_{m\times n}(\RR)$ such that $$\mathblack{A}+\mathblack{0}=\mathblack{A}$$ for all $\mathblack{A}\in\mathcal{M}_{m\times n}(\RR)$.
        \item Additive inverse element: $\forall\mathblack{A}\in\mathcal{M}_{m\times n}(\RR)$ $\exists(-\mathblack{A})\in\mathcal{M}_{m\times n}(\RR)$ such that $$\mathblack{A}+(-\mathblack{A})=\mathblack{0}$$
        \item Distributivity: $$(\alpha+\beta)\mathblack{A}=\alpha\mathblack{A}+\beta\mathblack{A}$$ for all $\mathblack{A}\in\mathcal{M}_{m\times n}(\RR)$ and all $\alpha,\beta\in\RR$.
    \end{enumerate}
\end{prop}
\begin{definition}
    Let $\mathblack{A}\in\mathcal{M}_{m\times n}(\RR)$ and $\mathblack{B}\in\mathcal{M}_{n\times p}(\RR)$. We define the \textit{product $\mathblack{A}\mathblack{B}$} as $$\mathblack{A}\mathblack{B}=(c_{ij})\quad\text{where }c_{ij}=\sum_{k=1}^na_{ik}b_{kj}$$
\end{definition}
\begin{prop}[Properties of matrix product]
    The following properties are satisfied:
    \begin{enumerate}
        \item Associativity: $$(\mathblack{A}\mathblack{B})\mathblack{C}=\mathblack{A}(\mathblack{B}\mathblack{C})$$ for all $\mathblack{A}\in\mathcal{M}_{m\times n}(\RR)$, $\mathblack{B}\in\mathcal{M}_{n\times p}(\RR)$ and $\mathblack{C}\in\mathcal{M}_{p\times q}(\RR)$.
        \item Multiplicative identity element: $\exists\mathblack{I}_n\in\mathcal{M}_n(\RR)$ such that 
        \begin{align*}
            &\mathblack{A}\mathblack{I}_n=\mathblack{A}\quad\forall\mathblack{A}\in\mathcal{M}_{m\times n}(\RR)\text{ and }\\
            &\mathblack{I}_n\mathblack{A}=\mathblack{A}\quad\forall\mathblack{A}\in\mathcal{M}_{n\times p}(\RR)
        \end{align*}
        \item Distributivity: $$(\mathblack{A}+\mathblack{B})\mathblack{C}=\mathblack{A}\mathblack{C}+\mathblack{B}\mathblack{C},$$ for all $\mathblack{A},\mathblack{B}\in\mathcal{M}_{m\times n}(\RR)$ and $\mathblack{C}\in\mathcal{M}_{n\times p}(\RR)$.
    \end{enumerate}
\end{prop}
\begin{definition}
    We say that a matrix $\mathblack{A}\in\mathcal{M}_n(\RR)$ is \textit{invertible} if there is a matrix $\mathblack{B}\in\mathcal{M}_n(\RR)$ satisfying $$\mathblack{A}\mathblack{B}=\mathblack{B}\mathblack{A}=\mathblack{I}_n$$
    The set of invertible matrices of size $n$ over $\RR$ is denoted by $\GL_n(\RR)$\footnote{Or more generally, the set of invertible matrices of size $n$ over a field (see definition \ref{AS-field}) $K$ is denoted by $\GL_n(K)$.}.
\end{definition}
\begin{lemma}
    The product of invertible matrices is invertible.
\end{lemma}
\subsubsection*{Echelon form of a matrix}
\begin{definition}
    Let $\mathblack{A}\in\mathcal{M}_{m\times n}(\RR)$ be a matrix. The \textit{$i$-th pivot of $\mathblack{A}$} is the first nonzero element in the $i$-th row of $\mathblack{A}$.
\end{definition}
\begin{definition}[Row echelon form]
    A matrix $\mathblack{A}\in\mathcal{M}_{m\times n}(\RR)$ is in \textit{row echelon form} if:
    \begin{itemize}
        \item All rows consisting of only zeros are at the bottom.
        \item The pivot of a nonzero row is always strictly to the right of the pivot of the row above it.
    \end{itemize}
\end{definition}
\begin{definition}[Reduced row echelon form]
    A matrix $\mathblack{A}\in\mathcal{M}_{m\times n}(\RR)$ is in \textit{reduced row echelon form} if:
    \begin{itemize}
        \item It is in row echelon form.
        \item Pivots are equal to 1.
        \item Each column containing a pivot has zeros in all its other entries.
    \end{itemize}
\end{definition}
\begin{theorem}[Gau\ss' theorem]
    Let $\mathblack{A}\in\mathcal{M}_{m\times n}(\RR)$ be a matrix. Then, there is a matrix $\mathblack{P}\in\GL_m(\RR)$ such that $\mathblack{P}\mathblack{A}=\mathblack{A'}$ is in reduced row echelon form. Moreover, $\mathblack{A'}$ is uniquely determined by $\mathblack{A}$.
\end{theorem}
\begin{theorem}[PAQ reduction theorem]
    Let $\mathblack{A}\in\mathcal{M}_{m\times n}(\RR)$ be a matrix. Then, there exist matrices $\mathblack{P}\in\GL_m(\RR)$ and $\mathblack{Q}\in\GL_n(\RR)$ such that 
    $$\mathblack{P}\mathblack{A}\mathblack{Q}=\left(
    \begin{array}{@{\,} c|c @{\,}}
        \mathblack{I}_r & \mathblack{0}\\
        \hline
        \mathblack{0} & \mathblack{0}
    \end{array}
    \right).$$
    The number $r$ is uniquely determined by $\mathblack{A}$.
\end{theorem}
\subsubsection*{Rank of a matrix}
\begin{definition}[Rank]
    Let $\mathblack{A}\in\mathcal{M}_{m\times n}(\RR)$ be a matrix and suppose 
    $$\mathblack{P}\mathblack{A}\mathblack{Q}=\left(
    \begin{array}{@{\,} c|c @{\,}}
        \mathblack{I}_r & \mathblack{0}\\
        \hline
        \mathblack{0} & \mathblack{0}
    \end{array}
    \right)$$ for some matrices $\mathblack{P}\in\mathcal{M}_m(\RR)$ and $\mathblack{Q}\in\mathcal{M}_n(\RR)$. We define the \textit{rank of $\mathblack{A}$}, denoted by $\rank \mathblack{A}$, as the number ones in the matrix $\mathblack{P}\mathblack{A}\mathblack{Q}$, that is, $\rank\mathblack{A}:=r$.
\end{definition}
\begin{prop}
    Let $\mathblack{A},\mathblack{A}'\in\mathcal{M}_{m\times n}(\RR)$, $\mathblack{B},\mathblack{B}'\in\mathcal{M}_{1\times n}(\RR)$ and $\mathblack{P}\in\GL_m(\RR)$ be matrices. Suppose we have a system of linear equations $\mathblack{A}\mathblack{x}=\mathblack{B}$. If $\mathblack{P}(\mathblack{A}\mid\mathblack{B})=(\mathblack{A}'\mid\mathblack{B}')$\footnote{Here $(\mathblack{A}\mid\mathblack{B})$ denotes the augmented matrix obtained by appending the columns of $\mathblack{B}$ to the columns of $\mathblack{A}$.}, then the systems $\mathblack{A}\mathblack{x}=\mathblack{B}$ and $\mathblack{A}'\mathblack{x}=\mathblack{B}'$ are equivalent.
\end{prop}
\begin{corollary}
    The reduced row echelon form of an invertible matrix is the identity matrix.
\end{corollary}
\begin{definition}[Transposition]
    Let $\mathblack{A}\in\mathcal{M}_{m\times n}(\RR)$ be a matrix. If $\mathblack{A}=(a_{ij})$, we define the \textit{transpose $\transpose{A}$ of $\mathblack{A}$} as the matrix $\transpose{A}=(b_{ij})$, where $b_{ij}=a_{ji}$ for $i=1,\ldots,m$ and $j=1,\ldots,n$.
\end{definition}
\begin{prop}
    Let $\mathblack{A}\in\mathcal{M}_{m\times n}(\RR)$ be a matrix. Then, $\rank \mathblack{A}=\rank\transpose{A}$.
\end{prop}
\begin{theorem}[Rouché-Frobenius theorem]
    Let $\mathblack{A}\mathblack{x}=\mathblack{B}$ be a system of equations with $n$ variables. The system is:
    \begin{itemize}
        \item \textit{determined and consistent} if and only if $$\rank \mathblack{A}=\rank (\mathblack{A}\mid \mathblack{B})=n$$
        \item \textit{indeterminate with $s$ free variables} if and only if $$\rank \mathblack{A}=\rank (\mathblack{A}\mid \mathblack{B})=n-s$$
        \item \textit{inconsistent} if and only if $$\rank \mathblack{A}\ne\rank (\mathblack{A}\mid \mathblack{B})$$
    \end{itemize}
\end{theorem}
\subsubsection*{Determinant of a matrix}
\begin{definition}[Determinant]
    A determinant is a function $\det:\mathcal{M}_n(\RR)\rightarrow\RR$ satisfying the following properties:
    \begin{enumerate}
        \item If $\mathblack{A}=(\mathblack{a}_1\mid\cdots\mid \mathblack{a}_n)$, where $\mathblack{a}_i$ are column vectors in $\RR^n$ for $i=1,\ldots,n$ and $\mathblack{a}_j=\lambda\mathblack{u}+\mu\mathblack{v}$ for some other column vectors $\mathblack{u}$ and $\mathblack{v}$, then:
        \begin{multline*}
            \det \mathblack{A}=\det(\mathblack{a}_1\mid\cdots\mid\mathblack{a}_j\mid\cdots\mid \mathblack{a}_n)=\\=\det(\mathblack{a}_1\mid\cdots\mid \mathblack{a}_{j-1}\mid\lambda \mathblack{u}+\mu \mathblack{v}\mid \mathblack{a}_{j+1}\mid\cdots\mid \mathblack{a}_n)=\\=\lambda\det(\mathblack{a}_1\mid\cdots\mid \mathblack{a}_{j-1}\mid \mathblack{u}\mid \mathblack{a}_{j+1}\mid\cdots\mid \mathblack{a}_n)+\\+\mu\det(\mathblack{a}_1\mid\cdots\mid \mathblack{a}_{j-1}\mid \mathblack{v}\mid \mathblack{a}_{j+1}\mid\cdots\mid \mathblack{a}_n)
        \end{multline*}
        \item The determinant changes its sign whenever two columns are swapped.
        \item $\det \mathblack{I}_n=1$ for all $n\in\NN$.
    \end{enumerate}
\end{definition}
\begin{lemma}
    Whenever two columns of a matrix are identical, the determinant is 0.
\end{lemma}
\begin{prop}
    Let $\mathblack{A}\in\mathcal{M}_n(\RR)$ be a matrix in its row echelon form. If $\mathblack{A}=(a_{ij})$, then: $$\det\mathblack{A}=\prod_{i=1}^na_{ii}$$ 
\end{prop}
\begin{prop}
    Let $\mathblack{A}\in\mathcal{M}_n(\RR)$ be a matrix. The following are equivalent:
    \begin{enumerate}
        \item $\mathblack{A}$ is not invertible.
        \item $\rank\mathblack{A}<n$.
        \item $\det\mathblack{A}=0$.
    \end{enumerate}
\end{prop}
\begin{theorem}
    Let $\det:\mathcal{M}_n(\RR)\rightarrow\RR$ be a determinant. Then, for all matrices $\mathblack{A},\mathblack{B}\in\mathcal{M}_n(\RR)$: $$\det (\mathblack{A}\mathblack{B})=\det\mathblack{A}\det\mathblack{B}.$$
\end{theorem}
\begin{corollary}
    Let $\det,\det':\mathcal{M}_n(\RR)\rightarrow\RR$ be two determinants. Then, for all matrix $\mathblack{A}\in\mathcal{M}_n(\RR)$: $$\det\mathblack{A}={\det}'\mathblack{A}$$
\end{corollary}
\begin{prop}
    Let $\mathblack{A}\in\mathcal{M}_n(\RR)$. Then: $$\det \mathblack{A}=\sum_{\sigma\in S_n}\varepsilon(\sigma)\prod_{i=1}^na_{i\sigma(i)}$$
\end{prop}
\begin{prop}
    For all matrix $\mathblack{A}\in\mathcal{M}_n(\RR)$: $$\det\mathblack{A}=\det\transpose{A}$$
\end{prop}
\begin{prop}
    Let $\mathblack{A}=(a_{ij})\in\mathcal{M}_n(\RR)$. We denote by $\mathblack{A}_{ij}$ the square matrix obtained from $\mathblack{A}$ by removing the $i$-th row and $j$-th column. Then, for every $i\in\{1,\ldots,n\}$, $$\det\mathblack{A}=\sum_{j=1}^n(-1)^{i+j}a_{ij}\det\mathblack{A}_{ij}.$$
\end{prop}
\begin{definition}
    Let $\mathblack{A}=(a_{ij})\in\mathcal{M}_n(\RR)$. We define the \textit{cofactor matrix $\mathblack{C}$ of $\mathblack{A}$} as: $$\mathblack{C}=(b_{ij}),\quad\text{where }b_{ij}=(-1)^{i+j}\det\mathblack{A}_{ij}\footnote{$\mathblack{C}$ is usually denoted as $\cofactor\mathblack{A}$.}.$$ We define the \textit{adjugate matrix $\adjugate\mathblack{A}$ of $\mathblack{A}$} as: $$\adjugate\mathblack{A}=\transpose{\mathblack{C}}.$$
\end{definition}
\begin{theorem}
    Let $\mathblack{A}\in\mathcal{M}_n(\RR)$. Then: $$\mathblack{A}\adjugate\mathblack{A}=(\det \mathblack{A})\mathblack{I}_n$$ Moreover if $\det \mathblack{A}\ne 0$, then: $$\mathblack{A}^{-1}=\frac{1}{\det \mathblack{A}}\adjugate\mathblack{A}$$
\end{theorem}
\subsection{Vector spaces}
\subsubsection*{Introduction and basic definitions}
\begin{definition}
    A \textit{vector space over a field\footnote{See definition \ref{AS-field}} $K$} is a set $E$ together with two operations
    \begin{align*}
        +:E\times E&\longrightarrow E&\cdot:K\times E&\longrightarrow E\\
        (\mathblack{u},\mathblack{v})&\longmapsto \mathblack{u}+ \mathblack{v}&(\lambda,\mathblack{v})&\longmapsto \lambda\cdot \mathblack{v}
    \end{align*}
    that satisfy the following properties:
    \begin{enumerate}
        \item $\mathblack{u}+(\mathblack{v}+\mathblack{w})=(\mathblack{u}+\mathblack{v})+\mathblack{w}\quad\forall\mathblack{u},\mathblack{v},\mathblack{w}\in E$.
        \item $\mathblack{u}+\mathblack{v}=\mathblack{u}+\mathblack{v}\quad\forall\mathblack{u},\mathblack{v}\in E$.
        \item $\exists\mathblack{0}\in E$ such that $\mathblack{v}+\mathblack{0}=\mathblack{v}\quad\forall\mathblack{v}\in E$.
        \item $\forall\mathblack{v}\in E$ there exists $-\mathblack{v}\in E$ such that $\mathblack{v}+(-\mathblack{v})=\mathblack{0}$.                  
        \item $\lambda\cdot(\mu\cdot\mathblack{v})=(\lambda\mu)\cdot\mathblack{v}\quad\forall\mathblack{v}\in E$ and $\forall\lambda,\mu\in K$. 
        \item $1\cdot\mathblack{v}=\mathblack{v}\quad\forall\mathblack{v}\in E$, where 1 denotes the multiplicative identity element in $K$.
        \item $\lambda\cdot(\mathblack{u}+\mathblack{v})=\lambda\cdot\mathblack{u}+\lambda\cdot\mathblack{v}\quad\forall\mathblack{u},\mathblack{v}\in E$ and $\forall\lambda\in K$.
        \item $(\lambda+\mu)\mathblack{v}=\lambda\mathblack{v}+\mu\mathblack{v}\quad\forall\mathblack{v}\in E$ and $\forall\lambda,\mu\in K$.
    \end{enumerate}
    In these conditions, we say that $(E,+,\cdot)$ is a vector space\footnote{For simplicity we will denote the vector space only by $V$ and if the context is clear we won't refer to its associated field. Moreover sometimes we will also omit the product $\cdot$ between scalars and vectors.}.
\end{definition}
\begin{definition}
    Let $E$ be a vector space over a field $K$ and $F\subseteq E$ be a subset of $E$. Then, $F$ is a vector space over $K$ if the following property is satisfied:
    $$\lambda \mathblack{u}+\mu \mathblack{v}\in F\quad\forall \mathblack{u},\mathblack{v}\in F\text{ and }\forall\lambda,\mu\in K$$
\end{definition}
\begin{definition}
    Let $E$ be a vector space and $F\subseteq E$. $F$ is a \textit{vector subspace of $E$} if it's itself a vector space with the operations defined in $E$.
\end{definition}
\begin{definition}
    Let $E$ be a vector space over a field $K$. A \textit{linear combination of the vectors $\mathblack{v}_1,\ldots,\mathblack{v}_n\in E$} is a vector of the form $$a_1\mathblack{v}_1+\cdots+a_n\mathblack{v}_n$$ where $a_i\in K$, $i=1,\ldots,n$. 
\end{definition}
\begin{definition}
    Let $E$ be a vector space over a field $K$ and $F\subseteq E$. The set $$\langle F\rangle=\{a_1\mathblack{v}_1+\cdots+a_n\mathblack{v}_n:a_i\in K,\mathblack{v}_i\in F,i=1,\ldots,n\}$$ is called \textit{subspace generated by $F$}.
\end{definition}
\begin{lemma}
    Let $E$ be a vector space and $F\subseteq E$. Then, $\langle F\rangle$ is a vector subspace of $E$. Moreover, $\langle F\rangle$ is the smallest subspace containing $F$.
\end{lemma}
\begin{definition}
    Let $E$ be a vector space and $F\subseteq E$. We say that $F$ is a \textit{generating set of $E$} if $\langle F\rangle=E$.
\end{definition}
\subsubsection*{Linear independence}
\begin{definition}
    Let $E$ be a vector space over a field $K$. The vectors $\mathblack{v}_1,\ldots,\mathblack{v}_n\in E$ are \textit{linearly independent} if the unique solution of the equation $$a_1\mathblack{v}_1+\cdots+a_n\mathblack{v}_n=0$$ for $a_i\in K$, $i=1,\ldots,n$, is $a_1=\cdots=a_n=0$. Otherwise we say that the vectors $\mathblack{v}_1,\ldots,\mathblack{v}_n$ are \textit{linearly dependent}.
\end{definition}
\begin{lemma}
    Let $E$ be a vector space. The vectors $\mathblack{v}_1,\ldots,\mathblack{v}_n$ are linearly dependent if and only if one of them is a linear combination of the others.
\end{lemma}
\begin{definition}
    Let $E$ be a vector space. A \textit{basis of $E$} is an ordered set $\mathfrak{B}=(\mathblack{v}_1,\ldots,\mathblack{v}_n)$ of vectors of $E$ such that: 
    \begin{enumerate}
        \item $\langle \mathblack{v}_1,\ldots,\mathblack{v}_n\rangle = E$.
        \item $\mathblack{v}_1,\ldots,\mathblack{v}_n$ are linearly independent.
    \end{enumerate}
\end{definition}
\begin{lemma}[Steinitz exchange lemma]
    Let $E$ be a vector space, $\mathfrak{B}$ be bases of $E$ be and $\mathblack{v}_1,\ldots,\mathblack{v}_k\in E$ be linearly independent vectors of $E$. Then, we can exchange $k$ appropriate vectors of $\mathfrak{B}$ by $\mathblack{v}_1,\ldots,\mathblack{v}_k$ to define a new basis.
\end{lemma}
\begin{corollary}
    Let $E$ be a vector space that has a finite basis $\mathfrak{B}=(\mathblack{v}_1,\ldots,\mathblack{v}_n)$. Then, all basis of $E$ be are finite and they have the same number ($n$) of vectors.
\end{corollary}
\begin{lemma}
    Let $E$ be a vector space. Suppose we have a generating set $S=\{\mathblack{v}_1,\ldots,\mathblack{v}_n\}$ of $E$. Then, $E$ be admits a basis formed with a subset of $S$.
\end{lemma}
\begin{definition}
    Let $E$ be a finite vector space. The \textit{dimension of $E$}, denoted by $\dim E$, is the number of vectors in any basis of $E$.
\end{definition}
\begin{definition}
    Let $E$ be a finite vector space over a field $K$, $\mathfrak{B}=(\mathblack{v}_1,\ldots,\mathblack{v}_n)$ be a basis of $E$ be and $\mathblack{u}\in E$. Suppose $$\mathblack{u}=a_1\mathblack{v}_1+\cdots+a_n\mathblack{v}_n$$ for some $a_i\in K$, $i=1,\ldots,n$. We call $(a_1,\ldots,a_n)\in K^n$ \textit{coordinates of $\mathblack{u}$ on the basis $\mathfrak{B}$} and we denote it by $[\mathblack{u}]_{\mathfrak{B}}$.
\end{definition}
\begin{prop}
    Let $E$ be a vector space. If $\dim E<\infty$, the maximum number of linearly independent vectors is equal to $\dim E$. If $\dim E=\infty$, there is no such maximum.
\end{prop}
\begin{prop}
    Let $E$ be a vector space of dimension $n$. Then, $n$ is de minimum size of a generating set of $E$.
\end{prop}
\begin{prop}
    Let $E$ be a finite vector space and $F$ be a vector subspace of $E$. Then, $\dim F\leq\dim E$ and $$\dim F=\dim E\iff F=E$$
\end{prop}
\subsubsection*{Sum of subspaces}
\begin{lemma}
    Let $E$ be a vector space and $F,G\subseteq E$ be two vector subspaces of $E$. Then, the intersection $F\cap G$ is a vector subspace of $E$.
\end{lemma}
\begin{definition}
    Let $E$ be a vector space and $F,G\subseteq E$ be two vector subspaces of $E$. The \textit{sum of $F$ and $G$} is: $$F+G=\langle F\cup G\rangle=\{\mathblack{u}+\mathblack{v}: \mathblack{u}\in F,\mathblack{v}\in G\}$$
\end{definition}
\begin{prop}[Gra\ss mann formula]
    Let $E$ be a finite vector space and $F,G\subseteq E$ be two vector subspace of $E$. Then: $$\dim (F+G)+\dim(F\cap G)=\dim F+\dim G$$
\end{prop}
\begin{lemma}
    Let $E$ be a vector space and $F,G\subseteq E$ be two vector subspaces of $E$. Then, $F\cap G=\{0\}$ if and only if all vector $\mathblack{w}\in F+G$ can be written uniquely as $\mathblack{w}=\mathblack{u}+\mathblack{v}$, with $\mathblack{u}\in F$ and $\mathblack{v}\in G$.
\end{lemma}
\begin{definition}[Direct sum]
    Let $E$ be a vector space and $F,G\subseteq E$ be two vector subspaces of $E$. Then, the sum $F+G$ is \textit{direct} if $F\cap G=\{0\}$. In this case we denote the sum as $F\oplus G$. More generally, if $F_1,\ldots,F_n\subseteq E$ are vector subspaces of $E$, the sum $F_1+\cdots+F_n$ is direct if all vector $\mathblack{w}\in F$ can be written uniquely as $\mathblack{w}=\mathblack{v}_1+\cdots+\mathblack{v}_n$, where $\mathblack{v}_i\in F_i$ for $i=1,\ldots,n$. In this case we denote the sum by $F_1\oplus\cdots\oplus F_n$.
\end{definition}
\subsubsection*{Rank of a matrix}
\begin{definition}
    Let $\mathblack{A}\in\mathcal{M}_{n\times m}(\RR)$. The \textit{row rank of $\mathblack{A}$} is the dimension of the subspace generated by the rows of $\mathblack{A}$ in $\RR^m$. Analogously, the \textit{column rank of $\mathblack{A}$} is the dimension of the subspace generated by the columns of $\mathblack{A}$ in $\RR^n$.
\end{definition}
\begin{prop}
    Let $\mathblack{A}\in\mathcal{M}_{n\times m}(\RR)$. Then, the row rank of $\mathblack{A}$ is equal to the column rank of $\mathblack{A}$. Therefore, we refer to it simply as \textit{rank of $\mathblack{A}$} or $\rank\mathblack{A}$.
\end{prop}
\begin{definition}
    Let $\mathblack{A}\in\mathcal{M}_n(\RR)$. A \textit{minor of order $k$ of $\mathblack{A}$} is a submatrix $\mathblack{A}'\in\mathcal{M}_k(\RR)$ obtained from $\mathblack{A}$ selecting $k$ rows and $k$ columns of $\mathblack{A}$.
\end{definition}
\begin{prop}
    Let $\mathblack{A}\in\mathcal{M}_{n\times m}(\RR)$. Then:
     $$\rank\mathblack{A}=\max\{k:\text{$\mathblack{A}$\ has an invertible minor of order $k$}\}$$
\end{prop}
\subsubsection*{Quotient vector space}
\begin{definition}
    Let $E$ be a vector space and $F\subseteq E$ be a vector subspace. We say that $G\subseteq E$ is a \textit{complementary subspace of $F$} if $F\oplus G=E$.
\end{definition}
\begin{definition}
    Let $E$ be a finite vector space of dimension $n$ and $F\subseteq E$ be a vector subspace of dimension $m$. Then, there exists a complementary subspace of $F$ and its dimension is $n-m$.
\end{definition}
\begin{definition}
    Let $E$ be a vector space and $F\subseteq E$ be a vector subspace. We say the vectors $\mathblack{u},\mathblack{v}\in E$ are \textit{equivalent modulo $F$}, $\mathblack{u}\sim_F\mathblack{v}$, if $\mathblack{u}-\mathblack{v}\in F$.
\end{definition}
\begin{lemma}
    Let $E$ be a vector space and $F\subseteq E$ be a vector subspace. Then, $\sim_F$ is an equivalence relation and, moreover, if $\mathblack{v}\in E$ the \textit{equivalence class $[\mathblack{v}]$ of $\mathblack{v}$} is: $$[\mathblack{v}]=\mathblack{v}+F:=\{\mathblack{v}+\mathblack{u}:\mathblack{u}\in F\}$$
\end{lemma}
\begin{definition}
    Let $E$ be a vector space over a field $K$ and $F\subseteq E$ be a vector subspace. We define the \textit{quotient space $\quot{E}{F}$ under $\sim_F$} as the set of equivalence classes with the operations defined as:
    $$[\mathblack{u}]+[\mathblack{v}]=[\mathblack{u}+\mathblack{v}]\qquad \lambda[\mathblack{v}]=[\lambda\mathblack{v}]$$
    for all $\mathblack{u},\mathblack{v}\in E$ and all $\lambda\in K$.
\end{definition}
\begin{prop}
    Let $E$ be a vector space over a field $K$ and $F\subseteq E$ be a vector subspace. The set $\quot{E}{F}$ together with the two operation defined above is a vector space over $K$.
\end{prop}
\begin{prop}
    Let $E$ be a finite vector space of dimension $n$ and $F\subseteq E$ be a vector subspace. Then: $$\dim\left(\quot{E}{F}\right)=\dim E-\dim F$$
\end{prop}
\subsection{Linear maps}
\begin{definition}
    Let $E$, $F$ be two vector spaces over a field $K$. A function $f:E\rightarrow F$ is a \textit{linear map} if $\forall\mathblack{u}\in E$ $\forall\mathblack{v}\in F$ and $\forall\lambda\in K$ the following two conditions are satisfied:
    \begin{enumerate}
        \item $f(\mathblack{u}+\mathblack{v})=f(\mathblack{u})+f(\mathblack{v})$.
        \item $f(\lambda \mathblack{u})=\lambda f(\mathblack{u})$.
    \end{enumerate}
\end{definition}
\begin{prop}
    Let $E$, $F$ be two vector spaces over a field $K$. Then, if $f:E\rightarrow F$ is a linear map, $\forall\mathblack{u}\in E$ $\forall\mathblack{v}\in F$ and $\forall\lambda,\mu\in K$ we have:
    \begin{enumerate}
        \item $f(\mathblack{0})=\mathblack{0}$.
        \item $f(-\mathblack{u})=-f(\mathblack{u})$.
        \item $f(\lambda\mathblack{u}+\mu\mathblack{v})=\lambda f(\mathblack{u})+\mu f(\mathblack{v})$.
    \end{enumerate}
\end{prop}
\begin{prop}
    Let $E$, $F$, $G$ be three vector spaces. If $f:E\rightarrow F$ and $g:F\rightarrow G$ are linear maps, then $g\circ f:E\rightarrow G$ is a linear map.
\end{prop}
\begin{prop}
    Let $E$, $F$ be two vector spaces. If $f:E\rightarrow F$ is a bijective linear map, then $f^{-1}:F\rightarrow E$ is a linear map.
\end{prop}
\begin{prop}
    Let $E$, $F$ be two vector spaces, $f:E\rightarrow F$ be a linear map and $G\subseteq E$ and $H\subseteq F$ be vector subspaces. Then:
    \begin{enumerate}
        \item $f(G)=\{f(\mathblack{u}): \mathblack{u}\in G\}\subseteq F$ is a vector subspace.
        \item $f^{-1}(H)=\{\mathblack{u}\in E: f(\mathblack{u})\in H\}\subseteq E$ is a vector subspace.
    \end{enumerate}
    In particular, $f(E)$ is denoted by $\im f$ and $f^{-1}(\{0\})$ is denoted by $\ker f$ and these subspaces are called \textit{image of $f$} and \textit{kernel of $f$}, respectively. More precisely, their definitions are:
    $$\im f=\{f(\mathblack{u}): \mathblack{u}\in E\}\qquad\ker f=\{\mathblack{u}\in E: f(\mathblack{u})=0\}$$
\end{prop}
\begin{prop}
    Let $E$, $F$ be two vector spaces and $f:E\rightarrow F$ be a linear map. Then:
    \begin{enumerate}
        \item $f$ is injective if and only if $\ker f=\{0\}$
        \item $f$ is surjective if and only if $\im f=F$.
    \end{enumerate}
\end{prop}
\begin{corollary}
    Let $E$, $F$ be two finite vector spaces and $f:E\rightarrow F$ be a linear map. Then:
    \begin{enumerate}
        \item $f$ is injective if and only if $\dim(\ker f)=0$
        \item $f$ is surjective if and only if $\dim(\im f)=\dim F$.
    \end{enumerate}
\end{corollary}
\begin{definition}
\hfill
\begin{itemize}
    \item A monomorphism is an injective linear map.
    \item An epimorphism is a surjective linear map.
    \item An isomorphism is a bijective linear map.
    \item An endomorphism is a linear map from a vector space to itself.
    \item An automorphism is a bijective endomorphism.
\end{itemize}
\end{definition}
\begin{definition}
    We say that two vector spaces $E$ and $F$ are \textit{isomorphic}, $E\cong F$, if there exists an isomorphism between them.
\end{definition}
\begin{prop}
    Let $E$, $F$ be two vector spaces and $f:E\rightarrow F$ be a monomorphism. If $\mathblack{u}_1,\ldots,\mathblack{u}_n\in E$ are linearly independent vectors, then $f(\mathblack{u}_1),\ldots,f(\mathblack{u}_n)$  are linearly independent.
\end{prop}
\begin{lemma}
    Let $E$, $F$ be two vector spaces and $f:E\rightarrow F$ be a linear map. If $\mathblack{u}_1,\ldots,\mathblack{u}_n\in E$, then: $$\langle f(\mathblack{u}_1),\ldots,f(\mathblack{u}_n)\rangle=f(\langle\mathblack{u}_1,\ldots,\mathblack{u}_n\rangle)$$
\end{lemma}
\begin{corollary}
    Let $E$, $F$ be two vector spaces and $f:E\rightarrow F$ be an epimorphism. If $\langle\mathblack{u}_1,\ldots,\mathblack{u}_n\rangle= E$, then $\langle f(\mathblack{u}_1),\ldots,f(\mathblack{u}_n)\rangle=F$.
\end{corollary}
\begin{corollary}
    Let $E$, $F$ be two vector spaces and $f:E\rightarrow F$ be an isomorphism. If $(\mathblack{u}_1,\ldots,\mathblack{u}_n)$ is a basis of $E$, then $(f(\mathblack{u}_1),\ldots,f(\mathblack{u}_n))$ is a basis of $F$. 
\end{corollary}
\begin{theorem}[Coordination theorem]
    Let $E$ be a finite vector space over a field $K$ of dimension $n$ and $\mathfrak{B}=(\mathblack{u}_1,\ldots,\mathblack{u}_n)$ be a basis of $E$. Then, the function $f:K^n\rightarrow E$ defined by $$f(a_1,\ldots,a_n)=a_1\mathblack{u}_1+\cdots a_n\mathblack{u}_n$$ is a isomorphism.
\end{theorem}
\begin{corollary}
    Two finite vector spaces are isomorphic if and only if they have the same dimension.
\end{corollary}
\subsubsection*{Isomorphism theorems}
\begin{theorem}[First isomorphism theorem]
    Let $E$, $F$ be two vector spaces and $f:E\rightarrow F$ be a linear map. Then, there exists an isomorphism $\Tilde{f}:\quot{E}{\ker f}\rightarrow \im f$ satisfying $f=\Tilde{f}\circ\pi$, where $\pi:E\rightarrow \quot{E}{\ker f}$, $\pi(\mathblack{u})=[\mathblack{u}]$.
    \illustration{0.35}{Images/first_isomorphism}{}{}
\end{theorem}
\begin{corollary}
    Let $E$, $F$ be two vector spaces such that $\dim E=n$ and let $f:E\rightarrow F$ be a linear map. Then: $$\dim(\ker f)+\dim(\im f)=n$$
\end{corollary}
\begin{corollary}
    Let $E$, $F$ be two finite vector spaces of dimensions $n$ and $f:E\rightarrow F$ be a linear map. Then: $$f\text{ is injective}\iff f\text{ is surjective}\iff f\text{ is bijective}$$
\end{corollary}
\begin{theorem}[Second isomorphism theorem]
    Let $E$, $F$ be two vector spaces and $G\subseteq E$ be a vector subspace. Then, there exists an isomorphism $$\quot{F}{F\cap G}\cong\quot{F+G}{G}$$
\end{theorem}
\begin{theorem}[Third isomorphism theorem]
    Let $E$, $F$, $G$ be three vector spaces such that $G\subseteq F\subseteq E$. Then, there exists an isomorphism $$\quot{{(\textstyle\quot{E}{G})}}{{(\textstyle\quot{F}{G})}}\cong\quot{E}{F}$$
\end{theorem}
\begin{theorem}
    Let $E$, $F$ be two vector spaces over a field $K$, $\mathfrak{B}=(\mathblack{u}_1,\ldots,\mathblack{u}_n)$ be a basis of $E$ and $\mathblack{v}_1,\ldots,\mathblack{v}_n\in F$ be any vectors of $F$. Then, there exists a unique linear map $f:E\rightarrow F$ such that $f(\mathblack{u}_i)=\mathblack{v}_i$, $i=1,\ldots,n$. 
\end{theorem}
\subsubsection*{Matrix of a linear map}
\begin{prop}
    Let $E$, $F$ be two finite vector spaces over a field $K$ with $\dim E=n$ and $\dim F=m$, $\mathfrak{B}$ and $\mathfrak{B}'$ be bases of $E$ be and $F$ respectively and $f:E\rightarrow F$ be a linear map. Then, there exists a matrix $\mathblack{A}\in\mathcal{M}_{m\times n}(K)$ such that $\forall\mathblack{u}\in E$ $$[f(\mathblack{u})]_{\mathfrak{B}'}=\mathblack{A}[\mathblack{u}]_\mathfrak{B}$$
    The matrix $\mathblack{A}$ is called \textit{matrix of $f$ in the basis $\mathfrak{B}$ and $\mathfrak{B}'$} and it is denoted by $[f]_{\mathfrak{B},\mathfrak{B}'}$\footnote{If $E=F$ and $\mathfrak{B}=\mathfrak{B}'$, we denote $[f]_{\mathfrak{B},\mathfrak{B}}$ simply by $[f]_{\mathfrak{B}}$.}. 
\end{prop}
\begin{corollary}
    Let $E$ be a finite vector space, $\mathfrak{B}$ and $\mathfrak{B}'$ be two basis of $E$ respectively and $\id:E\rightarrow E$ be the identity linear map. Then, $\forall\mathblack{u}\in E$ we have: $$[\mathblack{u}]_{\mathfrak{B}'}=[\id]_{\mathfrak{B},\mathfrak{B}'}[\mathblack{u}]_\mathfrak{B}$$ The matrix $[\id]_{\mathfrak{B},\mathfrak{B}'}$ is called \textit{change-of-basis matrix}. 
\end{corollary}
\begin{prop}
    Let $E$, $F$, $G$ be three vector spaces, $\mathfrak{B}$, $\mathfrak{B}'$, $\mathfrak{B}''$ be bases of $E$, $F$ and $G$ respectively and $f:E\rightarrow F$ and $g:F\rightarrow G$ be linear maps. Then, $g\circ f:E\rightarrow G$ has the following matrix in the basis $\mathfrak{B}$ and $\mathfrak{B}''$: $$[g\circ f]_{\mathfrak{B},\mathfrak{B}''}=[g]_{\mathfrak{B}',\mathfrak{B}''}[f]_{\mathfrak{B},\mathfrak{B}'}$$ 
\end{prop}
\begin{corollary}
    Let $E$ be a finite vector space, $\mathfrak{B}$ and $\mathfrak{B}'$ be two basis of $E$. Then, the matrix $[id]_{\mathfrak{B},\mathfrak{B}'}$ is invertible and $${\left([\id]_{\mathfrak{B},\mathfrak{B}'}\right)}^{-1}=[\id]_{\mathfrak{B}',\mathfrak{B}}$$
\end{corollary}
\begin{corollary}
    Let $E$, $F$ be two finite vector spaces, $\mathfrak{B}$ and $\mathfrak{B}'$ be bases of $E$ and $F$ respectively and $f:E\rightarrow F$ be a linear map. Then:
    \begin{enumerate}
        \item $f$ is injective $\iff\rank[f]_{\mathfrak{B},\mathfrak{B}'}=\dim E$.
        \item $f$ is surjective $\iff\rank[f]_{\mathfrak{B},\mathfrak{B}'}=\dim F$.
    \end{enumerate}
\end{corollary}
\begin{corollary}
    Let $E$, $F$ be two finite vector spaces. A linear map $f:E\rightarrow F$ is an isomorphism if and only if there exist basis $\mathfrak{B}$ and $\mathfrak{B}'$ of $E$ and $F$ respectively such that $[f]_{\mathfrak{B},\mathfrak{B}'}$ is invertible.
\end{corollary}
\begin{prop}[Change of basis formula]
    Let $E$, $F$ be two finite vector spaces, $\mathfrak{B}_1$ and $\mathfrak{B}_2$ be bases of $E$, $\mathfrak{B}_1'$ and $\mathfrak{B}_2'$ be bases of $F$ and $f:E\rightarrow F$ be a linear map. Then: $$[f]_{\mathfrak{B}_2,\mathfrak{B}_2'}=[\id]_{\mathfrak{B}_1',\mathfrak{B}_2'}[f]_{\mathfrak{B}_1,\mathfrak{B}_1'}[\id]_{\mathfrak{B}_2,\mathfrak{B}_1}$$
\end{prop}
\begin{lemma}
    Let $E$, $F$ be two finite vector spaces over a field $K$ with $\dim E=n$ and $\dim F=m$ and $\mathfrak{B}$ and $\mathfrak{B}'$ be bases of $E$ be and $F$ respectively. Then, any matrix $\mathblack{A}\in\mathcal{M}_{m\times n}(K)$ determines a linear map $f:E\rightarrow F$ with $[f]_{\mathfrak{B},\mathfrak{B}'}=\mathblack{A}$.
\end{lemma}
\begin{theorem}
    Let $E$, $F$ be two finite vector spaces and $f:E\rightarrow F$ be a linear map. Then, there exist basis $\mathfrak{B}_0$ of $E$ be and $\mathfrak{B}_0'$ of $F$ such that: 
    $$[f]_{\mathfrak{B}_0,\mathfrak{B}_0'}=\left(
    \begin{array}{c|c}
        \mathblack{I}_r & \mathblack{0}\\
        \hline
        \mathblack{0} & \mathblack{0}
    \end{array}\right)$$
    where $r=\dim\left(\im f\right)$.
\end{theorem}
\subsubsection*{Dual space}
\begin{lemma}
    Let $E$, $F$ be two finite vector spaces over a field $K$. Then, the set $$\mathcal{L}(E,F):=\{f: f\text{ is a linear map from $E$ to $F$}\}\footnote{If $E=F$, we denote $\mathcal{L}(E,E)$ simply as $\mathcal{L}(E)$.}$$ is a vector space over $K$ with the operations defined as:
    \begin{enumerate}
        \item $(f+g)(\mathblack{v})=f(\mathblack{u})+f(\mathblack{v})\quad\forall f,g\in\mathcal{L}(E,F)$.
        \item $(f\lambda)(\mathblack{v})=\lambda f(\mathblack{v})\quad\forall f,g\in\mathcal{L}(E,F)$ and $\forall \lambda\in K$.
    \end{enumerate}
\end{lemma}
\begin{prop}
    Let $E$, $F$ be two finite vector spaces over a field $K$ with $\dim E=n$, $\dim F=m$. Then, for all basis $\mathfrak{\mathfrak{B}}$ of $E$ be and $\mathfrak{B}'$ of $F$, the function 
    \begin{align*}
        \mathcal{L}(E,F)&\longrightarrow\mathcal{M}_{m\times n}(K)\\
        f&\longmapsto[f]_{\mathfrak{B},\mathfrak{B}'}
    \end{align*}
    is a isomorphism.
\end{prop}
\begin{corollary}
    Let $E$, $F$ be two finite vector spaces with $\dim E=n$, $\dim F=m$. Then, $\dim \mathcal{L}(E,F)=mn$.
\end{corollary}
\begin{definition}
    Let $E$ be a vector space over a field $K$. We define the \textit{dual space $E^*$ of $E$} as: $$E^*:=\mathcal{L}(E,K)$$
\end{definition}
\begin{prop}
    Let $E$ be a finite vector space over a field $K$ with $\dim E=n$ and $\mathfrak{B}$ be a basis of $E$. Then, the function
    \begin{align*}
        E^*&\longrightarrow\mathcal{M}_{1\times n}(K)\\
        \omega&\longmapsto[\omega]_{\mathfrak{B},1}
    \end{align*}
    is a isomorphism. Therefore, $\dim E^*=\dim E$.
\end{prop}
\begin{definition}
    We define the \textit{Kronecker delta $\delta_{ij}$} as the function: $$\delta_{ij}=\left\{
    \begin{array}{ccc}
        0 & \text{if} & i\ne j\\
        1 & \text{if} & i=j
    \end{array}
    \right.$$
\end{definition}
\begin{definition}
    Let $E$ be a finite vector space and $\mathfrak{B}=(\mathblack{u}_1,\ldots,\mathblack{u}_n)$ be a basis of $E$. We define the \textit{dual basis $\mathfrak{B}^*$ of $\mathfrak{B}$} as the basis of $E^*$ formed by $(\eta_1,\ldots,\eta_n)$ where $$\eta_i(\mathblack{u}_j)=\delta_{ij}$$
\end{definition}
\begin{lemma}
    Let $E$ be a vector space over a field $K$, $\mathfrak{B}$ be a basis of $E$ and $(\mathblack{u}_1^*,\ldots,\mathblack{u}_n^*)$ be the dual basis of $\mathfrak{B}$. Then, $\forall \mathblack{v}\in E$: $$[\mathblack{v}]_\mathfrak{B}=(\mathblack{u}_1^*(\mathblack{v}),\ldots,\mathblack{u}_n^*(\mathblack{v}))\in K^n$$
\end{lemma}
\begin{lemma}
    Let $E$ be a vector space over a field $K$, $\mathfrak{B}=(\mathblack{u}_1,\ldots,\mathblack{u}_n)$ be a basis of $E$ and $\mathfrak{B}^*$ be the dual basis of $\mathfrak{B}$. Then, $\forall \omega\in E^*$: $$[\omega]_{\mathfrak{B}^*}=(\omega(\mathblack{u}_1),\ldots,\omega(\mathblack{u}_n))\in K^n$$
\end{lemma}
\begin{definition}[Dual map]
    Let $E$, $F$ be two vector spaces over a field $K$ and $f\in \mathcal{L}(E,F)$. The function $f^*$ defined by
    \begin{align*}
        f^*:F^*&\longrightarrow E^*\\
        \omega &\longmapsto\omega\circ f
    \end{align*}
    is a linear map and it's called \textit{dual map of $f$}.
\end{definition}
\begin{theorem}
    Let $E$, $F$ be two finite vector spaces, $\mathfrak{B}$ and $\mathfrak{B}'$ be bases of $E$ and $F$ respectively and $f\in\mathcal{L}(E,F)$. Then: $$[f^*]_{\mathfrak{B}'^*,\mathfrak{B}^*}={([f]_{\mathfrak{B},\mathfrak{B}'})}^\mathrm{T}$$
\end{theorem}
\subsubsection*{Double dual space}
\begin{definition}[Double dual space]
    Let $E$ be a vector space over a field $K$. The \textit{double dual space $E^{**}$ of $E$} is defined as: $$E^{**}:=({E^*})^*=\mathcal{L}(E^*,K)$$
\end{definition}
\begin{prop}
    Let $E$ be a vector space over a field $K$ and $\mathblack{v}\in E$. We define the function:
    \begin{align*}
        \phi_{\mathblack{v}}:E^*&\longrightarrow K\\
        \omega &\longmapsto\omega(\mathblack{v})
    \end{align*}
    which is linear. This map induces an injective linear map $\Phi$ defined by:
    \begin{align*}
        \Phi:E&\longrightarrow E^{**}\\
        \mathblack{v}&\longmapsto\phi_{\mathblack{v}}
    \end{align*}
    Moreover, if $\dim E<\infty$, $\Phi$ is a natural isomorphism\footnote{This means that the definition of $\Phi$ does not depend on a choice of basis.}. 
\end{prop}
\subsubsection*{Annihilator space}
\begin{definition}
    Let $E$ be a vector space and $F\subseteq E^*$. We define the \textit{annihilator of $F$} as: 
    $$F^0=\{\mathblack{v}\in E:\omega(\mathblack{v})=0\;\forall\omega\in F\}$$
\end{definition}
\begin{lemma}
    Let $E$ be a vector space and $F\subseteq E^*$ be a vector subspace of $E^*$. If $F=\langle\omega_1,\ldots,\omega_n\rangle$, then $F^0$ is the set of solutions of the system:
    $$\left\{
    \begin{array}{c}
        \omega_1(\mathblack{v})=0\\
        \vdots\\
        \omega_n(\mathblack{v})=0
    \end{array}
    \right.$$
\end{lemma}
\begin{lemma}
    Let $E$ be a vector space and $F\subseteq E^*$ be a vector subspace of $E^*$. Then, $F^0$ is a vector subspace of $E^*$.
\end{lemma}
\begin{theorem}
    Let $E$ be a finite vector space and $F\subseteq E^*$ be a vector subspace of $E^*$. Then: $$\dim F^0+\dim F=\dim E$$
\end{theorem}
\begin{definition}
    Let $E$ be a vector space and $F\subseteq E$. We define the \textit{annihilator of $F$} as: 
    $$F^0=\{\omega\in E^*:\omega(\mathblack{v})=0\;\forall\mathblack{v}\in F\}$$
\end{definition}
\begin{lemma}
    Let $E$ be a vector space and $F\subseteq E$  be a vector subspace of $E$. If $F=\langle\mathblack{v}_1,\ldots,\mathblack{v}_n\rangle$, then: $$F^0=\{\omega\in E^*:\omega(\mathblack{v}_1)=\cdots=\omega(\mathblack{v}_n)=0\}$$
\end{lemma}
\begin{prop}
    Let $E$ be a vector space. Then, whether $F\subseteq E$ or $F\subseteq E^*$, we have: $${(F^0)}^0=F$$
\end{prop}
\subsection{Classification of endomorphisms}
\begin{definition}
    Let $E$ be a vector space over a field $K$ and $\lambda\in K$. A \textit{homothety of ratio $\lambda$} is a linear map $f:E\rightarrow E$ such that $f(\mathblack{v})=\lambda\mathblack{v}$ $\forall\mathblack{v}\in E$.
\end{definition}
\subsubsection*{Similarity}
\begin{definition}
    Let $E$ be a vector space and $f,g\in\mathcal{L}(E)$. We say that $f$ and $g$ are \textit{similar} if there are basis $\mathfrak{B}$ and $\mathfrak{B}'$ of $E$ such that $[f]_\mathfrak{B}=[g]_{\mathfrak{B}'}$.
\end{definition}
\begin{lemma}
    Let $E$ be a vector space, $\mathfrak{B}$ and $\mathfrak{B}'$ basis of $E$ and $f\in\mathcal{L}(E)$. If $\mathblack{M}=[f]_\mathfrak{B}$, $\mathblack{N}=[f]_{\mathfrak{B}'}$ and $\mathblack{P}=[\id]_{\mathfrak{B},\mathfrak{B}'}$, then: $$\mathblack{M}=\mathblack{P}^{-1}\mathblack{N}\mathblack{P}$$
\end{lemma}
\begin{definition}
    Let $K$ be a field. Two matrices $\mathblack{M},\mathblack{N}\in\mathcal{M}_n(K)$ are \textit{similar} if there exists a matrix $\mathblack{P}\in\GL_n(K)$ such that $\mathblack{M}=\mathblack{P}^{-1}\mathblack{N}\mathblack{P}$.
\end{definition}
\begin{prop}
    Let $E$ be a finite vector space and $f,g\in\mathcal{L}(E)$.:
    \begin{enumerate}
        \item $f$ and $g$ are similar if and only if for all basis $\mathfrak{B}$ of $E$ the matrices $[f]_\mathfrak{B}$ and $[g]_\mathfrak{B}$ are similar.
        \item $f$ and $g$ are similar if and only if there is an automorphism $h\in\mathcal{L}(E)$ such that $g=h^{-1}fh$.
    \end{enumerate}
\end{prop}
\subsubsection*{Diagonalization}
\begin{definition}
    Let $K$ be a field. A matrix $\mathblack{A}=(a_{ij})\in\mathcal{M}_n(K)$ is \textit{diagonal} if $a_{ij}=0$ whenever $i\ne j$. That is, $\mathblack{A}$ is of the form: 
    $$\mathblack{A}=
    \begin{pmatrix}
        a_{11} & 0 & \cdots & 0\\
        0 & a_{22} & \ddots & \vdots \\
        \vdots & \ddots & \ddots & 0\\
        0 & \cdots & 0 & a_{nn}
    \end{pmatrix}
    $$
\end{definition}
\begin{definition}
    Let $K$ be a field. A matrix $\mathblack{A}\in\mathcal{M}_n(K)$ is \textit{diagonalizable} if is similar to diagonal matrix.
\end{definition}
\begin{definition}
    An endomorphism is \textit{diagonalizable} if its associated matrix in some basis is diagonalizable.
\end{definition}
\begin{definition}
    Let $E$ be a vector space over a field $K$ and $f\in\mathcal{L}(E)$. We say that a nonzero vector $\mathblack{u}\in E$ is an \textit{eigenvector of $f$ with eigenvalue $\lambda\in K$} if $f(\mathblack{u})=\lambda \mathblack{u}$.
\end{definition}
\begin{lemma}
    Let $E$ be a vector space over a field $K$, $f\in\mathcal{L}(E)$ and $\lambda\in K$. The eigenvectors of $f$ of eigenvalue $\lambda$ are the nonzero vectors of the subspace $\ker(f-\lambda\id)$, called \textit{eigenspace corresponding to $\lambda$}.
\end{lemma}
\begin{lemma}
    Let $E$ be a vector space over a field $K$ of dimension $n$, $\mathfrak{B}$ be a basis of $E$ and $f\in\mathcal{L}(E)$. Then, $\det([f-x\id]_\mathfrak{B})$ is a polynomial on the variable $x$ of degree $n$ with dominant coefficient equal to $(-1)^n$ and constant term equal to $\det([f]_\mathfrak{B})$. 
\end{lemma}
\begin{corollary}
    Let $E$ be a vector space of dimension $n$ and $f\in\mathcal{L}(E)$. Then, $f$ has at most $n$ distinct eigenvalues.
\end{corollary}
\begin{corollary}
    Let $E$ be a vector space over $\CC$ and $f\in\mathcal{L}(E)$. Then, $f$ has at least one eigenvalue.
\end{corollary}
\begin{definition}
    Let $K$ be a field and $\mathblack{A}\in\mathcal{M}_n(K)$. The polynomial $p_{\mathblack{A}}(\lambda)=\det(\mathblack{A}-\lambda \mathblack{I}_n)$ is called \textit{characteristic polynomial of $\mathblack{A}$}.
\end{definition}
\begin{prop}
    Let $E$ be a vector space and $f\in\mathcal{L}(E)$. For all basis $\mathfrak{B}$ of $E$, the characteristic polynomial of $[f]_\mathfrak{B}$ is the same. Therefore, we denote it $p_f(\lambda)$ and we refer to it as \textit{characteristic polynomial of $f$}.
\end{prop}
\begin{prop}
    Let $E$ be a vector space and $f\in\mathcal{L}(E)$. Then, eigenvectors of $f$ of distinct eigenvalues are linearly independent.
\end{prop}
\begin{corollary}
    Let $E$ be a vector space and $f\in\mathcal{L}(E)$. Suppose $\lambda_1,\ldots,\lambda_n$ are the distinct eigenvalues of $f$ and $E_{\lambda_1},\ldots,E_{\lambda_n}$ are their corresponded eigenspaces. Then, $$E_{\lambda_1}+\cdots+E_{\lambda_n}$$ is a direct sum. 
\end{corollary}
\begin{prop}
    Let $E$ be a finite vector space of dimension $n$, $f\in\mathcal{L}(E)$ and $\lambda$ be a root of multiplicity $m$ of the characteristic polynomial $p_f(x)$. Then: $$1\leq \dim(\ker(f-\lambda\id))\leq m$$
\end{prop}
\begin{theorem}[Diagonalization theorem]
    Let $E$ be a finite vector space and $f\in\mathcal{L}(E)$. $f$ is diagonalizable if and only if:
    \begin{enumerate}
        \item $p_f(x)=(-1)^n(x-\lambda_1)^{m_1}\cdots(x-\lambda_k)^{m_k}$ with distinct $\lambda_1,\ldots,\lambda_k\in K$.
        \item $\dim(\ker(f-\lambda_i \id))=m_i$, $i=1,\ldots,k$.
    \end{enumerate}
\end{theorem}
\begin{corollary}
    Let $E$ be a finite vector space with $\dim E=n$ and $f\in\mathcal{L}(E)$. If $f$ has $n$ distinct eigenvalues, $f$ is diagonalizable.
\end{corollary}
\begin{prop}
    Let $E$ be a finite vector space and $f,g\in\mathcal{L}(E)$ such that $f$ and $g$ are similar. Then: $$f\text{ is diagonalizable}\iff g\text{ is diagonalizable}$$
\end{prop}
\begin{lemma}
    Let $K$ be a field and $\mathblack{A},\mathblack{B}\in\mathcal{M}_n(K)$ be similar matrices. Then, $\forall k\in\NN$, $\mathblack{A}^k$ and $\mathblack{B}^k$ are similar.
\end{lemma}
\begin{lemma}
    Let $E$ be a finite vector space over a field $K$ with $\dim E=n$ and $f\in\mathcal{L}(E)$. Then, the function $\phi_f:K[x]\rightarrow\mathcal{L}(E)$ defined by $$\phi_f(a_0+a_1x+\cdots+a_nx^n)=a_0+a_1f+\cdots+a_nf^n$$
    is linear and satisfies: $$\phi_f((pq)(x))=\phi_f(p(x))\phi_f(q(x))\quad\forall p(x),q(x)\in K[x]$$
\end{lemma}
\begin{definition}
    Let $E$ be a finite vector space with $\dim E=n$ and $f\in\mathcal{L}(E)$. The \textit{minimal polynomial $m_f(x)\in K[x]$ of $f$} is the unique a polynomial satisfying:
    \begin{itemize}
        \item $m_f(f)=0$.
        \item $m_f$ is monic.
        \item $m_f$ is of minimum degree.
    \end{itemize}
\end{definition}
\begin{prop}
    Let $E$ be a vector space over a field $K$ and $f\in\mathcal{L}(E)$. If $p(x)\in K[x]$ is such that $p(f)=0$, then $m_f(x)\mid p(x)$.
\end{prop}
\subsubsection*{Cayley-Hamilton theorem}
\begin{theorem}[Cayley-Hamilton theorem]
    Let $K$ be a field, $n\geq 1$ and $\mathblack{A}\in\mathcal{M}_n(K)$. Then: $$m_{\mathblack{A}}(x)\mid p_{\mathblack{A}}(x)\mid m_{\mathblack{A}}(x)^n$$ Therefore $p_{\mathblack{A}}(\mathblack{A})=0$ and $m_{\mathblack{A}}(x)$ and $p_{\mathblack{A}}(x)$ have the same irreducible factors.
\end{theorem}
\begin{corollary}
    Let $K$ be a field and $\mathblack{A}\in\GL_n(K)$ be a matrix with $p_{\mathblack{A}}(x)=a_0+a_1x+\cdots+(-1)^nx^n$. Then: $$\mathblack{A}^{-1}=-\frac{1}{a_0}\left(\mathblack{A}^{n-1}+a_{n-1}\mathblack{A}^{n-2}+\cdots+a_2\mathblack{A}+a_1\mathblack{I}_n\right)$$
\end{corollary}
\begin{lemma}
    Let $E$ be a finite vector space over a field $K$, $\mathfrak{B}$ be a basis of $f$ and $f\in\mathcal{L}(E)$. Then $\forall\lambda,\mu\in K$ and $\forall r,s\in\NN$: 
    \begin{enumerate}
        \item $[f^r]_\mathfrak{B}={\left([f]_\mathfrak{B}\right)}^r$.
        \item $[\lambda f]_\mathfrak{B}=\lambda[f]_\mathfrak{B}$.
        \item $[\lambda f^r+\mu f^s]_\mathfrak{B}=[\lambda f^r]_\mathfrak{B}+[\mu f^s]_\mathfrak{B}$.
    \end{enumerate}
\end{lemma}
\begin{lemma}
    Let $E$ be a finite vector space over a field $K$, $f\in\mathcal{L}(E)$ and $\mathblack{v}$ be an eigenvector of $f$ of eigenvalue $\lambda$. Then, $\forall p(x)\in K[x]$ we have: $$p(f)(\mathblack{v})=p(\lambda)\mathblack{v}$$
\end{lemma}
\begin{theorem}[Cayley-Hamilton theorem]
    Let $E$ be a finite vector space over a field $K$ such that $\dim E=n$ and $f\in\mathcal{L}(E)$. Then: $$m_f(x)\mid p_f(x)\mid m_f(x)^n$$ 
\end{theorem}
\begin{definition}
    A field $K$ satisfying that all polynomial with coefficient in $K$ of degree greater o equal to 1 factorizes as a product of linear factors is called an \textit{algebraically closed field}.
\end{definition}
\begin{definition}
    Let $E$ be a vector space and $f\in\mathcal{L}(E)$. We say that $F\subseteq E$ is an \textit{invariant subspace of $E$ under $f$} if $f(F)\subseteq F$.
\end{definition}
\begin{lemma}
    Let $E$ be a vector space and $f\in\mathcal{L}(E)$.
    \begin{enumerate}
        \item If $F\subseteq E$ is an invariant subspace of $E$ under $f$, then: $$p_{f|F}(x)\mid p_f(x)\footnote{Here $f|F$ is the function $f$ restricted to the subspace $F$.}$$
        \item If $F_1$ and $F_2$ are invariant subspaces of $E$ under $f$ such that $E=F_1\oplus F_2$, then:
        \begin{itemize}
            \item $p_f(x)=p_{f|F_1}(x)\cdot p_{f|F_2}(x)$.
            \item $m_f(x)=\lcm(m_{f|F_1}(x),m_{f|F_2}(x))$.
        \end{itemize}
    \end{enumerate}
\end{lemma}
\begin{lemma}
    Let $E$ be a vector space, $f\in\mathcal{L}(E)$ and $a(x),b(x)\in K[x]$. Suppose $m(x)=\lcm(a(x),b(x))$ and $d(x)=gcd(a(x),b(x))$. Then:
    \begin{enumerate}
        \item $\ker(a(f))+\ker(b(f))=\ker(m(f))$.
        \item $\ker(a(f))\cap\ker(b(f))=\ker(d(f))$.
    \end{enumerate}
    In particular, if $a(x)$ and $b(x)$ are coprime and $a(f)b(f)=0$, then: $$E=\ker(a(x))\oplus\ker(b(x))$$
\end{lemma}
\begin{theorem}
    Let $E$ be a finite vector space such that $\dim E=n$ and $f\in\mathcal{L}(E)$. If $p_f(x)={q_1(x)}^{n_1}\cdots q_r(x)^{n_r}$ and $m_f(x)={q_1(x)}^{m_1}\cdots {q_r(x)}^{m_r}$ with $q_i(x)$ distinct irreducible factors, then: $$E=\ker({q_1(f)}^{m_1})\oplus\cdots\oplus\ker({q_r(f)}^{m_r})$$ Moreover, $\dim\left(\ker({q_i(f)}^{m_i})\right)=n_i\text{deg}(q_i(x))$.
\end{theorem}
\subsubsection*{Jordan form}
\begin{definition}
    Let $K$ be a field and $\mathblack{A}\in\mathcal{M}_n(K)$. A \textit{jordan bloc of $\mathblack{A}$} is a square submatrix composed by a value $\lambda\in K$ on the principal diagonal, ones on the diagonal just below the principal diagonal and zeros elsewhere. That is, a Jordan bloc is a matrix of the form: 
    $$
    \begin{pmatrix}
        \lambda & 0 & 0 & \cdots & 0 \\
        1 & \lambda & 0 & \ddots & \vdots \\
        0 & 1 & \lambda & \ddots & 0 \\
        \vdots & \ddots &\ddots & \ddots & 0 \\
        0 & \cdots & 0 & 1 & \lambda   
    \end{pmatrix}
    $$
\end{definition}
\begin{prop}
    Let $E$ be a finite vector space over a field $K$ with $\dim E=n$ and $f\in\mathcal{L}(E)$. If $p_f(x)=\pm(x-\lambda_1)^{n_1}\cdots(x-\lambda_k)^{n_k}$, there exists a basis $\mathfrak{B}$ of $E$ such that
    $$[f]_{\mathfrak{B}}=
    \begin{pmatrix}
        \mathblack{J}_1 & \mathblack{0} & \cdots & \mathblack{0} \\
        \mathblack{0} & \mathblack{J}_2 & \ddots & \vdots \\
        \vdots & \ddots & \ddots & \mathblack{0} \\
        \mathblack{0} & \cdots & \mathblack{0} & \mathblack{J}_r \\
    \end{pmatrix}
    $$
    where $\mathblack{J}_1,\ldots,\mathblack{J}_r$ are Jordan blocs associated with eigenvalues $\lambda_1,\ldots,\lambda_k$ satisfying:
    \begin{enumerate}
        \item\label{LA_diag1} For $i=1,\ldots,k$, the sum of the sizes of Jordan blocs associated with the eigenvalue $\lambda_i$ is $n_i$.
        \item\label{LA_diag2} The sizes of Jordan blocs are determined by $\dim(\ker((f-\lambda_i\id)^r))$, $r=1,\ldots,n_i-1$.
    \end{enumerate}
\end{prop}
\begin{prop}
    Let $E$ be a finite vector space over a field $K$ with $\dim E=n$ and $\mathblack{A}\in\mathcal{M}_n(K)$. If $p_{\mathblack{A}}(x)=\pm(x-\lambda_1)^{n_1}\cdots(x-\lambda_k)^{n_k}$, there exist a matrix $\mathblack{P}\in\GL_n(K)$ such that: 
    $$\mathblack{P}^{-1}\mathblack{A}\mathblack{P}=
    \begin{pmatrix}
        \mathblack{J}_1 & \mathblack{0} & \cdots & \mathblack{0} \\
        \mathblack{0} & \mathblack{J}_2 & \ddots & \vdots \\
        \vdots & \ddots & \ddots & \mathblack{0} \\
        \mathblack{0} & \cdots & \mathblack{0} & \mathblack{J}_r \\
    \end{pmatrix}
    $$
    where $\mathblack{J}_1,\ldots,\mathblack{J}_r$ are Jordan blocs associated with eigenvalues $\lambda_1,\ldots,\lambda_k$ satisfying properties \ref{LA_diag1} and \ref{LA_diag2}.
\end{prop}
\begin{theorem}
    Let $E$ be a vector space, $f,g\in\mathcal{L}(E)$ such that $p_f(x)=(x-\lambda_1)^{n_1}\cdots(x-\lambda_k)^{n_k}$. If $g$ satisfies:
    \begin{enumerate}
        \item $p_f(x)=p_g(x)$
        \item $m_f(x)=m_g(x)$
        \item $\dim(\ker((f-\lambda \id)^r))=\dim(\ker((g-\lambda \id)^r))$ $\forall\lambda\in K$ $\forall r\geq 1$
    \end{enumerate}
    then $f$ is similar to $g$.
\end{theorem}
\subsection{Symmetric bilinear forms}
\subsubsection*{First definitions}
\begin{definition}
    Let $E$, $F$, $G$ be three vector spaces over a field $K$. We say that a function $\varphi:E\times F\rightarrow G$ is \textit{bilinear} if $\forall\mathblack{u}_1,\mathblack{u}_2,\mathblack{u}\in E$, $\forall \mathblack{v}_1,\mathblack{v}_2,\mathblack{v}\in F$ and $\forall\lambda\in K$ we have:
    \begin{enumerate}
        \item $\varphi(\mathblack{u}_1+\mathblack{u}_2,\mathblack{v})=\varphi(\mathblack{u}_1,\mathblack{v})+\varphi(\mathblack{u}_2,\mathblack{v})$.
        \item $\varphi(\lambda \mathblack{u},\mathblack{v})=\lambda \varphi(\mathblack{u},\mathblack{v})$.
        \item $\varphi(\mathblack{u},\mathblack{v}_1+\mathblack{v}_2)=\varphi(\mathblack{u},\mathblack{v}_1)+\varphi(\mathblack{u},\mathblack{v}_2)$.
        \item $\varphi(\mathblack{u},\lambda \mathblack{v})=\lambda \varphi(\mathblack{u},\mathblack{v})$.
    \end{enumerate}
\end{definition}
\begin{definition}
    Let $E$ be a vector space over a field $K$. A \textit{bilinear form from $E$ onto $K$} is a bilinear map $\varphi:E\times E\rightarrow K$.
\end{definition}
\begin{definition}
    Let $E$ be a vector space over a field $K$. A bilinear form $\varphi:E\times E\rightarrow K$ is \textit{symmetric} if $$\varphi(\mathblack{u},\mathblack{v})=\varphi(\mathblack{v},\mathblack{u})\quad\forall \mathblack{u},\mathblack{v}\in E$$
\end{definition}
\subsubsection*{Matrix associated with a bilinear form}
\begin{definition}
    Let $E$ be a finite vector space over a field $K$, $\mathfrak{B}=(\mathblack{v}_1,\ldots,\mathblack{v}_n)$ be a basis of $E$ and $\varphi:E\times E\rightarrow K$ be a symmetric bilinear form. We define the \textit{matrix of the bilinear form $\varphi$ with respect to the basis $\mathfrak{B}$} as the matrix $[\varphi]_\mathfrak{B}\in\mathcal{M}_n(K)$ defined as: $$[\varphi]_\mathfrak{B}=
    \begin{pmatrix}
        \varphi(\mathblack{v}_1,\mathblack{v}_1) & \varphi(\mathblack{v}_1,\mathblack{v}_2) & \cdots & \varphi(\mathblack{v}_1,\mathblack{v}_n) \\
        \varphi(\mathblack{v}_2,\mathblack{v}_1) & \varphi(\mathblack{v}_2,\mathblack{v}_2) & \cdots & \varphi(\mathblack{v}_2,\mathblack{v}_n) \\
        \vdots & \vdots & \ddots & \vdots \\
        \varphi(\mathblack{v}_n,\mathblack{v}_1) & \varphi(\mathblack{v}_n,\mathblack{v}_2) & \cdots & \varphi(\mathblack{v}_n,\mathblack{v}_n) \\
    \end{pmatrix}$$ 
\end{definition}
\begin{lemma}
    Let $E$ be a finite vector space over a field $K$, $\mathfrak{B}$ be a basis of $E$ and $\varphi:E\times E\rightarrow K$ be a symmetric bilinear form. Then:
    $$\varphi(\mathblack{u},\mathblack{v})={\left([\mathblack{u}]_\mathfrak{B}\right)}^\mathrm{T}[\varphi]_\mathfrak{B}[\mathblack{v}]_\mathfrak{B}\quad\forall\mathblack{u},\mathblack{v}\in E$$
\end{lemma}
\begin{prop}
    Let $E$ be a finite vector space over a field $K$, $\mathfrak{B}$ be a basis of $E$ and $\varphi:E\times E\rightarrow K$ be a symmetric bilinear form. Then: $$\varphi\text{ is symmetric}\iff[\varphi]_\mathfrak{B}\text{ is symmetric}$$
\end{prop}
\begin{prop}
    Let $E$ be a finite vector space over a field $K$, $\mathfrak{B}$ and $\mathfrak{B}'$ be bases of $E$ and $\varphi:E\times E\rightarrow K$ be a symmetric bilinear form. Then: $$[\varphi]_{\mathfrak{B}'}={([\id]_{\mathfrak{B}',\mathfrak{B}})}^\mathrm{T}[\varphi]_\mathfrak{B}[\id]_{\mathfrak{B}',\mathfrak{B}}$$
\end{prop}
\subsubsection*{Orthogonal basis}
\begin{definition}\label{ALG-isotrop}
    Let $E$ be a finite vector space over a field $K$, $\varphi:E\times E\rightarrow K$ be a symmetric bilinear form and $\mathblack{u},\mathblack{v}\in E$. 
    \begin{itemize}
        \item We say that $\mathblack{u}$ and $\mathblack{v}$ are \textit{orthogonal} if $\varphi(\mathblack{u},\mathblack{v})=0$.
        \item If $\mathblack{v}\ne 0$, we say that $\mathblack{v}$ is \textit{isotropic} if $\varphi(\mathblack{v},\mathblack{v})=0$.
    \end{itemize} 
\end{definition}
\begin{definition}
    Let $E$ be a finite vector space over a field $K$, $\mathfrak{B}=(\mathblack{v}_1,\ldots,\mathblack{v}_n)$ be a basis of $E$ and $\varphi:E\times E\rightarrow K$ be a symmetric bilinear form.
    \begin{itemize}
        \item We say that $\mathfrak{B}$ is \textit{orthogonal with respect to $\varphi$} if $\varphi(\mathblack{v}_i,\mathblack{v}_j)=0$ $\forall i\ne j$.
        \item We say that $\mathfrak{B}$ is \textit{orthonormal with respect to $\varphi$} if $\varphi(\mathblack{v}_i,\mathblack{v}_j)=\delta_{ij}$.
    \end{itemize}
\end{definition}
\begin{theorem}
    Let $E$ be a finite vector space over a field $K$, $\mathfrak{B}$ be a basis of $E$ and $\varphi:E\times E\rightarrow K$ be a symmetric bilinear form. Then, $E$ has an orthogonal basis with respect to $\varphi$ and an orthonormal basis with respect to $\varphi$.
\end{theorem}
\begin{corollary}
    Let $K$ be a field with $\ch K\ne 2$ and $\mathblack{A}\in\mathcal{M}_n(K)$ be a symmetric matrix. Then, there exists a matrix $\mathblack{P}\in\GL_n(K)$ such that $\transpose{P}\mathblack{A}\mathblack{P}$ is diagonal.
\end{corollary}
\subsubsection*{Orthogonal decompositions}
\begin{definition}\label{ALG-singular}
    Let $E$ be a finite vector space over a field $K$, $F\subseteq E$ be a vector subspace of $E$ and $\varphi:E\times E\rightarrow K$ be a symmetric bilinear form. We define the \textit{orthogonal complement of $F$} as: $$F^\perp=\{\mathblack{u}\in E:\varphi(\mathblack{u},\mathblack{v})=0\;\forall\mathblack{v}\in F\}$$
\end{definition}
\begin{definition}
    Let $E$ be a finite vector space over a field $K$ and $\varphi:E\times E\rightarrow K$ be a symmetric bilinear form. We define the \textit{radical of $\varphi$} as: $$\rad\varphi=E^\perp$$ We say that $\varphi$ is \textit{nonsingular} if $\rad\varphi=\{0\}$.
\end{definition}
\begin{definition}
    Let $E$ be a finite vector space over a field $K$, $\varphi:E\times E\rightarrow K$ be a nonsingular symmetric bilinear form and $\mathblack{u}\in E$. We define $\varphi_{\mathblack{u}}:E\rightarrow K$, $\varphi_{\mathblack{u}}(v)=\varphi(\mathblack{u},\mathblack{v})$. Then, the function 
    \begin{align*}
        E&\longrightarrow E^*\\
        \mathblack{u}&\longmapsto\varphi_{\mathblack{u}}
    \end{align*} is a isomorphism. 
\end{definition}
\begin{definition}
    Let $E$ be a finite vector space over a field $K$, $F\subseteq E$ be a vector subspace of $E$ and $\varphi:E\times E\rightarrow K$ be a nonsingular symmetric bilinear form. Then:
    \begin{enumerate}
        \item $\dim E=\dim F+\dim F^\perp$.
        \item ${(F^\perp)}^\perp=F$.
        \item If $\varphi_{|F}$ is nonsingular, then $E=F\oplus F^\perp$.
    \end{enumerate}
\end{definition}
\begin{definition}
    Let $E$ be a finite vector space over a field $K$, $F_1,F_2\subseteq E$ be vector subspaces of $E$ and $\varphi:E\times E\rightarrow K$ be a symmetric bilinear form. We say that the sum $F_1+F_2$ is \textit{orthogonal} if it's direct and $\varphi(\mathblack{u},\mathblack{v})=0$ $\forall \mathblack{u}\in F_1$ and $\mathblack{v}\in F_2$. In this case, we denote by $F_1\perp F_2=F_1+F_2$.
\end{definition}
\begin{prop}
    Let $E$ be a finite vector space over a field $K$, $F_1,F_2\subseteq E$ be vector subspaces of $E$ such that $E=F_1\perp F_2$ and $\varphi:E\times E\rightarrow K$ be a symmetric bilinear form. Then, $\forall \mathblack{v}\in E$ there exist unique $\mathblack{v}_1\in F_1$ and $\mathblack{v}_2\in F_2$ such that $\mathblack{v}=\mathblack{v}_1+\mathblack{v}_2$.
\end{prop}
\begin{definition}\label{perpendicular}
    Let $E$ be a finite vector space over a field $K$, $F_1,F_2\subseteq E$ be vector subspaces of $E$ such that $E=F_1\perp F_2$ and $\varphi:E\times E\rightarrow K$ be a symmetric bilinear form. The function
    \begin{align*}
        \pi:E=F_1\perp F_2&\longrightarrow F_i\\
        \mathblack{v}=\mathblack{v}_1+\mathblack{v}_2&\longmapsto \mathblack{v}_i
    \end{align*}
    for $i=1,2$ is called \textit{orthogonal projection of $E$ onto $F_i$ according to the decomposition $E=F_1\perp F_2$}.
\end{definition}
\begin{method}[Gram-Schmidt process]
    Let $E$ be a finite vector space over a field $K$, $\mathfrak{B}=(\mathblack{v}_1,\ldots,\mathblack{v}_n)$ be a basis of $E$ and $\varphi:E\times E\rightarrow K$ be a symmetric bilinear form. $\forall\mathblack{u},\mathblack{v}\in E$, we define $$\text{proj}_{\mathblack{u}}(\mathblack{v})=\frac{\varphi(\mathblack{u},\mathblack{v})}{\varphi(\mathblack{u},\mathblack{u})}\mathblack{u}$$ We will create an orthogonal basis $(\mathblack{u}_1,\ldots,\mathblack{u}_n)$ of $E$ from $\mathfrak{B}$. We define $\mathblack{u}_i$, $i=1,\ldots,n$ to be:
    \begin{align*}
        \mathblack{u}_1&=\mathblack{v}_1\\
        \mathblack{u}_2&=\mathblack{v}_2-\text{proj}_{\mathblack{u}_1}(\mathblack{v}_2)\\
        \mathblack{u}_3&=\mathblack{v}_3-\text{proj}_{\mathblack{u}_1}(\mathblack{v}_3)-\text{proj}_{\mathblack{u}_2}(\mathblack{v}_3)\\
        &\;\;\vdots\\
        \mathblack{u}_n&=\mathblack{v}_n-\sum_{i=1}^{n-1}\text{proj}_{\mathblack{u}_i}(\mathblack{v}_n)
    \end{align*}
    To obtain an orthogonal basis $(\mathblack{e}_1,\ldots,\mathblack{e}_n)$ of $E$ from $\mathfrak{B}$, define $\mathblack{e}_i$, $i=1,\ldots,n$ to be: $$\mathblack{e}_i=\frac{\mathblack{u}_i}{\sqrt{\varphi(\mathblack{u}_i,\mathblack{u}_i)}}$$
\end{method}
\subsubsection*{Sylvester's law of inertia}
\begin{definition}
    An \textit{orthogonal geometry over a field $K$} is a pair $(E,\varphi)$, where $E$ is a vector space over $K$ and $\varphi$ is a symmetric bilinear form over $E$.
\end{definition}
\begin{definition}\label{isometry}
    Let $(E_1,\varphi_1)$, $(E_2,\varphi_2)$ be two orthogonal geometries over a field $K$. An \textit{isometry from $(E_1,\varphi_1)$ to $(E_2,\varphi_2)$} is an isomorphism $f:E_1\rightarrow E_2$ such that $$\varphi_2(f(\mathblack{u}),f(\mathblack{v}))=\varphi_1(\mathblack{u},\mathblack{v})\quad\forall\mathblack{u},\mathblack{v}\in E_1$$ We say that $(E_1,\varphi_1)$ and $(E_2,\varphi_2)$ are \textit{isometric} if there exists an isometry between them.
\end{definition}
\begin{definition}
    Let $E$ be a vector space over a field $K$ and $\varphi_1$, $\varphi_2$ be symmetric bilinear forms. We say that $\varphi_1$ and $\varphi_2$ are \textit{equivalent} if and only if $(E,\varphi_1)$ and $(E,\varphi_2)$ are isometric.
\end{definition}
\begin{definition}
    Let $\mathblack{A},\mathblack{B}\in\mathcal{M}_n(\RR)$. We say that $\mathblack{A}$ and $\mathfrak{B}$ are congruent if there exists a matrix $\mathblack{P}\in\GL_n(\RR)$ such that $$\mathblack{A}=\transpose{P}\mathblack{B}\mathblack{P}$$
\end{definition}
\begin{prop}
    Let $E$ be a finite vector space over a field $K$, $\mathfrak{B}_1$ be a basis of $E$ and $\varphi_1$, $\varphi_2$ be symmetric bilinear forms. Then the following statements are equivalent:
    \begin{enumerate}
        \item The orthogonal geometries $(E,\varphi_1)$ and $(E,\varphi_2)$ are isometric.
        \item There exists a basis $\mathfrak{B}_2$ of $E$ such that $[\varphi_1]_{\mathfrak{B}_1}=[\varphi_2]_{\mathfrak{B}_2}$.
        \item The matrices $[\varphi_1]_{\mathfrak{B}_1}$ and $[\varphi_2]_{\mathfrak{B}_2}$ are congruent.
    \end{enumerate}
\end{prop}
\begin{theorem}[Sylvester's law of inertia]
    Let $E$ be a finite vector space over $\RR$ and $\varphi$ be a symmetric bilinear form over $E$. Then, there exists a basis $\mathfrak{B}$ of $E$ such that $$[\varphi]_\mathfrak{B}=
    \begin{pmatrix}
        0 &&&&&&&&&\\
        & \ddots&&&&&&&\\
        && 0&&&&\bigzero&&\\
        &&& 1 &&&&&\\
        &&&& \ddots &&&&\\
        &&&&& 1 &&&\\
        &&\bigzero&&&& -1 &&\\
        &&&&&&&\ddots &\\
        &&&&&&&& -1
    \end{pmatrix}$$
    where in the diagonal there are $r_0$ zeros, $r_+$ ones and $r_-$ minus ones and the triplet $(r_0,r_+,r_-)$ doesn't depend on the basis $\mathfrak{B}$.
\end{theorem}
\begin{definition}
    Let $E$ be a finite vector space over $\RR$ and $\varphi$ be a symmetric bilinear form over $E$. Let $\mathfrak{B}$ be an orthogonal basis of $E$ with respect to $\varphi$. We define the \textit{rank of $\varphi$} as: $$\rank \varphi=\rank ([\varphi]_\mathfrak{B})$$ We define the \textit{signature of $\varphi$} as: $$\sig\varphi=(r_+,r_-)$$ where $r_+$ is el number of positive reals numbers on the diagonal of $[\varphi]_\mathfrak{B}$ and $r_-$ is el number of negative reals numbers on the diagonal of $[\varphi]_\mathfrak{B}$.
\end{definition}
\begin{theorem}
    Let $(E_1,\varphi_1)$, $(E_2,\varphi_2)$ be two orthogonal geometries over $\RR$ of finite dimension. Then, $(E_1,\varphi_1)$ and $(E_2,\varphi_2)$ are isometric if and only if $\dim E_1=\dim E_2$ and $\sig \varphi_1=\sig \varphi_2$.
\end{theorem}
\subsubsection*{Inner products}
\begin{definition}
    Let $E$ be a finite vector space over $\RR$ and $\varphi$ be a symmetric bilinear form over $E$. We say that $\varphi$ is \textit{positive-definite} if $$\varphi(\mathblack{v},\mathblack{v})>0\quad\forall\mathblack{v}\in E\setminus\{0\}$$ We say that $\varphi$ is \textit{negative-definite} if $$\varphi(\mathblack{v},\mathblack{v})<0\quad\forall\mathblack{v}\in E\setminus\{0\}\footnote{The terms \textit{positive-semidefinite} and \textit{negative-semidefinite} are used when $\forall\mathblack{v}\in E\setminus\{0\}$, $\varphi(\mathblack{v},\mathblack{v})\geq 0$ or $\varphi(\mathblack{v},\mathblack{v})\leq 0$, respectively.}$$
\end{definition}
\begin{definition}
    Let $E$ be a vector space over $\RR$. An \textit{inner product over $E$} is a positive-definite symmetric bilinear form over $E$.
\end{definition}
\begin{definition}\label{espai_euclidia}
    An \textit{Euclidean vector space} is a pair $(E,\varphi)$, where $E$ is a vector space over $\RR$ and $\varphi$ is an inner product over $E$.
\end{definition}
\begin{theorem}[Cauchy-Schwartz inequality]
    Let $(E,\varphi)$ be an Euclidean vector space. Then: $$\varphi(\mathblack{u},\mathblack{v})^2\leq \varphi(\mathblack{u},\mathblack{u})\varphi(\mathblack{v},\mathblack{v})\quad\forall \mathblack{u},\mathblack{v}\in E$$
\end{theorem}
\begin{definition}
    Let $E$ be a vector space over $\RR$. A \textit{norm on $E$} is a function 
    \begin{align*}
        \|\cdot\|:E&\longrightarrow\RR\\
        \mathblack{u}&\longmapsto\|\mathblack{u}\|
    \end{align*}
    such that:
    \begin{enumerate}
        \item $\|\mathblack{u}\|=0\iff \mathblack{u}=\mathblack{0}$.
        \item $\|\lambda \mathblack{u}\|=|\lambda|\|\mathblack{u}\|$, $\forall \mathblack{u}\in E$, $\lambda\in\RR$.
        \item $\|\mathblack{u}+\mathblack{v}\|\leq\|\mathblack{u}\|+\|\mathblack{v}\|$, $\forall \mathblack{u},\mathblack{v}\in E$\footnote{Note that $\forall\mathblack{u}\in E$ we have: $0=\|\mathblack{u}+(-\mathblack{u})\|\leq\|\mathblack{u}\|+\|-\mathblack{u}\|=2\|\mathblack{u}\|\implies\|\mathblack{u}\|\geq 0$.}.
    \end{enumerate}
\end{definition}
\begin{prop}
    Let $(E,\varphi)$ be an Euclidean vector space. Then, the function
    \begin{align*}
        \|\cdot\|_\varphi:E&\longrightarrow\RR\\
        u&\longmapsto\|u\|_\varphi=\sqrt{\varphi(u,u)}
    \end{align*}
    is a norm called \textit{norm associated with the inner product $\varphi$}.
\end{prop}
\begin{definition}
    Let $(E,\varphi)$ be an Euclidean vector space and $\mathblack{u},\mathblack{v}\in E\setminus\{0\}$. We define the angle with \textit{respect to $\varphi$ between $\mathblack{u}$ and $\mathblack{v}$} as the unique $\theta\in[0,\pi]$ such that: $$\cos{\theta}=\frac{\varphi(\mathblack{u},\mathblack{v})}{\|\mathblack{u}\|_\varphi\|\mathblack{v}\|_\varphi}$$
\end{definition}
\subsubsection*{Spectral theorem}
\begin{definition}
    Let $(E,\varphi)$ be a finite Euclidean vector space and $f\in\mathcal{L}(E)$. Then, there exists a unique $f'\in\mathcal{L}(E)$ such that $$\varphi(f(\mathblack{u}),\mathblack{v})=\varphi(\mathblack{u},f'(\mathblack{v}))\quad\forall \mathblack{u},\mathblack{v}\in E$$ This $f'$ is called \textit{adjoint of $f$}. 
\end{definition}
\begin{definition}
    Let $(E,\varphi)$ be a finite Euclidean vector space and $f\in\mathcal{L}(E)$. $f$ is called \textit{auto-adjoint} if $f=f'$.
\end{definition}
\begin{lemma}
    Let $(E,\varphi)$ be a finite Euclidean vector space of dimension $n$ and $f\in\mathcal{L}(E)$ be auto-adjoint. Then, there exist $\lambda_1,\ldots,\lambda_n\in\RR$ such that $$p_f(x)=(x-\lambda_1 )\cdots(x-\lambda_n)$$
\end{lemma}
\begin{definition}
    Let $K$ be a field and $A\in\GL_n(K)$ be a matrix. We say that $A$ is \textit{orthogonal} if and only if $$\mathblack{P}\transpose{P}=\transpose{P}\mathblack{P}=\mathblack{I}_n$$ The set of orthogonal matrices of size $n$ over $K$ is denoted by $\mathcal{O}_n(K)$.
\end{definition}
\begin{theorem}[Spectral theorem]
    Let $(E,\varphi)$ be a a finite Euclidean vector space and $f\in\mathcal{L}(E)$ be auto-adjoint. Then, $E$ has an orthonormal basis of eigenvectors of $f$. In particular, $f$ diagonalizes.
\end{theorem}
\begin{corollary}
    Let $K$ be a field. All symmetric matrices $A\in\mathcal{M}_n(K)$ are diagonalizable. More precisely, there exists $\mathblack{P}\in\mathcal{O}_n(K)$ such that $\transpose{P}\mathblack{A}\mathblack{P}$ is diagonal.
\end{corollary}
\begin{definition}
    Let $A=(a_{ij})\in\mathcal{M}_{m\times n}(\CC)$. We define the \textit{complex conjugate $\overline{\mathblack{A}}$ of $\mathblack{A}$} as $\overline{A}=(\overline{a_{ij}})$.
\end{definition}
\begin{prop}
    Let $\mathblack{A},\mathblack{B}\in\mathcal{M}_{m\times n}(\CC)$, $\mathblack{C}\in\mathcal{M}_{n\times p}(\CC)$ and $\lambda\in\CC$. Then:
    \begin{enumerate}
        \item $\overline{\mathblack{A}+\mathblack{B}}=\overline{\mathblack{A}}+\overline{\mathblack{B}}$.
        \item $\overline{\mathblack{A}\mathblack{C}}=\overline{\mathblack{A}}\cdot\overline{\mathblack{C}}$.
        \item $\overline{\lambda\cdot\mathblack{A}}=\overline{\lambda}\cdot\overline{\mathblack{A}}$.
    \end{enumerate}
\end{prop}
\begin{corollary}
    Let $\mathblack{A}\in\mathcal{M}_n(\RR)$ be a symmetric matrix. Then, there exist $\lambda_1,\ldots,\lambda_n\in\RR$ such that $$p_{\mathblack{A}}(x)=(x-\lambda_1 )\cdots(x-\lambda_n)$$
\end{corollary}
\begin{theorem}[Descartes' rule of signs]
    Let $P(x)=a_0+\cdots+a_nx^n\in\RR[x]$:
    \begin{enumerate}
        \item The number of positive roots of $P(x)$ is at most equal to the number of sign variations in the sequence $[a_d,a_{d-1},\ldots,a_1,a_0]$.
        \item If $P(x)=a_n(x-\alpha_1)^{n_1}\cdots(x-\alpha_r)^{n_r}$, then the number of positive roots of $P(x)$ is equal to the number of sign variations in the sequence (having in account multiplicity).
\end{enumerate}
\end{theorem}
\end{multicols}
\end{document}