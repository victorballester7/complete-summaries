\documentclass[../../../main_math.tex]{subfiles}

\begin{document}
\changecolor{INLP}
\begin{multicols}{2}[\section{Instabilities and nonlinear phenomena}]
  \subsection{Review of bifurcations}
  \subsubsection{Hopf bifurcation}
  \begin{definition}
    The normal form of a Hopf bifurcation is:
    \begin{equation*}
      \begin{cases}
        \dot{x}=\mu x-\omega y-x(x^2+y^2) \\
        \dot{y}=\omega x+\mu y-y(x^2+y^2)
      \end{cases}
    \end{equation*}
    Or in polar coordinates:
    \begin{equation*}
      \begin{cases}
        \dot{r}=\mu r-r^3 \\
        \dot{\theta}=-\omega
      \end{cases}
    \end{equation*}
  \end{definition}
  \subsubsection{Homoclinic bifurcation}
  \begin{definition}
    The normal form of a homoclinic bifurcation is:
    \begin{equation*}
      \begin{cases}
        \dot{x}=y \\
        \dot{y}=-\mu- x+x^2-xy
      \end{cases}
    \end{equation*}
  \end{definition}
\end{multicols}
\end{document}