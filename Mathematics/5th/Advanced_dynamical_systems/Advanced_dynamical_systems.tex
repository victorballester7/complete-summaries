\documentclass[../../../main_math.tex]{subfiles}

\begin{document}
\changecolor{ADS}
\begin{multicols}{2}[\section{Advanced dynamical sytems}]
  \subsection{Introduction}
  \subsubsection{Rotations in \texorpdfstring{$\S^1$}{S1}}
  \begin{proposition}
    Let $\alpha=\frac{p}{q}\in\QQ$ and let $R_\alpha:\S^1\to \S^1$ be the rotation of angle $\alpha$. Then, all the points of $\S^1$ are periodic for $R_\alpha$ with period $q$.
  \end{proposition}
  \begin{proof}
    We identify the elements of $\S^1$ as $\quot{\RR}{\ZZ}$. Let $x\in \S^1$. Then, ${R_\alpha}^q x=x+\alpha q=x+p=x$. And $q$ is the smallest integer such that ${R_\alpha}^q x=x$ because we assume that $p$ and $q$ are coprime.
  \end{proof}
  \begin{proposition}
    Let $\alpha\in\RR\setminus\QQ$ and let $R_\alpha:\S^1\to \S^1$ be the rotation of angle $\alpha$. Then, all the points of $\S^1$ are dense in $\S^1$.
  \end{proposition}
  \begin{proof}
    Let $\varepsilon>0$, $x,y\in \S^1$. Discretize $\S^1$ in intervals of length at most $\frac{1}{\varepsilon}$. Then, $\exists m,n\in \NN$ with $m< n\leq \frac{1}{\varepsilon}+1$ such that ${R_\alpha}^m x$ and ${R_\alpha}^nx$ are in the same interval. Thus, $\abs{{R_\alpha}^{n-m}x-x}<\varepsilon$. Now, concatenating ${R_\alpha}^{n-m}x$ repeatedly, we will eventually have $\abs{{R_\alpha}^{k(n-m)}x - y}<\varepsilon$ for some $k\in \NN$.
  \end{proof}
\end{multicols}
\end{document}