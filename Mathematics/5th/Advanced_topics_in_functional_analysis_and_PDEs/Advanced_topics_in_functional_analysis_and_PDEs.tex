\documentclass[../../../main_math.tex]{subfiles}

\begin{document}
\changecolor{ATFAPDE}
\begin{multicols}{2}[\section{Advanced topics in functional analysis and PDEs}]
  \subsection{\texorpdfstring{$L^p$}{Lp} spaces}
  \subsubsection{Topologies of \texorpdfstring{$L^p$}{Lp} spaces}
  \begin{definition}
    Let $E$ be a Banach space and ${(x_n)}_{n\in\NN}\in E$. We say that ${(x_n)}_{n\in\NN}$ \emph{converges weakly} to $x\in E$ if $\forall f\in E^*$ we have:
    $$
      \lim_{n\to\infty}{f(x_n)}=f(x)
    $$
    We denote this by $x_n\rightharpoonup x$.
  \end{definition}
  \begin{definition}
    Let $E$ be a Banach space and ${(x_n)}_{n\in\NN}\in E$. We say that ${(x_n)}_{n\in\NN}$ \emph{converges strongly} to $x\in E$ if $\forall f\in E^*$ we have:
    $$
      \lim_{n\to\infty}{\norm{x_n-x}}=0
    $$
    We denote this by $x_n\to x$.
  \end{definition}
  \begin{definition}
    Let $E$ be a Banach space and ${(x_n)}_{n\in\NN}\in E$. Assume $E=F^*$, where $F$ is a Banach space. Then, we say that ${(x_n)}_{n\in\NN}$ \emph{converges weakly-*} to $x\in E$ if $\forall f\in F$ we have:
    $$
      \lim_{n\to\infty}{x_n(f)}=x(f)
    $$
    We denote this by $x_n\overset{*}\rightharpoonup x$.
  \end{definition}
  \begin{theorem}
    Let $E$ be a Banach space and ${(x_n)}_{n\in\NN}\in E$. Then:
    \begin{enumerate}
      \item If $x_n\to x$, then $x_n\rightharpoonup x$.
      \item If $E$ is reflexive then weak convergence is equivalent to weak-* convergence.
      \item If $x_n\to x$, $x_n\rightharpoonup x$ or $x_n\overset{*}\rightharpoonup x$, then $(x_n)$ is bounded.
      \item Let ${(L_n)}_{n\in\NN}\in E^*$. If $x_n\to x$ in $E$ and $L_n\overset{*}\rightharpoonup L$ in $E^*$, then $L_n(x_n)\to L(x)$ in $\CC$.
    \end{enumerate}
  \end{theorem}
  \begin{theorem}[Banach-Alaoglu theorem]
    Let $\Omega\subseteq \RR^d$ be a set and $1<p<\infty$. If $(f_n)$ is a bounded sequence in $L^p(\Omega)$, then there is a subsequence $(f_{n_k})$ and $f\in L^p(\Omega)$ so that $f_{n_k}\rightharpoonup f$ in $L^p(\Omega)$. If $p=\infty$, then there is a subsequence $(f_{n_k})$ and $f\in L^\infty(\Omega)$ so that $f_{n_k}\overset{*}\rightharpoonup f$ in $L^\infty(\Omega)$.
  \end{theorem}
  \subsubsection{Lower-semicontinuous functions and convexity}
  \begin{definition}
    Let $E$ be a Banach space, $(x_n)\in E$ and $f:E\to\RR$. We say that $f$ is \emph{strongly lower-semicontinuous} if:
    $$
      x_n\to x\implies f(x)\leq \liminf_{n\to\infty}{f(x_n)}
    $$
  \end{definition}
  \begin{remark}
    Analogously, we can define \emph{weakly lower-semicontinuity} and \emph{weak-* lower-semicontinuity} by replacing $x_n\to x$ by $x_n\rightharpoonup x$ and $x_n\overset{*}\rightharpoonup x$ respectively.
  \end{remark}
  \begin{theorem}
    Let $E$ be a Banach space and $f:E\to\RR$ be convex. Then, $f$ is strongly lower-semicontinuous if and only if $f$ is weakly lower-semicontinuous. In particular, the map ${\norm{\cdot}}_E$ is weakly lower-semicontinuous.
  \end{theorem}
  \begin{theorem}
    Let $\Omega\subseteq \RR^d$ be a set and $1<p<\infty$. Then if $f_n\rightharpoonup f$ in $L^p(\Omega)$, then: $$\norm{f}_p\leq \liminf_{n\to\infty}{\norm{f_n}_p}$$ In addition, if $\displaystyle\norm{f}_p=\lim_{n\to\infty}{\norm{f_n}_p}$, then $f_n\to f$ in $L^p(\Omega)$.
  \end{theorem}
  \subsection{Sobolev spaces}
  \begin{definition}[Sobolev spaces]
    Let $\Omega\subseteq \RR^d$ be an open set, $m\in\NN$ and $1\leq p\leq \infty$. We define the \emph{Sobolev spaces} $W^{m,p}$ as:
    $$
      W^{m,p}(\Omega)\!:=\!\{f\in L^p(\Omega): \forall\alpha\in\NN^d, \abs{\alpha}\leq m, \partial^\alpha f\in L^p(\Omega)\}
    $$
    Moreover we define the associate norm $\norm{\cdot}_{W^{m,p}(\Omega)}$ as:
    $$
      \norm{f}_{W^{m,p}(\Omega)}:={\left(\sum_{\abs{\alpha}\leq m}{\norm{\partial^\alpha f}_p}^p\right)}^{1/p}
    $$
    If $p=2$, we denote $H^m(\Omega):=W^{m,2}(\Omega)$.
  \end{definition}
  \begin{theorem}
    Let $\Omega\subseteq \RR^d$ be an open set. Then, for all $m\in\NN$ and all $1\leq p\leq \infty$, $(W^{m,p}(\Omega),\norm{\cdot}_{W^{m,p}(\Omega)}$ is Banach. Moreover, if $p<\infty$, it is separable and if $1<p<\infty$, it is reflexive. Finally, $H^m(\Omega)$ is a separable Hilbert space.
  \end{theorem}
  \begin{definition}
    Let $\Omega\subseteq \RR^d$ be an open set, $m\in\NN$ and $1\leq p\leq \infty$. We define the space $W_0^{m,p}(\Omega):=\overline{\mathcal{C}_0^\infty(\Omega)}$, where the closure is taken with the norm of $W^{m,p}(\Omega)$. Similarly, we set $H_0^m(\Omega):=W_0^{m,2}(\Omega)$.
  \end{definition}
  \begin{remark}
    Note that $W_0^{m,p}(\Omega)$ is also Banach (with the same norm as $W^{m,p}(\Omega)$) because it is a closed subspace of a Banach space.
  \end{remark}
  \begin{theorem}
    For $1\leq p < \infty$ we have that $W_0^{m,p}(\RR^d)=W^{m,p}(\RR^d)$. In particular, $\mathcal{C}^\infty(\RR^d)\cap W^{m,p}(\RR^d)$ and $\mathcal{C}_0^{\infty}(\RR^d)$ are dense in $W^{m,p}(\RR^d)$ for $1\leq p<\infty$.
  \end{theorem}
  \begin{theorem}[Poincaré's inequality]\label{ATFAPDE:poincare_ineq}
    Let $\Omega$ be a bounded open set in $\RR^d$ and let $1\leq p<\infty$. Then, there is a constant $C=C(\Omega,p)$ so that $\forall u\in W_0^{m,p}(\Omega)$ we have:
    $$
      \norm{u}_p\leq C\norm{\grad u}_p
    $$
  \end{theorem}
  \begin{remark}
    \mnameref{ATFAPDE:poincare_ineq} is also valid when $\Omega$ is unbounded in one direction.
  \end{remark}
  \begin{corollary}
    If $\Omega$ is bounded, then the constant function $f(x)=C$ with $C\ne 0$ is not in $W_0^{1,p}$. Thus, we cannot approximate constant functions by $\mathcal{C}_0^\infty(\Omega)$ functions, with the $W^{1,p}(\Omega)$ norm.
  \end{corollary}
  \begin{definition}
    Let $\Omega$ be a bounded set. We define the average of $u$ in $\Omega$ as:
    $$
      \fint_\Omega u:=\frac{1}{\abs{\Omega}}\int_\Omega u
    $$
  \end{definition}
  \begin{theorem}[Poincaré-Wirtinger's inequality]
    Let $\Omega\subseteq \RR^d$ be a bounded connected open set with $\mathcal{C}^1$ boundary, and let $1\leq p<\infty$. Then, there is a constant $C=C(\Omega,p)$ so that $\forall u\in W^{1,p}(\Omega)$ we have:
    $$
      \norm{u-\fint_\Omega u}_p\leq C\norm{\grad u}_p
    $$
  \end{theorem}
\end{multicols}
\end{document}