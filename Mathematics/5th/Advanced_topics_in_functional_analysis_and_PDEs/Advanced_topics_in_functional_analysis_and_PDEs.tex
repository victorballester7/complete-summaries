\documentclass[../../../main_math.tex]{subfiles}

\begin{document}
\changecolor{ATFAPDE}
\begin{multicols}{2}[\section{Advanced topics in functional analysis and PDEs}]
  \subsection{\texorpdfstring{$L^p$}{Lp} spaces}
  \subsubsection{Topologies of \texorpdfstring{$L^p$}{Lp} spaces}
  \begin{definition}
    Let $E$ be a Banach space and ${(x_n)}_{n\in\NN}\in E$. We say that ${(x_n)}_{n\in\NN}$ \emph{converges weakly} to $x\in E$ if $\forall f\in E^*$ we have:
    $$
      \lim_{n\to\infty}{f(x_n)}=f(x)
    $$
    We denote this by $x_n\rightharpoonup x$.
  \end{definition}
  \begin{definition}
    Let $E$ be a Banach space and ${(x_n)}_{n\in\NN}\in E$. We say that ${(x_n)}_{n\in\NN}$ \emph{converges strongly} to $x\in E$ if we have:
    $$
      \lim_{n\to\infty}{\norm{x_n-x}}=0
    $$
    We denote this by $x_n\to x$.
  \end{definition}
  \begin{definition}
    Let $E$ be a Banach space and ${(L_n)}_{n\in\NN}\in E$. Assume $E=F^*$, where $F$ is a Banach space. Then, we say that ${(L_n)}_{n\in\NN}$ \emph{converges weakly-*} to $L\in E$ if $\forall x\in F$ we have:
    $$
      \lim_{n\to\infty}{L_n(x)}=L(x)
    $$
    We denote this by $L_n\overset{*}\rightharpoonup L$.
  \end{definition}
  \begin{theorem}
    Let $E$ be a Banach space and ${(x_n)}_{n\in\NN}\in E$. Then:
    \begin{enumerate}
      \item If $x_n\to x$, then $x_n\rightharpoonup x$.
      \item If $E$ is reflexive then weak convergence is equivalent to weak-* convergence.
      \item If $x_n\to x$, $x_n\rightharpoonup x$ or $x_n\overset{*}\rightharpoonup x$, then $(x_n)$ is bounded.
      \item Let ${(L_n)}_{n\in\NN}\in E^*$. If $x_n\to x$ in $E$ and $L_n\overset{*}\rightharpoonup L$ in $E^*$, then $L_n(x_n)\to L(x)$ in $\CC$.
    \end{enumerate}
  \end{theorem}
  \begin{proof}
    We only prove the first and last points.
    \begin{enumerate}
      \item Let $L\in E^*$. Then, $\norm{L}< \infty$ and:
            $$
              \abs{L(x_n)-L(x)}\leq \norm{L}\norm{x_n-x}\to 0
            $$
            \setcounter{enumi}{3}
      \item Let $(L_n)\in E^*$ converge to $L\in E^*$ weakly-* and let $x_n\to x$ in $E$. Then:
            \begin{align*}
              \abs{L_n(x_n)-L(x)} & \leq \abs{(L_n-L)(x_n)}+\abs{L(x_n)-L(x)}         \\
                                  & \leq \norm{L_n-L}\norm{x_n}+\norm{L}\norm{x_n -x}
            \end{align*}
            And the result follows.
    \end{enumerate}
  \end{proof}
  \begin{theorem}[Banach-Alaoglu theorem]
    Let $\Omega\subseteq \RR^d$ be a set and $1<p<\infty$. If $(f_n)$ is a bounded sequence in $L^p(\Omega)$, then there is a subsequence $(f_{n_k})$ and $f\in L^p(\Omega)$ so that $f_{n_k}\rightharpoonup f$ in $L^p(\Omega)$. If $p=\infty$, then there is a subsequence $(f_{n_k})$ and $f\in L^\infty(\Omega)$ so that $f_{n_k}\overset{*}\rightharpoonup f$ in $L^\infty(\Omega)$.
  \end{theorem}
  \subsubsection{Lower-semicontinuous functions and convexity}
  \begin{definition}
    Let $E$ be a Banach space, $(x_n)\in E$ and $f:E\to\RR$. We say that $f$ is \emph{strongly lower-semicontinuous} if:
    $$
      x_n\to x\implies f(x)\leq \liminf_{n\to\infty}{f(x_n)}
    $$
  \end{definition}
  \begin{remark}
    Analogously, we can define \emph{weakly lower-semicontinuity} and \emph{weak-* lower-semicontinuity} by replacing $x_n\to x$ by $x_n\rightharpoonup x$ and $x_n\overset{*}\rightharpoonup x$ respectively.
  \end{remark}
  \begin{theorem}\label{ATFAPDE:lower_semicontinuity_thm}
    Let $E$ be a Banach space and $f:E\to\RR$ be convex. Then, $f$ is strongly lower-semicontinuous if and only if $f$ is weakly lower-semicontinuous. In particular, the map ${\norm{\cdot}}_E$ is weakly lower-semicontinuous.
  \end{theorem}
  \begin{proof}
    We only prove one implication, and also we admit that if $f$ is convex and strongly lower-semicontinuous, then there is a continuous linear operator $L_x$ (called \emph{support plane}) so that $\forall y\in E$ we have:
    $$
      f(y)\geq f(x)+L_x(y-x)
    $$
    In particular, if $(x_n)\rightharpoonup x$, then:
    $$
      f(x_n)\geq f(x)+L_x(x_n-x)\implies \liminf_{n\to\infty}{f(x_n)}\geq f(x)
    $$
  \end{proof}
  \begin{theorem}
    Let $\Omega\subseteq \RR^d$ be a set and $1<p<\infty$. If $f_n\rightharpoonup f$ in $L^p(\Omega)$, then: $$\norm{f}_p\leq \liminf_{n\to\infty}{\norm{f_n}_p}$$ In addition, if $\displaystyle\norm{f}_p=\lim_{n\to\infty}{\norm{f_n}_p}$, then $f_n\to f$ in $L^p(\Omega)$.
  \end{theorem}
  \begin{proof}
    The first point is \mcref{ATFAPDE:lower_semicontinuity_thm} in the case $E = L^p(\Omega)$. We only prove the second point for $p=2$. If $\displaystyle\norm{f}_2=\lim_{n\to\infty}{\norm{f_n}_2}$, then:
    \begin{align*}
      {\norm{f_n-f}_2}^2 & =\norm{f_n}_2^2+\norm{f}_2^2-2\Re\int_\Omega f_n\overline{f}    \\
                         & \to {\norm{f}_2}^2+{\norm{f}_2}^2-2\Re\int_\Omega f\overline{f} \\
                         & =0
    \end{align*}
    where the convergence of the integral is due to the weakly convergence of $f_n$ to $f$.
  \end{proof}
  \subsection{Sobolev spaces}
  \subsubsection{Basic definitions}
  \begin{definition}[Sobolev spaces]
    Let $\Omega\subseteq \RR^d$ be an open set, $m\in\NN$ and $1\leq p\leq \infty$. We define the \emph{Sobolev spaces} $W^{m,p}$ as:
    $$
      W^{m,p}(\Omega)\!:=\!\{f\in L^p(\Omega): \forall\alpha\in\NN^d, \abs{\alpha}\leq m, \partial^\alpha f\in L^p(\Omega)\}
    $$
    Moreover we define the associate norm $\norm{\cdot}_{W^{m,p}(\Omega)}$ as:
    $$
      \norm{f}_{W^{m,p}(\Omega)}:={\left(\sum_{\abs{\alpha}\leq m}{\norm{\partial^\alpha f}_p}^p\right)}^{1/p}
    $$
    If $p=2$, we denote $H^m(\Omega):=W^{m,2}(\Omega)$.
  \end{definition}
  \begin{theorem}
    Let $\Omega\subseteq \RR^d$ be an open set. Then, for all $m\in\NN$ and all $1\leq p\leq \infty$, $(W^{m,p}(\Omega),\norm{\cdot}_{W^{m,p}(\Omega)})$ is Banach. Moreover, if $p<\infty$, it is separable and if $1<p<\infty$, it is reflexive. Finally, $H^m(\Omega)$ is a separable Hilbert space.
  \end{theorem}
  \begin{proof}
    Let $(f_j)$ be a Cauchy sequence in $W^{m,p}(\Omega)$. Then, $(f_j)$ is Cauchy in $L^p(\Omega)$, and for all $\abs{\alpha}\leq m$, $(\partial^\alpha f_j)$ is Cauchy in $L^p(\Omega)$. Since $L^p(\Omega)$ is complete, there are $f\in L^p(\Omega)$ and $f_\alpha\in L^p(\Omega)$, so that:
    $$
      f_j\overset{L^p}\longrightarrow f\qquad\partial^\alpha f_j\overset{L^p}\longrightarrow f_\alpha
    $$
    It remains to prove that $f_\alpha=\partial^\alpha f$. Since we have convergence in $L^p$, we also have convergence in the distributional sense, that is $f_j\overset{\mathcal{D}^*}\to f$. In particular, we must have $\partial^\alpha f_j\overset{\mathcal{D}^*}\to \partial^\alpha f$. By uniqueness of the limit in $\mathcal{D}^*(\Omega)$, we indeed have $f_\alpha=\partial^\alpha f$. This proves that $W^{m,p}(\Omega)$ is complete. Now, the map
    $$
      \function{}{W^{m,p}(\Omega)}{(L^p(\Omega))^N}{f}{(\partial^\alpha f)_{\abs{\alpha}\leq m}}
    $$
    with $N:=\abs{\{\alpha\in\NN^d:\abs{\alpha}\leq m\}}$ is an isometry. So $W^{m,p}(\Omega)$ can be identified with a closed vector space of $(L^p(\Omega))^N$. In particular, $W^{m,p}(\Omega)$ is separable for $p<\infty$, and it is reflexive for $1<p<\infty$.
  \end{proof}
  \begin{definition}
    Let $\Omega\subseteq \RR^d$ be an open set, $m\in\NN$ and $1\leq p\leq \infty$. We define the space $W_0^{m,p}(\Omega):=\overline{\mathcal{C}_0^\infty(\Omega)}$, where the closure is taken with the norm of $W^{m,p}(\Omega)$. Similarly, we set $H_0^m(\Omega):=W_0^{m,2}(\Omega)$.
  \end{definition}
  \begin{remark}
    Note that $W_0^{m,p}(\Omega)$ is also Banach (with the same norm as $W^{m,p}(\Omega)$) because it is a closed subspace in a Banach space.
  \end{remark}
  \begin{lemma}\label{ATFAPDE:smooth_convolution}
    Let $1\leq p\leq \infty$ and $(\phi_\varepsilon)$ be an approximation of identity. For all $f\in L^p(\RR^d)$, we set $f_\varepsilon:= f*\phi_\varepsilon$. Then:
    \begin{itemize}
      \item $f_\varepsilon$ is smooth.
      \item $f_\varepsilon\in L^p(\RR^d)$ with $\norm{f_\varepsilon}_p\leq \norm{f}_p$.
      \item If $p<\infty$, then $\norm{f_\varepsilon-f}_p\overset{\varepsilon\to 0}\longrightarrow 0$.
    \end{itemize}
  \end{lemma}
  \begin{theorem}
    For $1\leq p < \infty$ we have that $W_0^{m,p}(\RR^d)=W^{m,p}(\RR^d)$. In particular, $\mathcal{C}^\infty(\RR^d)\cap W^{m,p}(\RR^d)$ and $\mathcal{C}_0^{\infty}(\RR^d)$ are dense in $W^{m,p}(\RR^d)$ for $1\leq p<\infty$.
  \end{theorem}
  \begin{proof}
    Let us first prove that $\mathcal{C}^\infty(\RR^d)$ is dense in $W^{m,p}(\RR^d)$. Let $f\in W^{m,p}(\RR^d)$ and set $f_\varepsilon:=f*j_\varepsilon$ for an approximation of identity $j_\varepsilon$. By \mcref{ATFAPDE:smooth_convolution}, the functions $f_\varepsilon$ are smooth. For all $\abs{\alpha}\leq m$, the function $\partial^\alpha f$ is in $L^p(\RR^d)$, and we have $\partial^\alpha(f_\varepsilon)=(\partial^\alpha f)*j_\varepsilon$. By \mcref{HA:kernelConvLp}, we deduce that $\forall\abs{\alpha}\leq m$:
    $$
      \norm{\partial^\alpha f*j_\varepsilon-\partial^\alpha f}_p\to 0
    $$
    This already proves that $\mathcal{C}^\infty(\RR^d)\cap W^{m,p}(\RR^d)$ is dense in $W^{m,p}(\RR^d)$.
    For the second part, we take $f\in \mathcal{C}^\infty(\RR^d)\cap W^{m,p}(\RR^d)$ and set $f_n:=\chi(x/n)f$, where $\chi$ is a smooth cut-off function satisfying $\chi(x)=1$ for $\abs{x}\leq 1$. Then, $f_n\in \mathcal{C}_0^\infty(\RR^d)$. By \mnameref{RFA:dominated}, we have $\norm{f_n-f}_p\to 0$. Moreover, we have:
    \begin{align*}
      \norm{\grad f_n-\grad f}_p & =\norm{\grad\chi f+\grad f[\chi(x/n)-1]}_p                           \\
                                 & \leq \norm{\grad\chi}_\infty\norm{f}_p+\norm{\grad f[\chi(x/n)-1]}_p
    \end{align*}
    and the last term goes to 0 again by \mnameref{RFA:dominated}. So $\norm{\grad f_n-\grad f}_p\to 0$. We go on with all derivatives, which proves that $\norm{\partial^\alpha f_n-\partial^\alpha f}_p\to 0$ for all $\abs{\alpha}\leq m$. This shows that $f_n\to f$ in $W^{m,p}(\RR^d)$.
  \end{proof}
  \begin{theorem}[Poincaré's inequality]\label{ATFAPDE:poincare_ineq}
    Let $\Omega$ be a bounded open set in $\RR^d$ and let $1\leq p<\infty$. Then, there is a constant $C=C(\Omega,p)$ so that $\forall u\in W_0^{m,p}(\Omega)$ we have:
    $$
      \norm{u}_p\leq C\norm{\grad u}_p
    $$
  \end{theorem}
  % \begin{proof}
  %   Let $L:= \diam(\Omega)$. By density of $\mathcal{C}_0^\infty(\Omega)$ in $W_0^{1,p}(\Omega)$, it is enough to prove the result for $u\in \mathcal{C}_0^\infty(\Omega)$. Let $x\in \Omega$ and consider a point $a\in \partial\Omega$, so that $a_1=x_1$ (same first coordinate). On the segment $[a,x]$, we have the point-wise bound:
  %   $$
  %     \abs{u(x)}=\abs{u(x)-u(a)}\leq \int_a^x{\abs{\partial_{x_1}u}(s,x_2,\ldots,x_d)ds}\leq \int_a^x{\abs{\grad u}(s,x_2,\ldots,x_d)ds}
  %   $$
  %   where we used Hölder's inequality in the last line. We take the $p$ power and integrate. This gives, using Fubini, and the fact that the $dx_1$ integration can be performed on a segment of size $L$:
  %   $$
  %     \abs{u}^p\leq L\int_{\Omega}{\abs{\grad u}^p}
  %   $$
  % \end{proof}
  \begin{remark}
    \mnameref{ATFAPDE:poincare_ineq} is also valid when $\Omega$ is unbounded in one direction.
  \end{remark}
  \begin{corollary}
    If $\Omega$ is bounded, then the constant function $f(x)=C$ with $C\ne 0$ is not in $W_0^{1,p}$. Thus, we cannot approximate constant functions by $\mathcal{C}_0^\infty(\Omega)$ functions, with the $W^{1,p}(\Omega)$ norm.
  \end{corollary}
  \begin{definition}
    Let $\Omega$ be a bounded set. We define the average of $u$ in $\Omega$ as:
    $$
      \fint_\Omega u:=\frac{1}{\abs{\Omega}}\int_\Omega u
    $$
  \end{definition}
  \begin{theorem}[Poincaré-Wirtinger's inequality]
    Let $\Omega\subseteq \RR^d$ be a bounded connected open set with $\mathcal{C}^1$ boundary, and let $1\leq p<\infty$. Then, there is a constant $C=C(\Omega,p)$ so that $\forall u\in W^{1,p}(\Omega)$ we have:
    $$
      \norm{u-\fint_\Omega u}_p\leq C\norm{\grad u}_p
    $$
  \end{theorem}
  \subsubsection{Sobolev embeddings}
  \begin{definition}
    Let $E$, $F$ be Banach. We say that $F$ is \emph{embedded} in $E$ if $F\subseteq E$ and the inclusion map $i:F\hookrightarrow  E$ is continuous. We say that $F$ is \emph{compactly embedded} in $E$ if $F\subseteq E$ and the inclusion map $i:F\hookrightarrow  E$ is compact.
  \end{definition}
  \begin{theorem}[Gagliardo, Nirengerg and Sobolev's inequality]\label{ATFAPDE:gagliardo_nirengerg_sobolev}
    For all $1\leq p\leq\frac{d}{m}$, there is a constant $C=C(p,m,d)$ so that $\forall u\in \mathcal{C}_0^\infty(\RR^d)$ we have:
    $$
      \norm{u}_{q}\leq C\sum_{\abs{\alpha}= m}{\norm{\partial^\alpha u}_p}$$
    where $\displaystyle\frac{1}{q}=\frac{1}{p}-\frac{m}{d}$.
    That is, we have the continuous embedding $W^{m,p}(\RR^d)\hookrightarrow L^q(\RR^d)$. In particular for $m=1$, we have $W^{1,p}(\RR^d)\hookrightarrow L^{p^*}(\RR^d)$ with $\displaystyle\frac{1}{p^*}=\frac{1}{p}-\frac{1}{d}$.
  \end{theorem}
  \begin{proof}
    By induction, it suffices to prove the result only for $m=1$. We will prove only the case $d=2$. We start with $p=1$. Let $u\in \mathcal{C}_0^\infty(\RR^2)$. We have:
    \begin{multline*}
      \abs{u(x_1,x_2)}\leq \int_{-\infty}^{x_1}\abs{\partial_{x_1}u(s,x_2)}\dd{s}\leq\\\leq\int_{\RR}\abs{\partial_{x_1}u(s,x_2)}\dd{s}=:v_1(x_2)
    \end{multline*}
    Similarly, $\abs{u(x_1,x_2)}\leq v_2(x_1)$. So:
    \begin{multline*}
      {\norm{u}_2}^2\leq \int_{\RR^2} \abs{v_1(x_2)}\abs{v_2(x_1)}\dd{x_1}\dd{x_2}=\norm{v_1}_1 \norm{v_2}_1=\\=\norm{\partial_{x_1}u}_1\norm{\partial_{x_2}u}_1\leq {\norm{\grad u}_1}^2
    \end{multline*}
    For the case $1\leq p<2$, we apply the result to the function $u_t:=\abs{u}^{t-1}u$. This function satisfies $\abs{\grad u_t}=t\abs{u}^{t-1}\abs{\grad u}$ and so:
    \begin{multline*}
      {\norm{u}_{2t}}^t=\norm{u_t}_2\leq \norm{u_t}_1=t\norm{\abs{u}^{t-1}\grad u}\leq \\\leq t\norm{\abs{u}^{t-1}}_{p'}\norm{\grad u}_p=t{\norm{u}_{(t-1)p'}}^{t-1} \norm{\grad u}
    \end{multline*}
    where $p'$ is the Hölder conjugate of $p$. Now, we choose $t$ so that $2t=(t-1)p'$, that is, $t=\frac{p}{2-p}$ and so:
    $$
      \norm{u}_{\frac{2p}{2-p}}\leq \frac{p}{2-p}\norm{\grad u}_p
    $$
  \end{proof}
  \begin{definition}[Hölder continuity]
    Let $k\in\NN$ and $0\leq\theta\leq 1$. We say that a map $u:\Omega\to\CC$ is \emph{$\mathcal{C}^{k,\theta}$-Hölder continuous} if there is $C\geq 0$ such that $\forall\abs{\alpha}=k$ and all $x,y\in\Omega$ we have:
    $$
      \abs{\partial^\alpha u(x)-\partial^\alpha u(y)}\leq C\abs{x-y}^\theta
    $$
    The set of such functions is denoted by $\mathcal{C}^{k,\theta}(\Omega)$.
  \end{definition}
  \begin{remark}
    Note that $\mathcal{C}^{k,0}(\Omega)=\mathcal{C}^k(\Omega)$ and that $\mathcal{C}^{0,1}(\Omega)$ is the set of Lipschitz continuous functions.
  \end{remark}
  \begin{remark}
    Note that $\mathcal{C}^{k,\theta}(\Omega)$ together with the norm
    $$
      \norm{u}_{\mathcal{C}^{k,\theta}(\Omega)}:=\sup_{x\ne y}\sup_{\abs{\alpha}=k}\frac{\abs{\partial^\alpha u(x)-\partial^\alpha u(y)}}{\abs{x-y}^\theta}
    $$
    is a Banach space.
  \end{remark}
  \begin{theorem}[Morrey's embedding]\label{ATFAPDE:morrey_embedding}
    Let $m\geq 1$ and $p>\frac{d}{m}$. Then, $W^{m,p}(\RR^d)\subset L^\infty(\RR^d)$. In addition, let:
    $$
      k:=\left\lfloor m-\frac{d}{p}\right\rfloor\qquad\theta:=m-\frac{d}{p}-k\in [0,1)
    $$
    If $\theta\ne 0$, then $W^{m,p}(\RR^d)\subset \mathcal{C}^{k,\theta}(\RR^d)$.
  \end{theorem}
  \begin{theorem}
    For all $1\leq p\leq \infty$, $W^{1,p}(\RR)\hookrightarrow L^\infty(\RR)\cap \mathcal{C}^{0}(\RR)$.
  \end{theorem}
  \begin{proof}
    The proof for $p>1$ comes from \mnameref{ATFAPDE:morrey_embedding}. For $p=1$ we have that $\forall u\in \mathcal{C}_0^\infty(\RR)$ we have:
    $\abs{u(x)}\leq \int_{-\infty}^x{\abs{u'}}\leq \norm{u'}_1$. So $\norm{u}_\infty\leq \norm{u'}_1$. By density, this proves that $\forall u\in W^{1,1}(\RR)$ we have $u\in L^\infty(\RR)$. Now, let $(u_n)\in \mathcal{C}_0^\infty(\RR)$ be such that $u_n\to u$ in $W^{1,1}(\RR)$. Then, $u_n\to u$ in $L^\infty(\RR)$ and so $u$ is continuous because it is the uniform limit of continuous functions.
  \end{proof}
  \subsubsection{Extension operators}
  \begin{definition}
    Let $\Omega\subseteq \RR^d$ be an open set. An \emph{extension} of $u\in W\in W^{m,p}(\Omega)$ is a function $\tilde{u}\in W^{m,p}(\RR^d)$ so that $\tilde{u}\almoste{=}u$ in $\Omega$. An \emph{extension operator} is a bounded linear operator $E:W^{m,p}(\Omega)\to W^{m,p}(\RR^d)$ so that $Eu$ is an extension of $u$ $\forall u\in W^{m,p}(\Omega)$.
  \end{definition}
  \begin{remark}
    From now on, we will denote $\RR_{\pm}^d:=\RR^{d-1}\times\RR_{\pm}$ and $\RR_0^d:=\RR^{d-1}\times\{0\}$.
  \end{remark}
  \begin{theorem}
    For all $m\in\NN$ and all $1\leq p<\infty$, $\mathcal{C}^\infty(\overline{\RR_{\geq 0}^d})$ is dense in $W^{m,p}(\RR_{\geq 0}^d)$.
  \end{theorem}
  \begin{proof}
    Let $$
      \tau_h(u)(x_1,\ldots, x_d):=u(x_1,\ldots, x_{d-1},x_d+h)
    $$
    be the translation operator and set $u_\varepsilon:=\tau_{\varepsilon}(u)*\phi_\varepsilon$, where $\varepsilon>0$ and $\phi_\varepsilon$ is an approximation of identity. Then, $u_\varepsilon\in \mathcal{C}^\infty(\overline{\RR_{\geq 0}^d})$ by the properties of the convolution. Moreover:
    \begin{multline*}
      \norm{\partial^\alpha u_\varepsilon-\partial^\alpha u}_p \leq \norm{\partial^\alpha u_\varepsilon-\partial^\alpha (\tau_{\varepsilon}u)}_p+\norm{\partial^\alpha (\tau_{\varepsilon}u)-\partial^\alpha u}_p                         \\
      \leq \norm{(\partial^\alpha \tau_\varepsilon u)*\phi_\varepsilon-\partial^\alpha (\tau_{\varepsilon}u)}_p+\norm{\tau_{\varepsilon}(\partial^\alpha u)-\partial^\alpha u}_p
    \end{multline*}
    The first term goes to zero by the properties of smoothing sequences, and the second goes to zero since translations are continuous in $L^p$ (check \mcref{HA:translated}).
  \end{proof}
  \begin{remark}
    The same proof shows that $\mathcal{C}^\infty(\overline{\Omega})$ is dense in $W^{m,p}(\Omega)$, if $\Omega$ is bounded with $\Fr{\Omega}$ of class $\mathcal{C}^1$. This time, one needs to locally translate u along the normal direction.
  \end{remark}
  \begin{theorem}
    For all $m\in\NN$ and all $1\leq p<\infty$, there is an extension operator $E:W^{m,p}(\RR_{\geq 0}^d)\to W^{m,p}(\RR^d)$.
  \end{theorem}
  \begin{proof}
    We only do the proof for $d=1$ and $m=1$ to highlight the main ideas. Let $u\in W^{1,p}(\RR_{\geq 0})$. We define the \emph{first order reflection}:
    $$
      \bar{u}:=\begin{cases}
        u(x)             & \text{if }x\geq 0 \\
        -3u(-x)+4u(-x/2) & \text{if }x<0
      \end{cases}
    $$
    By density, it is enough to prove the result for $u\in \mathcal{C}^1(\RR_{\geq 0})$. An easy check shows that $\bar{u}\in \mathcal{C}^1(\RR)$. Moreover, we have:
    \begin{align*}
      {\norm{\bar{u}}_{W^{1,p}(\RR)}}^p & =\int_{\RR}{\abs{\bar{u}}^p}+{\abs{\bar{u}'}^p}                                                   \\
      \begin{split}
        & =\!\int_{\RR_{\geq 0}}\!{\abs{u}^p}\!+\!\!{\abs{u'}^p}\!+\!\int_{\RR_{\leq 0}}\![{\abs{-3u(-x)+4u(-x/2)}^p}+ \\
        & \hspace{2.75cm}+{\abs{3u'(-x)-2u'(-x/2)}^p}]
      \end{split} \\
                                        & \leq C{\norm{u}_{W^{1,p}(\RR_{\geq 0})}}^p
    \end{align*}
    for some constant $C>0$. Thus, $E$ is a bounded extension operator.
  \end{proof}
  \begin{remark}
    The reader can check that the same construction works on $\RR^d$, by setting
    $$
      \bar{u}(x_1,\ldots,x_d):=
      \begin{cases}
        u(x_1,\ldots,x_d)               & \text{if }x_d\geq 0 \\
        \begin{split}
          -3u(x_1,\ldots,x_{d-1}-x_d)+ \\
          +4u(x_1,\ldots,x_{d-1}-x_d/2)
        \end{split} & \text{if }x_d<0
      \end{cases}
    $$
    The proof for higher derivatives $m \geq 1$ needs to add more terms in order to make the junction smooth enough.
  \end{remark}
  \begin{definition}
    We say that a domain $\Omega\subseteq \RR^d$ has boundary of class $\mathcal{C}^k$ if $\forall x\in \Fr{\Omega}$ there is a neighborhood $\varepsilon,\delta>0$ and a $\mathcal{C}^k$-diffeomorphism $\phi:B(x,\varepsilon)\to B(0,\delta)$ so that $\phi(x)=0$ and $\phi(B(x,\varepsilon)\cap \Omega)=B(0,\delta)\cap \RR_{\geq 0}^d$. Note that in particular this implies that $\phi(\Fr{\Omega}\cap B(x,\varepsilon))=B(0,\delta)\cap \RR_0^d$.
  \end{definition}
  \begin{theorem}
    Let $\Omega\subseteq \RR^d$ be a bounded domain with $\mathcal{C}^k$ boundary. Then, $\forall m\leq k$ and all $1\leq p<\infty$, there is an extension operator $E:W^{m,p}(\Omega)\to W^{m,p}(\RR^d)$.
  \end{theorem}
  \begin{theorem}
    Let $\Omega\subseteq \RR^d$ be a bounded domain with $\mathcal{C}^k$ boundary. Then, $\forall m\leq k$, if $1\leq p<\frac{d}{m}$,
    % now add the conclusions of the gagliardo theorem
    there is an embedding $W^{m,p}(\Omega)\hookrightarrow L^q(\Omega)$, where $\displaystyle\frac{1}{q}=\frac{1}{p}-\frac{m}{d}$. If $p>\frac{d}{m}$, then $W^{m,p}(\Omega)\hookrightarrow \mathcal{C}^{\ell,\theta} (\overline{\Omega})$, where $\ell:=\left\lfloor m-\frac{d}{p}\right\rfloor$ and $\theta:=m-\frac{d}{p}-\ell\in [0,1)$.
  \end{theorem}
  \begin{theorem}[Reillich-Kondrachov's compactness theorem]\label{ATFAPDE:reillich_kondrachov_compactness}
    Let $\Omega\subseteq \RR^d$ be a bounded domain with $\mathcal{C}^k$ boundary. Then, $\forall m\leq k$ we have:
    \begin{itemize}
      \item If $1\leq p<\frac{d}{m}$, $\forall r\in [p,q)$, where $\displaystyle\frac{1}{q}=\frac{1}{p}-\frac{m}{d}$, the embedding $W^{m,p}(\Omega)\hookrightarrow L^r(\Omega)$ is compact.
      \item If $p\geq \frac{d}{m}$, then $\forall r\in [p,\infty)$, the embedding $W^{m,p}(\Omega)\hookrightarrow L^r(\Omega)$ is compact.
      \item If $p>\frac{d}{m}$, then the embedding $W^{m,p}(\Omega)\hookrightarrow \mathcal{C}^{0}(\overline{\Omega})$ is compact.
    \end{itemize}
  \end{theorem}
  \subsubsection{Trace operators}
  \begin{theorem}
    Let $1\leq p<\infty$ and $u\in W^{1,p}(\RR_{\geq 0}^d)$. Then, the function $u|_{\RR_0^d}:\RR^{d-1}\to\CC$ belongs to $L^p(\RR^{d-1})$.
  \end{theorem}
  \begin{definition}
    We define the \emph{trace operator} as the map:
    \begin{align*}
      \function{\Tr}{W^{1,p}(\RR_{\geq 0}^d)}{L^p(\RR^{d-1})}{u}{u|_{\RR_0^d}}
    \end{align*}
  \end{definition}
  \begin{theorem}
    Let $1\leq p<\infty$ and $u\in W^{1,p}(\RR_{\geq 0}^d)$. Then, $\Tr u=0$ if and only if $u\in W_0^{1,p}(\RR_{\geq 0}^d)$.
  \end{theorem}
  \begin{theorem}\label{ATFAPDE:trace_thm}
    Let $1\leq p<\infty$ and $\Omega\subset\RR^d$ be a bounded domain with $\mathcal{C}^1$ boundary. Then, the trace operator
    $$
      \function{\Tr}{W^{1,p}(\Omega)}{L^p(\Fr{\Omega})}{u}{u|_{\Fr{\Omega}}}
    $$
    is bounded. Here we are taking the norm of $L^p(\Fr{\Omega})$ as ${\norm{u}_{L^p(\Fr{\Omega})}}^p:=\int_{\Fr{\Omega}}{\abs{u}^p}$. In addition:
    \begin{itemize}
      \item $\forall u\in W^{1,p}(\Omega)$, $\Tr u=0$ if and only if $u\in W_0^{1,p}(\Omega)$.
      \item For $p=2$, $\Tr$ is surjective.
    \end{itemize}
  \end{theorem}
  \begin{lemma}
    Let $\Omega\subseteq \RR^d$ be an open set and $u\in W^{1,p}(\Omega)$ with $1\leq p\leq \infty$. Then, $\abs{u}\in W^{1,p}(\Omega)$ and $\norm{\grad \abs{u}}\almoste{\leq}\norm{\grad u}$.
  \end{lemma}
  \begin{proof}
    $$
      \norm{2\abs{u}\grad\abs{u}}=\norm{\grad \abs{u}^2}=2\norm{\Re(\overline{u}\grad u)}\leq 2\norm{\grad u}\abs{u}
    $$
    On the set $\{u\ne 0\}$, we can divide by $2\abs{u}$, which gives the result. The proof on the set $\{u=0\}$ is much difficult.
  \end{proof}
\end{multicols}
\end{document}