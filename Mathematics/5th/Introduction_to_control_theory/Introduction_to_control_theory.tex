\documentclass[../../../main_math.tex]{subfiles}

\begin{document}
\changecolor{ICT}
\begin{multicols}{2}[\section{Introduction to control theory}]
  \subsection{Control theory in ODEs}
  \subsubsection{Stability}
  \begin{definition}
    A function $\alpha: \RR_{\geq 0} \to \RR_{\geq 0}$ is said to be of \emph{class $\mathcal{K}$} if it is continuous, strictly increasing and $\alpha(0) = 0$. If, moreover, $\displaystyle \lim_{s \to \infty} \alpha(s) = \infty$, then $\alpha$ is said to be of \emph{class $\mathcal{K}^\infty$}.
  \end{definition}
  \begin{definition}
    A function $\beta: \RR_{\geq 0} \times \RR_{\geq 0} \to \RR_{\geq 0}$ is said to be of \emph{class $\mathcal{KL}$} if, for each fixed $t \geq 0$, the function $\beta(\cdot, t)$ is of class $\mathcal{K}$ and, for each fixed $s \geq 0$, the function $\beta(s, \cdot)$ is decreasing and $\displaystyle \lim_{t \to \infty} \beta(s, t) = 0$.
  \end{definition}
  \begin{remark}
    An example of a function class $\mathcal{K}$ not in $\mathcal{K}^\infty$ is for example $\alpha(s)=\arctan(s)$. Examples of functions of class $\mathcal{KL}$ are for instance $\beta(s, t) = s\exp{-t}$ or $\beta(s, t) = \arctan(s/(t+1))$.
  \end{remark}
  \begin{definition}
    Let $E\subseteq \RR^n$ be a neighbourhood of the origin and $V: E \to \RR_{\geq 0}$ be a function. We say that $V$ is \emph{positive definite} on $E$ if $\{V=0\} = \{0\}$. We say that $V$ is \emph{negative definite} on $E$ if $-V$ is positive definite on $E$.
  \end{definition}
  \begin{lemma}
    Let $E\subseteq \RR^n$ be a neighbourhood of the origin and $V: E \to \RR_{\geq 0}$ be positive definite on $E$. Then, for any compact set $K \subseteq E$ with $0\in \Int K$, there exists $\alpha \in \mathcal{K}$ such that $\alpha(\norm{\vf{x}}) \leq V(\vf{x})$ for all $\vf{x} \in K$.
  \end{lemma}
  \begin{remark}
    If $V$ is continuous, then it is uniformly continuous on compact sets, and so we have:
    $$
      \abs{V(\vf{x}) - V(\vf{y})} \leq \omega(\norm{\vf{x}-\vf{y}})
    $$
    where $\omega$ is a modulus of continuity of $V$. Then, we can find $\alpha_1 \in \mathcal{K}$ such that $\alpha_1\geq \omega$ and so we have an upper bound for $V(x)\leq \alpha_1(\norm{\vf{x}})$.
  \end{remark}
  \begin{definition}
    Let $E\subseteq \RR^n$ be a neighbourhood of the origin. We defined the \emph{penalized norm} on $E$ as the function:
    $$
      \function{\omega_E}{E}{\RR_{\geq 0}}{\vf{x}}{\norm{\vf{x}}\left(1+\frac{1}{d(\vf{x}, \Fr{E})}\right)}
    $$
  \end{definition}
  From now on, we will consider that the system
  \begin{equation}\label{ICT:ode}
    \begin{cases}
      \dot{\vf{x}} = \vf{f}(\vf{x}) \\
      \vf{x}(0) = \vf{x}_0
    \end{cases}
  \end{equation}
  has an equilibrium point at the origin. We will denote by $\vf{X}(\vf{x}_0, t)$ a solution of the system with initial condition $\vf{X}(\vf{x}_0, 0) = \vf{x}_0\in \mathcal{O}\subseteq \RR^n$.
  \begin{definition}
    The equilibrium $\vf{X}(0, t)=0$ of \mcref{ICT:ode} is said to be:
    \begin{itemize}
      \item \emph{stable} if $\exists\mu>0$ and $\alpha\in\mathcal{K}$ such that $\forall\norm{\vf{x}_0}<\mu$ any solution $\vf{X}(\vf{x}_0, \cdot)$ exists for all $t\geq 0$ and satisfies:
            $$
              \norm{\vf{X}(\vf{x}_0, t)}\leq \alpha(\norm{\vf{x}_0})\quad\forall t\geq 0
            $$
      \item \emph{attractive} if $\exists\mu>0$ such that $\forall\norm{\vf{x}_0}<\mu$ any solution $\vf{X}(\vf{x}_0, \cdot)$ exists for all $t\geq 0$ and satisfies:
            $$
              \lim_{t\to\infty}\norm{\vf{X}(\vf{x}_0, t)}=0
            $$
      \item \emph{asymptotically stable} if $\exists \mu>0$ and $\beta\in \mathcal{KL}$ such that $\forall\norm{\vf{x}_0}<\mu$ any solution $\vf{X}(\vf{x}_0, \cdot)$ exists for all $t\geq 0$ and satisfies:
            $$
              \norm{\vf{X}(\vf{x}_0, t)}\leq \beta(\norm{\vf{x}_0}, t)\quad\forall t\geq 0
            $$
      \item \emph{exponentially stable} if $\exists k,\lambda,\mu>0$ such that $\forall\norm{\vf{x}_0}<\mu$ any solution $\vf{X}(\vf{x}_0, \cdot)$ exists for all $t\geq 0$ and satisfies:
            $$
              \norm{\vf{X}(\vf{x}_0, t)}\leq k\norm{\vf{x}_0} \exp{-\lambda t}\quad\forall t\geq 0
            $$
    \end{itemize}
    Moreover, in the last two cases, if $\mu$ can be picked as large as we want, then the equilibrium is said to be \emph{globally stable}.
  \end{definition}
  \begin{remark}
    Note that exponential stability implies asymptotic stability, which implies stability, which implies attractivity. Moreover, it can be seen that asymptotically stability is equivalent to stability and attractivity.
  \end{remark}
  \begin{remark}
    An equivalent definition for stability is the following: $\forall \varepsilon>0$ $\exists \delta>0$ such that if $\norm{\vf{x}_0}<\delta$ then $\norm{\vf{X}(\vf{x}_0, t)}<\varepsilon$ for all $t\geq 0$.
  \end{remark}
  \begin{definition}
    The equilibrium $\vf{X}(0, t)=0$ of \mcref{ICT:ode} is said to be unstable if $\exists \varepsilon>0$ such that $\forall \delta>0$ $\exists \vf{x_0}\in B(\vf{0},\delta)$ and a solution $\vf{X}(\vf{x}_0, \cdot)$ such that $\norm{\vf{X}(\vf{x}_0, t^*)}\geq \varepsilon$ for some $t^*\geq 0$.
  \end{definition}
  \begin{remark}
    A solution may be unstable and attractive at the same time. For example, the system
    $$
      \begin{cases}
        \dot{x} = x^2(y-x) + y^5 \\
        \dot{y} = y^2(y-2x)
      \end{cases}
    $$
    exhibits the behaviour shown in \mcref{ICT:unstable_attractor}.
    \begin{figure}[H]
      \centering
      \includestandalone[mode=image|tex,width=0.5\linewidth]{Images/unstable_attractor}
      \caption{Unstable attractor}
      \label{ICT:unstable_attractor}
    \end{figure}
  \end{remark}
  \begin{definition}
    We define the \emph{basin of attraction} of the origin as the set $\mathcal{A}$ of all initial conditions $\vf{x}_0$ such that the solution $\vf{X}(\vf{x}_0, \cdot)$ exists for all $t\geq 0$ and satisfies $\displaystyle\lim_{t\to\infty}\vf{X}(\vf{x}_0, t)=0$.
  \end{definition}
  \begin{theorem}
    If the origin is asymptotically stable, then its basin of attraction is an open set included in $\mathcal{O}$. Besides, $\exists \beta_\mathcal{A}\in \mathcal{KL}$ such that $\forall \vf{x}_0\in\mathcal{A}$, any solution $\vf{X}(\vf{x}_0, \cdot)$ exists for all $t\geq 0$ and satisfies $\omega_{\mathcal{A}}(\norm{\vf{X}(\vf{x}_0, t)})\leq \beta_\mathcal{A}(\norm{\vf{x}_0}, t)$ for all $t\geq 0$.
  \end{theorem}
  \begin{theorem}
    Assume that $\vf{f}\in\mathcal{C}^1$. Then:
    \begin{enumerate}
      \item The zero solution is exponentially stable if and only if the zero solution of the system $\dot{\vf{y}}=\vf{Df}(\vf{0}) \vf{y}$ is exponentially stable.
      \item If $\vf{Df}(\vf{0})$ has an eigenvalue with positive real part, then the origin is unstable.
    \end{enumerate}
  \end{theorem}
  \begin{remark}
    In linear dynamics exponentially stability is equivalent to global exponentially stability, which in turn is equivalent to global asymptotic stability which is equivalent to asymptotic stability.
  \end{remark}
  \begin{corollary}
    If $\vf{f}\in\mathcal{C}^1$ and $\vf{Df}(\vf{0})$ has all its eigenvalues with negative real part, then the origin is asymptotically stable.
  \end{corollary}
  \begin{theorem}
    Let $V:\mathcal{O}\to \RR_{\geq 0}$ be a locally Lipschitz function which is positive definite on $\mathcal{O}$. Then, if
    $$
      D_f^+V(\vf{x}) = \limsup_{t\to 0^+}\frac{V(\vf{x} + \vf{f}(\vf{x})t) - V(\vf{x})}{t}
    $$
    is non-positive for all $\vf{x}\in \mathcal{O}$, then the origin is stable. The function $V$ is called a \emph{Lyapunov function}.
  \end{theorem}
  \begin{proof}
    Since $\mathcal{O}$ is a neighbourhood of the origin $\exists R>0$ such that $\overline{B(0,R)}\subseteq \mathcal{O}$. Then, since $V$ is continuous and positive definite, $\exists \alpha_1,\alpha_2\in\mathcal{K}$ such that $\alpha_1(\norm{\vf{x}})\leq V(\vf{x})\leq \alpha_2(\norm{\vf{x}})$ for all $\vf{x}\in B(0,R)$. Let $\mu:={\alpha_2}^{-1}(\alpha_1(R/2))$. Then, any solution with initial conditions $\norm{\vf{x}_0}<\mu$ belongs to $\overline{B(0,R)}$ at least for $t\in[0, T)$. Now if we consider $v(t) := V(\vf{X}(\vf{x}_0, t))$, then we have $\dot{v}(t) = D_f^+V(\vf{X}(\vf{x}_0, t))\leq 0$ for all $t\geq 0$. Thus, $\forall t \in [0, T)$ we have:
    \begin{multline*}
      \alpha_1(\norm{\vf{X}(\vf{x}_0, t)})\leq V(\vf{X}(\vf{x}_0, t))=v(t) \leq v(0)=\\ = V(\vf{x}_0) \leq \alpha_2(\norm{\vf{x}_0})
    \end{multline*}
    And so $\norm{\vf{X}(\vf{x}_0, t)}\leq \alpha_2^{-1}(\alpha_1(\norm{\vf{x}_0}))\leq R/2$ for all $t\in[0, T)$. This mean that in fact $T=\infty$ and so the origin is stable.
  \end{proof}
  \begin{theorem}
    Let $V:\mathcal{O}\to \RR_{\geq 0}$ be a locally Lipschitz function which is positive definite on $\mathcal{O}$. Then, if
    $$
      D_f^+V(\vf{x})\leq -w(\vf{x}), \quad \forall \vf{x}\in \mathcal{O}
    $$
    with $w:\mathcal{O}\to \RR_{\geq 0}$ continuous and positive definite, then the origin is globally asymptotically stable.
  \end{theorem}
  \begin{theorem}[Lasaalle's invariance principle]
    Let $K$ be a compact set contained in $\mathcal{O}$ and let $V:\mathcal{O}\to \RR_{\geq 0}$ be a locally Lipschitz function which is positive definite on $\mathcal{O}$ and such that $D_f^+V(\vf{x})\leq -w(\vf{x})$ for all $\vf{x}\in K$ with $w:\mathcal{O}\to \RR_{\geq 0}$ continuous (not necessarily positive definite). Then, for any solution $\vf{X}(\vf{x}_0, \cdot)$ with $\vf{x}_0\in K$ and defined on $K$ for all $t\geq 0$, $\exists v^*\in\RR_{\geq 0}$ such that $\vf{X}(\vf{x}_0, t)$ converges to the largest positively invariant set contained in:
    $$
      \{\vf{y}\in K: V(\vf{y})=v^*\text{ and }w(\vf{y})=0\}
    $$
  \end{theorem}
  \begin{theorem}[Chetaev's theorem]
    Let $V:\mathcal{O}\to \RR_{\geq 0}$ be a locally Lipschitz function such that:
    \begin{itemize}
      \item $0\in \Fr{G}$, with $G:=\{\vf{x}\in \mathcal{O}: V(\vf{x})=0\}$.
      \item There exists a neighbourhood $U$ (called Chetaev surface) of the origin such that $D_f^+V(\vf{x})>0$ for all $\vf{x}\in U\cap G$
    \end{itemize}
    Then, the origin is unstable.
  \end{theorem}
  \begin{theorem}
    If the origin is asymptotically stable, then $\forall\varepsilon>0$ $\{f(\vf{x}): \norm{\vf{x}}\leq \varepsilon\}$ is a neighbourhood of the origin.
  \end{theorem}
  \begin{theorem}
    If the origin is locally asymptotically stable with basin of attraction $\mathcal{A}$, then $\exists \lambda>0$ and $v:\mathcal{A}\to \RR_{\geq 0}$ $\mathcal{C}^\infty$, positive definite and proper (that is, $\displaystyle \lim_{d(\vf{x}, \Fr{\mathcal{A}})\to 0}v(\vf{x})=\infty$) such that:
    $$
      D_f^+v(\vf{x})\leq -\lambda v(\vf{x})\quad\forall \vf{x}\in \mathcal{A}
    $$
  \end{theorem}
  \subsubsection{Control design and stabilization of equilibrium points}
  \begin{definition}
    The system $\dot{\vf{x}} = \vf{f}(\vf{x}, \vf{u})$ is said to be \emph{controllable} in time $T>0$ if $\forall \vf{x}_0, \vf{x}_T\in \mathcal{O}$ $\exists \vf{u}: [0, T]\to \RR^p$ such that the solution $\vf{X}(\vf{x}_0,\cdot, \vf{u})$ of the system with initial condition $\vf{X}(\vf{x}_0, 0,\vf{u}) = \vf{x}_0$ satisfies $\vf{X}(\vf{x}_0, T,\vf{u}) = \vf{x}_T$.
  \end{definition}
  \begin{definition}
    The origin is said to be \emph{asymptotically stabilizable} if there exists $q\in \NN$, a neighbourhood $\mathcal{V}\subseteq \RR^q$ of the origin and $\vf\varphi:\RR\times\RR^n\times\mathcal{V}\to \RR^q$, $\vf\psi:\RR\times\RR^n\times\mathcal{V}\to \RR^p$ both continuous, such that the origin is an asymptotically stable solution of the system:
    $$
      \begin{cases}
        \dot{\vf{x}} = \vf{f}(\vf{x}, \vf{u})         \\
        \dot{\vf{u}} = \vf\varphi(t, \vf{x}, \vf\chi) \\
        \vf{\vf\chi} = \vf\psi(t, \vf{x}, \vf\chi)
      \end{cases}
    $$
  \end{definition}
  \begin{theorem}[Kalmann's theorem]
    Consider the linear system $\dot{\vf{x}} = \vf{Ax} + \vf{Bu}$ with $\vf{A}\in \RR^{n\times n}$ and $\vf{B}\in \RR^{n\times p}$. Then, the origin is asymptotically stabilizable if and only if
    $$
      \rank\vf{C}:=\rank\begin{pmatrix} \vf{B} & \vf{AB} & \cdots & \vf{A}^{n-1}\vf{B} \end{pmatrix} = n
    $$
    The matrix $\vf{C}$ is called the \emph{controllability matrix}.
  \end{theorem}

  \subsection{Control theory in PDEs}
\end{multicols}
\end{document}