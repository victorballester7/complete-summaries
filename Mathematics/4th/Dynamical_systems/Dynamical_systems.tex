\documentclass[../../../main.tex]{subfiles}


\begin{document}
\renewcommand{\col}{\apl}
\begin{multicols}{2}[\section{Dynamical systems}]
  \subsection{Dynamical systems in Euclidean spaces}
  \subsubsection{Introduction}
  \begin{definition}
    A \emph{dynamical system} is a triplet $(X,G,\Pi)$, where $G$ is a topological abelian group\footnote{That is, $G$ is an abelian group with an inherited topological structure.}, $X$ is a topological space and $\Pi:X\times G\rightarrow X$ is a function such that:
    \begin{itemize}
      \item $\Pi(\cdot,t)$ is continuous $\forall t\in G$.
      \item $\Pi(x,0)=x$ $\forall x\in X$.
      \item $\Pi(\Pi(t,x),s)=\Pi(x,t+s)$ $\forall s,t\in G$ and $\forall x\in X$.
    \end{itemize}
    We say that a dynamical system $(X,G,\Pi)$ is \emph{discrete} if $G=\ZZ$ and we say that it is \emph{continuous} if $G=\RR$.
  \end{definition}
  \begin{definition}
    Let $(X,G,\Pi)$ be a dynamical system and $x\in X$. We define the \emph{orbit} of $x$ as: $$\gamma_x=\gamma_{x,\Pi}:=\{\Pi(x,t):t\in G\}$$
  \end{definition}
  \begin{definition}
    Let $(X,G,\Pi)$ be a dynamical system. In the continuous case, we say that $x\in X$ is an \emph{equilibrium point} if $\Pi(x,t)=x$ $\forall t\in G$. In the discrete case, a point $x\in X$ satisfying this property is called a \emph{fixed point}.
  \end{definition}
  \begin{definition}
    Let $(X,G,\Pi)$ be a dynamical system. A \emph{periodic orbit of period $T$} is an orbit of the system that satisfies $\Pi(x,t+T)=\Pi(x,t)$ $\forall t\in G$ and for some $x\in X$.
  \end{definition}
  \begin{lemma}
    Let $\vf{f}:\RR^n\rightarrow\RR^n$ be a continuous function such that $\exists M,N\in\RR_{\geq 0}$ with $\|\vf{f}(\vf{x})\|\leq M\|\vf{x}\|+N$. Then, the solutions of the ode $\vf{x}' =\vf{f}(\vf{x})$ are defined for all $t\in\RR$.
  \end{lemma}
  \begin{theorem}
    Let $(X,G,\Pi)$ be a dynamical system and $\gamma_x$ be an orbit. Then, there are 3 possible cases for $\gamma_x$:
    In the continuous case:
    \begin{enumerate}
      \item $\gamma_x$ is an equilibrium point.
      \item $\gamma_x$ is a periodic orbit.
      \item $\gamma_x\cong\RR$.
    \end{enumerate}
    In the discrete case:
    \begin{enumerate}
      \item $\gamma_x$ is a fixed point.
      \item $\gamma_x$ is a periodic orbit.
      \item $\gamma_x\cong\ZZ$.
    \end{enumerate}
  \end{theorem}
  \subsubsection{Equivalence and conjugacy of dynamical systems}
  \begin{definition}
    Let $(X_1,G,\Pi_1)$ and $(X_2,G,\Pi_2)$ be two dynamical systems and $r\in\NN\cup\{0,\infty,\omega\}$. We say that $(X_1,G,\Pi_1)$ and $(X_2,G,\Pi_2)$ are \emph{equivalent dynamical systems} of class $\mathcal{C}^r$ if there exists a diffeomorphism $h:X_1\rightarrow X_2$ of class $\mathcal{C}^r$ such that $\forall x\in X_1$, $h(\gamma_{x,\Pi_1})=\gamma_{h(x),\Pi_2}$ and preserving the orientation of the orbits. In particular, if $r=0$ we say that $(X_1,G,\Pi_1)$ and $(X_2,G,\Pi_2)$ are \emph{topologically equivalent}. That diffeomorphism $h$ is called an \emph{equivalence} (of class $\mathcal{C}^r$) between $(X_1,G,\Pi_1)$ and $(X_2,G,\Pi_2)$.
  \end{definition}
  \begin{definition}
    Let $(X_1,G,\Pi_1)$ and $(X_2,G,\Pi_2)$ two be dynamical systems and $r\in\NN\cup\{0,\infty,\omega\}$. We say that $(X_1,G,\Pi_1)$ and $(X_2,G,\Pi_2)$ are \emph{conjugate dynamical systems} of class $\mathcal{C}^r$ if there exists a diffeomorphism $h:X_1\rightarrow X_2$ of class $\mathcal{C}^r$ such that $\forall (x,t)\in X_1\times G$, $h(\Pi_1(x,t))=\Pi_2(h(x),t)$. In particular, if $r=0$ we say that $(X_1,G,\Pi_1)$ and $(X_2,G,\Pi_2)$ are \emph{topologically conjugate}. That diffeomorphism $h$ is called a \emph{conjugacy} (of class $\mathcal{C}^r$) between $(X_1,G,\Pi_1)$ and $(X_2,G,\Pi_2)$.
  \end{definition}
  \begin{definition}
    Let $f,g:\RR^2\rightarrow\RR$ be two functions and consider the system of odes:
    \begin{equation}\label{DS_plane}
      \left\{
      \begin{aligned}
        x' & =p(x,y) \\
        y' & =q(x,y)
      \end{aligned}
      \right.
    \end{equation}
    We say that the system is \emph{symmetric with respect to the $x$-axis} if it is invariant under the transformation $(x,y,t)\rightarrow(-x,y,-t)$. Analogously, we say that the system is \emph{symmetric with respect to the $y$-axis} if it is invariant under the transformation $(x,y,t)\rightarrow(x,-y,-t)$.
  \end{definition}
  \begin{theorem}
    Consider the system \cref{DS_plane} and suppose that the origin is an equilibrium point. If the system is symmetric with respect to the $x$-axis or $y$-axis, and if the origin is a center of the linear part of the system, then the origin is a center of the nonlinear system \cref{DS_plane}.
  \end{theorem}
  \begin{theorem}
    Consider the differential system in $\CC$ defined by:
    $$z'=\ii z+ a_2z^2+a_3z^3+\cdots=:\ii z+f(z)$$ This is a \emph{holomorphic differential equation}. Then, there exists a conjugaction that conjugues this system with $z'=\ii z$. This process of finding the conjugacy is called \emph{linearization} of the first system.
  \end{theorem}
  \subsubsection{Invariant sets and limit sets}
  \begin{definition}
    Let $(X,G,\Pi)$ be a dynamical system and $A\subseteq \RR^n$ be a subset. We say that $A$ is \emph{invariant} if $\gamma_x\subseteq A$ $\forall x\in A$. We say that $A$ is \emph{positively invariant} if ${\gamma_x}^+\subseteq A$ $\forall x\in A$ Analogously, we say that $A$ is \emph{negatively invariant} if ${\gamma_x}^-\subseteq A$ $\forall x\in A$.
  \end{definition}
  \begin{definition}
    Let $(X,G,\Pi)$ be a dynamical system and $A\subseteq \RR^n$ be an invariant subset. We say that $A$ is \emph{minimal} if it doesn't contain any proper invariant subset.
  \end{definition}
  \begin{definition}
    Let $f(x,y)=0$ be an algebraic curve, $p,q\in\RR[x,y]$ and consider the polynomial system of degree $n$ of \cref{DS_plane}. We say that $f(x,y)=0$ is an \emph{invariant algebraic curve} under the system of \cref{DS_plane} if
    \begin{equation}
      \pdv{f}{x}(x,y)p(x,y)+\pdv{f}{y}q(x,y)=k(x,y)f(x,y)
    \end{equation}
    where $k\in\RR[x,y]$ is called \emph{cofactor} of the invariant curve $f(x,y)=0$\footnote{Note that $\deg k\leq n-1$.}.
  \end{definition}
  \begin{definition}
    A \emph{Kolmogorov system} is a dynamical system of the form:
    \begin{equation}
      \left\{
      \begin{aligned}
        {x_1}' & =x_1 f_1(\vf{x}) \\
               & \;\;\vdots       \\
        {x_n}' & =x_n f_n(\vf{x})
      \end{aligned}
      \right.
    \end{equation}
  \end{definition}
  \subsection{Study of local dynamics}
  \begin{theorem}[The stable manifold theorem]
    Let $E\subseteq\RR^n$ be an open subset containing the origin, $\vf{f}\in\mathcal{C}^1(E)$ and $\vf\phi_t$ be the flow of the following system of $n$ equations:
    \begin{equation}\label{DS_general}
      \vf{x}' =\vf{f}(\vf{x})
    \end{equation}
    Suppose that $\vf{f}(\vf{0})=\vf{0}$ and that $\vf{Df}(0)$ has $n_+$ eigenvalues with positive real part and $n_-=n-n_+$ eigenvalues with negative real part. Then, there exists a $n_+$-dimensional differentiable manifold $S$ tangent to the stable eigenspace $E^\text{s}$ of the linear system at $\vf{0}$ such that $\forall t\geq 0$, $\vf\phi_t(S)\subseteq S$ and $\forall \vf{x}_0\in S$: $$\lim_{t\to\infty}\vf\phi_t(\vf{x}_0)=\vf{0}$$
    Analogously, there exists a $n_-$-dimensional differentiable manifold $U$ tangent to the stable eigenspace $E^\text{u}$ of the linear system at $\vf{0}$ such that $\forall t\leq 0$, $\vf\phi_t(U)\subseteq U$ and $\forall \vf{x}_0\in U$: $$\lim_{t\to-\infty}\vf\phi_t(\vf{x}_0)=\vf{0}$$
  \end{theorem}
  \begin{definition}
    Let $E\subseteq\RR^n$ be an open subset containing the origin, $\vf{f}\in\mathcal{C}^1(E)$ and $\vf\phi_t$ be the flow of the system of \cref{DS_general}. The \emph{global stable manifold} and \emph{global unstable manifold} at $\vf{0}$ are defined respectively by:
    $$W^\text{s}(\vf{0})=\bigcup_{t\leq 0}\vf\phi_t(S)\qquad W^\text{u}(\vf{0})=\bigcup_{t\geq 0}\vf\phi_t(U)$$
  \end{definition}
  \begin{proposition}
    Let $E\subseteq\RR^n$ be an open subset containing the origin, $\vf{f}\in\mathcal{C}^1(E)$ and $\vf\phi_t$ be the flow of the system of \cref{DS_general}. Then, the manifolds $W^\text{s}$ and $W^\text{u}$ are unique and invariant with respect to the flow $\vf\phi_t$
  \end{proposition}
  \begin{theorem}[The center manifold theorem]
    Let $E\subseteq\RR^n$ be an open subset containing the origin, $\vf{f}\in\mathcal{C}^r(E)$, $r\geq 1$ and consider the system of \cref{DS_general}. Suppose that $\vf{f}(\vf{0})=\vf{0}$ and that $\vf{Df}(0)$ has $n_+$ eigenvalues with positive real part, $n_-$ eigenvalues with negative real part and $n_0=n-n_+-n_-$ eigenvalues with zero real part. Then, there exists a $n_0$-dimensional differentiable manifold $W^\text{c}$ of class $\mathcal{C}^r$ tangent to the center eigenspace $E^\text{c}$ of the linear system at $\vf{0}$; there exists a $n_+$-dimensional differentiable manifold $S$ tangent to the stable eigenspace $E^\text{s}$ of the linear system at $\vf{0}$, and there exists a $n_-$-dimensional differentiable manifold $U$ tangent to the stable eigenspace $E^\text{u}$ of the linear system at $\vf{0}$. Furthermore, $W^\text{c}$, $W^\text{s}$ and $W^\text{u}$ are invariant under the flow $\vf\phi_t$.
  \end{theorem}
  \subsection{Global dynamics in continuous systems}
  \subsection{Global dynamics in discrete systems}
\end{multicols}
\end{document}