\documentclass[../../../main_math.tex]{subfiles}


\begin{document}
\changecolor{HA}
\begin{multicols}{2}[\section{Harmonic analysis}]
  \subsection{Introduction}
  Refer to \mnameref{MA:fouriersection} for a reminder of the introductory concepts of Fourier series.
  \subsubsection{Uniform convergence}
  \begin{theorem}
    Let $f$ be a continuous $T$-periodic function such that $f'$ exists except for a finite number of points and it is continuous and bounded. Then, $Sf$ converges uniformly to $f$ on $[-T/2,T/2]$.
  \end{theorem}
  \begin{proof}
    We have pointwise convergence towards $f$. Moreover:
    \begin{align*}
      \sum_{n\in\ZZ}\abs{\widehat{f}(n)} & \leq\abs{\widehat{f}(0)}+ \sum_{n\in\ZZ\setminus\{0\}}\frac{1}{n}n\abs{\widehat{f}(n)}                                             \\
                                         & \leq\abs{\widehat{f}(0)}+\frac{1}{2}\sum_{n\in\ZZ\setminus\{0\}}\left(\frac{1}{n^2} + n^2\abs{\widehat{f}(n)}^2\right)             \\
                                         & =\abs{\widehat{f}(0)}+\frac{1}{2}\sum_{n\in\ZZ\setminus\{0\}}\frac{1}{n^2}+\frac{T^2}{8\pi^2}\sum_{n\in\ZZ}\abs{\widehat{f'}(n)}^2 \\
                                         & \leq\abs{\widehat{f}(0)}+\frac{1}{2}\sum_{n\in\ZZ\setminus\{0\}}\frac{1}{n^2}+\frac{T}{8\pi^2}\norm{f'}^2                          \\
                                         & <\infty
    \end{align*}
    by \mnameref{MA:bessel} and because $f'$ is bounded. Thus, the \mnameref{MA:Mweierstrass} implies that $Sf$ converges uniformly to $f$.
  \end{proof}
  \begin{corollary}
    Let $f\in\mathcal{C}^{r-1}$ be a $T$-periodic function such that $f^{(r)}$ exists except for a finite number of points and it is continuous and bounded. Then: $$\sup_{x\in[-T/2,T/2]}\abs{S_Nf(x)-f(x)}\leq \frac{\varepsilon_N}{N^{r-1/2}}$$ for some sequence $(\varepsilon_N)\overset{N\to\infty}{\longrightarrow}0$.
  \end{corollary}
  \begin{proof}
    By \mref{RFA:cauchyschwarz} we have:
    \begin{align*}
      \abs{S_Nf(x)-f(x)} & \leq\sum_{n>\abs{N}}\frac{1}{n^r}n^r\abs{\widehat{f}(n)}                                                                                         \\
                         & \leq{\left(\sum_{n>\abs{N}}\frac{1}{n^{2r}}\right)}^{\frac{1}{2}}{\left(\sum_{n>\abs{N}}n^{2r}\abs{\widehat{f}(n)}^2 \right)}^{\frac{1}{2}}      \\
                         & \lesssim{\left(\int_N^\infty\frac{1}{x^{2r}}\dd{x}\right)}^{\frac{1}{2}}{\left(\sum_{n>\abs{N}}\abs{\widehat{f^{r}}(n)}^2 \right)}^{\frac{1}{2}} \\
                         & =\frac{\tilde{C}}{N^{r-1/2}}\varepsilon_N
    \end{align*}
    with $\varepsilon_N\overset{N\to\infty}{\longrightarrow}0$ because it is the tail of a convergent sequence.
  \end{proof}
  \subsubsection{Poisson kernel}
  For most of the proofs in this section check the analogous ones with the \mnameref{MA:fejerdef}.
  \begin{definition}[Poisson kernel]
    Let $r\in[0,1]$. We define the \emph{Poisson kernel} as $$P_r(t)=\sum_{n\in\ZZ}r^{\abs{n}}\exp{\frac{2\pi\ii n t}{T}}$$
  \end{definition}
  \begin{lemma}\label{HA:poisskernelchar}
    Let $r\in[0,1]$. Then:
    $$P_r(t)=\frac{1-r^2}{1-2r\cos\left(\frac{2\pi t}{T}\right)+r^2}$$
  \end{lemma}
  \begin{sproof}
    Use the geometric progression formula.
  \end{sproof}
  \begin{proposition}\label{HA:poissprop}
    The Poisson kernel has the following properties:
    \begin{enumerate}
      \item $P_r$ is a $T$-periodic, even and non-negative function.
      \item $\displaystyle\frac{1}{T}\int_{-T/2}^{T/2}P_r(t)\dd{t}=1\quad\forall N$.
      \item $\forall\delta>0$, $\displaystyle\lim_{r\to 1^-}\sup\{\abs{P_r(t)}:\delta\leq\abs{t}\leq T/2\}=0$.
    \end{enumerate}
  \end{proposition}
  \begin{theorem}
    Let $f\in L^1([-T/2,T/2])$ be a function having left- and right-sided limits at point $x_0$. Then: $$\lim_{r\to 1^-}f*P_r=\frac{f({x_0}^+)+f({x_0}^-)}{2}$$ In particular, if $f$ is continuous at $x_0$, $\displaystyle\lim_{r\to 1^-}f*P_r=f(x_0)$.
  \end{theorem}
  \begin{theorem}
    Let $p\geq 1$ and $f\in L^p([-T/2,T/2])$. Then:
    \begin{gather*}
      \lim_{N\to\infty}\norm{\sigma_Nf-f}_p=0\\
      \lim_{r\to 1^-}\norm{f*P_r-f}_p=0
    \end{gather*}
  \end{theorem}
  \subsection{Fourier transform}
  \subsubsection{Definition and first properties}
  \begin{definition}
    Let $f\in L^1(\RR)$. We define the \emph{Fourier transform} of $f$ as:
    $$\widehat{f}(\xi)=\int_{-\infty}^{+\infty}f(x)\exp{-2\pi \ii\xi x}\dd{x}$$
    The function $f$ is also called \emph{inverse Fourier transform} of $\widehat{f}$.
  \end{definition}
  \begin{proposition}\label{HA:fourierTransProperties}
    Let $f,g\in L^1(\RR)$ and $\alpha,\beta\in\RR$. Then:
    \begin{enumerate}
      \item $\widehat{(\alpha f+\beta g)}(\xi)=\alpha\widehat{f}(\xi)+\beta \widehat{g}(\xi)$
            \item\label{HA:FTprop2} Let $h\in\RR$. We define $T_hf(x)=f(x+h)$. Then: $$\widehat{T_hf}(\xi)=\exp{2\pi\ii \xi h}\widehat{f}(\xi)$$
            \item\label{HA:FTprop3} If $g(x)=\exp{2\pi\ii x h}f(x)$, then: $$\widehat{g}(\xi)=\widehat{f}(\xi-h)$$
            \item\label{HA:FTprop4} If $\lambda\in\RR^*$, then: $$\frac{1}{\lambda}\widehat{f\left(\frac{x}{\lambda}\right)}(\xi)=\widehat{f}(\lambda\xi)$$
            \item\label{HA:FTprop5} If $g(x)=\overline{f(x)}$, then: $$\widehat{g}(\xi)=\overline{\widehat{f}(-\xi)}$$
    \end{enumerate}
  \end{proposition}
  \begin{sproof}
    They follow from the linearity of the integral and some change of variable.
  \end{sproof}
  \begin{definition}
    Let $f\in L^1(\RR)$. We define the \emph{Fourier transform operator} as $\F f=\widehat{f}$.
  \end{definition}
  \begin{theorem}[Riemann-Lebesgue lemma]\label{HA:riemannlebesgue}
    Let $f\in L^1(\RR)$. Then:
    \begin{enumerate}
      \item $\F f$ is uniformly continuous.
      \item $\F$ is a continuous linear operator from $L^1(\RR)$ to $L^\infty(\RR)$ and $\norm{\F f}_{\infty}\leq \norm{f}_1$.
      \item $\displaystyle\lim_{\abs{\xi}\to\infty} \abs{\widehat{f}(\xi)}=0$
    \end{enumerate}
  \end{theorem}
  \begin{sproof}
    \begin{enumerate}
      \item Using \mcref{HA:FTprop3} we have:
            $$\abs{\F f(\xi+h)-\F f(\xi)}\leq \int_{-\infty}^{+\infty}\abs{\exp{-2\pi\ii x h}-1}\abs{f(x)}\dd{x}$$
            By the \mnameref{RFA:dominated} we have that the integral is bounded by $2\norm{f}_1$ and so entering the limit we obtain the bound $\varepsilon \norm{f}_1$ $\forall \varepsilon>0$. As the bound does not depend on the point $\xi$, the convergence is uniform.
      \item Clearly $\norm{\F f}_{\infty}\leq \norm{f}_1$. Hence the operator is bounded and therefore continuous.
      \item Note that $2\abs{\widehat{f}(\xi)}=\abs{\widehat{f}(\xi)-\exp{\ii\pi}\widehat{f}(\xi)}$ and:
            \begin{align*}
              \exp{\ii\pi}\widehat{f}(\xi) & =\int_{-\infty}^{+\infty}f(x)\exp{-2\pi\ii\xi x+\ii\pi}\dd{x}                    \\
                                           & =\int_{-\infty}^{+\infty}f\left(u+\frac{1}{2\xi}\right)\exp{-2\pi\ii\xi u}\dd{u}
            \end{align*}
            So: $$\abs{\widehat{f}(\xi)}\leq\frac{1}{2}\int_{-\infty}^{+\infty}\left[f(x)-f\left(x+\frac{1}{2\xi}\right)\right]\exp{-2\pi\ii\xi x}\dd{x}$$
            Now use again the \mnameref{RFA:dominated}.
    \end{enumerate}
  \end{sproof}
  \begin{proposition}\label{HA:symmetryFT}
    Let $f,g\in L^1(\RR)$. Then, $f\widehat{g},\widehat{f}g\in L^1(\RR)$ and:
    $$\int_{-\infty}^{+\infty}\widehat{f}(x)g(x)\dd{x}=\int_{-\infty}^{+\infty}f(x)\widehat{g}(x)\dd{x}$$
  \end{proposition}
  \begin{sproof}
    By \mnameref{HA:riemannlebesgue}, $\widehat{g}$ is bounded. Hence, $f\widehat{g}\in L^1(\RR)$ and the same applies for $\widehat{f}g$. For the equality, use \mnameref{FSV:fubini}.
  \end{sproof}
  \begin{proposition}\label{HA:diffFourierXf}
    Let $f$ be a function such that $x^k f\in L^1(\RR)$ for $k=0,\ldots,r$. Then, $\widehat{f}$ is $r$ times differentiable and:
    $${(\F f)}^{(k)}=\F({(-2\pi\ii x)}^kf(x))$$
    for $k=0,1,\ldots, r$.
  \end{proposition}
  \begin{proof}
    Note that the function $h:\xi\rightarrow\exp{-2\pi\ii \xi x}f(x)$ is $\mathcal{C}^\infty(\RR)$ and $h^{(k)}(\xi)={(-2\pi\ii x)}^k\exp{-2\pi\ii \xi x}f(x)$. Since $\abs{h^{(k)}(\xi)}\leq \abs{x^kf(x)}$ we can use \mcref{RFA:diffUnderIntegralSign} to conclude the result.
  \end{proof}
  \begin{proposition}\label{HA:diffFourierTransf}
    Let $f\in L^1(\RR)$ be such that $f^{(k)}\in L^1(\RR)$ for $k=1,\ldots,r$. Then: $$\widehat{f^{(k)}}(\xi)={(2\pi\ii\xi)}^k\widehat{f}(\xi)$$ for $k=0,1,\ldots, r$.
  \end{proposition}
  \begin{proof}
    We'll prove it by induction on $k$. The case $k=0$ is clear.
    For the other ones note that $\exists(a_n),(b_n)\in\RR$ with $\displaystyle\lim_{n\to\infty}a_n=-\infty$ and $\displaystyle\lim_{n\to\infty}b_n=+\infty$ and such that:
    $$\lim_{n\to\infty}f^{(k-1)}(a_n)=\lim_{n\to\infty}f^{(k-1)}(b_n)=0$$
    Hence using integration by parts:
    \begin{align*}
      \widehat{f^{(k)}}(n) & =\lim_{n\to\infty}\int_{a_n}^{b_n} f^{(k)}(x)\exp{-2\pi\ii \xi x}\dd{x}     \\
      \begin{split}
        & =\lim_{n\to\infty}  f^{(k-1)}(x)\exp{-2\pi\ii \xi x}\Big|_{a_n}^{b_n}\dd{x} +\\
        &\hspace{2cm}+2\pi\ii \xi\lim_{n\to\infty}\int_{a_n}^{b_n}f^{(k-1)}(x)\exp{-2\pi\ii \xi x}\dd{x}
      \end{split}    \\
                           & =2\pi\ii \xi\int_{-\infty}^{+\infty} f^{(k-1)}(x)\exp{-2\pi\ii \xi x}\dd{x} \\
                           & =\left(2\pi\ii \xi\right)\widehat{f^{(k-1)}}(n)                             \\
                           & ={\left(2\pi\ii \xi\right)}^k\widehat{f}(\xi)
    \end{align*}
  \end{proof}
  \begin{remark}
    Note that there exists functions $f\in\mathcal{C}(\RR)\cap L^1(\RR)$ for which the limit $\displaystyle\lim_{x\to\infty} f(x)$ does not exist.
  \end{remark}
  \begin{proposition}
    Let $f\in L^1(\RR)$ be such that it has compact support. Then, $\F f\in\mathcal{C}^\omega(\RR)$.
  \end{proposition}
  \begin{sproof}
    Suppose $f(x)\in[-K,K]$, $K>0$. Then, expanding $\F f$ with the power series of $\exp{-2\pi\ii\xi x}$ centered at $a\in\RR$ we have:
    \begin{align*}
      \F f(\xi) & =\int_{-K}^Kf(x)\sum_{n=0}^{\infty}\frac{{(-2\pi\ii x)}^n\exp{-2\pi\ii \xi x}}{n!}{(\xi-a)}^n\dd{x} \\
                & =\sum_{n=0}^\infty c_n{(\xi-a)}^n
    \end{align*}
    where $\abs{c_n}\leq\frac{{(2\pi K)}^n}{n!}\norm{f}_1$. Finally, use this to show that the radius of convergences (see \mcref{MA:radius}) is $\infty$.
  \end{sproof}
  \begin{lemma}\label{HA:expX2}
    Let $f(x)=\exp{-a x^2}$. Then, $\F f(\xi)=\sqrt{\frac{\pi}{a}}\exp{-\frac{{(\pi x)}^2}{a}}$ and moreover $\F^2f=f$. In particular if $a=\pi$, then $\F f=f$, that is $\widehat{f}(\xi)=f(\xi)=\exp{-\pi \xi^2}$.
  \end{lemma}
  \begin{sproof}
    $f$ satisfies the ode $y'=-2a x y$. Taking $\ \widehat{}\ $ on this expression and using \mcref{HA:diffFourierXf,HA:diffFourierTransf} we obtain that $\widehat{f}$ must satisfy the following ode:
    $$y'=-\frac{2\pi^2\xi}{a} y$$
    with initial condition $y(0)=\int_{-\infty}^{+\infty}\exp{-a x^2}\dd{x}=\sqrt{\frac{\pi}{a}}$.
  \end{sproof}
  \begin{lemma}\label{HA:expAbsX}
    Let $f(x)=\exp{-a\abs{x}}$. Then, $\F f(\xi)=\frac{2a}{a^2+4\pi^2\xi^2}$ and moreover $\F^2f=f$.
  \end{lemma}
  \begin{sproof}
    $$\F f(\xi)=2\int_0^{+\infty}\exp{-ax}\cos(2\pi\xi x)\dd{x}=\frac{2a}{a^2+4\pi^2\xi^2}$$
  \end{sproof}
  \subsubsection{The inverse Fourier transform}
  \begin{theorem}[Inversion theorem]\label{HA:inverseFT}
    Let $f\in L^1(\RR)$ such that $\F f\in L^1(\RR)$. Then:
    $$f(x)\almoste{=}\int_{-\infty}^{+\infty}\widehat{f}(\xi)\exp{2\pi \ii \xi x}\dd{\xi}$$
    Moreover if $f$ is continuous we can remove the ``almost everywhere''.
  \end{theorem}
  \begin{proof}
    Consider the integral: $$I=\int_{-\infty}^{+\infty}f(x+y)\frac{1}{t}\exp{-\pi \frac{y^2}{t^2}}\dd{y}$$
    Note that using \mcref{HA:expX2} and \mcref{HA:FTprop4}, we have that $\F \left(\frac{1}{\lambda}\exp{-\pi \frac{x^2}{\lambda^2}}\right)=\exp{-\pi\lambda^2\xi^2}$. On the one hand and using this latter thing, \mcref{HA:symmetryFT} we have:
    \begin{multline*}
      I=\int_{-\infty}^{+\infty}f(x+y)\frac{1}{t}\exp{-\pi \frac{y^2}{t^2}}\dd{y}=\int_{-\infty}^{+\infty}f(x+\xi)\widehat{\exp{-\pi t^2\xi^2}}\dd{\xi}=\\
      =\int_{-\infty}^{+\infty}\exp{2\pi\ii\xi x}\widehat{f}(\xi)\exp{-\pi t^2\xi^2}\dd{\xi}
    \end{multline*}
    which by \mnameref{RFA:dominated} converges to $\int_{-\infty}^{+\infty}\widehat{f}(\xi)\exp{2\pi \ii \xi x}\dd{\xi}$ as $t\to 0$.

    On the other hand with a change of variable we have: $$I=\int_{-\infty}^{+\infty}f(x+ty)\exp{-\pi y^2}\dd{y}$$ Using \mcref{RFA:thmLpBanachC} it suffices to prove that $\displaystyle\lim_{t\to 0}\norm{I-f(x)}_1=0$. But using that $\int_{-\infty}^{+\infty}\exp{-\pi y^2}\dd{y}=1$:
    \begin{align*}
      \norm{I-f(x)}_1 & =\int_{-\infty}^{+\infty}\abs{\int_{-\infty}^{+\infty}(f(x+ty)-f(x))\exp{-\pi y^2}\dd{y}}\dd{x}  \\
                      & \leq\int_{-\infty}^{+\infty}\exp{-\pi y^2}\int_{-\infty}^{+\infty}\abs{f(x+ty)-f(x)}\dd{x}\dd{y}
    \end{align*}
    where we have used \mnameref{RFA:fubini}. Now use the \mnameref{RFA:dominated}.
  \end{proof}
  \begin{corollary}
    Let $f\in L^1(\RR)$ such that $\F f\almoste{=}0$. Then, $f\almoste{=}0$.
  \end{corollary}
  \begin{lemma}\label{HA:periodicity}
    Let $f\in L^1(\RR)$. Then, $\F^2f(x)\almoste{=}f(-x)$. Hence, $\F^4\almoste{=}\id$.
  \end{lemma}
  \begin{proof}
    By the \mnameref{HA:inverseFT} we have:
    $$f(-x)\almoste{=}\int_{-\infty}^{+\infty}\widehat{f}(\xi)\exp{-2\pi\ii\xi x}\dd{\xi}=\F\widehat{f}(x)=\F^2 f(x)$$
  \end{proof}
  \begin{lemma}
    Let $f,g\in L^1(\RR)$. Then, $f*g\in L^1(\RR)$, $\norm{f*g}_1\leq\norm{f}_1\norm{g}_1$ and $\F(f*g)=\F f \F g$.
  \end{lemma}
  \begin{sproof}
    Use \mnameref{RFA:fubini}.
  \end{sproof}
  \subsubsection{Pointwise convergence}
  \begin{definition}
    Let $f\in L^1(\RR)$. We define the \emph{partial Fourier transform} as: $$S_Rf(x)=\int_{-R}^{R}\widehat{f}(\xi)\exp{2\pi\ii \xi x}\dd{\xi}$$
  \end{definition}
  \begin{definition}[Dirichlet kernel]
    We define the \emph{Dirichlet kernel} of order $R\in\RR_{>0}$ as: $$D_R(t)=\int_{-R}^{R}\exp{-2\pi\ii \xi t}\dd{\xi}=\frac{\sin(2\pi Rt)}{\pi t}$$
  \end{definition}
  \begin{proposition}
    The Dirichlet kernel has the following properties:
    \begin{enumerate}
      \item $D_R$ is an even function.
      \item \begin{align*}
              S_Rf(x) & =(f*D_R)(x)                                  \\
                      & =\int_{-\infty}^{+\infty}f(x-t)D_R(t)\dd{t}  \\
                      & =\int_0^{+\infty}[f(x+t)+f(x-t)]D_R(t)\dd{t}
            \end{align*}
    \end{enumerate}
  \end{proposition}
  \begin{theorem}[Dini's theorem]\label{HA:dini}
    Let $f\in L^1(\RR)$ and $x,\ell\in \RR$ such that $h(t):=\frac{\abs{f(x+t)+f(x-t)-2\ell}}{t}\in L^1((0,\delta))$ for some $\delta>0$. Then, $\displaystyle\lim_{R\to\infty}S_Rf(x)=\ell$.
  \end{theorem}
  \begin{sproof}
    Note that $$S_Rf(x)-\ell=\int_0^\infty[f(x+t)+f(x-t)-2\ell]D_R(t)\dd{t}$$ Now split this integral as a sum of the following ones:
    \begin{align*}
      I_1 & =\int_0^N[f(x+t)+f(x-t)-2\ell]D_R(t)\dd{t} \\
      I_2 & =\int_N^\infty[f(x+t)+f(x-t)]D_R(t)\dd{t}  \\
      I_3 & =-2\ell \int_N^\infty D_R(t)\dd{t}
    \end{align*}
    Given $\varepsilon>0$ take $N$ such that $\int_N^\infty\abs{\frac{f(x+t)+f(x-t)}{\pi t}}\dd{t}<\varepsilon$. Since $h$ is integrable in $(0,N)$, by \mnameref{HA:riemannlebesgue} we have that $I_1\overset{R\to\infty}{\longrightarrow}0$. Then, as we can write $I_3=-2\ell \int_{2\pi RN}^\infty \frac{\sin(u)}{\pi u}\dd{u}$ we have that $I_3\overset{R\to\infty}{\longrightarrow}0$.
  \end{sproof}
  \begin{lemma}\label{HA:translated}
    Let $f\in L^p(\RR)$ with $1\leq p<\infty$. Then, $\displaystyle \lim_{a\to 0}\norm{f-T_af}_p=0$.
  \end{lemma}
  \begin{sproof}
    Clearly is is true if $f\in\mathcal{C}_0^\infty(\RR)$ using \mnameref{RFA:dominated}. Now use that since $\mathcal{C}_0^\infty(\RR)$ is dense in $\mathcal{C}_0(\RR)$, which is dense in $L^p(\RR)$, $\exists(f_n)\in \mathcal{C}_0^\infty(\RR)$ such that $\displaystyle\lim_{n\to\infty}\norm{f_n-f}_p=0$.
  \end{sproof}
  \subsubsection{Uniform convergence}
  \begin{definition}
    Let $f\in L^1(\RR)$ and $R>0$. We define the \emph{Fejér mean} $\sigma_Rf(x)$ as: $$\sigma_Rf(x)=\frac{1}{R}\int_{0}^RS_rf(x)\dd{r}$$
  \end{definition}
  \begin{definition}
    Let $f\in L^1(\RR)$ and $R>0$. We define the \emph{Fejér kernel} $F_Rf(x)$ as: $$F_R(x)=\frac{1}{R}\int_{0}^RD_r(x)\dd{r}$$
  \end{definition}
  \begin{lemma}
    Let $f\in L^1(\RR)$ and $R>0$. Then, $\sigma_Rf=f*F_R$ and moreover:
    $$F_R(x)=\frac{{\left(\sin\left(\pi R x\right)\right)}^2}{\pi^2 R x^2}$$
  \end{lemma}
  \begin{definition}
    Let $t>0$. We define the \emph{Poisson kernel} $P_t$ as $P_t(x):=\F^{-1}(\exp{-2\pi t\abs{\xi}})$.
  \end{definition}
  \begin{lemma}
    Let $f\in L^1(\RR)$ and $t>0$. Then:
    \begin{align*}
      P_t(x)     & =\frac{t^2}{\pi(t^2+x^2)}                                                                 \\
      (f*P_t)(x) & =\int_{-\infty}^{+\infty}\exp{-2\pi t\abs{\xi}}\widehat{f}(\xi)\exp{2\pi\ii\xi x}\dd{\xi}
    \end{align*}
  \end{lemma}
  \begin{proof}
    Check \mcref{HA:expAbsX} for the first equality. For the other one:
    \begin{align*}
      (f*P_t)(x) & =\int_{-\infty}^{+\infty}f(y)P_t(x-y)\dd{y}                                                                      \\
                 & =\int_{-\infty}^{+\infty}\int_{-\infty}^{+\infty}f(y)\exp{-2\pi t\abs{\xi}}\exp{2\pi\ii \xi (x-y)}\dd{\xi}\dd{y} \\
                 & =\int_{-\infty}^{+\infty}\exp{-2\pi t\abs{\xi}}\widehat{f}(\xi)\exp{2\pi\ii\xi x}\dd{\xi}
    \end{align*}
  \end{proof}
  \begin{definition}
    Let $t>0$. We define the \emph{Weierstra\ss\ kernel} $W_t$ as $W_t(x):=\F^{-1}(\exp{-4\pi^2 t\xi^2})$.
  \end{definition}
  \begin{lemma}\label{HA:weierstrassKernel2}
    Let $f\in L^1(\RR)$ and $t>0$. Then:
    \begin{align*}
      W_t(x)     & =\frac{1}{\sqrt{4\pi t}}\exp{-\frac{x^2}{4t}}                                           \\
      (f*W_t)(x) & =\int_{-\infty}^{+\infty}\exp{-4\pi^2 t\xi^2}\widehat{f}(\xi)\exp{2\pi\ii\xi x}\dd{\xi}
    \end{align*}
  \end{lemma}
  \begin{proof}
    Check \mcref{HA:expX2} for the first equality. For the other one:
    \begin{align*}
      (f*W_t)(x) & =\int_{-\infty}^{+\infty}f(y)W_t(x-y)\dd{y}                                                                    \\
                 & =\int_{-\infty}^{+\infty}\int_{-\infty}^{+\infty}f(y)\exp{-4\pi^2 t\xi^2}\exp{2\pi\ii \xi (x-y)}\dd{\xi}\dd{y} \\
                 & =\int_{-\infty}^{+\infty}\exp{-4\pi^2 t\xi^2}\widehat{f}(\xi)\exp{2\pi\ii\xi x}\dd{\xi}
    \end{align*}
  \end{proof}
  \begin{proposition}
    Let $R>0$ and $t>0$. Then:
    \begin{enumerate}
      \item $F_R$, $P_t$ and $W_t$ are non-negative even functions.
      \item $\int_{-\infty}^{+\infty}F_R(x)\dd{x}=\int_{-\infty}^{+\infty}P_t(x)\dd{x}=\int_{-\infty}^{+\infty}W_t(x)\dd{x}=1$
      \item For all $\delta>0$, we have:
            \begin{multline*}
              \lim_{R\to \infty}\sup_{\abs{x}\geq \delta}F_R(x)=\lim_{t\to 0}\sup_{\abs{x}\geq \delta}P_t(x)=\\=\lim_{t\to 0}\sup_{\abs{x}\geq \delta}W_t(x)=0
            \end{multline*}
            \item\label{HA:propsKernelsitem4} For all $\delta>0$, we have:
            \begin{multline*}
              \lim_{R\to \infty}\int_{\abs{x}\geq \delta}F_R(x)\dd{x}=\lim_{t\to 0}\int_{\abs{x}\geq \delta}P_t(x)\dd{x}=\\=\lim_{t\to 0}\int_{\abs{x}\geq \delta}W_t(x)\dd{x}=0
            \end{multline*}
    \end{enumerate}
  \end{proposition}
  \begin{sproof}
    The first two properties are straightforward. For the third one, note that:
    \begin{align*}
      \sup_{\abs{x}\geq \delta}F_R(x) & \leq \frac{1}{\pi^2 R \delta^2}                        \\
      \sup_{\abs{x}\geq \delta}P_t(x) & =\frac{t^2}{\pi(t^2+\delta^2)}                         \\
      \sup_{\abs{x}\geq \delta}W_t(x) & =\frac{1}{\sqrt{4\pi\delta}}\exp{-\frac{x^2}{4\delta}}
    \end{align*}

    The last one is a consequence of the previous ones.
  \end{sproof}
  \begin{theorem}
    Let $f\in L^1(\RR)$ be a function having left- and right-sided limits at point $x_0$. Then:
    \begin{multline*}
      \lim_{R\to\infty}\sigma_Rf(x_0)=\lim_{t\to 0}(f*P_t)(x_0)=\lim_{t\to 0}(f*W_t)(x_0)=\\=\frac{f({x_0}^+)+f({x_0}^-)}{2}
    \end{multline*}
    Moreover if $f$ is uniformly continuous, the convergence is uniform.
  \end{theorem}
  \begin{sproof}
    Copy the proofs of \mnameref{MA:fejerthm0} and \mnameref{MA:fejerthm}.
  \end{sproof}
  \begin{lemma}\label{HA:normLpEquiv}
    Let $E\subseteq\RR^n$ be a measurable space, $p\geq 1$, $f\in L^p(E)$ and $q$ be such that $\frac{1}{p}+\frac{1}{q}=1$. Then:
    $$\norm{f}_p=\sup\left\{\int_Efg:\norm{g}_q=1\right\}$$
  \end{lemma}
  \begin{proof}
    On the one hand using \mnameref{RFA:holder}: $$\int_Efg\leq\norm{fg}_1\leq\norm{f}_p\norm{g}_q=\norm{f}_p$$
    Now consider $g=\frac{f^{p-1}}{{\norm{f}_p}^{\frac{p}{q}}}$. Then, $\norm{g}_q=1$ and moreover: $$\int_Efg=\int_E\frac{f^p}{{\norm{f}_p}^{\frac{p}{q}}}={\norm{f}_p}^{p-\frac{p}{q}}=\norm{f}_p$$
  \end{proof}
  \begin{lemma}[Minkowski's integral inequality]\label{HA:minkowski}
    Let $E,F\subseteq\RR^n$ be measurable spaces, $p\geq 1$ and $f\in L^p(E\times F)$. Then:
    $$\norm{\int_F h(x,y)\dd{y}}_p\leq\int_F\norm{h(\cdot,y)}_p\dd{y}$$
  \end{lemma}
  \begin{proof}
    Let $q$ be such that $\frac{1}{p}+\frac{1}{q}=1$ and $g\in L^q(E)$ with $\norm{g}_q=1$. Then, using \mnameref{RFA:holder} and \mnameref{RFA:fubini}:
    \begin{align*}
      \int_Eg(x)\int_F h(x,y)\dd{y}\dd{x} & =\int_F\int_E h(x,y)g(x)\dd{x}\dd{y}           \\
                                          & \leq \int_F\norm{h(\cdot,y)}_p\norm{g}_q\dd{y} \\
                                          & =\int_F\norm{h(\cdot,y)}_p\dd{y}
    \end{align*}
    Now use \mcref{HA:normLpEquiv}.
  \end{proof}
  \begin{theorem}\label{HA:kernelConvLp}
    Let $f\in L^p(\RR)$, $1\leq p\leq\infty$, and $\phi_\varepsilon$ be an approximation of unity. Then:
    \begin{equation*}
      \lim_{\varepsilon\to 0}\norm{f*\phi_\varepsilon-f}_p =0 \\
    \end{equation*}
  \end{theorem}
  \begin{sproof}
    Using \mnameref{HA:minkowski}, we have:
    \begin{align*}
      \norm{f*\phi_\varepsilon-f}_p & =\norm{\int_{-\infty}^{\infty}\phi_\varepsilon(y)(f(x-y)-f(x))\dd{y}}_p                                                                       \\
                                    & \leq\int_{-\infty}^{\infty}\!\!{\left[\int_{-\infty}^{\infty}{\phi_\varepsilon(y)}^p\abs{f(x-y)-f(x)}^p\dd{x}\right]}^{\frac{1}{p}}\!\!\dd{y} \\
                                    & = \int_{-\infty}^{\infty}\phi_\varepsilon(y)\norm{f-T_{-y}f}_p\dd{y}                                                                          \\
      \begin{split}
        & \leq \int_{\abs{y}< \delta}\phi_\varepsilon(y)\norm{f-T_{-y}f}_p\dd{y}+\\
        &\hspace{3cm}+2\norm{f}_p\int_{\abs{y}\geq\delta}\phi_\varepsilon(y)\dd{y}
      \end{split}
    \end{align*}
    Given $\varepsilon>0$, by \mcref{HA:translated} $\exists\delta>0$ such that the first integral is bounded by $\varepsilon$. Now use this $\delta$ and \mcref{HA:propsKernelsitem4} to conclude that the second integral goes to 0 as $R\to\infty$.
  \end{sproof}
  \begin{corollary}
    Let $f\in L^p(\RR)$ with $1\leq p\leq\infty$. Then:
    \begin{align*}
      \lim_{R\to\infty}\norm{\sigma_Rf-f}_p & =0 \\
      \lim_{t\to 0}\norm{f*P_t-f}_p         & =0 \\
      \lim_{t\to 0}\norm{f*W_t-f}_p         & =0
    \end{align*}
  \end{corollary}
  \subsubsection{Fourier transform on \texorpdfstring{$\RR^n$}{Rn}}
  In this section we will only expose the most important results of extending the Fourier transform to $L^1(\RR^n)$. Moreover we will not prove any of the results of this section as they are completely analogous to the previous ones.
  \begin{definition}
    Let $f\in L^1(\RR^n)$. We define the \emph{Fourier transform} of $f$ as:
    $$\widehat{f}(\vf\xi)=\int_{\RR^n}f(\vf{x})\exp{-2\pi \ii\langle \vf\xi, \vf{x}\rangle}\dd{\vf{x}}$$
    The function $f$ is also called \emph{inverse Fourier transform} of $\widehat{f}$.
  \end{definition}
  \begin{proposition}
    Let $f,g\in L^1(\RR^n)$ and $\alpha,\beta\in\RR$. Then:
    \begin{enumerate}
      \item $\widehat{(\alpha f+\beta g)}(\vf\xi)=\alpha\widehat{f}(\vf\xi)+\beta \widehat{g}(\vf\xi)$
      \item Let $\vf{h}\in\RR^n$. We define $T_{\vf{h}}f(x)=f(\vf{x}+\vf{h})$. Then: $$\widehat{T_{\vf{h}}f}(\vf\xi)=\exp{2\pi\ii \langle\vf\xi, \vf{h}\rangle}\widehat{f}(\vf\xi)$$
      \item If $g(\vf{x})=\exp{2\pi\ii \langle\vf{x}, \vf{h}\rangle}f(\vf{x})$, then: $$\widehat{g}(\vf\xi)=\widehat{f}(\vf\xi-\vf{h})$$
      \item If $\lambda\in\RR^*$, then: $$\frac{1}{\lambda^n}\widehat{f\left(\frac{\vf{x}}{\lambda}\right)}(\vf\xi)=\widehat{f}(\lambda\vf\xi)$$
      \item If $g(\vf{x})=\overline{f(\vf{x})}$, then: $$\widehat{g}(\vf\xi)=\overline{\widehat{f}(-\vf\xi)}$$
    \end{enumerate}
  \end{proposition}
  \begin{theorem}[Riemann-Lebesgue lemma]
    Let $f\in L^1(\RR^n)$ and denote also by $\F$ the extension of the Fourier transform operator to $L^1(\RR^n)$. Then:
    \begin{enumerate}
      \item $\F f$ is uniformly continuous.
      \item $\F$ is a continuous linear operator from $L^1(\RR)$ to $L^\infty(\RR)$ and $\norm{\F f}_{\infty}\leq \norm{f}_1$.
      \item $\displaystyle\lim_{\abs{\vf\xi}\to\infty} \abs{\widehat{f}(\vf\xi)}=0$
    \end{enumerate}
  \end{theorem}
  \begin{proposition}
    Let $f$ be a function such that $\xi_j f\in L^1(\RR^n)$. Then, $\widehat{f}$ is differentiable with respect to $\xi_j$ and:
    $$\pdv{(\F f)}{\xi_j}(\vf\xi)=\F({(-2\pi\ii \xi_j)}f(\vf{x}))$$
  \end{proposition}
  \begin{proposition}
    Let $f\in L^1(\RR^n)$ be differentiable with respect to $x_j$ such that $\pdv{f}{x_j}\in L^1(\RR^n)$. Then: $$\widehat{\pdv{f}{x_j}}(\vf\xi)={2\pi\ii\xi_j}\widehat{f}(\vf\xi)$$
  \end{proposition}
  \subsubsection{Fourier transform on \texorpdfstring{$L^2(\RR)$}{L2(R)}}
  \begin{lemma}\label{HA:lemaPrePlancherel}
    Let $f,g\in L^2(\RR)$. Then, $f*g$ is continuous and bounded. Moreover, $\norm{f*g}_\infty\leq \norm{f}_2\norm{g}_2$.
  \end{lemma}
  \begin{sproof}
    The inequality follows from \mnameref{RFA:holder}. Moreover:
    \begin{multline*}
      \abs{(f*g)(x+h)-(f*g)(x)} \leq\\
      \leq\int_{-\infty}^{+\infty}\abs{f(x+h-y)-f(x-y)}\abs{g(y)}\dd{y}\leq\\
      \leq\norm{g}_2\norm{f-T_{-h}f}_2
    \end{multline*}
    So $f*g$ is continuous, by \mcref{HA:translated}.
  \end{sproof}
  \begin{theorem}[Plancherel theorem]\label{HA:plancherel}
    Let $f\in L^1(\RR)\cap L^2(\RR)$. Then, $\widehat{f}\in L^2(\RR)$ and:
    $$\int_{-\infty}^{+\infty}\abs{f(x)}^2\dd{x}=\int_{-\infty}^{+\infty}\abs{\widehat{f}(\xi)}^2\dd{\xi}$$
  \end{theorem}
  \begin{proof}
    Let $\tilde{f}(x):=\overline{f(-x)}$. Then, $\widehat{\tilde{f}}(\xi)=\overline{\widehat{f}(\xi)}$ and so by \mcref{HA:lemaPrePlancherel} we have that $g:=f*\tilde{f}$ is continuous and bounded. Moreover $\widehat{g}(\xi)=\widehat{f}(\xi)\widehat{\tilde{f}}(\xi)=\abs{\widehat{f}(\xi)}^2$ and $g(0)=\int_{-\infty}^{+\infty}\tilde{f}(-y)f(y)\dd{y}={\norm{f}_2}^2$. On the other hand, by \mcref{HA:weierstrassKernel2} we have:
    \begin{equation}\label{HA:planchEq}
      (g*W_t)(0)=\int_{-\infty}^{+\infty}\exp{-4\pi^2 t\xi^2}\widehat{g}(\xi)\dd{\xi}=\int_{-\infty}^{+\infty}\exp{-4\pi^2 t\xi^2}\abs{\widehat{f}(\xi)}\dd{\xi}
    \end{equation}
    And by \mcref{HA:kernelConvLp}, $\displaystyle\lim_{t\to 0^+}(g*W_t)(x)=g(0)={\norm{f}_2}^2$. Thus, by the definition of limit taking $\varepsilon = {\norm{f}_2}^2$, we have that $\abs{\int_{-\infty}^{+\infty}\exp{-4\pi^2 t\xi^2}\widehat{g}(\xi)\dd{\xi}}\leq 2{\norm{f}_2}^2$ for $t$ small enough. Finally, if $t$ is that small, then $1\leq 2\exp{-4\pi^2 t\xi^2}$ and so:
    $$\int_{-\infty}^{+\infty}\abs{\widehat{f}(\xi)}^2\dd{\xi}\leq 2\int_{-\infty}^{+\infty}\widehat{g}(\xi)\exp{-4\pi^2 t\xi^2}\dd{\xi}\leq 4{\norm{f}_2}^2<\infty$$
    Now use \mnameref{RFA:dominated} in \mcref{HA:planchEq} and make $t\to 0$.
  \end{proof}
  \begin{corollary}
    Let $f,g\in L^1(\RR)\cap L^2(\RR)$. Then:
    $$\int_{-\infty}^{+\infty}f(x)\overline{g(x)}\dd{x}=\int_{-\infty}^{+\infty}\widehat{f}(\xi)\overline{\widehat{g}(\xi)}\dd{\xi}$$
  \end{corollary}
  \begin{proof}
    Use \mnameref{HA:plancherel} and \mnameref{RFA:polarization}.
  \end{proof}
  \begin{proposition}\label{HA:preDefFTinL2}
    Let $f\in L^2(\RR)$. Then, $\exists(f_n)\in L^1(\RR)\cap L^2(\RR)$ such that $\displaystyle\lim_{n\to\infty}\norm{f-f_n}_2=0$.
  \end{proposition}
  \begin{sproof}
    Take the sequence $f_n(x)=f(x)\indi{[-n,n]}(x)$.
  \end{sproof}
  \begin{proposition}
    Let $f\in L^2(\RR)$ and $(f_n)\in L^1(\RR)\cap L^2(\RR)$ such that $\displaystyle\lim_{n\to\infty}\norm{f-f_n}_2=0$. Then, the limit $\displaystyle\lim_{n\to\infty}\widehat{f_n}(\xi)$ exists and we will call it $\widehat{f}(\xi)$.
  \end{proposition}
  \begin{proof}
    Since $L^2(\RR)$ is Hilbert, $(f_n)$ is Cauchy. But by \mnameref{HA:plancherel}, $(\widehat{f_n})$ is also Cauchy and so it has limit, because $(\widehat{f_n})\in L^2(\RR)$.

    To see that the definition is well-defined, suppose $(g_n)\in L^1(\RR)\cap L^2(\RR)$ is another sequence such that $\displaystyle\lim_{n\to\infty}\norm{f-g_n}_2=0$. But in that case:
    $$\norm{g_n-f_n}_2\leq\norm{g_n-f}_2+\norm{f-f_n}_2\overset{N\to\infty}{\longrightarrow}0$$
  \end{proof}
  \begin{remark}
    Note that the abuse of notation in the definition of the limit make sense as it coincides with the ordinary Fourier transform when $f\in L^1(\RR)$ (by taking $f_n=f$ $\forall n\in\NN$).
  \end{remark}
  \begin{theorem}
    Let $f,g\in L^2(\RR)$. Then:
    \begin{enumerate}
      \item $\displaystyle\widehat{f}(\xi)\overset{L^2}{=}\lim_{n\to\infty}\int_{-n}^nf(x)\exp{-2\pi\ii \xi x}\dd{x}$
      \item $\norm{f}_2=\norm{\widehat{f}}_2$
      \item $\displaystyle\int_{-\infty}^{+\infty}f(x)\widehat{g}(x)\dd{x}=\int_{-\infty}^{+\infty}\widehat{f}(x)g(x)\dd{x}$
      \item $\displaystyle\int_{-\infty}^{+\infty}f(x)\overline{g(x)}
              \dd{x}=\int_{-\infty}^{+\infty}\widehat{f}(x)\overline{\widehat{g}(x)}\dd{x}$
    \end{enumerate}
  \end{theorem}
  \begin{proof}
    The first property follows from its definition. For the second one, if $\displaystyle f(x)\overset{L^2}{=}\lim_{n\to\infty}f_n(x)$, by \mnameref{HA:plancherel} we have $\norm{f_n}_2=\norm{\widehat{f}}_2$. Now use the continuity of the norm. For the other properties, take the function given in the proof of \mcref{HA:preDefFTinL2} and use the \mnameref{RFA:monotone}.
  \end{proof}
  \begin{lemma}[Generalized Hölder's inequality]\label{HA:holderGeneralized}
    Let $E\subseteq\RR^n$ be a measurable set, $1\leq p_1,\ldots,p_n\leq \infty$ be such that $\sum_{i=1}^n\frac{1}{p_i}=1$ and ${f_i}\in L^{p_i}(E)$. Then:
    $$\norm{\prod_{i=1}^{n}f_i}_r\leq\prod_{i=1}^{n}\norm{{f_i}}_{p_i}$$
  \end{lemma}
  \begin{proof}
    We will prove it by induction on $n$. For $n=1$ the result is clear. For $n\geq 2$, note that the numbers $q_n=\frac{p_n}{p_n-1}$ and $p_n$ are Hölder conjugates. Moreover, if we define $r_i=p_i\left(1-\frac{1}{p_n}\right)$ we have that $\sum_{i=1}^{n-1}\frac{1}{r_i}=1$ and so using \mnameref{RFA:holder} we have:
    \begin{align*}
      \norm{{f_1\cdots f_n}}_1 & \leq \norm{{f_1\cdots f_{n-1}}}_{q_n}\norm{f_n}_{p_n}                                                                    \\
                               & = {\norm{\abs{f_1\cdots f_{n-1}}^{q_n}}_{1}}^{\frac{1}{q_n}}\norm{f_n}_{p_n}                                             \\
                               & \leq {\norm{\abs{f_1}^{q_n}}_{r_1}}^{\frac{1}{q_n}}\cdots {\norm{\abs{f_1}^{q_n}}_{r_1}}^{\frac{1}{q_n}}\norm{f_n}_{p_n} \\
                               & =\norm{f_1}_{p_1}\cdots\norm{f_n}_{p_n}
    \end{align*}
    where in the penultimate step we have used the induction hypothesis and in the last equality we have used the fact that $r_iq_n=p_i$.
  \end{proof}
  \begin{lemma}[Young's convolution inequality]\label{HA:youngConvolution}
    Let $f\in L^p(\RR^n)$, $g\in L^q(\RR^n)$ and take $r$ such that $$\frac{1}{p}+\frac{1}{q}=\frac{1}{r}+1$$ with $1\leq p,q,r\leq \infty$. Then: $$\norm{f*g}_r\leq\norm{f}_p\norm{g}_q$$
  \end{lemma}
  \begin{sproof}
    Note that:
    \begin{multline*}
      \abs{(f*g)(x)} \\
      \leq\int_{\RR^n}{\left(\abs{f(y)}^{p}\abs{g(x-y)}^{q}\right)}^\frac{1}{r}\abs{f(y)}^{1-\frac{p}{r}}\abs{g(x-y)}^{1-\frac{q}{r}}\dd{y}    \\
      \leq \norm{{\left(\abs{f(y)}^{p}\abs{g(x-y)}^{q}\right)}^\frac{1}{r}}_r\norm{\abs{f(y)}^{\frac{r-p}{r}}}_{\frac{pr}{r-p}}\cdot\\\cdot\norm{\abs{g(x-y)}^{\frac{r-q}{r}}}_{\frac{qr}{r-q}}
    \end{multline*}
    where in the second inequality we have used the \mnameref{HA:holderGeneralized} because:
    $$\frac{1}{r}+\frac{r-p}{pr}+\frac{r-q}{qr}=\frac{1}{p}+\frac{1}{q}-\frac{1}{r}=1$$
    Finally:
    \begin{align*}
      {\norm{f*g}_r}^r & \leq {\norm{f}_p}^{r-p}{\norm{g}_q}^{r-q} \int_{\RR^n}\int_{\RR^n}\abs{f(y)}^{p}\abs{g(x-y)}^{q}\dd{y}\dd{x} \\
                       & ={\norm{f}_p}^{r-p}{\norm{g}_q}^{r-q} {\norm{f}_p}^{p}{\norm{g}_q}^{q}                                       \\
                       & ={\norm{f}_p}^{r}{\norm{g}_q}^{r}
    \end{align*}
    where in the second equality we have used the \mnameref{RFA:fubini}.
  \end{sproof}
  \begin{theorem}
    Let $f\in L^2(\RR)$ and $g\in L^1(\RR)$. Then, $\widehat{f*g}\in L^2(\RR)$ and:$$\widehat{f*g}(\xi)=\widehat{f}(\xi)\widehat{g}(\xi)$$
  \end{theorem}
  \begin{proof}
    By \mnameref{HA:youngConvolution} with $p=r=2$ and $q=1$ we have:
    $$\norm{f*g}_2\leq\norm{f}_2\norm{g}_1<\infty$$
    The equality follows in the same way as in $L^1(\RR)$.
  \end{proof}
  \begin{lemma}
    Let $f\in L^p(\RR)$ with $1<p<2$. Then, there exist functions $f_1\in L^1(\RR)$ and $f_2\in L^2(\RR)$ such that $f=f_1+f_2$.
  \end{lemma}
  \begin{proof}
    The set $E:=\{\abs{f}\geq 1\}$ has finite measure because $f\in L^p(\RR)$. Now consider the Functions
    \begin{gather*}
      f_1(x):=\begin{cases}
        f(x) & \text{if }x\in E    \\
        0    & \text{if }x\notin E
      \end{cases}\qquad
      f_2(x):=\begin{cases}
        0    & \text{if }x\in E    \\
        f(x) & \text{if }x\notin E
      \end{cases}
    \end{gather*}
    By \mnameref{RFA:holder}, $f_1\in L^1(\RR)$ because $\m{E}<\infty$. On the other hand: $$\int_{-\infty}^{\infty}\abs{f_2(x)}^2\dd{x}=\int_{\RR\setminus E}\abs{f(x)}^2\dd{x}\leq \int_{\RR\setminus E}\abs{f(x)}^p\dd{x}<\infty$$
    because $\abs{f}<1$ in $\RR\setminus E$.
    So $f_2\in L^2(\RR)$.
  \end{proof}
  \begin{definition}
    Let $f=f_1+f_2\in L^p(\RR)$ with $1<p<2$ with $f_1\in L^1(\RR)$ and $f_2\in L^2(\RR)$. We define the Fourier transform of $f$ as:
    $$\widehat{f}(\xi):=\widehat{f_1}(\xi)+\widehat{f_2}(\xi)$$
  \end{definition}
  \begin{remark}
    This definition is well-defined. Indeed, suppose $f=g_1+g_2$ with $1<p<2$ with $g_1\in L^1(\RR)$ and $g_2\in L^2(\RR)$. Then, $f_1-g_1=g_2-f_2\in L^1(\RR)\cap L^2(\RR)$ and so $$\widehat{f_1}-\widehat{g_1}=\widehat{f_1-g_1}=\widehat{f_2-g_2}=\widehat{f_2}-\widehat{g_2}$$
    Hence, $\widehat{f}=\widehat{f_1}+\widehat{f_2}=\widehat{g_1}+\widehat{g_2}$.
  \end{remark}
  \subsubsection{Applications of the Fourier transform}
  \begin{remark}
    Probably the most important application of Fourier series is the resolution of pdes and it is a consequence of \mcref{HA:diffFourierTransf}, which reduces any order of a pde in the spatial variable to 1. The procedure is to compute the Fourier transform $\F$ of the pde, solve it, and then get back to the first function using the inverse transform.
  \end{remark}
  \begin{theorem}[Uncertainty principle]
    Let $f\in L^2(\RR)$ be differentiable such that $xf^2\in L^1(\RR)$ and $f'\in L^2(\RR)$. Then:
    $$\left(\int_{-\infty}^\infty x^2 \abs{f(x)}^2\dd{x}\right)\left(\int_{-\infty}^\infty \xi^2 \abs{\widehat{f}(\xi)}^2\dd{\xi}\right)\geq \frac{{\norm{f}_2}^4}{16\pi^2}$$
    and the equality holds if and only if $f(x)=\exp{-\lambda^2x^2}$, $\lambda\in\RR$.
  \end{theorem}
  \begin{proof}
    Let $I$ be the left-hand-side term of the inequality. First note that:
    $$\int_{-\infty}^\infty \!\xi^2 \abs{\widehat{f}(\xi)}^2\dd{\xi}=\frac{1}{4\pi^2}\int_{-\infty}^\infty \abs{\widehat{f'}(\xi)}^2\dd{\xi}=\frac{1}{4\pi^2}\int_{-\infty}^\infty \abs{{f'}(\xi)}^2\dd{\xi}$$
    where the first equality is by the analogous \mcref{HA:diffFourierTransf} in $L^2(\RR)$ and in the second one we have used \mnameref{HA:plancherel}.
    Now by the \mref{RFA:cauchyschwarz}, we have:
    \begin{align*}
      I & \geq \frac{1}{4\pi^2}{\left(\int_{-\infty}^\infty x\abs{f(x){f'}(x)}\dd{x}\right)}^2         \\
        & \geq \frac{1}{4\pi^2}{\left(\int_{-\infty}^\infty x\Re(f(x)\overline{f'(x)})\dd{x}\right)}^2 \\
        & = \frac{1}{16\pi^2}{\left(\int_{-\infty}^\infty x\dv{}{x}(\abs{f(x)}^2)\dd{x}\right)}^2      \\
        & = \frac{1}{16\pi^2}{\left(\int_{-\infty}^\infty \abs{f(x)}^2\dd{x}\right)}^2                 \\
    \end{align*}
    where in the third step we have used that:
    $$\dv{}{x}(\abs{f(x)}^2)=2\Re(f(x)\overline{f'(x)})$$
    and in the last step we have integrated by parts (here we use that $xf^2\in L^1(\RR)$).
  \end{proof}
  \begin{theorem}[Poisson summation formula]
    Let $f\in\mathcal{C}(\RR)\cap L^1(\RR)$ be such that $\sum_{k\in\ZZ}f(x+k)$ converges uniformly for $x\in[0,1]$ and such that $\sum_{k\in\ZZ}\abs{\widehat{f}(k)}<\infty$. Then:
    $$\sum_{k\in\ZZ}f(x+k)=\sum_{k\in\ZZ}\widehat{f}(k)\exp{2\pi\ii k x}$$
    In particular, for $x=0$ we have:
    $$\sum_{k\in\ZZ}f(k)=\sum_{k\in\ZZ}\widehat{f}(k)$$
  \end{theorem}
  \begin{proof}
    Let $F(x):=\sum_{k\in\ZZ}f(x+k)$. Note that $F$ is 1-periodic. If we see that $\widehat{F}(n)=\widehat{f}(n)$ $\forall n\in\ZZ$, the continuity of $f$ and the convergence of its Fourier series will imply $F(x)=\sum_{k\in\ZZ}\widehat{f}(k)\exp{2\pi\ii k x}$. But:
    \begin{align*}
      \widehat{F}(n) & =\int_{0}^{1}\sum_{k\in\ZZ}f(x+k)\exp{-2\pi\ii n x}\dd{x}     \\
                     & =\sum_{k\in\ZZ}\int_{0}^{1}f(x+k)\exp{-2\pi\ii n x}\dd{x}     \\
                     & =\sum_{k\in\ZZ}\int_{k}^{k+1}f(x)\exp{-2\pi\ii n (x-k)}\dd{x} \\
                     & =\sum_{k\in\ZZ}\int_{k}^{k+1}f(x)\exp{-2\pi\ii n x}\dd{x}     \\
                     & =\int_{-\infty}^{\infty}f(x)\exp{-2\pi\ii n x}\dd{x}          \\
                     & =\widehat{f}(n)
    \end{align*}
  \end{proof}
  \begin{definition}
    Let $f\in L^2(\RR)$. We say that $f$ is \emph{bandlimited} if $\exists B\in\RR$ such that $\supp\widehat{f}\subseteq[-B,B]$.
  \end{definition}
  \begin{theorem}[Nyquist-Shannon sampling theorem]
    Let $f\in L^2(\RR)$ be bandlimited with constant $B$. Then:
    $$f(x)\overset{L^2}{=}\sum_{k\in\ZZ}f\left(\frac{k}{2B}\right)\frac{\sin(\pi(2Bx-k))}{\pi(2Bx-k)}$$
    Moreover: $${\norm{f}_2}^2=\frac{1}{2B}\sum_{k\in\ZZ}\abs{f\left(\frac{k}{2B}\right)}^2$$
  \end{theorem}
  \begin{proof}
    An easy check shows that the Fourier series of $\xi\mapsto\exp{2\pi\ii x\xi}$ on $[-B, B]$ is: $$\exp{2\pi\ii x\xi}=\sum_{k\in\ZZ}\frac{\sin(\pi(2Bx-k))}{\pi(2Bx-k)}\exp{\frac{\pi\ii k\xi}{B}}$$
    Thus:
    \begin{align*}
      f(x) & =\int_{-B}^{B}\widehat{f}(\xi)\exp{2\pi\ii \xi x}\dd{\xi}                                                           \\
           & =\sum_{k\in\ZZ}\frac{\sin(\pi(2Bx-k))}{\pi(2Bx-k)}\int_{-B}^{B}\widehat{f}(\xi)\exp{\frac{\pi\ii k\xi}{B}} \dd{\xi} \\
           & =\sum_{k\in\ZZ}f\left(\frac{k}{2B}\right)\frac{\sin(\pi(2Bx-k))}{\pi(2Bx-k)}
    \end{align*}
    The second equality follows from both \mnameref{HA:plancherel} and \mnameref{MA:parseval}:
    $${\norm{f}_2}^2={\norm{\widehat{f}}_2}^2=\frac{1}{2B}\sum_{k\in\ZZ}\abs{f\left(\frac{k}{2B}\right)}^2$$
    because by a similar argument as before, the Fourier coefficients of $\widehat{f}(k)$ (thought as periodically extended) are $\frac{1}{2B}f\left(\frac{-k}{2B}\right)$.
  \end{proof}
  \subsubsection{Discrete Fourier transform}
  \begin{definition}
    Consider a function $f$ with support $\{0,\ldots,N-1\}$. We can think $f$ as:
    $$\function{f}{\ZZ}{\CC}{k}{f(k\mod{n})=:f[k]}$$
    Note that with this definition, $f$ is $N$-periodic. We define the \emph{discrete Fourier transform} (\emph{DFT}) of $f$ as:
    $$\widehat{f}[k]:=\sum_{n=0}^{N-1}f[n]\exp{-\frac{2\pi\ii nk}{N}}$$
    If we denote $\omega_N:=\exp{-\frac{2\pi\ii}{N}}$ we can write:
    $$\widehat{f}[k]=\sum_{n=0}^{N-1}f[n]{\omega_N}^{kn}$$
    We will denote $\vf{f}:=(f[0],\ldots,f[N-1])$
  \end{definition}
  \begin{proposition}
    Let $f,g:\ZZ\rightarrow\CC$. Then:
    \begin{enumerate}
      \item $\widehat{f}$ is linear.
      \item If $n\in\ZZ$ and $g[k]=f[k-n]$ $\forall k\in\ZZ$, then: $$\widehat{g}[k]=\widehat{f}[k]\exp{-\frac{2\pi\ii k n}{N}}$$
      \item If $g[k]=\overline{f[k]}$ $\forall k\in\ZZ$, then: $$\widehat{g}[k]=\overline{\widehat{f}[N-k]}$$
    \end{enumerate}
  \end{proposition}
  \begin{proposition}
    Let $f:\ZZ\rightarrow\CC$. Then, $\vf{\widehat{f}}=\vf{A}(\omega_N)\vf{f}$, where
    $$\vf{A}(\omega_N)=\!\begin{pmatrix}
        1      & 1                & 1                   & \cdots & 1                       \\
        1      & \omega_N         & {\omega_N}^2        & \cdots & {\omega_N}^{N-1}        \\
        1      & {\omega_N}^2     & {\omega_N}^4        & \cdots & {\omega_N}^{2(N-1)}     \\
        \vdots & \vdots           & \vdots              & \ddots & \vdots                  \\
        1      & {\omega_N}^{N-1} & {\omega_N}^{2(N-1)} & \cdots & {\omega_N}^{(N-1)(N-1)}
      \end{pmatrix}$$
  \end{proposition}
  is a symmetric matrix.
  \begin{lemma}
    Let $N\in\NN$. Then, $$\vf{A}(\omega_N)\vf{A}(\overline{\omega_N})=\vf{A}(\overline{\omega_N})\vf{A}(\omega_N)=N\vf{I}_N$$
  \end{lemma}
  \begin{sproof}
    Remember that both ${\omega_N}$ and $\overline{\omega_N}$ are roots of $1+x+\cdots+x^{N-1}$.
  \end{sproof}
  \begin{definition}
    Let $f:\ZZ\rightarrow\CC$. We define the \emph{inverse discrete Fourier transform} as:
    $$\vf{f}=\frac{1}{N}\vf{A}(\overline{\omega_N})\vf{\widehat{f}}$$
  \end{definition}
  \begin{theorem}[Plancherel theorem]
    Let $f:\ZZ\rightarrow\CC$. Then:
    $$\sum_{k=0}^{N-1}f[k]\overline{g[k]}=\frac{1}{N}\sum_{k=0}^{N-1}\widehat{f}[k]\overline{\widehat{g}[k]}$$
    In particular, if $f=g$, we have:
    $$\sum_{k=0}^{N-1}\abs{f[k]}^2=\frac{1}{N}\sum_{k=0}^{N-1}\abs{\widehat{f}[k]}^2$$
  \end{theorem}
  \begin{proof}
    Using vector notation:
    \begin{align*}
      \dotp{\transpose{\vf{f}}}{\overline{\vf{g}}} & =\transpose{\left(\frac{1}{N}\vf{A}(\overline{\omega_N})\vf{\widehat{f}}\right)}\left(\frac{1}{N}\overline{\vf{A}(\overline{\omega_N})}\vf{\widehat{\overline{g}}}\right) \\
                                                   & =\frac{1}{N^2}\transpose{\vf{\widehat{f}}}\transpose{\vf{A}(\overline{\omega_N})}\vf{A}({\omega_N})\vf{\widehat{\overline{g}}}                                            \\
                                                   & =\frac{1}{N}\dotp{\transpose{\vf{\widehat{f}}}}{\vf{\widehat{\overline{g}}}}
    \end{align*}
    because $\vf{A}(\overline{\omega_N})$ is symmetric.
  \end{proof}
  \begin{definition}
    Let $f,g:\ZZ\rightarrow\CC$. We define the \emph{convolution} of $f$ and $g$ as:
    $$(f*g)[k]:=\sum_{n=0}^{N-1}f[n]g[k-n]$$
  \end{definition}
  \begin{lemma}
    Let $f,g:\ZZ\rightarrow\CC$. Then:
    $$\widehat{f*g}[k]=\widehat{f}[k]\widehat{g}[k]$$
  \end{lemma}
  \begin{proof}
    \begin{align*}
      \widehat{f*g}[k] & =\sum_{n=0}^{N-1}\sum_{j=0}^{N-1}f[j]g[n-j]{\omega_N}^{nk}                    \\
                       & =\sum_{j=0}^{N-1}f[j]{\omega_N}^{jk}\sum_{n=0}^{N-1}g[n-j]{\omega_N}^{(n-j)k} \\
                       & =\widehat{f}[k]\widehat{g}[k]
    \end{align*}
  \end{proof}
  \begin{theorem}[Poisson summation formula]
    Let $f:\ZZ\rightarrow\CC$. Then:
    $$\sum_{k=0}^{N-1}\widehat{f}[k]=Nf[0]$$
  \end{theorem}
  \begin{proof}
    $$\sum_{k=0}^{N-1}\widehat{f}[k]=\sum_{k,n=0}^{N-1}f[n]{\omega_N}^{kn}=Nf[0]$$
    because $\sum_{k=0}^{N-1}{\omega_N}^{kn}=N$ if $n=0$ and 0 otherwise because ${\omega_N}^{n}$ are roots of $1+x+\cdots+x^{N-1}$.
  \end{proof}
  \subsubsection{Fast Fourier transform}
  \begin{definition}
    Let $f:\ZZ\rightarrow\CC$. Note that we need $\O{N^2}$ operations in order to compute $\widehat{f}$. The \emph{fast Fourier transform} (\emph{FFT}) aims to minimize that number by maxing use of some tricks.
  \end{definition}
  \begin{definition}[Radix-2 DIT Cooley-Tukey FFT algorithm]
    Let $f:\ZZ\rightarrow\CC$ and assume that $N=2m$. The \emph{radix-2 decimation-in-time (DIT) FFT} is defined as follows. We can write:
    \begin{multline*}
      \widehat{f}[k] =\sum_{n=0}^{N/2-1}f[2n]{\omega_N}^{2nk}+\sum_{n=0}^{N/2-1}f[2n+1]{\omega_N}^{(2n+1)k}    \\ =\sum_{n=0}^{N/2-1}f[2n]{(\exp{-\frac{2\pi\ii}{N/2}})}^{nk}+\exp{-\frac{2\pi\ii k}{N}}\sum_{n=0}^{N/2-1}f[2n+1]{(\exp{-\frac{2\pi\ii}{N/2}})}^{nk} \\
      =:E_k+\exp{-\frac{2\pi\ii k}{N}}O_k
    \end{multline*}
    for $k=0,\ldots,N/2-1$ even though the equality holds for $k=0,\ldots,N-1$. For the other cases, we use the periodicity of $\exp{-\frac{2\pi\ii k}{N}}$ to get:
    $$\widehat{f}[k+N/2] =E_k-\exp{-\frac{2\pi\ii k}{N}}O_k$$
    for $k=0,\ldots,N/2-1$.
    Note that $E_k$ and $O_k$ are both $N/2$-dimensional DFT of the even terms of $f$ and the odd terms of $f$, respectively. We can thus compute them recursively until the respective $m$ is odd. Using this method we can get the DFT of $f$ in at most $O{N\log N}$ time.
  \end{definition}
  \subsection{Distributions}
  \begin{definition}
    Let $\Omega\subseteq \RR^n$ and $(\varphi_n),\varphi\in\mathcal{D}(\Omega):=\mathcal{C}_0^\infty(\Omega)$. We say that $\varphi_n\rightarrow \varphi$ in $\mathcal{D}(\Omega)$ if:
    \begin{enumerate}
      \item There exists a compact set such that $\supp \varphi_n\subseteq K$ $\forall n\in\NN$.
      \item $\displaystyle\lim_{n\to\infty}\norm{\partial^\alpha\varphi_n-\partial^\alpha\varphi}_{L^\infty(K)}=0$ $\forall \alpha$.
    \end{enumerate}
  \end{definition}
  \begin{definition}[Distribution]
    Let $\Omega\subseteq \RR^n$ be a set. A \emph{distribution} on $\Omega$ is a continuous linear form on $\mathcal{D}(\Omega)$. The vector space of all distributions on $\Omega$ is denoted by $\mathcal{D}^*(\Omega)$.
  \end{definition}
  \begin{lemma}
    Let $\Omega\subseteq\RR^n$ and $T:\mathcal{D}(\Omega)\rightarrow\CC$ be linear. Then, $T$ is continuous if $\forall (\varphi_n)\in\mathcal{D}(\Omega)$ with $\varphi_n\rightarrow 0$ in $\mathcal{D}(\Omega)$ we have that $T(\varphi_n)\rightarrow 0$.
  \end{lemma}
  \begin{lemma}[Fundamental lemma of calculus of variations]\label{HA:fundamentallemma}
    Let $\Omega\subseteq\RR^n$ be a domain and $f\in L_\mathrm{loc}^1(\Omega)$ such that $$\int_\Omega f(\vf{x})\varphi(\vf{x})\dd{\vf{x}}=0$$ for all $\varphi\in\mathcal{D}(\Omega)$. Then, $f\almoste{=}0$ in $\Omega$.
  \end{lemma}
  \begin{proposition}
    Let $\Omega\subseteq\RR^n$ and $T:\mathcal{D}(\Omega)\rightarrow\CC$ be linear. Then, $T\in \mathcal{D}^*(\Omega)$ if and only if for all compact set $K\subseteq \Omega$, there exist $C>0$ and $m\in\NN\cup\{0\}$ such that $\forall \varphi\in\mathcal{D}(\Omega)$ we have:
    $$\abs{T(\varphi)}\leq C\sum_{\abs{\alpha}\leq m}\norm{\partial^\alpha\varphi}_{L^\infty(K)}$$
  \end{proposition}
  \begin{proof}
    The right-to-left implication is clear. For the other one, suppose that there exists a compact set $K$ such that $\forall C>0$ and all $m\in\NN\cup\{0\}$ there exists a sequence $(\varphi_k)\in\mathcal{D}(\Omega)$ such that
    $$\abs{T(\varphi_k)}> C\sum_{\abs{\alpha}\leq m}\norm{\partial^\alpha\varphi_k}_{L^\infty(K)}=:C\norm{\varphi_k}_{m,K}$$
    Now consider $\psi_k:=\frac{\varphi_k}{k\norm{\varphi_k}_{m,K}}$. Clearly $\norm{\psi_k}_{L^\infty(K)}\leq\frac{1}{k}\overset{k\to\infty}{\longrightarrow}0$ but $\abs{T(\psi_k)}=\frac{\abs{T(\varphi_k)}}{k\norm{\varphi_k}_{m,K}}>1$ by considering the particular case of $C=m$. Hence, $T$ cannot be continuous, which is a contradiction.
  \end{proof}
  \begin{proposition}
    Let $\Omega\subseteq \RR^n$ be a set and $f\in L_{\mathrm{loc}}^1(\Omega)$. Then, the map
    \begin{equation}\label{HA:regularDistribution}
      \function{\Lambda_f}{\mathcal{D}(\Omega)}{\CC}{\varphi}{\displaystyle\int_\Omega f(\vf{x})\varphi(\vf{x})\dd{\vf{x}}}
    \end{equation}
    is a distribution. Hence, $\Lambda_f(\varphi)$ is usually denoted by $\dotp{f}{\varphi}$. Sometimes we will do an abuse of notation denoting $\Lambda_f$ as $f$ (in view of the \mnameref{HA:fundamentallemma}).
  \end{proposition}
  \begin{proof}
    $\Lambda_f$ is clearly linear. Moreover: $$\abs{\Lambda_f(\varphi)}\leq\int_\Omega\abs{f(\vf{x})\varphi(\vf{x})}\leq \norm{f}_1\norm{\varphi}_\infty$$
    Hence, $\Lambda_f$ is bounded and therefore continuous.
  \end{proof}
  \begin{definition}
    The distributions that can be expressed as in \mcref{HA:regularDistribution} are called \emph{regular distributions}.
  \end{definition}
  \begin{proposition}[Dirac's $\delta$ distribution]
    Let $\Omega\subseteq \RR^n$ be a set and $\vf{x}_0\in\Omega$. Then, the map
    $$
      \function{\delta_{\vf{x}_0}}{\mathcal{D}(\Omega)}{\RR}{\varphi}{\varphi(\vf{x}_0)}
    $$
    is a distribution. We will denote $\delta_{\vf{0}}$ simply by $\delta$.
  \end{proposition}
  \begin{proof}
    Clearly $\delta_{\vf{x}_0}$ is linear and bounded because $\abs{\delta_{\vf{x}_0}(\varphi)}=\abs{\varphi(\vf{x}_0)}\leq \norm{\varphi}_\infty$.
  \end{proof}
  \begin{lemma}
    The Dirac's $\delta_{\vf{0}}$ distribution is not regular.
  \end{lemma}
  \begin{proof}
    Suppose it is regular. Then, $\exists f\in L_{\mathrm{loc}}^1(\Omega)$ such that $\delta=\Lambda_f$. Hence, $\varphi(\vf{0})=\delta(\varphi)=\int_\Omega f(\vf{x})\varphi(\vf{x})\dd{\vf{x}}$ for all $\varphi\in\mathcal{D}(\Omega)$. Then, if we take $\varphi_n(x):=\varphi(nx)$, where:
    $$\varphi(x)=\begin{cases}
        \exp{-\frac{1}{1-\norm{x}^2}} & \text{if } \norm{x}\leq 1 \\
        0                             & \text{if } \norm{x}> 1
      \end{cases}
    $$
    then $\varphi_n\in\mathcal{D}(\Omega)$ and have support $\overline{B(\vf{0},1/n)}$. So:
    $$\exp{-1}=\abs{\int_{\Omega\cap \overline{B(\vf{0},1/n)}} f(\vf{x})\varphi_n(\vf{x})\dd{\vf{x}}}\leq \int_{\norm{x}<\frac{1}{n}}\abs{f(\vf{x})}\dd{\vf{x}}\overset{n\to\infty}{\longrightarrow}0$$
  \end{proof}
  \begin{proposition}[Cauchy principal value]
    We define the \emph{Cauchy principal value} $T:=\pv\left(\frac{1}{x}\right)$ as the distribution
    $$T(\varphi)=\lim_{\varepsilon\to 0}\int_{\abs{x}\geq \varepsilon}\frac{\varphi(x)}{x}\dd{x}$$
  \end{proposition}
  \begin{proof}
    First of all note that it is well defined because we can write:
    $$T(\varphi)=\lim_{\varepsilon\to 0}\int_{\varepsilon}^\infty\frac{\varphi(x)-\varphi(-x)}{x}\dd{x}=\int_{0}^\infty\frac{\varphi(x)-\varphi(-x)}{x}\dd{x}$$
    which is well-defined because $\varphi$ has compact support and in a neighborhood of $0$ the integrand is bounded (by the \mnameref{RVF:meanvaluetheorem}). Moreover it is clearly linear and continuous because $$\abs{T(\varphi)}\leq \m{K} \norm{\varphi'}_\infty$$
    where $\abs{K}$ is the measure of the support of $\varphi$.
  \end{proof}
  \begin{definition}
    Let $\Omega\subseteq \RR^n$. We say that a distribution $T\in\mathcal{D}^*(\Omega)$ is a \emph{distribution of order $N\in\NN\cup\{0\}$} if $\exists N\in\NN\cup\{0\}$ such that for all compact set $K$ $\exists C_K>0$ with
    $$\abs{T(\varphi)}\leq C_K\norm{\varphi}_{N,K}$$
    for all $\varphi\in\mathcal{D}(\Omega)$. We say that $T$ is has \emph{infinite order} if it is not of order $N$ for any $N\in\NN$.
  \end{definition}
  \subsubsection{Convergency of distributions}
  \begin{definition}
    Let $\Omega\subseteq \RR^n$ be a set and $(T_n)\in \mathcal{D}^*(\Omega)$. We say that $(T_n)$ converges to $T\in\mathcal{D}^*(\Omega)$ if $T_n(\varphi)\overset{n\to\infty}{\longrightarrow}T(\varphi)$ for all $\varphi\in\mathcal{D}(\Omega)$.
  \end{definition}
  \begin{definition}
    Let $\Omega\subseteq \RR^n$ be a set. We say that a sequence of functions $(\phi_\varepsilon)\in L_\mathrm{loc}^1(\Omega)$ is an \emph{approximation of unity} if
    \begin{enumerate}
      \item $\displaystyle\int_\Omega \phi_\varepsilon=1$
      \item $\displaystyle\int_\Omega \abs{\phi_\varepsilon}\leq M$ $\forall \varepsilon>0$
      \item $\displaystyle\lim_{\varepsilon\to 0}\int_{\norm{\vf{x}}\geq \delta}\phi_\varepsilon(\vf{x})\dd{\vf{x}}=0$ $\forall \delta>0$.
    \end{enumerate}
  \end{definition}
  \begin{proposition}
    Let $\Omega\subseteq \RR^n$ be a set and $\phi\in L^1(\Omega)$ such that $\int_\Omega\phi=1$. Let $\phi_\varepsilon:=\frac{1}{\varepsilon^n}\phi(\frac{x}{\varepsilon})$. Then, $(\phi_\varepsilon)$ is an approximation of unity, $\phi_\varepsilon \in L_\mathrm{loc}^1$ $\forall \varepsilon$ and $\phi_\varepsilon\to \delta_{\vf{0}}$ in $\mathcal{D}^*(\Omega)$.
  \end{proposition}
  \begin{sproof}
    Let $\varphi\in\mathcal{D}(\Omega)$. Then:
    \begin{align*}
      \abs{\phi_\varepsilon(\varphi)-\delta_{\vf{0}}(\varphi)} & \leq \int_\Omega \abs{\phi_\varepsilon(x)}\abs{\varphi(\vf{x})-\varphi(\vf{0})}\dd{\vf{x}} \\
      \begin{split}
        & =\int_{\norm{\vf{x}}<\delta} \abs{\phi_\varepsilon(x)}\abs{\varphi(\vf{x})-\varphi(\vf{0})}\dd{\vf{x}}+\\
        &\hspace{1cm}+\int_{\norm{\vf{x}}\geq\delta} \abs{\phi_\varepsilon(x)}\abs{\varphi(\vf{x})-\varphi(\vf{0})}\dd{\vf{x}}
      \end{split}
    \end{align*}
    Now use the properties of approximation of unity to see that each interval goes to zero as $\varepsilon\to 0$.
  \end{sproof}
  \begin{theorem}
    Let $\Omega\subseteq \RR^n$ be a set and $(f_n)\in L_\mathrm{loc}^p$ such that $f_n\overset{L_\mathrm{loc}^p}{f}$ (which means that $\int_K\abs{f_n-f}\to 0$ for any compact set $K$). Then, $T_{f_n}$ converges to $T_f$ in $\mathcal{D}(\Omega)$.
  \end{theorem}
  \begin{proof}
    Use \mnameref{RFA:holder}.
  \end{proof}
  \begin{remark}
    Clearly if $f_n$ converge uniformly to $f$, the condition of the theorem holds and we get the same result. But it can be seen that only with pointwise convergence is not enough (consider $f_n(x)=n^kx^n(1-x)\indi{[0,1]}$ for $k\in\NN$). Moreover, $T_{f_n}$ converges to $T_f$ in $\mathcal{D}(\Omega)$ does not imply pointwise convergence of $f_n$ towards $f$.
  \end{remark}
  % \begin{definition}
  %   Let $\Omega\subseteq \RR^n$ be a set and $n\in\NN$. We define the \emph{differentiation operator} $D^n:\mathcal{D}^*(\Omega)\rightarrow\mathcal{D}^*(\Omega)$ by: $$\dotp{D^n\Lambda}{\varphi}=\dotp{\Lambda}{{(-1)}^nD^n\varphi}$$
  %   for all $\Lambda\in\mathcal{D}^*(\Omega)$ and all $\varphi\in\mathcal{D}(\Omega)$. The distribution $D^n\Lambda$ is called \emph{distributional derivative}.
  % \end{definition}
  % \begin{definition}
  %   We define the \emph{Heaviside step function} as the function $H(x)=\indi{x>0}$.
  % \end{definition}
  % \begin{proposition}
  %   We have that $\Lambda_{H}=:H\in\mathcal{D}^*(\RR)$ and: $$H'=\delta$$
  % \end{proposition}
  % \begin{proof}
  %   For all $\varphi\in\mathcal{D}(\Omega)$ we have:
  %   \begin{align*}
  %     \dotp{H'}{\varphi}=-\dotp{H}{\varphi'} & =-\int_{-\infty}^\infty H(x)\varphi'(x)\dd{x}                  \\
  %                                            & =-\int_{0}^\infty\varphi'(x)\dd{x}= \varphi(0)=\delta(\varphi)
  %   \end{align*}
  %   because $\varphi$ has compact support.
  % \end{proof}
\end{multicols}
\end{document}