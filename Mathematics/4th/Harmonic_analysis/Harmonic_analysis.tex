\documentclass[../../../main_math.tex]{subfiles}


\begin{document}
\changecolor{HA}
\begin{multicols}{2}[\section{Harmonic analysis}]
  \subsection{Introduction}
  Refer to \mnameref{MA:fouriersection} for a reminder of the introductory concepts of Fourier series.
  \subsubsection{Uniform convergence}
  \begin{theorem}
    Let $f$ be a continuous $T$-periodic function such that $f'$ exists except for a finite number of points and it is continuous and bounded. Then, $S_Nf$ converges uniformly to $f$ on $[-T/2,T/2]$.
  \end{theorem}
  \begin{proof}
    We have pointwise convergence towards $f$. Moreover:
    \begin{align*}
      \sum_{n\in\ZZ}\abs{\widehat{f}(n)} & \leq\abs{\widehat{f}(0)}+ \sum_{n\in\ZZ\setminus\{0\}}\frac{1}{n}n\abs{\widehat{f}(n)}                                             \\
                                         & \leq\abs{\widehat{f}(0)}+\frac{1}{2}\sum_{n\in\ZZ\setminus\{0\}}\left(\frac{1}{n^2} + n^2\abs{\widehat{f}(n)}^2\right)             \\
                                         & =\abs{\widehat{f}(0)}+\frac{1}{2}\sum_{n\in\ZZ\setminus\{0\}}\frac{1}{n^2}+\frac{T^2}{8\pi^2}\sum_{n\in\ZZ}\abs{\widehat{f'}(n)}^2 \\
                                         & \leq\abs{\widehat{f}(0)}+\frac{1}{2}\sum_{n\in\ZZ\setminus\{0\}}\frac{1}{n^2}+\frac{T}{8\pi^2}\norm{f'}^2                          \\
                                         & <\infty
    \end{align*}
    by \mnameref{MA:bessel} and because $f'$ is bounded. Thus, the \mnameref{MA:Mweierstrass} implies that $S_Nf$ converges uniformly to $f$.
  \end{proof}
  \begin{corollary}
    Let $f\in\mathcal{C}^{r-1}$ be a $T$-periodic function such that $f^{(r)}$ exists except for a finite number of points and it is continuous and bounded. Then: $$\sup_{x\in[-T/2,T/2]}\abs{S_Nf(x)-f(x)}\leq \frac{\varepsilon_N}{N^{r-1/2}}$$ for some sequence $(\varepsilon_N)\overset{N\to\infty}{\longrightarrow}0$.
  \end{corollary}
  \begin{proof}
    By \mref{RFA:cauchyschwarz} we have:
    \begin{align*}
      \abs{S_Nf(x)-f(x)} & \leq\sum_{n>\abs{N}}\frac{1}{n^r}n^r\abs{\widehat{f}(n)}                                                                                         \\
                         & \leq{\left(\sum_{n>\abs{N}}\frac{1}{n^{2r}}\right)}^{\frac{1}{2}}{\left(\sum_{n>\abs{N}}n^{2r}\abs{\widehat{f}(n)}^2 \right)}^{\frac{1}{2}}      \\
                         & \lesssim{\left(\int_N^\infty\frac{1}{x^{2r}}\dd{x}\right)}^{\frac{1}{2}}{\left(\sum_{n>\abs{N}}\abs{\widehat{f^{r}}(n)}^2 \right)}^{\frac{1}{2}} \\
                         & =\frac{\tilde{C}}{N^{r-1/2}}\varepsilon_N
    \end{align*}
    with $\varepsilon_N\overset{N\to\infty}{\longrightarrow}0$ because it is the tail of a convergent sequence.
  \end{proof}
  \subsubsection{Poisson kernel}
  For most of the proofs in this section check the analogous ones with the \mnameref{MA:fejerdef}.
  \begin{definition}[Poisson kernel]
    Let $r\in[0,1]$. We define the \emph{Poisson kernel} as $$P_r(t)=\sum_{n\in\ZZ}r^{\abs{n}}\exp{\frac{2\pi\ii n t}{T}}$$
  \end{definition}
  \begin{lemma}\label{HA:poisskernelchar}
    Let $r\in[0,1]$. Then:
    $$P_r(t)=\frac{1-r^2}{1-2r\cos\left(\frac{2\pi t}{T}\right)+r^2}$$
  \end{lemma}
  \begin{sproof}
    Use the geometric progression formula.
  \end{sproof}
  \begin{proposition}\label{HA:poissprop}
    The Poisson kernel has the following properties:
    \begin{enumerate}
      \item $P_r$ is a $T$-periodic, even and non-negative function.
      \item $\displaystyle\frac{1}{T}\int_{-T/2}^{T/2}P_r(t)\dd{t}=1\quad\forall N$.
      \item $\forall\delta>0$, $\displaystyle\lim_{r\to 1^-}\sup\{\abs{P_r(t)}:\delta\leq\abs{t}\leq T/2\}=0$.
    \end{enumerate}
  \end{proposition}
  \begin{theorem}
    Let $f\in L^1([-T/2,T/2])$ be a function having left- and right-sided limits at point $x_0$. Then: $$\lim_{r\to 1^-}f*P_r=\frac{f({x_0}^+)+f({x_0}^-)}{2}$$ In particular, if $f$ is continuous at $x_0$, $\displaystyle\lim_{r\to 1^-}f*P_r=f(x_0)$.
  \end{theorem}
  \begin{theorem}
    Let $p\geq 1$ and $f\in L^p([-T/2,T/2])$. Then:
    \begin{gather*}
      \lim_{N\to\infty}\norm{\sigma_Nf-f}_p=0\\
      \lim_{r\to 1^-}\norm{f*P_r-f}_p=0
    \end{gather*}
  \end{theorem}
  \subsection{Fourier transform}
  \subsubsection{Definition and first properties}
  \begin{definition}
    Let $f\in L^1(\RR)$. We define the \emph{Fourier transform} of $f$ as:
    $$\widehat{f}(\xi)=\int_{-\infty}^{+\infty}f(x)\exp{-2\pi \ii\xi x}\dd{x}$$
    The function $f$ is also called \emph{inverse Fourier transform} of $\widehat{f}$.
  \end{definition}
  \begin{proposition}\label{HA:fourierTransProperties}
    Let $f,g\in L^1(\RR)$ and $\alpha,\beta\in\RR$. Then:
    \begin{enumerate}
      \item $\widehat{(\alpha f+\beta g)}(\xi)=\alpha\widehat{f}(\xi)+\beta \widehat{g}(\xi)$
      \item\label{HA:FTprop2} Let $h\in\RR$. We define $T_hf(x)=f(x+h)$. Then: $$\widehat{T_hf}(\xi)=\exp{2\pi\ii \xi h}\widehat{f}(\xi)$$
      \item\label{HA:FTprop3} If $g(x)=\exp{2\pi\ii x h}f(x)$, then: $$\widehat{g}(\xi)=\widehat{f}(\xi-h)$$
      \item\label{HA:FTprop4} If $\lambda\in\RR^*$, then: $$\frac{1}{\lambda}\widehat{f\left(\frac{x}{\lambda}\right)}(\xi)=\widehat{f}(\lambda\xi)$$
      \item\label{HA:FTprop5} If $g(x)=\overline{f(x)}$, then: $$\widehat{g}(\xi)=\overline{\widehat{f}(-\xi)}$$
    \end{enumerate}
  \end{proposition}
  \begin{sproof}
    They follow from the linearity of the integral and some change of variable.
  \end{sproof}
  \begin{definition}
    Let $f\in L^1(\RR)$. We define the \emph{Fourier transform operator} as $\F f=\widehat{f}$.
  \end{definition}
  \begin{proposition}\label{HA:boundedFT}
    Let $f\in L^1(\RR)$. Then:
    \begin{enumerate}
      \item $\F f$ is uniformly continuous.
      \item $\F$ is a continuous linear operator from $L^1(\RR)$ to $L^\infty(\RR)$ and $\norm{\F f}_{\infty}\leq \norm{f}_1$.
    \end{enumerate}
  \end{proposition}
  \begin{proof}
    \begin{enumerate}
      \item Using \mcref{HA:FTprop3} we have:
            $$\abs{\F f(\xi+h)-\F f(\xi)}\leq \int_{-\infty}^{+\infty}\abs{\exp{-2\pi\ii x h}-1}\abs{f(x)}\dd{x}$$
            By the \mnameref{RFA:dominated} we have that the integral is bounded by $2\norm{f}_1$ and so entering the limit we obtain the bound $\varepsilon \norm{f}_1$ $\forall \varepsilon>0$. As the bound does not depend on the point $\xi$, the convergence is uniform.
      \item Clearly $\norm{\F f}_{\infty}\leq \norm{f}_1$. Hence the operator is bounded and therefore continuous.
    \end{enumerate}
  \end{proof}
  \begin{theorem}[Riemann-Lebesgue lemma]\label{HA:riemannlebesgue}
    Let $f\in L^1(\RR)$. Then:
    $$\lim_{\abs{\xi}\to\infty} \abs{\widehat{f}(\xi)}=0$$
  \end{theorem}
  \begin{sproof}
    Note that $2\abs{\widehat{f}(\xi)}=\abs{\widehat{f}(\xi)-\exp{\ii\pi}\widehat{f}(\xi)}$ and:
    \begin{align*}
      \exp{\ii\pi}\widehat{f}(\xi) & =\int_{-\infty}^{+\infty}f(x)\exp{-2\pi\ii\xi x+\ii\pi}\dd{x}                    \\
                                   & =\int_{-\infty}^{+\infty}f\left(u+\frac{1}{2\xi}\right)\exp{-2\pi\ii\xi u}\dd{u}
    \end{align*}
    So: $$\abs{\widehat{f}(\xi)}\leq\frac{1}{2}\int_{-\infty}^{+\infty}\left|f(x)-f\left(x+\frac{1}{2\xi}\right)\right|\dd{x}$$
    Now use again the \mnameref{RFA:dominated}.
  \end{sproof}
  \begin{proposition}\label{HA:symmetryFT}
    Let $f,g\in L^1(\RR)$. Then, $f\widehat{g},\widehat{f}g\in L^1(\RR)$ and:
    $$\int_{-\infty}^{+\infty}\widehat{f}(x)g(x)\dd{x}=\int_{-\infty}^{+\infty}f(x)\widehat{g}(x)\dd{x}$$
  \end{proposition}
  \begin{sproof}
    By \mcref{HA:boundedFT}, $\widehat{g}$ is bounded. Hence, $f\widehat{g}\in L^1(\RR)$ and the same applies for $\widehat{f}g$. For the equality, use \mnameref{FSV:fubini}.
  \end{sproof}
  \begin{proposition}\label{HA:diffFourierXf}
    Let $f$ be a function such that $x^k f\in L^1(\RR)$ for $k=0,\ldots,r$. Then, $\widehat{f}$ is $r$ times differentiable and:
    $${(\F f)}^{(k)}=\F({(-2\pi\ii x)}^kf(x))$$
    for $k=0,1,\ldots, r$.
  \end{proposition}
  \begin{proof}
    Note that the function $h:\xi\rightarrow\exp{-2\pi\ii \xi x}f(x)$ is $\mathcal{C}^\infty(\RR)$ and $h^{(k)}(\xi)={(-2\pi\ii x)}^k\exp{-2\pi\ii \xi x}f(x)$. Since $\abs{h^{(k)}(\xi)}\leq \abs{x^kf(x)}$ we can use \mcref{RFA:diffUnderIntegralSign} to conclude the result.
  \end{proof}
  \begin{proposition}\label{HA:diffFourierTransf}
    Let $f\in L^1(\RR)$ be such that $f^{(k)}\in L^1(\RR)$ for $k=1,\ldots,r$. Then: $$\widehat{f^{(k)}}(\xi)={(2\pi\ii\xi)}^k\widehat{f}(\xi)$$ for $k=0,1,\ldots, r$.
  \end{proposition}
  \begin{proof}
    We'll prove it by induction on $k$. The case $k=0$ is clear.
    For the other ones note that $\exists(a_n),(b_n)\in\RR$ with $\displaystyle\lim_{n\to\infty}a_n=-\infty$ and $\displaystyle\lim_{n\to\infty}b_n=+\infty$ and such that:
    $$\lim_{n\to\infty}f^{(k-1)}(a_n)=\lim_{n\to\infty}f^{(k-1)}(b_n)=0$$
    Hence using integration by parts:
    \begin{align*}
      \widehat{f^{(k)}}(\xi) & =\lim_{n\to\infty}\int_{a_n}^{b_n} f^{(k)}(x)\exp{-2\pi\ii \xi x}\dd{x}   \\
      \begin{split}
         & =\lim_{n\to\infty}  f^{(k-1)}(x)\exp{-2\pi\ii \xi x}\Big|_{a_n}^{b_n}\dd{x} +                   \\
         & \hspace{2cm}+2\pi\ii \xi\lim_{n\to\infty}\int_{a_n}^{b_n}f^{(k-1)}(x)\exp{-2\pi\ii \xi x}\dd{x}
      \end{split} \\
                             & =\left(2\pi\ii \xi\right)\widehat{f^{(k-1)}}(n)                           \\
                             & ={\left(2\pi\ii \xi\right)}^k\widehat{f}(\xi)
    \end{align*}
  \end{proof}
  \begin{remark}
    Note that there exists functions $f\in\mathcal{C}(\RR)\cap L^1(\RR)$ for which the limit $\displaystyle\lim_{x\to\infty} f(x)$ does not exist.
  \end{remark}
  \begin{proposition}
    Let $f\in L^1(\RR)$ be such that it has compact support. Then, $\F f\in\mathcal{C}^\omega(\RR)$.
  \end{proposition}
  \begin{sproof}
    Suppose $f(x)\in[-K,K]$, $K>0$. Then, expanding $\F f$ with the power series of $\exp{-2\pi\ii\xi x}$ centered at $a\in\RR$ we have:
    \begin{align*}
      \F f(\xi) & =\int_{-K}^Kf(x)\sum_{n=0}^{\infty}\frac{{(-2\pi\ii x)}^n\exp{-2\pi\ii a x}}{n!}{(\xi-a)}^n\dd{x} \\
                & =\sum_{n=0}^\infty c_n{(\xi-a)}^n
    \end{align*}
    where $\abs{c_n}\leq\frac{{(2\pi K)}^n}{n!}\norm{f}_1$. Finally, use this to show that the radius of convergences (see \mcref{MA:radius}) is $\infty$.
  \end{sproof}
  \begin{lemma}\label{HA:expX2}
    Let $f(x)=\exp{-a x^2}$. Then, $\F f(\xi)=\sqrt{\frac{\pi}{a}}\exp{-\frac{{(\pi \xi)}^2}{a}}$ and moreover $\F^2f=f$. In particular if $a=\pi$, then $\F f=f$.
  \end{lemma}
  \begin{sproof}
    $f$ satisfies the ODE $y'=-2a x y$. Taking $\ \widehat{}\ $ on this expression and using \mcref{HA:diffFourierXf,HA:diffFourierTransf} we obtain that $\widehat{f}$ must satisfy the following ODE:
    $$y'=-\frac{2\pi^2\xi}{a} y$$
    with initial condition $y(0)=\int_{-\infty}^{+\infty}\exp{-a x^2}\dd{x}=\sqrt{\frac{\pi}{a}}$.
  \end{sproof}
  \begin{lemma}\label{HA:expAbsX}
    Let $f(x)=\exp{-a\abs{x}}$. Then, $\F f(\xi)=\frac{2a}{a^2+4\pi^2\xi^2}$ and moreover $\F^2f=f$.
  \end{lemma}
  \begin{sproof}
    $$\F f(\xi)=2\int_0^{+\infty}\exp{-ax}\cos(2\pi\xi x)\dd{x}=\frac{2a}{a^2+4\pi^2\xi^2}$$
  \end{sproof}
  \begin{lemma}
    Let $f(x)=\indi{[-a,a]}(x)$, $a>0$. Then, $\F f(\xi)=\frac{\sin(2\pi a\xi)}{\pi\xi}$.
  \end{lemma}
  \subsubsection{The inverse Fourier transform}
  \begin{theorem}[Inversion theorem]\label{HA:inverseFT}
    Let $f\in L^1(\RR)$ such that $\F f\in L^1(\RR)$. Then:
    $$f(x)\almoste{=}\int_{-\infty}^{+\infty}\widehat{f}(\xi)\exp{2\pi \ii \xi x}\dd{\xi}$$
    Moreover if $f$ is continuous we can remove the ``almost everywhere''.
  \end{theorem}
  \begin{proof}
    Consider the integral: $$I_t(x)=\int_{-\infty}^{+\infty}f(x+y)\frac{1}{t}\exp{-\pi \frac{y^2}{t^2}}\dd{y}$$
    Note that using \mcref{HA:expX2} and \mcref{HA:FTprop4}, we have that $\F \left(\frac{1}{\lambda}\exp{-\pi \frac{x^2}{\lambda^2}}\right)=\exp{-\pi\lambda^2\xi^2}$. On the one hand, using this latter thing and \mcref{HA:symmetryFT} we have:
    \begin{multline*}
      I_t(x)=\int_{-\infty}^{+\infty}f(x+y)\frac{1}{t}\exp{-\pi \frac{y^2}{t^2}}\dd{y}=\int_{-\infty}^{+\infty}f(x+\xi)\widehat{\exp{-\pi t^2\xi^2}}\dd{\xi}=\\
      =\int_{-\infty}^{+\infty}\exp{2\pi\ii\xi x}\widehat{f}(\xi)\exp{-\pi t^2\xi^2}\dd{\xi}
    \end{multline*}
    which by \mnameref{RFA:dominated} converges to $\int_{-\infty}^{+\infty}\widehat{f}(\xi)\exp{2\pi \ii \xi x}\dd{\xi}$ as $t\to 0$.

    On the other hand with a change of variable we have: $$I_t(x)=\int_{-\infty}^{+\infty}f(x+ty)\exp{-\pi y^2}\dd{y}$$ Using \mcref{RFA:thmLpBanachC} it suffices to prove that $\displaystyle\lim_{t\to 0}\norm{I_t(x)-f(x)}_1=0$. But using that $\int_{-\infty}^{+\infty}\exp{-\pi y^2}\dd{y}=1$:
    \begin{align*}
      \norm{I_t(x)-f(x)}_1 & =\int_{-\infty}^{+\infty}\abs{\int_{-\infty}^{+\infty}(f(x+ty)-f(x))\exp{-\pi y^2}\dd{y}}\dd{x}  \\
                           & \leq\int_{-\infty}^{+\infty}\exp{-\pi y^2}\int_{-\infty}^{+\infty}\abs{f(x+ty)-f(x)}\dd{x}\dd{y}
    \end{align*}
    where we have used \mnameref{RFA:fubini}. Now use the \mnameref{RFA:dominated}.
  \end{proof}
  \begin{corollary}
    Let $f\in L^1(\RR)$ such that $\F f\almoste{=}0$. Then, $f\almoste{=}0$.
  \end{corollary}
  \begin{corollary}\label{HA:periodicity}
    Let $f\in L^1(\RR)$. Then, $\F^2f(x)\almoste{=}f(-x)$. Hence, $\F^4\almoste{=}\id$.
  \end{corollary}
  \begin{proof}
    By the \mnameref{HA:inverseFT} we have:
    $$f(-x)\almoste{=}\int_{-\infty}^{+\infty}\widehat{f}(\xi)\exp{-2\pi\ii\xi x}\dd{\xi}=\F\widehat{f}(x)=\F^2 f(x)$$
  \end{proof}
  \begin{lemma}
    Let $f,g\in L^1(\RR)$. Then, $f*g\in L^1(\RR)$, $\norm{f*g}_1\leq\norm{f}_1\norm{g}_1$ and $\F(f*g)=\F f \F g$. In particular if $g(x)=\overline{f(-x)}$ then $\F(f*g)=\abs{\widehat{f}}^2$.
  \end{lemma}
  \begin{sproof}
    Show first that $f(x-y)g(y)\in L^1(\RR^2)$ and then use \mnameref{RFA:fubini}.
  \end{sproof}
  \subsubsection{Pointwise convergence}
  \begin{definition}
    Let $f\in L^1(\RR)$. We define the \emph{partial inverse Fourier transform} as: $$S_Rf(x)=\int_{-R}^{R}\widehat{f}(\xi)\exp{2\pi\ii \xi x}\dd{\xi}$$
  \end{definition}
  \begin{definition}[Dirichlet kernel]
    We define the \emph{Dirichlet kernel} of order $R\in\RR_{>0}$ as: $$D_R(t)=\int_{-R}^{R}\exp{-2\pi\ii \xi t}\dd{\xi}=\frac{\sin(2\pi Rt)}{\pi t}$$
  \end{definition}
  \begin{proposition}
    The Dirichlet kernel has the following properties:
    \begin{enumerate}
      \item $D_R$ is an even function.
      \item $\displaystyle \int_{-\infty}^{+\infty}D_R(t)\dd{t}=1$ for all $R>0$.
      \item \begin{align*}
              S_Rf(x) & =(f*D_R)(x)                                  \\
                      & =\int_{-\infty}^{+\infty}f(x-t)D_R(t)\dd{t}  \\
                      & =\int_0^{+\infty}[f(x+t)+f(x-t)]D_R(t)\dd{t}
            \end{align*}
    \end{enumerate}
  \end{proposition}
  \begin{theorem}[Dini's theorem]\label{HA:dini}
    Let $f\in L^1(\RR)$ and $x,\ell\in \RR$ such that $h(t):=\frac{\abs{f(x+t)+f(x-t)-2\ell}}{t}\in L^1((0,\delta))$ for some $\delta>0$. Then, $\displaystyle\lim_{R\to\infty}S_Rf(x)=\ell$.
  \end{theorem}
  \begin{sproof}
    Note that $$S_Rf(x)-\ell=\int_0^\infty[f(x+t)+f(x-t)-2\ell]D_R(t)\dd{t}$$ Now split this integral as a sum of the following ones:
    \begin{align*}
      I_1 & =\int_0^N[f(x+t)+f(x-t)-2\ell]D_R(t)\dd{t} \\
      I_2 & =\int_N^\infty[f(x+t)+f(x-t)]D_R(t)\dd{t}  \\
      I_3 & =-2\ell \int_N^\infty D_R(t)\dd{t}
    \end{align*}
    Given $\varepsilon>0$ take $N$ such that $\int_N^\infty\abs{\frac{f(x+t)+f(x-t)}{\pi t}}\dd{t}<\varepsilon$. Since $h$ is integrable in $(0,N)$, by \mnameref{HA:riemannlebesgue} we have that $I_1\overset{R\to\infty}{\longrightarrow}0$. Then, as we can write $I_3=-2\ell \int_{2\pi RN}^\infty \frac{\sin(u)}{\pi u}\dd{u}$ we have that $I_3\overset{R\to\infty}{\longrightarrow}0$.
  \end{sproof}
  \begin{lemma}\label{HA:translated}
    Let $f\in L^p(\RR)$ with $1\leq p<\infty$. Then, $\displaystyle \lim_{a\to 0}\norm{f-T_af}_p=0$.
  \end{lemma}
  \begin{sproof}
    Clearly is is true if $f\in\mathcal{C}_0^\infty(\RR)$ using \mnameref{RFA:dominated}. Now use that since $\mathcal{C}_0^\infty(\RR)$ is dense in $\mathcal{C}_0(\RR)$, which is dense in $L^p(\RR)$, $\exists(f_n)\in \mathcal{C}_0^\infty(\RR)$ such that $\displaystyle\lim_{n\to\infty}\norm{f_n-f}_p=0$.
  \end{sproof}
  \subsubsection{Uniform convergence}
  \begin{definition}
    Let $f\in L^1(\RR)$ and $R>0$. We define the \emph{Fejér mean} $\sigma_Rf(x)$ as: $$\sigma_Rf(x)=\frac{1}{R}\int_{0}^RS_rf(x)\dd{r}$$
  \end{definition}
  \begin{definition}
    Let $f\in L^1(\RR)$ and $R>0$. We define the \emph{Fejér kernel} $F_Rf(x)$ as: $$F_R(x)=\frac{1}{R}\int_{0}^RD_r(x)\dd{r}$$
  \end{definition}
  \begin{lemma}
    Let $f\in L^1(\RR)$ and $R>0$. Then, $\sigma_Rf=f*F_R$ and moreover:
    $$F_R(x)=\frac{{\left(\sin\left(\pi R x\right)\right)}^2}{\pi^2 R x^2}$$
  \end{lemma}
  \begin{definition}
    Let $t>0$. We define the \emph{Poisson kernel} $P_t$ as $P_t(x):=\F^{-1}(\exp{-2\pi t\abs{\xi}})$.
  \end{definition}
  \begin{lemma}
    Let $f\in L^1(\RR)$ and $t>0$. Then:
    \begin{align*}
      P_t(x)     & =\frac{t}{\pi(t^2+x^2)}                                                                   \\
      (f*P_t)(x) & =\int_{-\infty}^{+\infty}\exp{-2\pi t\abs{\xi}}\widehat{f}(\xi)\exp{2\pi\ii\xi x}\dd{\xi}
    \end{align*}
  \end{lemma}
  \begin{proof}
    Check \mcref{HA:expAbsX} for the first equality. For the other one:
    \begin{align*}
      (f*P_t)(x) & =\int_{-\infty}^{+\infty}f(y)P_t(x-y)\dd{y}                                                                      \\
                 & =\int_{-\infty}^{+\infty}\int_{-\infty}^{+\infty}f(y)\exp{-2\pi t\abs{\xi}}\exp{2\pi\ii \xi (x-y)}\dd{\xi}\dd{y} \\
                 & =\int_{-\infty}^{+\infty}\exp{-2\pi t\abs{\xi}}\widehat{f}(\xi)\exp{2\pi\ii\xi x}\dd{\xi}
    \end{align*}
  \end{proof}
  \begin{definition}
    Let $t>0$. We define the \emph{Weierstra\ss\ kernel} $W_t$ as $W_t(x):=\F^{-1}(\exp{-4\pi^2 t\xi^2})$.
  \end{definition}
  \begin{lemma}\label{HA:weierstrassKernel2}
    Let $f\in L^1(\RR)$ and $t>0$. Then:
    \begin{align*}
      W_t(x)     & =\frac{1}{\sqrt{4\pi t}}\exp{-\frac{x^2}{4t}}                                           \\
      (f*W_t)(x) & =\int_{-\infty}^{+\infty}\exp{-4\pi^2 t\xi^2}\widehat{f}(\xi)\exp{2\pi\ii\xi x}\dd{\xi}
    \end{align*}
  \end{lemma}
  \begin{proof}
    Check \mcref{HA:expX2} for the first equality. For the other one:
    \begin{align*}
      (f*W_t)(x) & =\int_{-\infty}^{+\infty}f(y)W_t(x-y)\dd{y}                                                                    \\
                 & =\int_{-\infty}^{+\infty}\int_{-\infty}^{+\infty}f(y)\exp{-4\pi^2 t\xi^2}\exp{2\pi\ii \xi (x-y)}\dd{\xi}\dd{y} \\
                 & =\int_{-\infty}^{+\infty}\exp{-4\pi^2 t\xi^2}\widehat{f}(\xi)\exp{2\pi\ii\xi x}\dd{\xi}
    \end{align*}
  \end{proof}
  \begin{proposition}
    Let $R>0$ and $t>0$. Then:
    \begin{enumerate}
      \item $F_R$, $P_t$ and $W_t$ are non-negative even functions.
      \item $\int_{-\infty}^{+\infty}F_R(x)\dd{x}=\int_{-\infty}^{+\infty}P_t(x)\dd{x}=\int_{-\infty}^{+\infty}W_t(x)\dd{x}=1$
      \item For all $\delta>0$, we have:
            \begin{multline*}
              \lim_{R\to \infty}\sup_{\abs{x}\geq \delta}F_R(x)=\lim_{t\to 0}\sup_{\abs{x}\geq \delta}P_t(x)=\\=\lim_{t\to 0}\sup_{\abs{x}\geq \delta}W_t(x)=0
            \end{multline*}
      \item\label{HA:propsKernelsitem4} For all $\delta>0$, we have:
            \begin{multline*}
              \lim_{R\to \infty}\int_{\abs{x}\geq \delta}F_R(x)\dd{x}=\lim_{t\to 0}\int_{\abs{x}\geq \delta}P_t(x)\dd{x}=\\=\lim_{t\to 0}\int_{\abs{x}\geq \delta}W_t(x)\dd{x}=0
            \end{multline*}
    \end{enumerate}
    That is, $F_R$, $P_t$ and $W_t$ are \emph{approximations of the identity}.
  \end{proposition}
  \begin{sproof}
    The first two properties are straightforward. For the third one, note that:
    \begin{align*}
      \sup_{\abs{x}\geq \delta}F_R(x) & \leq \frac{1}{\pi^2 R \delta^2}                        \\
      \sup_{\abs{x}\geq \delta}P_t(x) & =\frac{t^2}{\pi(t^2+\delta^2)}                         \\
      \sup_{\abs{x}\geq \delta}W_t(x) & =\frac{1}{\sqrt{4\pi\delta}}\exp{-\frac{x^2}{4\delta}}
    \end{align*}

    The last one is a consequence of the previous ones.
  \end{sproof}
  \begin{theorem}
    Let $f\in L^1(\RR)$ be a function having left- and right-sided limits at point $x_0$. Then:
    \begin{multline*}
      \lim_{R\to\infty}\sigma_Rf(x_0)=\lim_{t\to 0}(f*P_t)(x_0)=\lim_{t\to 0}(f*W_t)(x_0)=\\=\frac{f({x_0}^+)+f({x_0}^-)}{2}
    \end{multline*}
    Moreover if $f$ is uniformly continuous, the convergence is uniform.
  \end{theorem}
  \begin{sproof}
    Copy the proofs of \mnameref{MA:fejerthm0} and \mnameref{MA:fejerthm}.
  \end{sproof}
  \begin{lemma}\label{HA:normLpEquiv}
    Let $E\subseteq\RR^n$ be a measurable space, $p\geq 1$, $f\in L^p(E)$ and $q$ be such that $\frac{1}{p}+\frac{1}{q}=1$. Then:
    $$\norm{f}_p=\sup\left\{\int_Efg:\norm{g}_q=1\right\}$$
  \end{lemma}
  \begin{proof}
    On the one hand using \mnameref{RFA:holder}: $$\int_Efg\leq\norm{fg}_1\leq\norm{f}_p\norm{g}_q=\norm{f}_p$$
    Now consider $g=\frac{\abs{f}^{p-1}\sign f}{{\norm{f}_p}^{\frac{p}{q}}}$. Then, $\norm{g}_q=1$ and moreover: $$\int_Efg=\int_E\frac{\abs{f}^p}{{\norm{f}_p}^{\frac{p}{q}}}={\norm{f}_p}^{p-\frac{p}{q}}=\norm{f}_p$$
  \end{proof}
  \begin{lemma}[Minkowski's integral inequality]\label{HA:minkowski}
    Let $E,F\subseteq\RR^n$ be measurable spaces, $p\geq 1$ and $f\in L^p(E\times F)$. Then:
    $$\norm{\int_F h(\cdot,\vf{y})\dd{\vf{y}}}_p\leq\int_F\norm{h(\cdot,\vf{y})}_p\dd{\vf{y}}$$
  \end{lemma}
  \begin{proof}
    Let $q$ be such that $\frac{1}{p}+\frac{1}{q}=1$ and $g\in L^q(E)$ with $\norm{g}_q=1$. Then, using \mnameref{RFA:holder} and \mnameref{RFA:fubini}:
    \begin{align*}
      \int_Eg(\vf{x})\int_F h(\vf{x},\vf{y})\dd{\vf{y}}\dd{\vf{x}} & =\int_F\int_E h(\vf{x},\vf{y})g(\vf{x})\dd{\vf{x}}\dd{\vf{y}} \\
                                                                   & \leq \int_F\norm{h(\cdot,\vf{y})}_p\norm{g}_q\dd{\vf{y}}      \\
                                                                   & =\int_F\norm{h(\cdot,\vf{y})}_p\dd{\vf{y}}
    \end{align*}
    Now use \mcref{HA:normLpEquiv}.
  \end{proof}
  \begin{theorem}\label{HA:kernelConvLp}
    Let $f\in L^p(\RR)$, $1\leq p\leq\infty$, and $\phi_\varepsilon$ be an approximation of identity. Then:
    \begin{equation*}
      \lim_{\varepsilon\to 0}\norm{f*\phi_\varepsilon-f}_p =0 \\
    \end{equation*}
  \end{theorem}
  \begin{sproof}
    Using \mnameref{HA:minkowski}, we have:
    \begin{align*}
      \norm{f*\phi_\varepsilon-f}_p & =\norm{\int_{-\infty}^{\infty}\phi_\varepsilon(\vf{y})(f(\vf{x}-\vf{y})-f(\vf{x}))\dd{\vf{y}}}_p                                                                                \\
                                    & \leq\int_{-\infty}^{\infty}\!\!{\left[\int_{-\infty}^{\infty}\!{\phi_\varepsilon(\vf{y})}^p\abs{f(\vf{x}-\vf{y})-f(\vf{x})}^p\dd{\vf{x}}\right]}^{\!\frac{1}{p}}\!\!\dd{\vf{y}} \\
                                    & = \int_{-\infty}^{\infty}\phi_\varepsilon(\vf{y})\norm{f-T_{-\vf{y}}f}_p\dd{\vf{y}}                                                                                             \\
      \begin{split}
         & \leq \int_{\abs{\vf{y}}< \delta}\phi_\varepsilon(\vf{y})\norm{f-T_{-\vf{y}}f}_p\dd{\vf{y}}+ \\
         & \hspace{3cm}+2\norm{f}_p\int_{\abs{\vf{y}}\geq\delta}\phi_\varepsilon(\vf{y})\dd{\vf{y}}
      \end{split}
    \end{align*}
    Given $\varepsilon>0$, by \mcref{HA:translated} $\exists\delta>0$ such that the first integral is bounded by $\varepsilon$. Now use this $\delta$ and \mcref{HA:propsKernelsitem4} to conclude that the second integral goes to 0 as $R\to\infty$.
  \end{sproof}
  \begin{corollary}
    Let $f\in L^p(\RR)$ with $1\leq p\leq\infty$. Then:
    \begin{align*}
      \lim_{R\to\infty}\norm{\sigma_Rf-f}_p & =0 \\
      \lim_{t\to 0}\norm{f*P_t-f}_p         & =0 \\
      \lim_{t\to 0}\norm{f*W_t-f}_p         & =0
    \end{align*}
  \end{corollary}
  \subsubsection{Fourier transform on \texorpdfstring{$L^2(\RR)$}{L2(R)}}
  \begin{lemma}\label{HA:lemaPrePlancherel}
    Let $f,g\in L^2(\RR)$. Then, $f*g$ is continuous and bounded. Moreover, $\norm{f*g}_\infty\leq \norm{f}_2\norm{g}_2$.
  \end{lemma}
  \begin{sproof}
    The inequality follows from \mnameref{RFA:holder}. Moreover:
    \begin{multline*}
      \abs{(f*g)(x+h)-(f*g)(x)} \leq\\
      \leq\int_{-\infty}^{+\infty}\abs{f(x+h-y)-f(x-y)}\abs{g(y)}\dd{y}\leq\\
      \leq\norm{g}_2\norm{f-T_{-h}f}_2
    \end{multline*}
    So $f*g$ is continuous, by \mcref{HA:translated}.
  \end{sproof}
  \begin{theorem}[Plancherel theorem]\label{HA:plancherel}
    Let $f\in L^1(\RR)\cap L^2(\RR)$. Then, $\widehat{f}\in L^2(\RR)$ and:
    $$\int_{-\infty}^{+\infty}\abs{f(x)}^2\dd{x}=\int_{-\infty}^{+\infty}\abs{\widehat{f}(\xi)}^2\dd{\xi}$$
  \end{theorem}
  \begin{proof}
    Let $\tilde{f}(x):=\overline{f(-x)}$. Then, $\widehat{\tilde{f}}(\xi)=\overline{\widehat{f}(\xi)}$ and so by \mcref{HA:lemaPrePlancherel} we have that $g:=f*\tilde{f}$ is continuous and bounded. Moreover $\widehat{g}(\xi)=\widehat{f}(\xi)\widehat{\tilde{f}}(\xi)=\abs{\widehat{f}(\xi)}^2$ and $g(0)=\int_{-\infty}^{+\infty}\tilde{f}(-y)f(y)\dd{y}={\norm{f}_2}^2$. On the other hand, by \mcref{HA:weierstrassKernel2} we have:
    \begin{equation}\label{HA:planchEq}
      (g*W_t)(0)=\int_{-\infty}^{+\infty}\exp{-4\pi^2 t\xi^2}\widehat{g}(\xi)\dd{\xi}=\int_{-\infty}^{+\infty}\exp{-4\pi^2 t\xi^2}\abs{\widehat{f}(\xi)}^2\dd{\xi}
    \end{equation}
    And by \mcref{HA:kernelConvLp}, $\displaystyle\lim_{t\to 0^+}(g*W_t)(0)=g(0)={\norm{f}_2}^2$. Thus, by the definition of limit taking $\varepsilon = {\norm{f}_2}^2$, we have that $\abs{\int_{-\infty}^{+\infty}\exp{-4\pi^2 t\xi^2}\widehat{g}(\xi)\dd{\xi}}\leq 2{\norm{f}_2}^2$ for $t$ small enough. Finally, if $t$ is that small, then $1\leq 2\exp{-4\pi^2 t\xi^2}$ and so:
    $$\int_{-\infty}^{+\infty}\abs{\widehat{f}(\xi)}^2\dd{\xi}\leq 2\int_{-\infty}^{+\infty}\widehat{g}(\xi)\exp{-4\pi^2 t\xi^2}\dd{\xi}\leq 4{\norm{f}_2}^2<\infty$$
    Now use \mnameref{RFA:dominated} in \mcref{HA:planchEq} and make $t\to 0$.
  \end{proof}
  \begin{corollary}
    Let $f,g\in L^1(\RR)\cap L^2(\RR)$. Then:
    $$\int_{-\infty}^{+\infty}f(x)\overline{g(x)}\dd{x}=\int_{-\infty}^{+\infty}\widehat{f}(\xi)\overline{\widehat{g}(\xi)}\dd{\xi}$$
  \end{corollary}
  \begin{proof}
    Use \mnameref{HA:plancherel} and \mnameref{RFA:polarization}.
  \end{proof}
  \begin{proposition}\label{HA:preDefFTinL2}
    Let $f\in L^2(\RR)$. Then, $\exists(f_n)\in L^1(\RR)\cap L^2(\RR)$ such that $\displaystyle\lim_{n\to\infty}\norm{f-f_n}_2=0$.
  \end{proposition}
  \begin{sproof}
    Take the sequence $f_n(x)=f(x)\indi{[-n,n]}(x)$.
  \end{sproof}
  \begin{proposition}
    Let $f\in L^2(\RR)$ and $(f_n)\in L^1(\RR)\cap L^2(\RR)$ such that $\displaystyle\lim_{n\to\infty}\norm{f-f_n}_2=0$. Then, the limit $\displaystyle\lim_{n\to\infty}\widehat{f_n}(\xi)$ exists and we will call it $\widehat{f}(\xi)$.
  \end{proposition}
  \begin{proof}
    Since $L^2(\RR)$ is Hilbert, $(f_n)$ is Cauchy. But by \mnameref{HA:plancherel}, $(\widehat{f_n})$ is also Cauchy and so it has limit, because $(\widehat{f_n})\in L^2(\RR)$.

    To see that the definition is well-defined, suppose $(g_n)\in L^1(\RR)\cap L^2(\RR)$ is another sequence such that $\displaystyle\lim_{n\to\infty}\norm{f-g_n}_2=0$. But in this case:
    $$\norm{g_n-f_n}_2\leq\norm{g_n-f}_2+\norm{f-f_n}_2\overset{n\to\infty}{\longrightarrow}0$$
  \end{proof}
  \begin{remark}
    Note that the abuse of notation in the definition of the limit make sense as it coincides with the ordinary Fourier transform when $f\in L^1(\RR)$ (by taking $f_n=f$ $\forall n\in\NN$).
  \end{remark}
  \begin{theorem}
    Let $f,g\in L^2(\RR)$. Then:
    \begin{enumerate}
      \item $\displaystyle\widehat{f}(\xi)\overset{L^2}{=}\lim_{n\to\infty}\int_{-n}^nf(x)\exp{-2\pi\ii \xi x}\dd{x}$
      \item $\norm{f}_2=\norm{\widehat{f}}_2$
      \item $\displaystyle\int_{-\infty}^{+\infty}f(x)\widehat{g}(x)\dd{x}=\int_{-\infty}^{+\infty}\widehat{f}(x)g(x)\dd{x}$
      \item $\displaystyle\int_{-\infty}^{+\infty}f(x)\overline{g(x)}
              \dd{x}=\int_{-\infty}^{+\infty}\widehat{f}(x)\overline{\widehat{g}(x)}\dd{x}$
    \end{enumerate}
  \end{theorem}
  \begin{proof}
    The first property follows from its definition. For the second one, if $\displaystyle f(x)\overset{L^2}{=}\lim_{n\to\infty}f_n(x)$, by \mnameref{HA:plancherel} we have $\norm{f_n}_2=\norm{\widehat{f}}_2$. Now use the continuity of the norm. For the other properties, take the function given in the proof of \mcref{HA:preDefFTinL2} and use the \mnameref{RFA:monotone}.
  \end{proof}
  \begin{proposition}[Jensen's inequality]
    Let $J:\RR\to\RR$ be a convex function, $f$ be a measurable function, and $\mu:\Omega\to\RR$ be measurable with $\int_\Omega\dd{\mu}=1$. Then:
    $$
      \int_\Omega J(f)\dd{\mu}\geq J\left(\int_\Omega f\dd{\mu}\right)
    $$
  \end{proposition}
  \begin{sproof}
    We assume differentiability on $J$ for simplicity. Since $J$ is convex we have that $\forall a,b\in\RR$:
    $$J(b)\geq J(a)+J'(a)(b-a)$$
    Taking $a=\int_\Omega f\dd{\mu}$ and $b=f(x)$, we have:
    $$J(f(x))\geq J\left(\int_\Omega f\dd{\mu}\right)+J'\left(\int_\Omega f\dd{\mu}\right)\!\!\left(f(x)-\int_\Omega f\dd{\mu}\right)$$
    Multiplying by $\dd{\mu}$ and integrating, yields the result.
  \end{sproof}
  \begin{lemma}[Generalized Hölder's inequality]\label{HA:holderGeneralized}
    Let $E\subseteq\RR^n$ be a measurable set, $1\leq p_1,\ldots,p_n\leq \infty$ be such that $\sum_{i=1}^n\frac{1}{p_i}=1$ and ${f_i}\in L^{p_i}(E)$. Then:
    $$\norm{\prod_{i=1}^{n}f_i}_1\leq\prod_{i=1}^{n}\norm{{f_i}}_{p_i}$$
  \end{lemma}
  \begin{proof}
    We will prove it by induction on $n$. For $n=1$ the result is clear. For $n\geq 2$, note that the numbers $q_n=\frac{p_n}{p_n-1}$ and $p_n$ are Hölder conjugates. Moreover, if we define $r_i=p_i\left(1-\frac{1}{p_n}\right)=\frac{p_i}{q_n}$ we have that $\sum_{i=1}^{n-1}\frac{1}{r_i}=1$ and so using \mnameref{RFA:holder} we have:
    \begin{align*}
      \norm{{f_1\cdots f_n}}_1 & \leq \norm{{f_1\cdots f_{n-1}}}_{q_n}\norm{f_n}_{p_n}                                                                    \\
                               & = {\norm{\abs{f_1\cdots f_{n-1}}^{q_n}}_{1}}^{\frac{1}{q_n}}\norm{f_n}_{p_n}                                             \\
                               & \leq {\norm{\abs{f_1}^{q_n}}_{r_1}}^{\frac{1}{q_n}}\cdots {\norm{\abs{f_1}^{q_n}}_{r_1}}^{\frac{1}{q_n}}\norm{f_n}_{p_n} \\
                               & =\norm{f_1}_{p_1}\cdots\norm{f_n}_{p_n}
    \end{align*}
    where in the penultimate step we have used the induction hypothesis and in the last equality we have used the fact that $r_iq_n=p_i$.
  \end{proof}
  \begin{lemma}[Young's convolution inequality]\label{HA:youngConvolution}
    Let $f\in L^p(\RR^n)$, $g\in L^q(\RR^n)$ and take $r$ such that $$\frac{1}{p}+\frac{1}{q}=\frac{1}{r}+1$$ with $1\leq p,q,r\leq \infty$. Then: $$\norm{f*g}_r\leq\norm{f}_p\norm{g}_q$$
  \end{lemma}
  \begin{sproof}
    Note that:
    \begin{multline*}
      \abs{(f*g)(\vf{x})} \\
      \leq\int_{\RR^n}{\left(\abs{f(\vf{y})}^{p}\abs{g(\vf{x}-\vf{y})}^{q}\right)}^\frac{1}{r}\abs{f(\vf{y})}^{1-\frac{p}{r}}\abs{g(\vf{x}-\vf{y})}^{1-\frac{q}{r}}\dd{\vf{y}}    \\
      \leq \norm{{\left(\abs{f(\vf{y})}^{p}\abs{g(\vf{x}-\vf{y})}^{q}\right)}^\frac{1}{r}}_r\norm{\abs{f(\vf{y})}^{\frac{r-p}{r}}}_{\frac{pr}{r-p}}\cdot\\\cdot\norm{\abs{g(\vf{x}-\vf{y})}^{\frac{r-q}{r}}}_{\frac{qr}{r-q}}
    \end{multline*}
    where in the second inequality we have used the \mnameref{HA:holderGeneralized} because:
    $$\frac{1}{r}+\frac{r-p}{pr}+\frac{r-q}{qr}=\frac{1}{p}+\frac{1}{q}-\frac{1}{r}=1$$
    Finally:
    \begin{align*}
      {\norm{f*g}_r}^r & \leq {\norm{f}_p}^{r-p}{\norm{g}_q}^{r-q} \!\int_{\RR^n}\int_{\RR^n}\abs{f(\vf{y})}^{p}\abs{g(\vf{x}-\vf{y})}^{q}\dd{\vf{y}}\dd{\vf{x}} \\
                       & ={\norm{f}_p}^{r-p}{\norm{g}_q}^{r-q} {\norm{f}_p}^{p}{\norm{g}_q}^{q}                                                                  \\
                       & ={\norm{f}_p}^{r}{\norm{g}_q}^{r}
    \end{align*}
    where in the second equality we have used the \mnameref{RFA:fubini}.
  \end{sproof}
  \begin{theorem}
    Let $f\in L^2(\RR)$ and $g\in L^1(\RR)$. Then, $\widehat{f*g}\in L^2(\RR)$ and:$$\widehat{f*g}(\xi)=\widehat{f}(\xi)\widehat{g}(\xi)$$
  \end{theorem}
  \begin{proof}
    By \mnameref{HA:youngConvolution} with $p=r=2$ and $q=1$ we have:
    $$\norm{f*g}_2\leq\norm{f}_2\norm{g}_1<\infty$$
    The equality follows in the same way as in $L^1(\RR)$.
  \end{proof}
  \subsubsection{Fourier transform on \texorpdfstring{$L^p(\RR)$}{Lp(R)}}
  \begin{lemma}
    Let $f\in L^p(\RR)$ with $1<p<2$. Then, there exist functions $f_1\in L^1(\RR)$ and $f_2\in L^2(\RR)$ such that $f=f_1+f_2$.
  \end{lemma}
  \begin{proof}
    The set $E:=\{\abs{f}\geq 1\}$ has finite measure because $f\in L^p(\RR)$. Now consider the Functions
    \begin{gather*}
      f_1(x):=\begin{cases}
        f(x) & \text{if }x\in E    \\
        0    & \text{if }x\notin E
      \end{cases}\qquad
      f_2(x):=\begin{cases}
        0    & \text{if }x\in E    \\
        f(x) & \text{if }x\notin E
      \end{cases}
    \end{gather*}
    By \mnameref{RFA:holder}, $f_1\in L^1(\RR)$ because $\m{E}<\infty$. On the other hand: $$\int_{-\infty}^{\infty}\abs{f_2(x)}^2\dd{x}=\int_{\RR\setminus E}\abs{f(x)}^2\dd{x}\leq \int_{\RR\setminus E}\abs{f(x)}^p\dd{x}<\infty$$
    because $\abs{f}<1$ in $\RR\setminus E$.
    So $f_2\in L^2(\RR)$.
  \end{proof}
  \begin{definition}
    Let $f=f_1+f_2\in L^p(\RR)$ with $1<p<2$, $f_1\in L^1(\RR)$ and $f_2\in L^2(\RR)$. We define the \emph{Fourier transform} of $f$ as:
    $$\widehat{f}(\xi):=\widehat{f_1}(\xi)+\widehat{f_2}(\xi)$$
  \end{definition}
  \begin{remark}
    This definition is well-defined. Indeed, suppose $f=g_1+g_2$ with $1<p<2$ with $g_1\in L^1(\RR)$ and $g_2\in L^2(\RR)$. Then, $f_1-g_1=g_2-f_2\in L^1(\RR)\cap L^2(\RR)$ and so $$\widehat{f_1}-\widehat{g_1}=\widehat{f_1-g_1}=\widehat{f_2-g_2}=\widehat{f_2}-\widehat{g_2}$$
    Hence, $\widehat{f}=\widehat{f_1}+\widehat{f_2}=\widehat{g_1}+\widehat{g_2}$.
  \end{remark}
  \subsubsection{Fourier transform on \texorpdfstring{$\RR^n$}{Rn}}
  In this section we will only expose the most important results of extending the Fourier transform to $L^1(\RR^n)$. Moreover we will not prove any of the results of this section as they are completely analogous to the previous ones.
  \begin{definition}
    Let $f\in L^1(\RR^n)$. We define the \emph{Fourier transform} of $f$ as:
    $$\widehat{f}(\vf\xi)=\int_{\RR^n}f(\vf{x})\exp{-2\pi \ii\langle \vf\xi, \vf{x}\rangle}\dd{\vf{x}}$$
    The function $f$ is also called \emph{inverse Fourier transform} of $\widehat{f}$.
  \end{definition}
  \begin{proposition}\label{HA:fourierPropertiesRn}
    Let $f,g\in L^1(\RR^n)$ and $\alpha,\beta\in\RR$. Then:
    \begin{enumerate}
      \item $\widehat{(\alpha f+\beta g)}(\vf\xi)=\alpha\widehat{f}(\vf\xi)+\beta \widehat{g}(\vf\xi)$
      \item Let $\vf{h}\in\RR^n$. We define $T_{\vf{h}}f(x)=f(\vf{x}+\vf{h})$. Then: $$\widehat{T_{\vf{h}}f}(\vf\xi)=\exp{2\pi\ii \langle\vf\xi, \vf{h}\rangle}\widehat{f}(\vf\xi)$$
      \item If $g(\vf{x})=\exp{2\pi\ii \langle\vf{x}, \vf{h}\rangle}f(\vf{x})$, then: $$\widehat{g}(\vf\xi)=\widehat{f}(\vf\xi-\vf{h})$$
      \item If $\lambda\in\RR^*$, then: $$\frac{1}{\lambda^n}\widehat{f\left(\frac{\vf{x}}{\lambda}\right)}(\vf\xi)=\widehat{f}(\lambda\vf\xi)$$
      \item If $g(\vf{x})=\overline{f(\vf{x})}$, then: $$\widehat{g}(\vf\xi)=\overline{\widehat{f}(-\vf\xi)}$$
    \end{enumerate}
  \end{proposition}
  \begin{theorem}
    Let $f\in L^1(\RR^n)$ and denote also by $\F$ the extension of the Fourier transform operator to $L^1(\RR^n)$. Then:
    \begin{enumerate}
      \item $\F f$ is uniformly continuous.
      \item $\F$ is a continuous linear operator from $L^1(\RR^n)$ to $L^\infty(\RR^n)$ and $\norm{\F f}_{\infty}\leq \norm{f}_1$.
    \end{enumerate}
  \end{theorem}
  \begin{theorem}[Riemann-Lebesgue lemma]
    Let $f\in L^1(\RR^n)$. Then:
    $$\lim_{\norm{\vf\xi}\to\infty} \abs{\widehat{f}(\vf\xi)}=0$$
  \end{theorem}
  \begin{proposition}
    Let $f$ be a function such that $\xi_j f\in L^1(\RR^n)$. Then, $\widehat{f}$ is differentiable with respect to $\xi_j$ and:
    $$\pdv{(\F f)}{\xi_j}(\vf\xi)=\F({(-2\pi\ii \xi_j)}f(\vf{x}))$$
  \end{proposition}
  \begin{proposition}
    Let $f\in L^1(\RR^n)$ be differentiable with respect to $x_j$ such that $\pdv{f}{x_j}\in L^1(\RR^n)$. Then: $$\widehat{\pdv{f}{x_j}}(\vf\xi)={2\pi\ii\xi_j}\widehat{f}(\vf\xi)$$
  \end{proposition}
  \begin{theorem}[Plancherel theorem]
    Let $f\in L^1(\RR^n)\cap L^2(\RR^n)$. Then, $\widehat{f}\in L^2(\RR^n)$ and:
    $$\int_{\RR^n}\abs{f(\vf{x})}^2\dd{\vf{x}}=\int_{\RR^n}\abs{\widehat{f}(\vf\xi)}^2\dd{\vf\xi}$$
  \end{theorem}
  \subsubsection{Applications of the Fourier transform}
  \begin{remark}
    Probably the most important application of Fourier series is the resolution of PDEs and it is a consequence of \mcref{HA:diffFourierTransf}, which reduces any order of a PDE in the spatial variable to 1. The procedure is to compute the Fourier transform $\F$ of the PDE, solve it, and then get back to the first function using the inverse transform.
  \end{remark}
  \begin{theorem}[Uncertainty principle]
    Let $f\in L^2(\RR)$ be differentiable such that $x\abs{f}^2\in L^1(\RR)$ and $f'\in L^2(\RR)$. Then:
    $$\left(\int_{-\infty}^\infty x^2 \abs{f(x)}^2\dd{x}\right)\left(\int_{-\infty}^\infty \xi^2 \abs{\widehat{f}(\xi)}^2\dd{\xi}\right)\geq \frac{{\norm{f}_2}^4}{16\pi^2}$$
    and the equality holds if and only if $f(x)=\exp{-\lambda^2x^2}$, $\lambda\in\RR$.
  \end{theorem}
  \begin{proof}
    Let $I$ be the left-hand-side term of the inequality. First note that:
    $$\int_{-\infty}^\infty \!\xi^2 \abs{\widehat{f}(\xi)}^2\dd{\xi}=\frac{1}{4\pi^2}\int_{-\infty}^\infty \abs{\widehat{f'}(\xi)}^2\dd{\xi}=\frac{1}{4\pi^2}\int_{-\infty}^\infty \abs{{f'}(\xi)}^2\dd{\xi}$$
    where the first equality is by the analogous \mcref{HA:diffFourierTransf} in $L^2(\RR)$ and in the second one we have used \mnameref{HA:plancherel}.
    Now by the \mref{RFA:cauchyschwarz}, we have:
    \begin{align*}
      I & \geq \frac{1}{4\pi^2}{\left(\int_{-\infty}^\infty x\abs{f(x){f'}(x)}\dd{x}\right)}^2         \\
        & \geq \frac{1}{4\pi^2}{\left(\int_{-\infty}^\infty x\Re(f(x)\overline{f'(x)})\dd{x}\right)}^2 \\
        & = \frac{1}{16\pi^2}{\left(\int_{-\infty}^\infty x\dv{}{x}(\abs{f(x)}^2)\dd{x}\right)}^2      \\
        & = \frac{1}{16\pi^2}{\left(\int_{-\infty}^\infty \abs{f(x)}^2\dd{x}\right)}^2                 \\
    \end{align*}
    where in the third step we have used that:
    $$\dv{}{x}(\abs{f(x)}^2)=2\Re(f(x)\overline{f'(x)})$$
    and in the last step we have integrated by parts (here we use that $x\abs{f}^2\in L^1(\RR)$).
  \end{proof}
  \begin{theorem}[Uncertainty principle in $\RR^n$]
    Let $f\in L^2(\RR^n)$ be regular enough. Then:
    $$\left(\int_{-\infty}^\infty \norm{\vf{x}}^2 \abs{f(\vf{x})}^2\dd{\vf{x}}\right)\left(\int_{-\infty}^\infty \norm{\vf\xi}^2 \abs{\widehat{f}(\vf\xi)}^2\dd{\vf\xi}\right)\geq \frac{n^2{\norm{f}_2}^4}{16\pi^2}$$
  \end{theorem}
  \begin{theorem}[Poisson summation formula]
    Let $f\in\mathcal{C}(\RR)\cap L^1(\RR)$ be such that $\sum_{k\in\ZZ}f(x+k)$ converges uniformly for $x\in[0,1]$ and such that $\sum_{k\in\ZZ}\abs{\widehat{f}(k)}<\infty$. Then:
    $$\sum_{k\in\ZZ}f(x+k)=\sum_{k\in\ZZ}\widehat{f}(k)\exp{2\pi\ii k x}$$
    In particular, for $x=0$ we have:
    $$\sum_{k\in\ZZ}f(k)=\sum_{k\in\ZZ}\widehat{f}(k)$$
  \end{theorem}
  \begin{proof}
    Let $F(x):=\sum_{k\in\ZZ}f(x+k)$. Note that $F$ is 1-periodic. If we see that $\widehat{F}(n)=\widehat{f}(n)$ $\forall n\in\ZZ$, the continuity of $f$ and the convergence of its Fourier series will imply $F(x)=\sum_{k\in\ZZ}\widehat{f}(k)\exp{2\pi\ii k x}$. But:
    \begin{align*}
      \widehat{F}(n) & =\int_{0}^{1}\sum_{k\in\ZZ}f(x+k)\exp{-2\pi\ii n x}\dd{x}     \\
                     & =\sum_{k\in\ZZ}\int_{0}^{1}f(x+k)\exp{-2\pi\ii n x}\dd{x}     \\
                     & =\sum_{k\in\ZZ}\int_{k}^{k+1}f(x)\exp{-2\pi\ii n (x-k)}\dd{x} \\
                     & =\sum_{k\in\ZZ}\int_{k}^{k+1}f(x)\exp{-2\pi\ii n x}\dd{x}     \\
                     & =\int_{-\infty}^{\infty}f(x)\exp{-2\pi\ii n x}\dd{x}          \\
                     & =\widehat{f}(n)
    \end{align*}
  \end{proof}
  \begin{definition}
    Let $f\in L^2(\RR)$. We say that $f$ is \emph{bandlimited} if $\exists B\in\RR$ such that $\supp\widehat{f}\subseteq[-B,B]$.
  \end{definition}
  \begin{theorem}[Nyquist-Shannon sampling theorem]
    Let $f\in L^2(\RR)$ be bandlimited with constant $B$. Then:
    $$f(x)\overset{L^2}{=}\sum_{k\in\ZZ}f\left(\frac{k}{2B}\right)\frac{\sin(\pi(2Bx-k))}{\pi(2Bx-k)}$$
    Moreover: $${\norm{f}_2}^2=\frac{1}{2B}\sum_{k\in\ZZ}\abs{f\left(\frac{k}{2B}\right)}^2$$
  \end{theorem}
  \begin{proof}
    An easy check shows that the Fourier series of $\xi\mapsto\exp{2\pi\ii x\xi}$ on $[-B, B]$ is: $$\exp{2\pi\ii x\xi}=\sum_{k\in\ZZ}\frac{\sin(\pi(2Bx-k))}{\pi(2Bx-k)}\exp{\frac{\pi\ii k\xi}{B}}$$
    Thus:
    \begin{align*}
      f(x) & =\int_{-B}^{B}\widehat{f}(\xi)\exp{2\pi\ii \xi x}\dd{\xi}                                                           \\
           & =\sum_{k\in\ZZ}\frac{\sin(\pi(2Bx-k))}{\pi(2Bx-k)}\int_{-B}^{B}\widehat{f}(\xi)\exp{\frac{\pi\ii k\xi}{B}} \dd{\xi} \\
           & =\sum_{k\in\ZZ}f\left(\frac{k}{2B}\right)\frac{\sin(\pi(2Bx-k))}{\pi(2Bx-k)}
    \end{align*}
    The second equality follows from both \mnameref{HA:plancherel} and \mnameref{MA:parseval}:
    $${\norm{f}_2}^2={\norm{\widehat{f}}_2}^2=\frac{1}{2B}\sum_{k\in\ZZ}\abs{f\left(\frac{k}{2B}\right)}^2$$
    because by a similar argument as before, the Fourier coefficients of $\widehat{f}(k)$ (thought as periodically extended) are $\frac{1}{2B}f\left(\frac{-k}{2B}\right)$.
  \end{proof}
  \subsubsection{Discrete Fourier transform}
  \begin{definition}
    Consider a function $f$ with support $\{0,\ldots,N-1\}$. We can think $f$ as:
    $$\function{f}{\ZZ}{\CC}{k}{f(k\mod{N})=:f[k]}$$
    Note that with this definition, $f$ is $N$-periodic. We define the \emph{discrete Fourier transform} (\emph{DFT}) of $f$ as:
    $$\widehat{f}[k]:=\sum_{n=0}^{N-1}f[n]\exp{-\frac{2\pi\ii nk}{N}}$$
    If we denote $\omega_N:=\exp{-\frac{2\pi\ii}{N}}$ we can write:
    $$\widehat{f}[k]=\sum_{n=0}^{N-1}f[n]{\omega_N}^{kn}$$
    We will denote $\vf{f}:=(f[0],\ldots,f[N-1])$
  \end{definition}
  \begin{proposition}
    Let $f,g:\ZZ\rightarrow\CC$. Then:
    \begin{enumerate}
      \item $\widehat{f}$ is linear.
      \item If $n\in\ZZ$ and $g[k]=f[k-n]$ $\forall k\in\ZZ$, then: $$\widehat{g}[k]=\widehat{f}[k]\exp{-\frac{2\pi\ii k n}{N}}$$
      \item If $g[k]=\overline{f[k]}$ $\forall k\in\ZZ$, then: $$\widehat{g}[k]=\overline{\widehat{f}[N-k]}$$
    \end{enumerate}
  \end{proposition}
  \begin{proposition}
    Let $f:\ZZ\rightarrow\CC$. Then, $\vf{\widehat{f}}=\vf{A}(\omega_N)\vf{f}$, where
    $$\vf{A}(\omega_N)=\!\begin{pmatrix}
        1      & 1                & 1                   & \cdots & 1                       \\
        1      & \omega_N         & {\omega_N}^2        & \cdots & {\omega_N}^{N-1}        \\
        1      & {\omega_N}^2     & {\omega_N}^4        & \cdots & {\omega_N}^{2(N-1)}     \\
        \vdots & \vdots           & \vdots              & \ddots & \vdots                  \\
        1      & {\omega_N}^{N-1} & {\omega_N}^{2(N-1)} & \cdots & {\omega_N}^{(N-1)(N-1)}
      \end{pmatrix}$$
  \end{proposition}
  is a symmetric matrix.
  \begin{lemma}
    Let $N\in\NN$. Then: $$\vf{A}(\omega_N)\vf{A}(\overline{\omega_N})=\vf{A}(\overline{\omega_N})\vf{A}(\omega_N)=N\vf{I}_N$$
  \end{lemma}
  \begin{sproof}
    Remember that both ${\omega_N}$ and $\overline{\omega_N}$ are roots of $1+x+\cdots+x^{N-1}$.
  \end{sproof}
  \begin{definition}
    Let $f:\ZZ\rightarrow\CC$. We define the \emph{inverse discrete Fourier transform} as:
    $$\vf{f}=\frac{1}{N}\vf{A}(\overline{\omega_N})\vf{\widehat{f}}$$
  \end{definition}
  \begin{theorem}[Plancherel theorem]
    Let $f:\ZZ\rightarrow\CC$. Then:
    $$\sum_{k=0}^{N-1}f[k]\overline{g[k]}=\frac{1}{N}\sum_{k=0}^{N-1}\widehat{f}[k]\overline{\widehat{g}[k]}$$
    In particular, if $f=g$, we have:
    $$\sum_{k=0}^{N-1}\abs{f[k]}^2=\frac{1}{N}\sum_{k=0}^{N-1}\abs{\widehat{f}[k]}^2$$
  \end{theorem}
  \begin{proof}
    Using vector notation:
    \begin{align*}
      \dotp{\transpose{\vf{f}}}{\overline{\vf{g}}} & =\transpose{\left(\frac{1}{N}\vf{A}(\overline{\omega_N})\vf{\widehat{f}}\right)}\left(\frac{1}{N}\overline{\vf{A}(\overline{\omega_N})}\vf{\overline{\widehat{g}}}\right) \\
                                                   & =\frac{1}{N^2}\transpose{\vf{\widehat{f}}}\transpose{\vf{A}(\overline{\omega_N})}\vf{A}({\omega_N})\vf{\overline{\widehat{g}}}                                            \\
                                                   & =\frac{1}{N}\dotp{\transpose{\vf{\widehat{f}}}}{\vf{\overline{\widehat{g}}}}
    \end{align*}
    because $\vf{A}(\overline{\omega_N})$ is symmetric.
  \end{proof}
  \begin{definition}
    Let $f,g:\ZZ\rightarrow\CC$. We define the \emph{convolution} of $f$ and $g$ as:
    $$(f*g)[k]:=\sum_{n=0}^{N-1}f[n]g[k-n]$$
  \end{definition}
  \begin{lemma}
    Let $f,g:\ZZ\rightarrow\CC$. Then:
    $$\widehat{f*g}[k]=\widehat{f}[k]\widehat{g}[k]$$
  \end{lemma}
  \begin{proof}
    \begin{align*}
      \widehat{f*g}[k] & =\sum_{n=0}^{N-1}\sum_{j=0}^{N-1}f[j]g[n-j]{\omega_N}^{nk}                    \\
                       & =\sum_{j=0}^{N-1}f[j]{\omega_N}^{jk}\sum_{n=0}^{N-1}g[n-j]{\omega_N}^{(n-j)k} \\
                       & =\widehat{f}[k]\widehat{g}[k]
    \end{align*}
  \end{proof}
  \begin{theorem}[Poisson summation formula]
    Let $f:\ZZ\rightarrow\CC$. Then:
    $$\sum_{k=0}^{N-1}\widehat{f}[k]=Nf[0]$$
  \end{theorem}
  \begin{proof}
    $$\sum_{k=0}^{N-1}\widehat{f}[k]=\sum_{k,n=0}^{N-1}f[n]{\omega_N}^{kn}=Nf[0]$$
    because $\sum_{k=0}^{N-1}{\omega_N}^{kn}=N$ if $n=0$ and 0 otherwise because ${\omega_N}^{n}$ are roots of $1+x+\cdots+x^{N-1}$.
  \end{proof}
  \subsubsection{Fast Fourier transform}
  \begin{definition}
    Let $f:\ZZ\rightarrow\CC$. Note that we need $\O{N^2}$ operations in order to compute $\widehat{f}$. The \emph{fast Fourier transform} (\emph{FFT}) aims to minimize that number by using some tricks.
  \end{definition}
  \begin{definition}[Radix-2 DIT Cooley-Tukey FFT algorithm]
    Let $f:\ZZ\rightarrow\CC$ and assume that $N=2m$. The \emph{radix-2 decimation-in-time (DIT) FFT} is defined as follows. We can write:
    \begin{multline*}
      \widehat{f}[k] =\sum_{n=0}^{N/2-1}f[2n]{\omega_N}^{2nk}+\sum_{n=0}^{N/2-1}f[2n+1]{\omega_N}^{(2n+1)k}    \\ =\!\sum_{n=0}^{N/2-1}\!f[2n]{\left(\exp{-\frac{2\pi\ii}{N/2}}\right)}^{nk}\!+\exp{-\frac{2\pi\ii k}{N}}\!\sum_{n=0}^{N/2-1}\!f[2n+1]{\left(\exp{-\frac{2\pi\ii}{N/2}}\right)}^{nk} \\
      =:E_k+\exp{-\frac{2\pi\ii k}{N}}O_k
    \end{multline*}
    for $k=0,\ldots,N/2-1$ even though the equality holds for $k=0,\ldots,N-1$. For the other cases, we use the periodicity of $\exp{-\frac{2\pi\ii k}{N}}$ to get:
    $$\widehat{f}[k+N/2] =E_k-\exp{-\frac{2\pi\ii k}{N}}O_k$$
    for $k=0,\ldots,N/2-1$.
    Note that $E_k$ and $O_k$ are both $N/2$-dimensional DFT of the even terms of $f$ and the odd terms of $f$, respectively. We can thus compute them recursively until the respective $m$ is odd. Using this method we can get the DFT of $f$ in at most (when $N=2^\ell$) $\O{N\log N}$ time.
  \end{definition}
  \subsection{Distributions}
  \subsubsection{Introduction}
  \begin{definition}
    Let $\Omega\subseteq \RR^n$ and $(\varphi_n),\varphi\in\mathcal{D}(\Omega):=\mathcal{C}_0^\infty(\Omega)$. The functions on $\mathcal{D}(\Omega)$ are usually called \emph{bump functions} or \emph{test functions}. We say that $\varphi_n\rightarrow \varphi$ in $\mathcal{D}(\Omega)$ if:
    \begin{enumerate}
      \item There exists a compact set such that $\supp \varphi_n, \supp \varphi\subseteq K$ $\forall n\in\NN$.
      \item $\displaystyle\lim_{n\to\infty}\norm{\partial^\alpha\varphi_n-\partial^\alpha\varphi}_{L^\infty(K)}=0$ $\forall \alpha\in{(\NN\cup\{0\})}^d$.
    \end{enumerate}
  \end{definition}
  \begin{definition}[Distribution]
    Let $\Omega\subseteq \RR^d$ be a set. A \emph{distribution} on $\Omega$ is a continuous linear form on $\mathcal{D}(\Omega)$. The vector space of all distributions on $\Omega$ is denoted by $\mathcal{D}^*(\Omega)$.
  \end{definition}
  \begin{lemma}
    Let $\Omega\subseteq\RR^d$ and $T:\mathcal{D}(\Omega)\rightarrow\CC$ be linear. Then, $T$ is continuous if and only if $\forall (\varphi_n)\in\mathcal{D}(\Omega)$ with $\varphi_n\rightarrow 0$ in $\mathcal{D}(\Omega)$ we have that $T(\varphi_n)\rightarrow 0$\footnote{Sometimes we will denote $T(\varphi)$ as $\langle T,\varphi\rangle$.}.
  \end{lemma}
  \begin{lemma}[Fundamental lemma of calculus of variations]\label{HA:fundamentallemma}
    Let $\Omega\subseteq\RR^d$ be a domain and $f\in L_\mathrm{loc}^1(\Omega)$ such that $$\int_\Omega f(\vf{x})\varphi(\vf{x})\dd{\vf{x}}=0$$ for all $\varphi\in\mathcal{D}(\Omega)$. Then, $f\almoste{=}0$ in $\Omega$.
  \end{lemma}
  \begin{proposition}
    Let $\Omega\subseteq\RR^d$ and $T:\mathcal{D}(\Omega)\rightarrow\CC$ be linear. Then, $T\in \mathcal{D}^*(\Omega)$ if and only if for all compact set $K\subseteq \Omega$, there exist $C>0$ and $m\in\NN\cup\{0\}$ such that $\forall \varphi\in\mathcal{D}(K)$ we have:
    $$\abs{T(\varphi)}\leq C\sum_{\abs{\alpha}\leq m}\norm{\partial^\alpha\varphi}_{L^\infty(K)}$$
  \end{proposition}
  \begin{proof}
    The right-to-left implication is clear. For the other one, suppose that there exists a compact set $K$ such that $\forall C>0$ and all $m\in\NN\cup\{0\}$ there exists a sequence $(\varphi_k)\in\mathcal{D}(\Omega)$ such that:
    $$\abs{T(\varphi_k)}> C\sum_{\abs{\alpha}\leq m}\norm{\partial^\alpha\varphi_k}_{L^\infty(K)}=:C\norm{\varphi_k}_{m,K}$$
    Now consider $\psi_k:=\frac{\varphi_k}{k\norm{\varphi_k}_{m,K}}$. Clearly $\forall \alpha\in {(\NN\cup\{0\})}^d$ $\norm{\partial^\alpha\psi_k}_{L^\infty(K)}\leq\frac{1}{k}\overset{k\to\infty}{\longrightarrow}0$ but $\abs{T(\psi_k)}=\frac{\abs{T(\varphi_k)}}{k\norm{\varphi_k}_{m,K}}>1$ by considering the particular case of $C=k$. Hence, $T$ cannot be continuous, which is a contradiction.
  \end{proof}
  \begin{proposition}
    Let $\Omega\subseteq \RR^n$ and $f\in L_{\mathrm{loc}}^1(\Omega)$. Then, the map
    \begin{equation}\label{HA:regularDistribution}
      \function{T_f}{\mathcal{D}(\Omega)}{\CC}{\varphi}{\displaystyle\int_\Omega f(\vf{x})\varphi(\vf{x})\dd{\vf{x}}}
    \end{equation}
    is a distribution. Hence, $T_f(\varphi)$ is usually denoted by $\dotp{f}{\varphi}$. Sometimes we will do an abuse of notation denoting $T_f$ as $f$ (in view of the \mnameref{HA:fundamentallemma}).
  \end{proposition}
  \begin{proof}
    $T_f$ is clearly linear. Moreover: $$\abs{T_f(\varphi)}\leq\int_\Omega\abs{f(\vf{x})\varphi(\vf{x})}\leq \norm{f}_1\norm{\varphi}_\infty$$
    Hence, $T_f$ is bounded and therefore continuous.
  \end{proof}
  \begin{definition}
    The distributions that can be expressed as in \mcref{HA:regularDistribution} are called \emph{regular distributions}.
  \end{definition}
  \begin{proposition}[Dirac's $\delta$ distribution]
    Let $\Omega\subseteq \RR^d$ be a set and $\vf{x}_0\in\Omega$. Then, the map
    $$
      \function{\delta_{\vf{x}_0}}{\mathcal{D}(\Omega)}{\RR}{\varphi}{\varphi(\vf{x}_0)}
    $$
    is a distribution and it is called \emph{Dirac's $\delta$ distribution}. We will denote $\delta_{\vf{0}}$ simply by $\delta$.
  \end{proposition}
  \begin{proof}
    Clearly $\delta_{\vf{x}_0}$ is linear and bounded because $\abs{\delta_{\vf{x}_0}(\varphi)}=\abs{\varphi(\vf{x}_0)}\leq \norm{\varphi}_\infty$.
  \end{proof}
  \begin{lemma}
    The Dirac's $\delta_{\vf{0}}$ distribution is not regular.
  \end{lemma}
  \begin{proof}
    Suppose it is regular. Then, $\exists f\in L_{\mathrm{loc}}^1(\Omega)$ such that $\delta=T_f$. Hence, $\varphi(\vf{0})=\delta(\varphi)=\int_\Omega f(\vf{x})\varphi(\vf{x})\dd{\vf{x}}$ for all $\varphi\in\mathcal{D}(\Omega)$. Then, if we take $\varphi_n(\vf{x}):=\varphi(n\vf{x})$, where:
    $$\varphi(x)=\begin{cases}
        \exp{-\frac{1}{1-\norm{\vf{x}}^2}} & \text{if } \norm{\vf{x}}\leq 1 \\
        0                                  & \text{if } \norm{\vf{x}}> 1
      \end{cases}
    $$
    then $\varphi_n\in\mathcal{D}(\Omega)$ and have support $\overline{B(\vf{0},1/n)}$. So:
    $$\exp{-1}=\abs{\int_{\Omega\cap \overline{B(\vf{0},1/n)}} f(\vf{x})\varphi_n(\vf{x})\dd{\vf{x}}}\leq \int_{\norm{x}<\frac{1}{n}}\abs{f(\vf{x})}\dd{\vf{x}}\overset{n\to\infty}{\longrightarrow}0$$
  \end{proof}
  \begin{proposition}[Cauchy principal value]
    We define the \emph{Cauchy principal value} $T:=\pv\left(\frac{1}{x}\right)$ as the distribution
    $$T(\varphi)=\lim_{\varepsilon\to 0}\int_{\abs{x}\geq \varepsilon}\frac{\varphi(x)}{x}\dd{x}$$
  \end{proposition}
  \begin{proof}
    First of all note that it is well defined because we can write:
    $$T(\varphi)=\lim_{\varepsilon\to 0}\int_{\varepsilon}^\infty\frac{\varphi(x)-\varphi(-x)}{x}\dd{x}=\int_{0}^\infty\frac{\varphi(x)-\varphi(-x)}{x}\dd{x}$$
    which is well-defined because $\varphi$ has compact support and in a neighborhood of $0$ the integrand is bounded (by the \mnameref{RVF:meanvaluetheorem}). Moreover it is clearly linear and continuous because $$\abs{T(\varphi)}\leq 2 \m{K} \norm{\varphi'}_\infty$$
    where $\abs{K}$ is the measure of the support of $\varphi$.
  \end{proof}
  \begin{definition}
    Let $\Omega\subseteq \RR^n$. We say that a distribution $T\in\mathcal{D}^*(\Omega)$ is a \emph{distribution of order $N\in\NN\cup\{0\}$} if $\exists N\in\NN\cup\{0\}$ such that for all compact set $K$ $\exists C_K>0$ with
    $$\abs{T(\varphi)}\leq C_K\norm{\varphi}_{N,K}$$
    for all $\varphi\in\mathcal{D}(\Omega)$. We say that $T$ is has \emph{infinite order} if it is not of order $N$ for any $N\in\NN$.
  \end{definition}
  \begin{definition}
    Let $\Omega\subseteq \RR^n$, $T,S\in\mathcal{D}^*(\Omega)$, $a\in\RR$ and $f\in\mathcal{C}^\infty(\RR^n)$. We define the distributions $T+S$, $aT$ and $fT$ as:
    \begin{align*}
      \langle T+S,\varphi\rangle & := \langle T, \varphi\rangle+\langle S, \varphi\rangle \\
      \langle aT,\varphi\rangle  & := \langle T, a\varphi\rangle                          \\
      \langle fT,\varphi\rangle  & := \langle T, f\varphi\rangle
    \end{align*}
  \end{definition}
  \begin{remark}
    In general the product of two distributions is not associative. For example, one can check that $\delta x=0$ and $x\pv\left(\frac{1}{x}\right)=1$. So:
    $$\left(\delta x\right)\pv\left(\frac{1}{x}\right)\ne\delta \left(x\pv\left(\frac{1}{x}\right)\right)$$
  \end{remark}
  \subsubsection{Convergence of distributions}
  \begin{definition}
    Let $\Omega\subseteq \RR^n$ be a set and $(T_n)\in \mathcal{D}^*(\Omega)$. We say that $(T_n)$ converges to $T\in\mathcal{D}^*(\Omega)$ if $T_n(\varphi)\overset{n\to\infty}{\longrightarrow}T(\varphi)$ for all $\varphi\in\mathcal{D}(\Omega)$.
  \end{definition}
  \begin{definition}
    Let $\Omega\subseteq \RR^n$ be a set. We say that a sequence of functions $(\phi_\varepsilon)\in L_\mathrm{loc}^1(\Omega)$ is an \emph{approximation of identity} if
    \begin{enumerate}
      \item $\displaystyle\int_\Omega \phi_\varepsilon=1$
      \item $\displaystyle\int_\Omega \abs{\phi_\varepsilon}\leq M$ $\forall \varepsilon>0$
      \item $\displaystyle\lim_{\varepsilon\to 0}\int_{\norm{\vf{x}}\geq \delta}\phi_\varepsilon(\vf{x})\dd{\vf{x}}=0$ $\forall \delta>0$.
    \end{enumerate}
  \end{definition}
  \begin{proposition}
    Let $\Omega\subseteq \RR^n$ be a set and $\phi\in L^1(\Omega)$ such that $\int_\Omega\phi=1$. Let $\phi_\varepsilon:=\frac{1}{\varepsilon^n}\phi(\frac{\vf{x}}{\varepsilon})$. Then, $(\phi_\varepsilon)$ is an approximation of identity, $\phi_\varepsilon \in L_\mathrm{loc}^1(\Omega)$ $\forall \varepsilon>0$ and $\phi_\varepsilon\overset{\varepsilon\to 0}{\longrightarrow} \delta_{\vf{0}}$ in $\mathcal{D}^*(\Omega)$.
  \end{proposition}
  \begin{sproof}
    Let $\varphi\in\mathcal{D}(\Omega)$. Then:
    \begin{align*}
      \abs{\phi_\varepsilon(\varphi)-\delta_{\vf{0}}(\varphi)} & \leq \int_\Omega \abs{\phi_\varepsilon(x)}\abs{\varphi(\vf{x})-\varphi(\vf{0})}\dd{\vf{x}} \\
      \begin{split}
         & =\int_{\norm{\vf{x}}<\delta} \abs{\phi_\varepsilon(x)}\abs{\varphi(\vf{x})-\varphi(\vf{0})}\dd{\vf{x}}+               \\
         & \hspace{1cm}+\int_{\norm{\vf{x}}\geq\delta} \abs{\phi_\varepsilon(x)}\abs{\varphi(\vf{x})-\varphi(\vf{0})}\dd{\vf{x}}
      \end{split}
    \end{align*}
    Now use the properties of approximation of identity to see that each interval goes to zero as $\varepsilon\to 0$.
  \end{sproof}
  \begin{theorem}
    Let $\Omega\subseteq \RR^n$ and $(f_n)\in L_\mathrm{loc}^p(\Omega)$ such that $f_n\overset{L_\mathrm{loc}^p}{\longrightarrow}f$ (which means that $\norm{f_n-f}_{L^p(K)}\to 0$ for any compact set $K\subseteq \Omega$). Then, $T_{f_n}$ converges to $T_f$ in $\mathcal{D}(\Omega)$.
  \end{theorem}
  \begin{proof}
    Use \mnameref{RFA:holder}.
  \end{proof}
  \begin{remark}
    Clearly if $f_n$ converge uniformly to $f$, the condition of the theorem holds and we get the same result. But it can be seen that only with pointwise convergence is not enough (consider $f_n(x)=n^kx^n(1-x)\indi{[0,1]}$ for $k\in\NN$). Moreover, $T_{f_n}$ converges to $T_f$ in $\mathcal{D}(\Omega)$ does not imply pointwise convergence of $f_n$ towards $f$.
  \end{remark}
  \subsubsection{Support of a distribution}
  \begin{definition}
    Let $T\in\mathcal{D}^*(\RR^n)$. We define the \emph{support} of $T$, $\supp T$, as the intersection of all closed sets $K$ such that if $\varphi\in\mathcal{D}(\RR^n)$ has support in $\RR^n\setminus K$, then $\langle T, \varphi\rangle=0$.
  \end{definition}
  \begin{lemma}\label{HA:supp_well_def}
    Let $T\in\mathcal{D}^*(\RR^n)$ and $\varphi\in\mathcal{D}(\RR^n)$ be such that $\supp \varphi \cap \supp T=\varnothing$. Then, $\langle T, \varphi\rangle=0$.
  \end{lemma}
  \begin{proof}
    Assume $\supp T=\bigcap_{i\in I} C_i$. Then, by the compactness of $\supp\varphi$, there exists $i_1,\ldots,i_n\in I$ such that:
    $$
      \supp\varphi\subseteq \bigcup_{j=1}^n (\RR^n\setminus C_{i_j})
    $$
    Now take a partition of unity $\psi_1,\ldots,\psi_n\in\mathcal{D}(\RR^n)$ subordinated to the open cover $\{\RR^n\setminus C_{i_j}:j=1,\ldots,n\}$ (check \mcref{DG:partition_of_unity}). These $\psi_j$ satisfy (by definition) that $\supp\psi_j\subseteq \RR^n\setminus C_{i_j}$ for all $j=1,\ldots,n$ and $\sum_{j=1}^{n}\psi_j= 1$ on $\supp\varphi$. Therefore, defining $\psi:=\sum_{j=1}^n \psi_j$ we have that $\psi\in\mathcal{D}(\RR^n)$ and $\varphi\psi=\varphi$ on $\supp\varphi$. Therefore:
    $$
      \langle T,\varphi\rangle = \langle T,\varphi\psi\rangle = \sum_{j=1}^n \langle T,\varphi\psi_j\rangle = 0
    $$
  \end{proof}
  \begin{definition}
    We denote $\mathcal{E}(\RR^n):=\mathcal{C}^\infty(\RR^n)$ and $\mathcal{E}^*(\RR^n)$ its dual space.
  \end{definition}
  \begin{definition}
    Let $T\in\mathcal{D}^*(\RR^n)$ with compact support. We can extent the definition of $T$ to $\mathcal{E}(\RR^n)$ in the following way. Let $\varphi\in\mathcal{E}(\RR^n)$ and take $\rho\in\mathcal{D}(\RR^n)$ such that $\rho=1$ on $\supp T$. Then, we define:
    $$
      \langle T,\varphi\rangle := \langle T,\rho\varphi\rangle
    $$
  \end{definition}
  \begin{remark}
    Note that in view of \mcref{HA:supp_well_def}, this definition is well defined because if $\rho,\omega\in\mathcal{D}(\RR^n)$ are two different test functions such that $\rho,\omega=1$ on $\supp T$, then $\varphi(\rho- \omega)=0$ on $\supp T$ and therefore $\langle T,\varphi(\rho- \omega)\rangle=0$.
  \end{remark}
  \begin{proposition}
    Let $T\in\mathcal{D}^*(\RR^n)$ with compact support. Then, $T\in \mathcal{E}^*(\RR^n)$ if and only if $\exists C>0$, $N\in\NN$ and $m\in \NN\cup\{0\}$ such that:
    $$
      \abs{\langle T,\varphi\rangle}\leq C\sum_{\abs{\alpha}\leq m} \sup_{\norm{\vf{x}}\leq N}\abs{\partial^\alpha\varphi(\vf{x})}
    $$
    for all $\varphi\in\mathcal{E}(\RR^n)$.
  \end{proposition}
  \begin{proof}
    The implication to the left is clear. For the other one, from the continuity in $\mathcal{D}^*(\RR^n)$ we know that for all compact $K$, there exist $C>0$ and $m\in\NN\cup\{0\}$ such that:
    $$
      \abs{\langle T,\varphi\rangle}= \abs{\langle T,\rho\varphi\rangle}\leq C\sum_{\abs{\alpha}\leq m} \sup_{\vf{x}\in K}\abs{\partial^\alpha(\rho\varphi)(\vf{x})}
    $$
    Now take $N>0$ such that $\supp\rho\subseteq\supp T\subseteq B(0,N)$. Thus:
    \begin{multline*}
      \abs{\langle T,\varphi\rangle}\leq C\sum_{\abs{\alpha}\leq m} \sup_{\vf{x}\in \supp\rho\leq N}\abs{\partial^\alpha\varphi(\vf{x})}\leq\\
      \leq C\sum_{\abs{\alpha}\leq m} \sup_{\norm{\vf{x}}\leq N}\abs{\partial^\alpha\varphi(\vf{x})}
    \end{multline*}
  \end{proof}
  \subsubsection{Differentiation of distributions}
  \begin{definition}
    Let $\Omega\subseteq \RR^n$ be a set, $T\in\mathcal{D}^*(\Omega)$ and $\alpha$ be a multiindex. We define the distribution $\partial^\alpha T$ as: $$\dotp{\partial^\alpha T}{\varphi}=\dotp{T}{{(-1)}^\abs{\alpha}\partial^\alpha\varphi}$$
    for all $\varphi\in\mathcal{D}(\Omega)$. The distribution $\partial^\alpha T$ is called \emph{distributional derivative}.
  \end{definition}
  \begin{definition}
    We define the \emph{Heaviside step function} as the function $H(x)=\indi{x>0}$.
  \end{definition}
  \begin{proposition}
    We have that $T_{H}=:H\in\mathcal{D}^*(\RR)$ and: $$H'=\delta$$
  \end{proposition}
  \begin{proof}
    For all $\varphi\in\mathcal{D}(\Omega)$ we have:
    \begin{equation*}
      \dotp{H'}{\varphi}=-\dotp{H}{\varphi'} =-\int_{0}^\infty\varphi'(x)\dd{x}= \varphi(0)=\delta(\varphi)
    \end{equation*}
    because $\varphi$ has compact support.
  \end{proof}
  \begin{lemma}
    Let $f\in L_{\text{loc}}^1(\RR^n)$. Then, ${(T_f)}'=T_{f'}$.
  \end{lemma}
  \begin{proposition}[Schwarz theorem]
    Let $\Omega\subseteq \RR^n$ be a set and $T\in\mathcal{D}^*(\Omega)$. Then:
    $$
      \frac{\partial^2 T}{\partial x_i\partial x_j}=\frac{\partial^2 T}{\partial x_j\partial x_i}
    $$
  \end{proposition}
  \begin{proposition}[Leibnitz rule]
    Let $\Omega\subseteq \RR^n$ be a set, $T\in\mathcal{D}^*(\Omega)$, $f\in \mathcal{C}^\infty(\Omega)$ and $\alpha$ be a multiindex. Then:
    $$
      \partial^\alpha (fT)=\sum_{\beta\leq \alpha}\binom{\alpha}{\beta}\partial^\beta f\partial^{\alpha-\beta} T
    $$
  \end{proposition}
  \begin{proposition}
    Let $T\in \mathcal{D}^*(\RR)$ be such that $T'=0$. Then, $T$ is constant (in the sense of distributions).
  \end{proposition}
  \begin{proof}
    Let $\varphi\in\mathcal{D}(\RR)$ with $\int_\RR\varphi=0$. Then, $\phi(x):=\int_{-\infty}^x\varphi(t)\dd{t}\in\mathcal{D}(\RR)$ and $\phi'=\varphi$. Thus:
    $$
      \dotp{T}{\varphi}=\dotp{T}{\phi'}=\dotp{T'}{\phi}=0
    $$
    Now consider a general $\varphi\in \mathcal{D}(\RR)$ and $\omega \in \mathcal{D}(\RR)$ such that $\int_\RR \omega = 1$. Then, $\phi(x):=\varphi - \omega \int_\RR\varphi$ integrates 0 and thus:
    $$
      \dotp{T}{\varphi} =\int_\RR\varphi\dotp{T}{\omega} = \dotp{C}{\varphi}
    $$
    with $C:=\dotp{T}{\omega}$.
  \end{proof}
  \begin{proposition}\label{HA:xmT_equal_0}
    Let $T\in \mathcal{D}^*(\RR)$ be such that $x^m T=0$ for some $m\in\NN$. Then, $T=\sum_{j=0}^{m-1}a_j\delta^{(j)}$ for some $a_j\in\RR$.
  \end{proposition}
  \begin{proof}
    Let $\varphi\in\mathcal{D}(\RR)$ with Taylor polynomial:
    $$
      P_\varphi (x)=\sum_{j=0}^{m-1}\frac{\varphi^{(j)}(0)}{j!}x^j+\frac{\varphi^{(m)}(\xi_x)}{m!}x^m
    $$
    with $\xi_x\in(0,x)$. Then:
    \begin{align*}
      \langle T,\varphi\rangle & =\sum_{j= 0}^{m-1} \frac{\varphi^{(j)}(0)}{j!}\langle T,x^j\rangle+\frac{1}{m!}\langle x^m T,\varphi^{(m)}(\xi_x)\rangle \\
                               & =\sum_{j= 0}^{m-1} a_j\langle \delta^{(j)},\varphi\rangle
    \end{align*}
    with $a_j=\frac{{(-1)^j}}{j!}\langle T,x^j\rangle$.
  \end{proof}
  \subsubsection{Schwartz class of functions}
  \begin{definition}
    Let $d\in\NN$. The \emph{Schwartz space} or \emph{space of rapidly decreasing functions} on $\RR^n$ is defined as:
    $$
      \mathcal{S}(\RR^d):=\{f\in \mathcal{C}^\infty(\RR^n):\norm{f}_{\alpha,\beta}<\infty\ \forall\alpha,\beta\in{(\NN\cup\{0\})}^d\}
    $$
    where:
    $$
      \norm{f}_{\alpha,\beta}:=\sup_{\vf{x}\in\RR^n}\abs{\vf{x}^\alpha (\partial^\beta f)(\vf{x})}
    $$
  \end{definition}
  \begin{lemma}
    Let $f\in \mathcal{S}(\RR^d)$. Then, ${\vf{x}}^\alpha f,\partial^\alpha f\in \mathcal{S}(\RR^d)$ for all $\alpha\in{(\NN\cup\{0\})}^d$.
  \end{lemma}
  \begin{lemma}
    Let $d\in\NN$. Then, $\mathcal{D}(\RR^d)\subset\mathcal{S}(\RR^d)\subset \mathcal{E}(\RR^d)$.
  \end{lemma}
  \begin{definition}
    Let $f,(f_n)\in \mathcal{S}(\RR^d)$. We say that $f_n\overset{\mathcal{S}}{\longrightarrow}f$ if $\norm{f_n-f}_{\alpha,\beta}\to 0$ for all $\alpha,\beta\in{(\NN\cup\{0\})}^d$.
  \end{definition}
  \begin{proposition}
    Let $d\in\NN$. Then, $\mathcal{S}(\RR^d)\subset L^p(\RR^d)$ for all $p\in[1,\infty]$.
  \end{proposition}
  \begin{proof}
    For $p=\infty$ the result is clear. Now suppose that $p\in[1,\infty)$ and let $\phi\in \mathcal{S}(\RR^d)$. Then:
    \begin{align*}
      \int_{\RR^d}\abs{\phi}^p & = \int_{B(0,1)}\abs{\phi}^p+\int_{\RR^d\setminus B(0,1)}\frac{\abs{\phi}^p\norm{\vf{x}}^{kp}}{\norm{\vf{x}}^{kp}} \\
                               & \leq C_1+C_2\int_{\RR^d\setminus B(0,1)}\frac{1}{\norm{\vf{x}}^{kp}}
    \end{align*}
    for some $k\in \NN$ yet to be determined. Here in the last step we have used $\abs{\abs{\phi}\norm{\vf{x}}^{k}}\leq \norm{f}_{k,0}=:C_2$. Now if $R_j:=\{\vf{x}\in\RR^d:2^j\leq \norm{\vf{x}}\leq 2^{j+1}\}$, $j\in\NN\cup\{0\}$, then:
    \begin{align*}
      \int_{\RR^d\setminus B(0,1)}\frac{1}{\norm{\vf{x}}^{kp}} & \leq \sum_{j=0}^\infty \int_{R_j}\frac{1}{\norm{\vf{x}}^{kp}} \\
                                                               & \leq\sum_{j=0}^\infty \frac{C2^{(j+1)d}}{2^{kpj}}             \\
                                                               & <\infty
    \end{align*}
    if and only if $kp-d>0$. So take $k> \frac{d}{p}$.
  \end{proof}
  \begin{remark}
    In $\RR^n$, the integrals of the form $\displaystyle\int_{B(0,1)}\frac{1}{\norm{\vf{x}}^k}\dd{\vf{x}}$ converge if and only if $k<n=\dim\RR^n$ whereas the integrals of the form $\displaystyle\int_{\RR^n\setminus B(0,1)}\frac{1}{\norm{\vf{x}}^k}\dd{\vf{x}}$ converge if and only if $k>n$. The limit case $k=n$ is diverges in both cases.
  \end{remark}
  \begin{lemma}
    Let $f$ be a function that has Fourier transform. Then, $f\in \mathcal{S}(\RR^d)\iff \widehat{f} \in \mathcal{S}(\RR^d)$.
  \end{lemma}
  \begin{proof}
    By symmetry, it suffices to do one implication. Moreover we will only do the case $d=1$ in order to keep the notation simple. Let $f\in \mathcal{S}(\RR)$ and $\alpha,\beta\in\NN\cup\{0\}$. Then, using \mcref{HA:diffFourierTransf,HA:diffFourierXf}:
    \begin{align*}
      \abs{\vf\xi^\alpha\partial^\beta \widehat{f}(\vf\xi)} & =\abs{\vf\xi^\alpha\F({(-2\pi\ii \vf{x})}^\beta f)(\vf\xi)}                                    \\
                                                            & =\frac{1}{\abs{2\pi\ii}^\alpha}\abs{\F[\partial^\alpha ({(-2\pi\ii \vf{x})}^\beta f)](\vf\xi)} \\
                                                            & \leq \norm{\partial^\alpha ({(-2\pi\ii \vf{x})}^\beta f)}_1                                    \\
                                                            & <\infty
    \end{align*}
    where in the last inequality we have used that $\norm{\widehat{g}}_\infty\leq \norm{g}_1$.
  \end{proof}
  \subsubsection{Tempered distributions}
  \begin{definition}
    A \emph{tempered distribution} is a linear and continuous operator $T:\mathcal{S}(\RR^d)\to \CC$. The space of all tempered distributions is denoted by $\mathcal{S}^*(\RR^d)$.
  \end{definition}
  \begin{lemma}
    Let $T:\mathcal{S}(\RR^d)\to \CC$ be linear. Then, $T\in\mathcal{S}^*(\RR^d)$ if and only if there exists $C>0$ and $m\in\NN\cup\{0\}$ such that $\forall \varphi\in \mathcal{S}(\RR^d)$ we have:
    $$
      \abs{T(\varphi)}\leq C\sum_{\abs{\alpha}+\abs{\beta}\leq m}\norm{\varphi}_{\alpha,\beta}
    $$
  \end{lemma}
  \begin{lemma}\label{HA:LpSD}
    $L^p(\RR^d)\subset\mathcal{S}^*(\RR^d)\subset \mathcal{D}^*(\RR^d)$.
  \end{lemma}
  \begin{lemma}
    Let $T\in \mathcal{S}^*(\RR^d)$. Then, $\partial^\alpha T\in \mathcal{S}^*(\RR^d)$ for all $\alpha\in{(\NN\cup\{0\})}^d$.
  \end{lemma}
  \begin{definition}
    Let $T\in \mathcal{S}^*(\RR^d)$ and $\psi\in \mathcal{S}(\RR^d)$. We define the \emph{convolution} $T*\psi$ as:
    $$
      \langle T*\psi,\varphi\rangle:=\langle T,\tilde\psi*\varphi\rangle
    $$
    for any $\varphi\in \mathcal{S}(\RR^d)$. Here $\tilde\psi(\vf{x}):=\psi(-\vf{x})$.
  \end{definition}
  \begin{lemma}\label{HA:lemma_aMbM}
    Let $a,b\geq 0$ and $m\in\NN\cup\{0\}$. Then:
    $$
      {(a+b)}^m\leq 2^{m-1}(a^m+b^m)
    $$
    and the equality holds if and only if $a=b$ or $m=0,1$.
  \end{lemma}
  \begin{proof}
    For $m=0,1$, the equality is true. Now suppose $m\geq 2$ and $b=\lambda a$ with $\lambda\in [0,\infty)$. We need to show that:
    $$
      {(1+\lambda)}^m\leq 2^{m-1}(1+\lambda^m)
    $$
    Consider $f(\lambda):=2^{m-1}(1+\lambda^m)-{(1+\lambda)}^m$. Then:
    $$
      f'(\lambda)=m\left[{(2\lambda)}^{m-1}-{(1+\lambda)}^{m-1}\right]
    $$
    Note that $\forall m\geq 2$, $f'(\lambda)<0$ for $\lambda\in[0,1)$, $f'(1)=0$ and $f'(\lambda)>0$ for $\lambda\in(1,\infty)$. Moreover $f(1)=0$. So $f(\lambda)\geq 0$ for all $\lambda\in [0,\infty)$ and the equality holds if and only if $\lambda=1$.
  \end{proof}
  \begin{lemma}
    Let $T\in \mathcal{S}^*(\RR^d)$ and $\psi\in \mathcal{S}(\RR^d)$. Then, $T*\psi\in \mathcal{S}^*(\RR^d)$.
  \end{lemma}
  \begin{proof}
    Clearly $T*\psi$ is linear. Let $\varphi_n\overset{\mathcal{S}}{\longrightarrow}0$. Then, it suffices to see that $\tilde\psi*\varphi_n\overset{\mathcal{S}}{\longrightarrow}0$. For the sake of simplicity we only do the case $d=1$. For all $\alpha,\beta\in{\NN\cup\{0\}}$ we have:
    \begin{align*}
      \abs{\vf{x}^\alpha \partial^\beta(\tilde\psi*\varphi_n)(\vf{x})} & =\abs{\vf{x}^\alpha (\partial^\beta\tilde\psi*\varphi_n)(\vf{x})}                                                                  \\
                                                                       & \leq \int_{\RR^d} \abs{\vf{x}^\alpha \partial^\beta\psi(\vf{y})\varphi_n(\vf{x}-\vf{y})}\dd{\vf{y}}                                \\
      \begin{split}
         & \leq 2^m\!\!\int_{\RR^d}\!\abs{\vf{x}-\vf{y}}^\alpha \abs{\partial^\beta\psi(\vf{y})}\abs{\varphi_n(\vf{x}-\vf{y})}\!\dd{\vf{y}} \\
         & \quad\;+2^m\int_{\RR^d} \abs{\vf{y}}^\alpha\abs{\partial^\beta\psi(\vf{y})} \abs{\varphi_n(\vf{x}-\vf{y})}\dd{\vf{y}}            \\
      \end{split} \\
      \begin{split}
         & \leq 2^m\sup_{\vf{x}\in\RR^d}\abs{\vf{x}}^\alpha\abs{\varphi_n(\vf{y})}\int_{\RR^d}  \abs{\partial^\beta\psi(\vf{y})}\dd{\vf{y}}  \\
         & \;\;\;+2^m\sup_{\vf{x}\in\RR^d}\abs{\varphi_n(\vf{y})}\int_{\RR^d} \abs{\vf{y}}^\alpha\abs{\partial^\beta\psi(\vf{y})}\dd{\vf{y}} \\
      \end{split}
    \end{align*}
    where in the second inequality we have used \mcref{HA:lemma_aMbM} with $m=\abs{\alpha}+1$. Note that this latter terms tend to zero as $n\to\infty$ because of the properties of the Schwartz space.
  \end{proof}
  \begin{lemma}
    Let $T\in \mathcal{S}^*(\RR^d)$, $\psi\in \mathcal{S}(\RR^d)$ and $\alpha\in{(\NN\cup\{0\})}^d$. Then:
    $$
      \partial^\alpha(T*\psi)=\partial^\alpha T*\psi=T*\partial^\alpha\psi
    $$
  \end{lemma}
  \begin{proof}
    \begin{multline*}
      \langle \partial^\alpha(T*\psi),\varphi\rangle={(-1)}^{\abs{\alpha}}\langle T*\psi,\partial^\alpha\varphi\rangle={(-1)}^{\abs{\alpha}}\langle T,\tilde\psi*\partial^\alpha\varphi\rangle\\={(-1)}^{\abs{\alpha}}\langle T,\partial^\alpha(\tilde\psi*\varphi)\rangle
      =\langle \partial^\alpha T,\tilde\psi*\varphi\rangle   =\langle \partial^\alpha T*\psi,\varphi\rangle
    \end{multline*}
    The other equality is analogous.
  \end{proof}
  \subsubsection{Fourier transform of distributions}
  \begin{definition}
    Let $T\in \mathcal{S}^*(\RR^d)$. We define the \emph{Fourier transform} $\widehat{T}$ (or $\F{T}$) of $T$ as
    $$
      \langle \widehat{T},\varphi\rangle:=\langle T,\widehat{\varphi}\rangle
    $$
    for all $\varphi\in \mathcal{S}(\RR^d)$. We define the \emph{inverse Fourier transform} $\F^{-1}{T}$ of $T$ as
    $$
      \langle \F^{-1}{T},\varphi\rangle:=\langle T,\F^{-1}{\varphi}\rangle
    $$
  \end{definition}
  \begin{lemma}
    Let $T\in \mathcal{S}^*(\RR^d)$. Then, $\F{T},\F^{-1}T\in \mathcal{S}^*(\RR^d)$.
  \end{lemma}
  \begin{proposition}\label{HA:diff_FourierTrans_distribution}
    Let $T\in \mathcal{S}^*(\RR^d)$, $\psi \in \mathcal{S}(\RR^d)$ and $\alpha\in{(\NN\cup\{0\})}^d$. Then:
    \begin{enumerate}
      \item $\partial^\alpha\widehat{T}=\F({{(-2\pi\ii \vf{x})}^{\alpha} T})$
      \item $\widehat{\partial^\alpha T}={(2\pi\ii \vf\xi)}^{\alpha}\widehat{T}$
      \item $\widehat{T*\psi}=\widehat{T}\widehat{\psi}$
    \end{enumerate}
  \end{proposition}
  \begin{proof}
    We prove the third one. The other are similar. We have:
    \begin{multline*}
      \langle \widehat{T*\psi},\varphi\rangle=\langle T*\psi,\widehat{\varphi}\rangle=\langle T,\tilde\psi*\widehat{\varphi}\rangle=\langle \widehat{T},\F^{-1}(\tilde{\psi}\widehat{\varphi})\rangle=\\
      =\langle \widehat{T},\F^{-1}(\tilde{\psi})\varphi\rangle=\langle \widehat{T},\widehat{\psi}\varphi\rangle=\langle \widehat{T}\widehat{\psi},\varphi\rangle
    \end{multline*}
  \end{proof}
  \begin{lemma}\label{HA:lemma_propdelta}
    We have that:
    \begin{enumerate}
      \item $\widehat{\delta_{\vf{a}}}=\exp{-2\pi\ii \vf{a}\cdot\vf{x}}$
      \item $\widehat{\indi{(0,\infty)}}=\frac{1}{2\pi\ii}\pv\left(\frac{1}{x}\right)+\frac{1}{2}\delta_0$
      \item $\widehat{\pv\left(\frac{1}{x}\right)}=-\pi \ii \sign(\xi)$
      \item $\delta*f=f$ $\forall f\in \mathcal{D}(\RR^d)$
    \end{enumerate}
  \end{lemma}
  \begin{proof}
    We prove the second one, the others are easier. Note that ${(\indi{(0,\infty)})}'=\delta$. Taking Fourier transform and using \mcref{HA:diff_FourierTrans_distribution}, we get
    $ 2\pi\ii x \widehat{\indi{(0,\infty)}}=1$. Hence, since $x\pv\left(\frac{1}{x}\right)=1$, we have:
    $$
      x\left(2\pi\ii \widehat{\indi{(0,\infty)}}-\pv\left(\frac{1}{x}\right)\right)=0
    $$
    By \mcref{HA:xmT_equal_0}, we get that $\widehat{\indi{(0,\infty)}}=\frac{1}{2\pi\ii}\pv\left(\frac{1}{x}\right)+C\delta_0$, for some $C\in \RR$. To find the constant change $x\to -x$ in the equation or alternatively apply the distribution to the function $\exp{-\pi x^2}$.
  \end{proof}
  \begin{definition}
    Let $T\in \mathcal{D}^*(\RR^n)$ and $S\in \mathcal{D}^*(\RR^m)$. We define the \emph{direct product} $TS$ as the distribution in $\mathcal{D}^*(\RR^{n+m})$ given by:
    $$
      \langle TS,\varphi\rangle=\langle T,\langle S ,\varphi(\vf{x},\cdot)\rangle\rangle
    $$
    for all $\varphi\in \mathcal{D}(\RR^{n+m})$. Usually we will dentote
    $$\langle TS,\varphi(\vf{x},\vf{y})\rangle=\langle T(\vf{x}),\langle S(\vf{y}) ,\varphi(\vf{x},\vf{y})\rangle\rangle$$
    in order to distinguish the variables
  \end{definition}
  \begin{lemma}
    Let $T\in \mathcal{D}^*(\RR^n)$ and $S\in \mathcal{D}^*(\RR^m)$. Then:
    \begin{enumerate}
      \item $\phi(\vf{x}):=\langle S(\vf{y}), \varphi(\vf{x},\vf{y})\rangle\in \mathcal{D}(\RR^n)$ and $\forall \alpha\in{(\NN\cup\{0\})}^n$ we have $$\partial_{\vf{x}}^\alpha\phi(\vf{x})=\langle S(\vf{y}), \partial_{\vf{x}}^\alpha\varphi(\vf{x},\vf{y})\rangle$$
      \item $TS$ is indeed a distribution.
      \item $TS=ST$
    \end{enumerate}
  \end{lemma}
  \begin{proposition}
    Let $T\in \mathcal{D}^*(\RR^n)$ and $S\in \mathcal{D}^*(\RR^m)$. Then, $\widehat{TS}=\widehat{T}\widehat{S}$.
  \end{proposition}
  \begin{proof}
    Given $\varphi(\vf{x},\vf{y})\in \mathcal{D}(\RR^{n+m})$ we have:
    \begin{align*}
      \F(\varphi(\vf{x},\vf{y})) & =\int_{\RR^{n+m}}\varphi(\vf{x},\vf{y})e^{-2\pi\ii(\vf{x},\vf{y})\cdot(\vf{\xi},\vf{\eta})}\dd{\vf{x}}\dd{\vf{y}}                                      \\
                                 & = \int_{\RR^{n}}e^{-2\pi\ii\vf{x}\cdot\vf{\xi}}\left(\int_{\RR^{m}}\varphi(\vf{x},\vf{y})e^{-2\pi\ii\vf{y}\cdot\vf{\eta}}\dd{\vf{y}}\right)\dd{\vf{x}} \\
                                 & =:\F_{\vf{x}}(\F_{\vf{y}}(\varphi))=\F_{\vf{y}}\F_{\vf{x}}\varphi
    \end{align*}
    by \mnameref{RFA:fubini}.
    Therefore:
    \begin{multline*}
      \langle \F(TS),\varphi\rangle=\langle T,\langle S,\F_{\vf{y}}\F_{\vf{x}}\varphi\rangle\rangle= \langle T,\langle \widehat{S}, \F_{\vf{x}}\varphi\rangle\rangle=\\=\langle \widehat{S}T, \F_{\vf{x}}\varphi \rangle=\langle \widehat{S}, \langle T, \F_{\vf{x}}\varphi \rangle\rangle=\langle \widehat{S}, \langle \widehat{T}, \varphi \rangle\rangle=\langle \widehat{S}\widehat{T}, \varphi \rangle
    \end{multline*}
  \end{proof}
  \subsubsection{Homogeneous distributions}
  \begin{definition}[Homogeneous distribution]
    A distribution $T\in \mathcal{S}^*(\RR^n)$ is said to be \emph{homogeneous of degree $r\in \RR$} if:
    $$
      \langle T,\varphi(\lambda\vf{x})\rangle=\lambda^{- n -r}\langle T,\varphi(\vf{x})\rangle
    $$
    for all $\lambda>0$ and all $\varphi\in \mathcal{S}(\RR^n)$.
  \end{definition}
  \begin{proposition}
    Let $T\in \mathcal{S}^*(\RR^n)$ be a homogeneous distribution of degree $r\in\RR$. Then, $\partial^\alpha T$ is homogeneous of degree $r-\abs{\alpha}$ for all $\alpha\in{(\NN\cup\{0\})}^n$.
  \end{proposition}
  \begin{proof}
    Let $\varphi\in \mathcal{S}(\RR^n)$ and $\lambda>0$. Then:
    \begin{multline*}
      \langle \partial^\alpha T,\varphi(\lambda\vf{x})\rangle={(-1)}^{\abs{\alpha}}\langle T,(\partial^\alpha\varphi)(\lambda\vf{x})\lambda^{\abs{\alpha}}\rangle=\\={(-1)}^{\abs{\alpha}}\lambda^{-n-r+\abs{\alpha}}\langle T,\partial^\alpha\varphi\rangle=\lambda^{-n-(r-\abs{\alpha})}\langle \partial^\alpha T,\varphi\rangle
    \end{multline*}
  \end{proof}
  \begin{proposition}
    Let $T\in \mathcal{S}^*(\RR^n)$ and $r\in \RR$. Then, $T$ is homogeneous of degree $r$ if and only if $\widehat{T}$ is homogeneous of degree $-n-r$.
  \end{proposition}
  \begin{proof}
    We only check one implication, the other is analogous. Using \mcref{HA:fourierPropertiesRn} we have:
    \begin{multline*}
      \langle \widehat{T}, \varphi(\lambda\vf{x})\rangle=\left\langle T, \widehat{\varphi}(\vf{\xi}/\lambda)\frac{1}{\lambda^n}\right\rangle =\\=\lambda^{n+r - n}\left\langle T, \widehat{\varphi}\right\rangle=\lambda^{-n-(-r-n)}\langle \widehat{T}, \varphi\rangle
    \end{multline*}
  \end{proof}
  \begin{corollary}\label{lem:preRiesz2}
    Let $k\in\RR$, $n\in\NN$ with $k<n$. Then, $\frac{1}{\norm{\vf{x}}^k}\in \mathcal{S}^*(\RR^n)$ and it is homogeneous of degree $-k$. Moreover, $\F\left(\frac{1}{\norm{\vf{x}}^k}\right) = C_{k,n}\frac{1}{\norm{\vf{\xi}}^{n-k}}$ with $C_{k,n}=\frac{{(2\pi)}^{\frac{n}{2}}}{{2}^{\frac{\alpha}{2}}}\frac{\Gamma\left(\frac{n-k}{2}\right)}{\Gamma\left(\frac{k}{2}\right)}$.
  \end{corollary}
  \subsubsection{Differential operators over distributions}
  \begin{definition}
    A \emph{differential operator over distributions} is an operator of the form:
    $$
      L(x,\partial):=\sum_{\abs{\alpha}\leq m}a_\alpha(x)\partial^\alpha
    $$
    If the coefficients $a_\alpha$ are constant, we will omit the $x$ and write $L(\partial)$ instead of $L(x,\partial)$.
  \end{definition}
  \begin{definition}
    Let $L(x,\partial)$ be a differential operator on an open set $U\subseteq\RR^d$ and $f\in \mathcal{D}^*(U)$. We say that $u\in\mathcal{D}^*(\RR^n)$ is a \emph{generalized solution} of $L(\vf{x},\partial)u=f$ in $U$ if
    $$
      \langle L(\vf{x},\partial)u,\varphi\rangle=\langle f,\varphi\rangle
    $$
    for all $\varphi\in \mathcal{D}(U)$.
  \end{definition}
  \begin{definition}
    Let $L(\partial)$ be a differential operator. We say that $E\in\mathcal{D}^*(\RR^n)$ is a \emph{fundamental solution} of $L(\partial)$ if $L(\partial)E=\delta$.
  \end{definition}
  \begin{theorem}
    Let $E$ be a fundamental solution of $L(\partial)u=f$, $f\in\mathcal{S}(\RR^n)$. Then, $E*f$ is a generalized solution of $L(\partial)u=f$.
  \end{theorem}
  \begin{proof}
    \begin{multline*}
      L(\partial)(E*f)=\sum_{\abs{\alpha}\leq m}a_\alpha\partial^\alpha(E*f)=\sum_{\abs{\alpha}\leq m}a_\alpha(\partial^\alpha E)*f=\\=\delta*f=f
    \end{multline*}
    where in the last equality we have used \mcref{HA:lemma_propdelta}.
  \end{proof}
  \begin{remark}
    In general we don't have unicity of fundamental solutions. Indeed if $E_0$ solves $L(\partial)u=0$ and $E$ is a fundamental solution, then $E+E_0$ is also a fundamental solution.
  \end{remark}
  \begin{theorem}
    Let $L(\partial)$ be a differential operator and $E\in\mathcal{S}^*(\RR^n)$. Then, $E$ is a fundamental solution of $L(\partial)$ if and only if $L(2\pi\ii\vf\xi)\widehat{E}=1$.
  \end{theorem}
  \begin{proof}
    Suppose $E$ is a fundamental solution. Then, $L(\partial)E=\delta$. Taking Fourier transforms we have:
    $$
      \sum_{\abs{\alpha}\leq m}a_\alpha\F{(\partial^\alpha E)}=\sum_{\abs{\alpha}\leq m}a_\alpha{(2\pi\ii\vf\xi)}^\alpha\widehat{E}=L(2\pi\ii\vf\xi)\widehat{E}=1
    $$
    The other implication is similar using $\F^{-1}$ instead.
  \end{proof}
  \begin{definition}
    Let $L(\partial)$ be a differential operator. We say that $E$ is a \emph{fundamental solution} of the Cauchy problem
    \begin{equation}\label{HA:cauchy}
      \begin{cases}
        L(\partial)u(t,\vf{x})=0 \\
        u(0,\vf{x})= f(\vf{x})
      \end{cases}
    \end{equation}
    if $L(\partial)E(t,\vf{x})=0$ and $E(0,\vf{x})=\delta(\vf{x})$.
  \end{definition}
  \begin{theorem}
    Let $L(\partial)$ be a differential operator and $E$ be a fundamental solution of the Cauchy problem of \mcref{HA:cauchy}. Then, $E*f$ is a solution of it.
  \end{theorem}
  \subsubsection{Applications to some PDEs}
  \begin{proposition}\label{HA:heat}
    Consider the operator:
    $$
      L(\partial)=\partial_t-a^2\sum_{j=1}^{n}\partial_{x_j}^2
    $$
    Then, $$
      E(t)=\frac{\indi{[0,\infty)}(t)}{{(4\pi a^2t)}^{n/2}}\exp{-\frac{\norm{\vf{x}}^2}{4a^2t}}
    $$
    is a fundamental solution the heat equation $L(\partial)u=0$.
  \end{proposition}
  \begin{proof}
    Taking $\F_{\vf{x}}$ on the equation $L(\partial) E=\delta$ we can transform it to:
    $$
      \partial_t \widehat{E}+4\pi^2a^2\norm{\vf\xi}^2\widehat{E}=\delta_t
    $$
    because $\delta=\delta_{\vf{x}}\delta_t$. It can be seen that a solution of this ODE is:
    $$
      \widehat{E}(t,\xi)=\indi{[0,\infty)}(t)\exp{-4\pi^2a^2\norm{\vf\xi}^2t}
    $$
    Taking the Fourier transform (in this case $\F^{-1}=\F$) we have:
    $$
      E(t,x)=\frac{\indi{[0,\infty)}(t)}{{(4\pi a^2t)}^{n/2}}e^{-\frac{\norm{\vf{x}}^2}{4a^2t}}
    $$
    where we have used \mcref{HA:FTprop4,HA:expX2}.
  \end{proof}
  \begin{proposition}
    Consider the Laplace operator:
    $$
      L(\partial)=\sum_{j=1}^{n}\partial_{x_j}^2
    $$
    Then, $$
      E(t)=\begin{cases}
        \frac{\Gamma\left(\frac{n}{2}-1\right)}{\pi^{\frac{n}{2}-2}}\frac{1}{\norm{\vf{x}}^{n-2}} & \text{if }n\geq 3 \\
        \frac{\log\norm{\vf{x}}}{2\pi}                                                            & \text{if }n=2     \\
      \end{cases}
    $$
    is a fundamental solution the Laplace equation $L(\partial)u=0$.
  \end{proposition}
  \begin{proof}
    Taking $\F_{\vf{x}}$ on the equation $L(\partial) E=\delta$ we see that $\widehat{E}$ satisfies:
    $$
      \widehat{E}(\vf\xi) = \frac{-1}{4\pi^2\norm{\vf\xi}^2}
    $$
    Let's study the integrability of this latter function in a neighbourhood of $0$. Let $R_j:=\{\vf{x}\in\RR^n:2^{-j}\leq \norm{x}\leq 2^{-j+1}\}$. Then:
    \begin{multline*}
      \int_{B(0,1)}\frac{1}{\norm{\vf{x}}^2}=\sum_{j=1}^{\infty}\int_{R_j} \frac{1}{\norm{\vf{x}}^2}\lesssim \sum_{j=1}^{\infty}\frac{{(2^{-j})}^n}{2^{-2j}}=\\= \sum_{j=1}^{\infty}2^{j(n-2)}<\infty\iff n\geq 3
    \end{multline*}
    Let's study first the case $n\geq 3$. We need to compute $\F^{-1}(\widehat{E})=\F(\widehat{E})$. Recall that $\F(\exp{-k\norm{\vf{x}}^2})= {\left(\frac{\pi}{k}\right)}^{\frac{n}{2}}\exp{-\frac{\pi^2\norm{\vf\xi}^2}{k}}$ (try to generalize \mcref{HA:expX2}). Therefore:
    \begin{multline*}
      {\left(\frac{\pi}{k}\right)}^{\frac{n}{2}}\int_{\RR^n} \exp{-\frac{\pi^2\norm{\vf\xi}^2}{k}}\varphi(\vf{\xi})\dd{\vf{\xi}}=\langle \F(\exp{-k\norm{\vf{x}}^2}),\varphi\rangle=\\=\int_{\RR^n} \exp{-k\norm{\vf{x}}^2}\widehat{\varphi}(\vf{x})\dd{\vf{x}}
    \end{multline*}
    Integrating both sides with respect to $k$ and using \mnameref{RFA:fubini} we have that, on the one hand:
    \begin{equation*}
      \int_{\RR^n} \widehat{\varphi}(\vf{\xi})\int_0^\infty\exp{-k\norm{\vf{\xi}}^2}\dd{k}\dd{\vf{\xi}}=\int_{\RR^n} \widehat{\varphi}(\vf{\xi})\frac{1}{\norm{\vf\xi}^2}\dd{\vf{\xi}}=\left\langle \frac{1}{\norm{\vf\xi}^2}, \widehat\varphi\right\rangle
    \end{equation*}
    On the other hand:
    $$\int_{\RR^n}\varphi(\vf{x})\int_{0}^\infty{\left(\frac{\pi}{k}\right)}^{\frac{n}{2}} \exp{-\frac{\pi^2\norm{\vf{x}}^2}{k}}\dd{k}\dd{\vf{x}}=\frac{\Gamma\left(\frac{n}{2}-1\right)}{\pi^{\frac{n}{2}-2}}\int_{\RR^n}\frac{\varphi(\vf{x})}{\norm{\vf{x}}^{n-2}}$$
    where we have used the change of variable $r=\frac{\pi^2\vf{\xi}^2}{k}$. Let's do now the case $n=2$. Consider $E_n=\frac{1}{2}\log(\norm{\vf{x}}^2+1/n^2)\overset{\mathcal{S}}{\longrightarrow}\log\norm{\vf{x}}$ (by the \mnameref{RFA:dominated}). Hence, we have that $\laplacian E_n = \frac{2n^2}{(n^2\norm{\vf{x}}^2+1)^2}\overset{\mathcal{S}}{\longrightarrow}\laplacian\log\norm{\vf{x}}$. Thus, $\forall \varphi\in \mathcal{S}(\RR^2)$:
    \begin{align*}
      \langle \laplacian\log\norm{\vf{x}},\varphi\rangle & = \lim_{n\to\infty}\langle \laplacian E_n,\varphi\rangle                                        \\
                                                         & = \lim_{n\to\infty}\int_{\RR^2} \frac{2n^2}{(n^2\norm{\vf{x}}^2+1)^2}\varphi(\vf{x})\dd{\vf{x}} \\
                                                         & = \lim_{n\to\infty}\int_{\RR^2} \frac{2}{(\norm{\vf{x}}^2+1)^2}\varphi(\vf{x}/n)\dd{\vf{x}}     \\
                                                         & =\varphi(0)\int_{\RR^2}\frac{2}{(\norm{\vf{x}}^2+1)^2}\dd{\vf{x}}                               \\
                                                         & =2\pi\varphi(0)
    \end{align*}
    where in the forth equality we have used the \mnameref{RFA:dominated} and at the end we have calculated the integral using polar coordinates.
  \end{proof}
  \begin{corollary}
    Consider the Cauchy-Riemann operators:
    $$
      \partial_z=\frac{1}{2}\left(\partial_x-i\partial_y\right)\qquad \partial_{\overline{z}}=\frac{1}{2}\left(\partial_x+i\partial_y\right)
    $$
    The fundamental solutions to $\partial_zu=f$ and $\partial_{\overline{z}}u=f$ are respectively:
    $$
      E=\frac{1}{\pi}\frac{1}{\overline{z}}\qquad E=\frac{1}{\pi}\frac{1}{z}
    $$
  \end{corollary}
  \begin{sproof}
    Recall that $\partial_z\partial_{\overline{z}}=\partial_{\overline{z}}\partial_z=\frac{1}{4}\laplacian$. $E=\frac{\log(z\overline{z})}{4\pi}$ is a fundamental solution of the Laplace equation and $\partial_z E=\frac{1}{4\pi z}$, $\partial_{\overline{z}} E=\frac{1}{4\pi \overline{z}}$.
  \end{sproof}
  \begin{proposition}
    Consider the Cauchy problem:
    $$
      \begin{cases}
        u_t=a^2\laplacian u   & \text{in }(0,\infty)\times\RR^n \\
        u(0,\vf{x})=f(\vf{x}) & \text{in }\RR^n
      \end{cases}
    $$
    Then, a fundamental solution is given by:
    $$
      E(t,\vf{x})=\frac{1}{{(4\pi a^2t)}^{\frac{n}{2}}}\exp{-\frac{\norm{\vf{x}}^2}{4a^2t}}
    $$
    And the general solution is:
    $$
      u(t,\vf{x})=(E*f)(t,\vf{x})=\int_{\RR^n} \frac{f(\vf{y})}{{(4\pi a^2t)}^{\frac{n}{2}}}\exp{-\frac{\norm{\vf{x}-\vf{y}}^2}{4a^2t}}\dd{\vf{y}}
    $$
  \end{proposition}
  \begin{proof}
    Taking $\F_{\vf{x}}$ on the equation we obtain:
    $$
      \widehat{E}_t=-4\pi^2\norm{\vf{\xi}}^2\widehat{E}
    $$
    Solving it we obtain, $\widehat{E}=C\exp{-4\pi^2\norm{\vf{\xi}}^2t}$. Using the initial condition we see that $C=1$. Now proceeding as in the proof of the \mcref{HA:heat} we obtain the result.
  \end{proof}
  \begin{theorem}[Malgrange-Ehrenpreis theorem]
    Every non-zero linear partial differential operator with constant coefficients has a fundamental solution.
  \end{theorem}
  \subsection{Singular intergals}
  \subsubsection{Hilbert transform}
  \begin{definition}
    Let $f\in L^p(\RR)$, $1\leq p<\infty$. The \emph{truncated Hilbert transform} is defined as:
    $$
      \H^\varepsilon f(x)=\frac{1}{\pi}\int_{\abs{x-y}>\varepsilon}\frac{f(y)}{x-y}\dd{y}
    $$
  \end{definition}
  \begin{definition}
    Let $f\in \mathcal{S}(\RR)$. We define the \emph{Hilbert transform} of $f$ as:
    $$
      \H f(x)=\frac{1}{\pi}\left(\pv\left(\frac{1}{x}\right)*f\right)(x)=\lim_{\varepsilon\to 0} \H^\varepsilon f(x)
    $$
  \end{definition}
  \begin{remark}
    We can extend the definition of $\H$ to functions that satisfy locally a \emph{Hölder condition}: $\forall x\in\RR$, $\exists C_x,\alpha_x,\delta_x>0$ such that $$\abs{f(x)-f(y)}\leq C_x\abs{x-y}^{\alpha_x}\qquad\text{for all}\abs{x-y}<\delta_x$$
    In that case we write:
    $$
      \H^\varepsilon f(x)=\frac{1}{\pi}\!\int_{\varepsilon<\abs{x-y}<\delta_x}\!\!\frac{f(y)-f(x)}{x-y}\dd{y} + \frac{1}{\pi}\!\int_{\abs{x-y}>\delta_x}\!\!\frac{f(y)}{x-y}\dd{y}
    $$

  \end{remark}
  \begin{lemma}
    Let $a,b\in\RR$, $a<b$. Then:
    $$
      \H(\indi{[a,b]})(x)=\frac{1}{\pi}\log\left| \frac{x-a}{x-b}\right|
    $$
  \end{lemma}
  \begin{proposition}
    Let $f\in L^p(\RR)$, $1\leq p<\infty$. Then:
    $$
      \F(\H f)(\xi)=-\ii\sign(\xi)\F f(\xi)=:m(\xi)\F f(\xi)
    $$
  \end{proposition}
  \begin{proof}
    \begin{multline*}
      \F(\H f)(\xi)=\frac{1}{\pi}\F\left(\pv\left(\frac{1}{x}\right)*f\right)(\xi)=\\
      =\frac{1}{\pi}\ii \widehat{\pv\left(\frac{1}{x}\right)}\widehat{f}(\xi)=m(\xi)\widehat{f}(\xi)
    \end{multline*}
  \end{proof}
  \begin{lemma}
    Let $f\in L^2(\RR)$. Then, $\H f\in L^2(\RR)$ and $\norm{\H f}_{2}= \norm{f}_{2}$.
  \end{lemma}
  \begin{proof}
    Using \mnameref{HA:plancherel}:
    $$
      \norm{\H f}_{2} = \norm{\widehat{\H f}}_{2}=\norm{m\widehat{f}}_{2}=\norm{\widehat{f}}_{2}=\norm{f}_{2}
    $$
  \end{proof}
  \begin{lemma}
    We have that $\H^2=-\id$ on $L^p(\RR)$, $1\leq p<\infty$.
  \end{lemma}
  \begin{proof}
    $\displaystyle
      \H^2 f=\H(\F^{-1}(m\widehat{f}))=\F^{-1}(m^2\widehat{f})=-f
    $
  \end{proof}
  \begin{lemma}\label{HA:prethmSingInt}
    Let $f\in \mathcal{S}(\RR)$. Then, ${(\H f)}^2=f^2+2\H(f\H f)$.
  \end{lemma}
  \begin{proof}
    We'll prove the equality using the Fourier transform and the uniqueness of it will imply the result.

    In general we have that $\widehat{fg}=\widehat{f}*\widehat{g}$ because:
    $$
      \F^{-1}(\widehat{f}*\widehat{g})=\F^3(\widehat{f}*\widehat{g})=\F^2(\F^2 f \F^2 g)=fg
    $$
    Thus:
    $$
      \widehat{f^2}= \widehat{f}*\widehat{f}\qquad
      2\F{(\H(f\H f))}=2m \widehat{f}*(m \widehat{f})
    $$
    The first term is $\int_{\RR} \widehat{f}(\eta)\widehat{f}(\xi-\eta)\dd{\eta}$ whereas the second one is $2\int_{\RR} m(\xi)\widehat{f}(\eta)m(\xi-\eta)\widehat{f}(\xi-\eta)\dd{\eta}=2\int_{\RR} m(\xi)\widehat{f}(\xi-\eta)\widehat{f}(\eta)m(\eta)\dd{\eta}$.
    Averaging those terms we have:
    \begin{align*}
      \begin{split}
        \widehat{f^2}+2\F{(\H(f\H f))} & =\int_{\RR} \widehat{f}(\eta)\widehat{f}(\xi-\eta)[1+m(\xi)\cdot \\
                                       & \hspace{2cm}\cdot(m(\xi-\eta)+m(\eta))]\dd{\eta}
      \end{split} \\
       & =\int_{\RR} \widehat{f}(\eta)\widehat{f}(\xi-\eta)m(\xi-\eta)m(\eta)\dd{\eta}                              \\
       & =\widehat{\H f}*\widehat{\H f}=\widehat{(\H f)^2}
    \end{align*}
    where the second equality follows for all $\xi,\eta\in\RR^2\setminus\{(0,0)\}$
  \end{proof}
  \begin{theorem}[Riesz theorem]
    Let $f\in L^p(\RR)$, $1<p<\infty$. Then, $\exists C_p>0$ such that:
    $$
      \norm{\H f}_p \leq C_p \norm{f}_p
    $$
  \end{theorem}
  \begin{proof}
    We will prove only the cases $p=2^k$, $k\in\NN$ and we'll do it by induction. The case $k=1$ is clear. Using \mcref{HA:prethmSingInt} we have:
    \begin{multline*}
      {\norm{\H f}_{2p}}^2=\norm{{(\H f)}^2}_p\leq \norm{f^2}_p+2\norm{\H(f\H f)}_p\leq\\\leq {\norm{f}_{2p}}^2+2C_p\norm{f\H f}_{p}\leq {\norm{f}_{2p}}^2+2C_p\norm{f}_{2p}\norm{\H f}_{2p}
    \end{multline*}
    where the last inequality follows from the \mref{RFA:cauchyschwarz}. This reduces to find for which $y$ we have $y^2 - 2C_p \alpha y - \alpha^2\leq 0$, where $\alpha={\norm{f}_{2p}}^2$. An easy check shows that:
    $$
      \norm{\H f}_{2p}\leq \left(C_p + \sqrt{C_p^2+1}\right)\norm{f}_{2p}
    $$
  \end{proof}
  \begin{lemma}
    Let $P_y$ be the Poisson kernel and $f\in L^p(\RR)$. Then, if $z=x+\ii y$:
    $$
      (P_y*f)(z)=\Re\left(\frac{\ii}{\pi}\int_\RR\frac{f(t)}{z-t}\dd{t}\right)=:\Re F_f(z)
    $$
    Moreover, $F_f\in\mathcal{H}(\{\Im f>0\})$.
  \end{lemma}
  \begin{sproof}
    The first part follows from:
    $$
      (P_y*f)(z)=\frac{y}{\pi}\int_\RR\frac{f(t)}{{(x-t)}^2+y^2}\dd{t}
    $$
    To show the last part, note that $F_f$ is $\RR$-differentiable and $\overline{\partial}F_f=0$.
  \end{sproof}
  \begin{definition}
    We define the \emph{conjugate Poisson kernel} $Q_y$ as:
    $$
      Q_y(x)=\frac{x}{\pi(x^2+y^2)}
    $$
  \end{definition}
  \begin{lemma}
    Let $f\in L^p(\RR)$, $1\leq p <\infty$. Then, if $z=x+\ii y$:
    $$
      \Im F_f(z) =\int_{\RR} \frac{f(t)(x-t)}{{(x-t)}^2+y^2}\dd{t}
    $$
  \end{lemma}
  \begin{theorem}
    Let $f\in L^p(\RR)$, $1\leq p <\infty$. Then:
    $$
      f*Q_\varepsilon -\H^\varepsilon f\overset{L^p}{\underset{\varepsilon\to 0}{\longrightarrow}} 0
    $$
    In particular:
    $$
      F_\varphi(x+\ii y) \overset{y\to 0}{\longrightarrow} \varphi(x) + \ii \H \varphi(x)
    $$
  \end{theorem}
  \subsubsection{Riesz transform}
  \begin{definition}
    We define $W_j:=\pv\left(\frac{x_j}{\norm{\vf{x}}^{n+1}}\right)$, $j=1,\ldots,n$.
  \end{definition}
  \begin{lemma}
    For each $n\in\NN$ and $j=1,\ldots,n$, $W_j\in\mathcal{S}^*(\RR^n)$.
  \end{lemma}
  \begin{definition}
    We define the \emph{Riesz transform} $R_jf$ as:
    \begin{multline*}
      R_jf(\vf{x}):=c_n(W_j*f)(\vf{x})=\\=\lim_{\varepsilon\to 0}c_n\int_{\norm{\vf{x}-\vf{y}} > \varepsilon}\frac{x_j-y_j}{\norm{\vf{x}-\vf{y}}^{n+1}}f(\vf{y})\dd{\vf{y}}
    \end{multline*}
    with $c_n=\frac{\Gamma\left(\frac{n+1}{2}\right)}{\pi^{\frac{n+1}{2}}}$.
  \end{definition}
  \begin{lemma}\label{lem:preRiesz1}
    $\partial_j\left(\frac{1}{\norm{\vf{x}}^{n-1}}\right)=(1-n)W_j$
  \end{lemma}
  \begin{proof}
    Let $\varphi\in\mathcal{S}(\RR^n)$ and write it as $\varphi=\varphi_\mathrm{e}+\varphi_\mathrm{o}$, where $\varphi_\mathrm{e}$ is even and $\varphi_\mathrm{o}$ is odd. Then:
    \begin{align*}
      \left\langle \partial_j\left(\frac{1}{\norm{\vf{x}}^{n-1}}\right),\varphi\right\rangle & = -\int_{\RR^n} \frac{\partial_j\varphi_\mathrm{o}}{\norm{\vf{x}}^{n-1}}\dd{\vf{x}}                                                   \\
                                                                                             & =-\lim_{\varepsilon \to 0}\int_{\RR^n\setminus B(0,\varepsilon)} \frac{\partial_j\varphi_\mathrm{o}}{\norm{\vf{x}}^{n-1}} \dd{\vf{x}} \\
                                                                                             & = (1-n)\lim_{\varepsilon \to 0}\int_{\RR^n\setminus B(0,\varepsilon)} \frac{x_j\varphi_\mathrm{o}}{\norm{\vf{x}}^{n+1}} \dd{\vf{x}}   \\
                                                                                             & = (1-n)\lim_{\varepsilon \to 0}\int_{\RR^n\setminus B(0,\varepsilon)} \frac{x_j\varphi}{\norm{\vf{x}}^{n+1}} \dd{\vf{x}}
    \end{align*}
  \end{proof}
  \begin{theorem}
    For each $j=1, \ldots, n$ we have: $$\widehat{W_j}=-\frac{\ii}{c_n}\frac{\xi_j}{\norm{\vf\xi}}$$
  \end{theorem}
  \begin{proof}
    Using \mcref{lem:preRiesz1,lem:preRiesz2} we have:
    \begin{multline*}
      \widehat{W_j}=\frac{1}{1-n}\F{(\partial_j\norm{\vf{x}}^{1-n})}=\frac{2\pi\ii\xi_j}{1-n} \widehat{\norm{\vf{x}}^{1-n}}=\\=\frac{2\pi\ii\xi_j}{1-n}\frac{\pi^{\frac{n-1}{2}}}{\Gamma\left(\frac{n-1}{2}\right)}\frac{1}{\norm{\vf{\xi}}}=\frac{\ii\pi^{\frac{n+1}{2}}}{\Gamma\left(\frac{n+1}{2}\right)}\frac{\xi_j}{\norm{\vf{\xi}}}
    \end{multline*}
  \end{proof}
  \begin{corollary}
    For each $j=1, \ldots, n$ we have: $$\F{(R_jf)}(\vf{\xi})=-\ii\frac{\xi_j}{\norm{\vf{\xi}}} \widehat{f}(\vf{\xi})$$
  \end{corollary}
  \begin{proposition}\hfill
    \begin{enumerate}
      \item $\displaystyle\sum_{j=1}^{n}R_j^2=-\id$
      \item For all $1\leq j,k\leq n$, $\partial_j\partial_k=R_jR_k\Delta$
    \end{enumerate}
  \end{proposition}
  \begin{sproof}
    Apply $\F$ on each of the equations and use the Fourier transform properties.
  \end{sproof}
\end{multicols}
\end{document}