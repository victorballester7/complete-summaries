\documentclass[../../../main.tex]{subfiles}


\begin{document}
\renewcommand{\col}{\ana}
\begin{multicols}{2}[\section{Real and functional analysis}]
  \subsection{Lebesgue integral}
  \subsubsection{Measures}
  \begin{definition}[$\sigma$-algebra]
    Let $\Omega$ be a set and $\Sigma\subseteq\mathcal{P}(\Omega)$. We say that $\Sigma$ is a \emph{$\sigma$-algebra} over $\Omega$ if:
    \begin{enumerate}
      \item $\Omega\in\Sigma$.
      \item If $A\in\Sigma$, then $A^c\in\Sigma$.
      \item If $A_1,A_2,\ldots\in\Sigma$, then: $$\bigcup_{n=1}^\infty A_n\in\Sigma$$
    \end{enumerate}
  \end{definition}
  \begin{proposition}
    Let $\Sigma$ be an $\sigma$-algebra over a set $\Omega$. Then:
    \begin{enumerate}
      \item $\varnothing\in\Sigma$.
      \item If $A,B\in\Sigma$, then $A\setminus B\in\Sigma$.
      \item If $A_1,A_2,\ldots\in\Sigma$, then: $$\bigcap_{n=1}^\infty A_n\in\Sigma$$
    \end{enumerate}
  \end{proposition}
  \begin{definition}[Measure]
    Let $\Sigma$ be a $\sigma$-algebra over a set $\Omega$. A \emph{measure} over $\Omega$ is any function $$\mu:\Sigma\longrightarrow[0,\infty]$$ satisfying the following properties:
    \begin{itemize}
      \item $\mu(\varnothing)=0$.
      \item \emph{$\sigma$-additivity}: If $\{A_n:n\geq1\}\subseteq\Sigma$ are pairwise disjoint, then: $$\mu\left(\bigsqcup_{n=1}^\infty A_n\right)=\sum_{n=1}^\infty \mu(A_n)$$
    \end{itemize}
  \end{definition}
  \begin{definition}
    Let $\Sigma$ be a set and $\{A_n:n\geq1\}\subseteq\Sigma$ be subsets. We say that $A_n\nearrow A$ if $A_n\subseteq A_{n+1}$ $\forall n\in\NN$ and $A=\bigcup_{n=1}^\infty A_n$. Analogously, we say that $A_n\searrow A$ if $A_n\supseteq A_{n+1}$ $\forall n\in\NN$ and $A=\bigcap_{n=1}^\infty A_n$.
  \end{definition}
  \begin{proposition}
    Let $\Sigma$ be a $\sigma$-algebra over a set $\Omega$, $\mu:\Sigma\longrightarrow[0,\infty]$ be a measure over $\Omega$ and $A_n,A,B\in\Sigma$, $n\in\NN$. Then:
    \begin{itemize}
      \item If $A\subseteq B$, then $\mu(A)\subseteq\mu(B)$.
      \item If $A_n\nearrow A$, then $\displaystyle\mu(A)=\lim_{n\to\infty} A_n$.
      \item If $A_n\searrow A$ and $\mu(A_1)<\infty$, then $\displaystyle\mu(A)=\lim_{n\to\infty} A_n$.
    \end{itemize}
  \end{proposition}
  \begin{definition}
    An \emph{interval} $I\subseteq\RR^n$ is a set of the form:
    $$I=\abs{a_1,b_1}\times\cdots\times\abs{a_n,b_n}$$
    where $a_i,b_i\in\RR_\infty$ and the notation $\abs{a,b}$ represents either $(a,b)$, $[a,b)$, $(a,b]$ or $[a,b]$.
  \end{definition}
  \begin{definition}
    Let $I=\prod_{i=1}^n\abs{a_i,b_i}\subseteq\RR^n$ be an interval. We define its \emph{volume} as:
    $$\vol(I):=\prod_{i=1}^n(b_i-a_i)$$
  \end{definition}
  \begin{definition}
    Let $m\in\NN\cup\{0\}$. We define the \emph{$m$-th dyadic cube} as the set: $$\mathcal{D}_m:=[a_1,a_1+2^{-m})\times [a_n,a_n+2^{-m})$$
    where $a_i\in 2^{-m}\ZZ$\footnote{Note that each $\mathcal{D}_m$ forms a partition of $\RR^n$}.
  \end{definition}
  \begin{lemma}
    Let $m\in\NN\cup\{0\}$. Then the sidelength of $\mathcal{D}_m$ is $2^{-m}$, $\vol(\mathcal{D}_m)= 2^{-mn}$ and its diameter is $(\mathcal{D}_m)= 2^{-m}\sqrt{n}$.
  \end{lemma}
  \begin{proposition}
    All nonempty open set $U\subseteq\RR^n$ can be written as a countable union of disjoint dyadic cubes whose closure is in $U$.
  \end{proposition}
  \subsection{Measure of sets in \texorpdfstring{$\RR^n$}{Rn}}
  \subsubsection{Outer measure}
  \begin{definition}
    Let $A\subseteq\RR^n$ be a set. We denote by $\mathcal{I}(A)$ the set of covers of $A$ that are intervals. Analogously, we denote by $\mathcal{I}_0(A)$ the set of open covers of $A$ that are intervals.
  \end{definition}
  \begin{definition}[Outer measure]
    Let $A\subseteq\RR^n$ be a set. We define its \emph{outer measure} as the measure $m^*$ defined by:
    $$m^*(A):=\inf\left\{\sum_{k= 1}^\infty \vol(I_k):\{I_k:k\geq 1\}\in \mathcal{I}(A)\right\}$$
  \end{definition}
  \begin{proposition}
    Let $A\subseteq\RR^n$ be a set. Then:
    $$m^*(A)=\inf\left\{\sum_{k= 1}^\infty \vol(I_k):\{I_k:k\geq 1\}\in \mathcal{I}_0(A)\right\}$$
  \end{proposition}
  \begin{theorem}
    The outer measure has the following properties:
    \begin{enumerate}
      \item $m^*(\varnothing)=0$.
      \item If $A\subseteq B\subseteq\RR^n$, then $m^*(A)\leq m^*(B)$.
      \item If $A_1, A_2,\ldots, \subseteq \RR^n$, then: $$ m^*\left(\bigcup_{k=1}^\infty A_k\right)\leq \sum_{k=1}^\infty m^*(A_k)$$
      \item If $I\subseteq \RR^n$ is an open interval and $I\subseteq A\subseteq \cl{I}$, then $m^*(A)=\vol(I)$.
      \item If $I_1,\ldots,I_N\subseteq \RR^n$ are disjoint intervals, then $\displaystyle m^*\left(\bigcup_{k=1}^N I_k\right)\leq \sum_{k=1}^N \vol(I_k)$.
      \item If $A,B\subseteq \RR^n$ and $d(A,B):=\inf\{d(a,b):a\in A,b\in B\}>0$, then $m^*(A\cup B)=m^*(A)+m^*(B)$.
      \item If $A\subseteq\RR^n$ and $x\in\RR^n$, then $m^*(A+x)=m^*(-A)=m^*(A)$\footnote{Here $A+x:=\{a+x:a\in A\}$ and $-A:=\{-a:a\in A\}$}.
    \end{enumerate}
  \end{theorem}
  \begin{lemma}
    Let $I,J_1,\ldots,J_N\subseteq \RR^n$ be intervals such that $I\subseteq \bigcup_{k=1}^N J_k$. Then, $\vol(I)\leq\sum_{k=1}^\infty \vol(J_k)$.
  \end{lemma}
\end{multicols}
\end{document}