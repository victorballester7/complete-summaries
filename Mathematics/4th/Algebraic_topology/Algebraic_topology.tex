\documentclass[../../../main.tex]{subfiles}


\begin{document}
\renewcommand{\col}{\geo}
\begin{multicols}{2}[\section{Algebraic topology}]
  	\subsection{Homotopy and fundamental group}
  	\subsubsection{Homotopy between maps and spaces}
  	
  	\begin{definition}
	  	Let $X,Y$ be two topological spaces \footnote{From now on $X$, $Y$ and $Z$ will always be topological spaces} and let $f,g: X \to Y$ be two continuous maps. A homotopy between $f$ and $g$ is a continuous map $H:X\times [0,1] \to Y$ such that for all $x\in X$
	  	\begin{enumerate}
		  		\item $H(x,0)=f(x)$
		  		\item $H(x,1)=g(x)$.
		\end{enumerate}
		We say that $f$ and $g$ are homotopic if there exists a homotopy between them, and we denote it by $f\simeq g$.
	\end{definition}
	  
	\begin{lemma}
	    Let $X=A\cup B$, where $A$ and $B$ are closed sets. If $\varphi: A \to Y$ and $\psi: B \to Y$ are continuous maps such that $\left.\varphi \right|_{A\cap B}=\left.\psi \right|_{A\cap B}$, then the map $$x\mapsto \begin{cases}
	    	\varphi(x) & \quad\text{if  } x\in A \\
	    	\psi(x)  & \quad\text{otherwise}\\
		\end{cases}$$ is also continuous.
	\end{lemma}
	
	\begin{proposition}
		The relation $\simeq$ is an equivalence relation between continuous maps.
    \end{proposition}
		
	\begin{definition}
		A continuous map $f:X\to Y$ is called a homotopy equivalence if there exists a map $g:Y\to X$ such that $f\circ g\simeq id_Y$ and $g\circ f \simeq id_X$. We say that $X$ and $Y$ are homotopy equivalent or that they have the same type of homotopy if there exists a homotopy equivalence between them, and we denote it by $X\simeq Y$.
	\end{definition}

  	\begin{proposition}
		The relation $\simeq$ is an equivalence relation between topological spaces.
	\end{proposition}

	\begin{definition}
		A space having the homotopy type of a point is called contractible.
	\end{definition}
	
	\subsubsection{Paths and foundamental group}
	
	\begin{definition}
		A path in $X$ is a continuous map $\sigma:[0,1]\to X$.
	\end{definition}

	\begin{definition}
		Let $\sigma, \tau : [0,1]\to X$ be two paths such that $\sigma(0)=\tau(0)$ and $\sigma(1)=\tau(1)$. A homotopy between them is a continuous map $H:[0,1]\times [0,1] \to X$ such that
		\begin{enumerate}
			\item $H(s,0)=\sigma(s) $ $\forall s\in [0,1]$
			\item $H(s,1)=\tau(s)$ $\forall s\in [0,1]$
			\item $H(0,t)=\sigma(0)=\tau(0)$ $\forall t\in [0,1]$
			\item $H(1,t)=\sigma(1)=\tau(1)$ $\forall t\in [0,1]$
		\end{enumerate}
	When there exists a homotopy between $\sigma$ and $\tau$ we say that they are homotopic, and we write $\sigma \simeq \tau $.
	\end{definition}

	\begin{proposition}
		The relation $\simeq$ is an equivalence relation between paths.
	\end{proposition}

	\begin{definition}
		Let $\sigma, \tau: [0,1] \to X$ such that $\sigma(1)=\tau(0)$. We define the product of $\sigma, \tau$ as 
		\begin{align*}
			\sigma \cdot \tau: [0,1] &\longrightarrow X \\
			x&\longmapsto \begin{cases}
				\sigma(2s) & 0\leq s \leq \frac{1}{2}\\
				\tau(2s-1)  & \frac{1}{2}\leq s \leq 1\\
			\end{cases}.
		\end{align*}
	
	\end{definition}

	\begin{lemma}\label{ProdCamins}
		Let $\sigma, \sigma', \tau, \tau':[0,1]\to X$ be paths such that 
		\begin{enumerate}
			\item $\sigma(0)=\sigma'(0)$
			\item $\sigma(1)=\sigma'(1)=\tau(0)=\tau'(0)$
			\item $\tau(1)=\tau'(1)$.
		\end{enumerate}
		If $\sigma \simeq \sigma'$ and $\tau \simeq \tau'$ then $\sigma \cdot \tau \simeq \sigma' \cdot \tau'$.
	\end{lemma}

	\begin{definition}
		A loop at the basepoint $x_0$ is a path $\sigma: [0,1]\to X$ such that $\sigma(0)=\sigma(1)=x_0$.
	\end{definition}

	\begin{definition}
		The foundamental group of $X$ at the basepoint $x_0$ is the set of all homotopy classes of loops at the basepoint $x_0$, and it is denoted by $\pi_1 (X,x_0)$. We define $[\sigma]\cdot [\tau]:=[\sigma \cdot \tau]$, which is well-defined by \cref{ProdCamins}.
	\end{definition}

	\begin{proposition}
		$\big(\pi_1(X,x_0), \cdot \big)$ is a group.
	\end{proposition}
	
	\begin{proposition}
		Let $x_0, y_0 \in X$. If there exists a path $\gamma: [0,1] \to X$ from $x_0$ to $y_0$ then $\pi_1 (X,x_0) \cong \pi_1 (X,y_0)$.
	\end{proposition}
	
	\begin{proof}
		\begin{align*}
			\varphi: \pi_1(X,x_0) &\longrightarrow \pi_1(X,y_0) \\
			[\sigma]&\longmapsto [\gamma \cdot \sigma \cdot \gamma^{-1}]\\
		\end{align*}
		is an isomorphism.
	\end{proof}
	
	\begin{definition}
		Let $f:X\to Y$ be a continuous map, and $x_0 \in X$. We define 
		\begin{align*}
			f_*: \pi_1 (X,x_0) & \longrightarrow \pi_1 (Y,f(x_0)) \\
			[\alpha] &\longmapsto [f\circ \alpha].
		\end{align*}
		
	\end{definition}

	\begin{proposition}
		Let $f,g:X\to Y$ and $h:Y \to Z $ be continuous maps. Then
		\begin{enumerate}
			\item $f_*$ is well-defined and it is a group morphism
			\item if $f(x_0)=g(x_0)$ then $f_*=g_*$
			\item $(h\circ f)_*=h_*\circ f_*$
		\end{enumerate}
	\end{proposition}

	\begin{corollary}
		$X\simeq Y \implies \pi_1 (X,x_0) \cong \pi_1 (Y,f(x_0))$.
	\end{corollary}


	\subsubsection{Foundamental group of $S^1$}
	
	\begin{lemma}
		Let (X,d) be a compact metric space, and $U_1, ..., U_n$ a finite cover of open sets of $X$. Then there exists $\epsilon>0$ such that $\forall x \in X$, $B(x,\epsilon)\subset U_i$ for some $i$.
	\end{lemma}

	\begin{proof}
		Let $C_i=X\setminus U_i$.  Then the map 
		
		\begin{align*}			
			f:X &\longrightarrow [0, +\infty) \\
			x &\longmapsto \frac{1}{n} \sum d(x,C_i)			
		\end{align*}
		
		is a continuous map on a compact set and therefore reaches the minimum at $X$. $\varepsilon=\min f$ satisfies the desired properties.
	\end{proof}
	

\end{multicols}
\end{document}